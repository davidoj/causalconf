%!TEX root = main.tex


\section{Variables and Probability Models}\label{sec:vague_variables}

\subsection{Section outline}

This section introduces the mathematical foundations used throughout the rest of the paper. The first subsection briefly introduces probability theory, which is likely to be familiar to many readers, as well as how string diagrams can be used to represent probabilistic functions (or \emph{Markov kernels}), which may be less familiar. We use string diagrams for probabilstic reasoning in a number of places, and this section is intended to help interpret mathematical statements in this form.

The second subsection discusses the interpretation of probabilistic variables. Our formalisation of probabilistic variables is standard -- we define them as measurable functions on a fundamental probability set $\Omega$. We discusses how this formalisation can be connected to statements about the real world via \emph{measurement processes}, and distinguishes observed variables (which are associated with measurement processes) from unobserved variables (which are not associated with measurement processes). This section is not part of the mathematical theory of probability gap models, but it is relevant when one wants to apply this theory to real problems or to understand how the theory of probability gap models relates to other theories of causal inference.

Finally, we introduce \emph{probability gap models}. Probability gap models are a generalisation of probability models, and to understand the rest of this paper a reader needs to understand what a probability gap model is, how we define the common kinds of probability gap models used in this paper and what conditional probabilities and conditional independence statements mean for probability gap models.

\subsubsection{Brief outline of probability gap models}

We consider a probability model to be a probability space $(\Omega,\sigalg{F},\mu)$ along with a collection of random variables. However, if I want to use probabilistic models to support decision making, then I need function from options to probability models. For example, suppose I have two options $A=\{0,1\}$, and I want to compare these options based on what I expect to happen if I choose them. If I choose option $0$, then I can (perhaps) represent my expectations about the consequences with a probability model, and if I choose option $1$ I can represent my expectations about the consequences with a different probability model. I can compare the two consequences, then decide which option seems to be better. To make this comparison, I have used a function from elements of $A$ to probability models. A function that takes elements of some set as inputs (which may or may not be decisions) and returns probability models is a \emph{probability gap model}, and the set of inputs it accepts is a \emph{probability gap}.

We are particularly interested in probability gap models where the consequences of all inputs share some marginal or conditional probabilities. The simplest example of a model like this can be represented by a probability distribution $\prob{P}^{\RV{X}}$ for some variable $\RV{X}:\Omega\to X$. Such a probability distribution is consistent with many base measures on the fundamental probability set $\Omega$, and so we can consider the choice of base measure to be a probability gap. Not every probability distribution over $X$ can define a probability gap model in this way. In particular, we need $\prob{P}^{\RV{X}}$ to assign probability 0 to outcomes that are mathematically impossible according to the definition of $\RV{X}$ to ensure that there is some base measure that features $\prob{P}^{\RV{X}}$ as a marginal. We call probability gap models represented by probability distributions \emph{order 0 probability gap models}.

Higher order probability gap models can be represented by conditional probabilities $\prob{P}^{\RV{Y}|\RV{X}}$ or pairs of conditional probabilities $\{\prob{P}^{\RV{X}|\RV{W}},\prob{P}^{\RV{Z}|\RV{WXY}}\}$, which we call \emph{order 1} and \emph{order 2} models respectively. Decision functions in data-driven decision problems correspond to probability gaps in order 2 models, as we discuss in Section \ref{sec:seedo_models}, which makes this type of model particularly interesting for our purposes. We also require these to be valid, and we define conditions for validity and prove that they are sufficient to ensure that models represented by conditional probabilities can in fact be mapped to base measures on the fundamental probability set.

A conditional independence statement in a probability gap model means that the corresponding conditional independence statement holds for all base measures in the range of the function defined by the model. It is possible to deduce conditional independences from ``independences'' in the conditional probabilities that we use to represent these models, and conditional independences can imply the existence of conditional probabilities with certain independence properties.

We can consider causal Bayesian networks to represent order 2 probability gap models. That is, a causal Bayesian network represents a function $\prob{P}$ that take inserts from some set $A$ of conditional probabilities and returns a probability model, and it does so in such a way that there are a pair of conditional probabilities $\{\prob{P}^{\RV{X}|\RV{W}},\prob{P}^{\RV{Z}|\RV{WXY}}\}$ shared by all models in the codomain of $\prob{P}$. The observational distribution is the value of $\prob{P}(\text{obs})$ for some \emph{observational insert} $\text{obs}\in A$, and other choices of inserts yield interventional distributions. Defining causal Bayesian networks in this manner resolves two areas of difficulty with causal Bayesian networks. First, under the standard definition of causal Bayesian networks interventional probabilities may fail to exist; with our perspective we can see that this arises due to misunderstanding the domain of $\prob{P}$. Secondly, there may be multiple distributions that differ in important ways that all satisfy the standard definition of ``interventional distributions''. The one-to-many relationship between observations and interventions is a basic challenge of causal inference, the problem arises when this relationship is obscured by calling multiple different things ``the interventional distribution''. If we consider causal Bayesian networks to represent order 2 probability gap models, we avoid doing this. 


\subsection{Semantics of observed and unobserved variables}\label{sec:variables}

We are interested in constructing \emph{probabilistic models} which explain some part of the world. In a model, variables play the role of ``pointing to the parts of the world the model is explaining''. Both observed an unobserved variables play important roles in causal modelling and we think it is worth clarifying what variables of either type refer to. Our approach is a standard one: a probabilistic model is associated with an experiment or measurement procedure that yields values in a well-defined set. Observable variables are obtained by applying well-defined functions to the result of this total measurement. We use a richer fundamental probability set that includes ``unobserved variables'' that are formally treated the same way as observed variables, but aren't associated with any real-world counterparts.

Consider Newton's second law in the form $\proc{F}=\proc{MA}$ as a simple example of a model that relates ``variables'' $\proc{F}$, $\proc{M}$ and $\proc{A}$. As \citet{feynman_feynman_1979} noted, this law is incomplete -- in order to understand it, we must bring some pre-existing understanding of force, mass and acceleration as independent things. Furthermore, the nature of this knowledge is somewhat perculiar. Acknowledging that physicists happen to know a great deal about forces on an object, it remains true that in order to actually say what the net force on a real object is, even a highly knowledgeable physicist will still have to go and do some measurements, and the result of such measurements will be a vector representing the net forces on that object.

This suggests that we can think about ``force'' $\proc{F}$ (or mass or acceleration) as a kind of procedure that we apply to a particular real world object and which returns a mathematical object (in this case, a vector). We will call $\proc{F}$ a \emph{procedure}. Our view of $\proc{F}$ is akin to \citet{menger_random_2003}'s notion of variables as ``consistent classes of quantities'' that consist of pairing between real-world objects and quantities of some type. Force $\proc{F}$ itself is not a well-defined mathematical thing, as measurement procedures are not mathematically well-defined. At the same time, the set of values it may yield \emph{are} well-defined mathematical things. No actual procedure can be guaranteed to return elements of a mathematical set known in advance -- anything can fail -- but we assume that we can study procedures reliable enough that we don't lose much by making this assumption.

\begin{definition}[Measurement procedure]
A \emph{measurement procedure} is a procedure that involves interacting with the real world somehow and delivering an element of a mathematical set as a result. The set of possible values is known prior to the measurement taking place, but the value that it will yield is not known. A procedure is given the font $\proc{B}$, we say it takes values in $X$ and $\proc{B}\yields x$ is the proposition that the the procedure $\proc{B}$ will yield the value $x\in X$. $\proc{B}\yields A$ for $A\subset X$ is the proposition $\lor_{x\in A} \proc{B}\yields x$. Two procedures $\proc{B}$ and $\proc{C}$ are the same if $\proc{B}\yields x\iff \proc{C}\yields x$ for all $x\in B$ (note that $\proc{B}$ and $\proc{C}$ could involve different actions in the real world).
\end{definition}

Measurement procedures are like functions without well-defined domains. We can compose measurement procedures with functions to produce new measurement procedures.

\begin{definition}[Composition of functions with procedures]
Given a procedure $\proc{B}$ that takes values in some set $B$, and a function $f:B\to C$, define the ``composition'' $f\circ \proc{B}$ to be any procedure $\proc{C}$ that yields $f(x)$ whenever $\proc{B}$ yields $x$. We can construct such a procedure by describing the steps: first, do $\proc{B}$ and secondly, apply $f$ to the value yielded by $\proc{B}$.
\end{definition}

For example, $\proc{MA}$ is the composition of $h:(x,y)\mapsto xy$ with the procedure $(\proc{M},\proc{A})$ that yields the mass and acceleration of the same object. Measurement procedure composition is associative:

\begin{align}
	(g\circ f)\circ\proc{B}\text{ yields } x &\iff B\text{ yields } (g\circ f)^{-1}(x) \\
	&\iff B\text{ yields } f^{-1}(g^{-1}(x))\\
	&\iff f\circ B \text{ yields } g^{-1}(x)\\
	&\iff g\circ(f\circ B)\text{ yields } x
\end{align}


One might whether there is also some kind of ``append'' operation that takes a standalone $\proc{M}$ and a standalone $\proc{A}$ and returns a procedure $(\proc{M},\proc{A})$. Unlike function composition, this would be an operation that acts on two procedures rather than a procedure and a function. Unlike composition, we can't easily reason about such an operation mathematically, because of the fact that measurment procedures have a foot in the real world. Our approach here is to suppose that there is some master measurement procedure $\proc{S}$ which takes values in $\Psi$ that handles all of the ``real world'' interaction relevant to our problem. Specifically, we assume that any measurement procedure of interest to our problem can be written as the composition $f\circ \proc{S}$ for some $f$.

For the model $\proc{F}=\proc{MA}$, for example, we could assume $\proc{F}=f\circ \proc{S}$ for some $f$ and $(\proc{M},\proc{A})=g\circ \proc{S}$ for some $g$. In this case, we can get $\proc{MA}=h\circ(\proc{M},\proc{A})=(h\circ g)\circ\proc{S}$. Note that each procedure is associated with a unique function with domain $\Psi$.

Given that measurement processes are in practice finite precision and with finite range, $\Psi$ will generally be a finite set. We can therefore equip $\Psi$ with the collection of measurable sets given by the power set $\sigalg{E}:=\mathscr{P}(\Psi)$, and $(\Psi,\sigalg{E})$ is a standard measurable space. $\sigalg{E}$ stands for a complete collection of logical propositions we can generate that depend on the results yielded by the measurement procedure $\proc{S}$.

$(\Psi,\sigalg{E})$ defines is a ``sample space'' limited to observable variables. That is, $\Psi$ is associated with a measurement procedure. Unobserved variables need not be associated with measurement procedures, and to accommodate these we use instead of $\Psi$ a richer fundamental probability set $\Omega$ which represents both observed and unobserved variables.

\begin{definition}[Sample space]
The sample space $(\Omega,\sigalg{F})$ is a set $\Omega$ along with with a $\sigma$-algebra $\sigalg{F}$ of subsets of $\Omega$.
\end{definition}

Observables are represented by a function $\RV{S}:\Omega\to \Psi$, and values of $\omega$ are related to propositions about measurement procedures via the criterion of \emph{consistency with observation}.

\begin{definition}[Consistency with observation]\label{def:observable}
An element $\omega\in \Omega$ is is \emph{consistent with observation} if the result yielded by $\proc{S}\yields \RV{S}(\omega)$
\end{definition}

Thus the procedure $\proc{S}$ restricts the observationally consistent elements of $\Omega$. If $\proc{S}$ yield the result $s$, then the consistent values of $\Omega$ will be $\RV{S}^{-1}(s)$. While two different sets of measurement outcomes $\Psi$ and $\Psi'$ entail a different mesurement procedures $\proc{S}$ and $\proc{S}'$, but different fundamental probability sets $\Omega$ and $\Omega'$ may be used to model a single procedure $\proc{S}$.

As far as we know, distinguishing variables from procedures is somewhat nonstandard, but we feel it is useful to distinguish the formal elements of our theory (variables) from the semi-formal elements (measurement procedures). Both variables and procedures are often discussed in statistical texts. For example, \citet{pearl_causality:_2009} offers the following two, purportedly equivalent, definitions of variables:
\begin{quote}
By a \emph{variable} we will mean an attribute, measurement or inquiry that may take on one of several possible outcomes, or values, from a specified domain. If we have beliefs (i.e., probabilities) attached to the possible values that a variable may attain, we will call that variable a random variable.
\end{quote}

\begin{quote}
This is a minor generalization of the textbook definition, according to which a random variable is a mapping from the fundamental probability set (e.g., the set of elementary events) to the real line. In our definition, the mapping is from the fundamental probability set to any set of objects called ``values,'' which may or may not be ordered.
\end{quote}

Our view is that the first definition is a definition of a procedure, while the second is a definition of a variable. Variables model procedures, but they are not the same thing. We can establish this by noting that, under our definition, every procedure of interest -- that is, all procedures that can be written $f\circ \proc{S}$ for some $f$ -- is modeled by a variable, but there may be variables defined on $\Omega$ that do not factorise through $\proc{S}$, and these variables do not model procedures.

\subsection{Events}

To recap, we have a procedure $\proc{S}$ yielding values in $\Psi$ that measures everything we are interested in, a fundamental probability set $\Omega$ and a function $\RV{S}$ that models $\proc{S}$ in the sense of Definition \ref{def:observable}. We assume also that $\Psi$ has a $\sigma$-algebra $\sigalg{E}$ (this may be the power set of $\Psi$, as measurement procedures are typically limited to finite precision). $\Omega$ is equipped with a $\sigma$-algebra $\sigalg{F}$ such that $\sigma(\RV{S})\subset \sigalg{F}$. If a procedure $\proc{X}=f\circ \RV{S}$ then we define $\RV{X}:\Omega\to X$ by $\RV{X}:=f\circ\RV{S}$.

If a particular procedure $\proc{X}=f\circ \proc{S}$ eventually yields a value $x$, then the values of $\Omega$ consistent with observation must be a subset of $\RV{X}^{-1}(x)$. We define an \emph{event} $\RV{X}\yields x:\equiv \RV{X}^{-1}(x)$, which we read ``the event that $\RV{X}$ yields x''. An event $\RV{X}\yields x$ occurs if the consistent values of $\Omega$ are a subset of $\RV{X}\yields x$, thus ``the event that $\RV{X}$ yields x occurs$\equiv \proc{X}$ yields $x$''. The definition of events applies to all types of variables, not just observables, but we only provide an interpretation of events ``occurring'' when the variable $\RV{X}$ is associated with some $\proc{X}$.

For measurable $A\in \sigalg{X}$, $\RV{X}\yields A=\bigcup_{x\in A} \RV{X}\yields x$. 

Given $\RV{Y}:\Omega\to X$, we can define a sequence of variables: $(\RV{X},\RV{Y}):=\omega\mapsto (\RV{X}(\omega),\RV{Y}(\omega))$. $(\RV{X},\RV{Y})$ has the property that $(\RV{X},\RV{Y})\yields (x,y)= \RV{X}\yields x\cap \RV{Y}\yields y$, which supports the interpretation of $(\RV{X},\RV{Y})$ as the values yielded by $\RV{X}$ and $\RV{Y}$ together.

It is common to use the symbol $=$ instead of $\bowtie$, but we want to avoid this because $\RV{Y}=y$ already has a meaning, namely that $\RV{Y}$ is a constant function everywhere equal to $y$. 

\subsection{Standard probability theory}

\begin{definition}[Probability measure]
Given a measure space $(X,\sigalg{X})$, a probability measure is a $\sigma$-additive function $\mu:\sigalg{X}\to [0,1]$ such that $\mu(\emptyset)=0$ and $\mu(X)=1$. We write $\Delta(X)$ for the set of all probability measures on $(X,\sigalg{X})$.
\end{definition}

\begin{definition}[Markov kernel]
Given measure spaces $(X,\sigalg{X})$, $(Y,\sigalg{Y})$ $\RV{Y}:\Omega\to Y$, a Markov kernel $\prob{Q}:X\kto Y$ is a map $Y\times \sigalg{X}\to [0,1]$ such that
\begin{enumerate}
	\item $y\mapsto \prob{Q}(A|y)$ is $\sigalg{B}$-measurable for all $A\in \sigalg{X}$
	\item $A\mapsto \prob{Q}(A|y)$ is a probability measure on $(X,\sigalg{X})$ for all $y\in Y$
\end{enumerate}
\end{definition}

\begin{definition}[Probability measures as Markov kernel]
Given $(X,\sigalg{X})$ and $\mu\in \Delta(X)$, the Markov kernel $\kernel{K}:\{*\}\kto X$ given by $\kernel{K}(A|*)=\mu(A)$ for all $A\in \sigalg{X}$ is the Markov kernel associated with the probability measure $\mu$. We will use probability measures and their associated Markov kernels interchangeably, as it is transparent how to get from one to another.
\end{definition}

\begin{definition}[Delta measure]
Given a measure space $(X,\sigalg{X})$ and $x\in X$, $\delta_x\in \Delta(X)$ is the measure defined by $\delta_x(A)=\llbracket x\in A \rrbracket$.
\end{definition}

\begin{definition}[Probability space]
A probability space is a triple $(\mu,\Omega,\sigalg{F})$, where $\mu$ is a base measure on $\sigalg{F}$.
\end{definition}

\begin{definition}[Marginal distribution with respect to a probability space]\label{def:pushforward}
Given a probability space $(\mu,\Omega,\sigalg{F})$ and a random variable $\RV{X}:\Omega\to (X,\sigalg{X})$, we can define the \emph{marginal distribution} of $\RV{X}$ with respect to $\mu$, $\mu^{\RV{X}}:\sigalg{X}\to [0,1]$ by $\mu^{\RV{X}}(A):=\mu(\RV{X}\yields A)$ for any $A\in \sigalg{X}$.
\end{definition}

\begin{definition}[Disintegration]
Given a Markov kernel $\kernel{K}:W\kto X\times Y$, with $W$, $X$ and $Y$ standard measurable, any kernel $\kernel{L}:W\times X\kto Y$ satisfying
\begin{align}
	\kernel{K} = \tikzfig{disintegration_existence}\label{eq:disint_def}
\end{align}
is a $W\times X\kto Y$ \emph{disintegration} of $\kernel{K}$.
\end{definition}

\begin{definition}[Conditional probability with respect to a probability space]\label{def:disint}
Given a probability space $(\mu,\Omega)$ and random variables $\RV{X}:\Omega\to (X,\sigalg{X})$, $\RV{Y}:\Omega\to (Y,\sigalg{Y})$, the probability of $\RV{Y}$ given $\RV{X}$ is any $X\kto Y$ disintegration of $\mu^{\RV{XY}}$. That is,
\begin{align}
	\int_{A} \mu^{\RV{Y}|\RV{X}}(B|x) \mathrm{d}\mu^{\RV{X}}(x)&= \mu^{\RV{XY}}(x,y) &\forall A\in \sigalg{X}, B\in \sigalg{Y}\\
	&\iff\\
	\tikzfig{disint_def} &= \mu^{\RV{XY}}
\end{align}
\end{definition}

\begin{lemma}[Marginal distribution as a kernel product]
Given a probability space $(\mu,\Omega,\sigalg{F})$ and a random variable $\RV{X}:\Omega\to (X,\sigalg{X})$, define $\kernel{F}_{\RV{X}}:\Omega\kto X$ by $\kernel{F}_{\RV{X}}(A|\omega)=\delta_{\RV{X}(\omega)}(A)$, then
\begin{align}
	\mu^{\RV{X}} = \mu\kernel{F}_{\RV{X}}
\end{align}
\end{lemma}

\begin{proof}
Consier any $A\in \sigalg{X}$.
\begin{align}
	\mu \kernel{F}_{\RV{X}}(A) &= \int_\Omega \delta_{\RV{X}(\omega)}(A) \mathrm{d}\mu(\omega)\\
	&= \int_{\RV{X}^{-1}(\omega)} \mathrm{d}\mu(\omega)\\
	&= \mu^{\RV{X}}(A)
\end{align}
\end{proof}

Disintegration of arbitrary Markov kernels is possible in standard measurable spaces. We will assume that all spaces are standard measurable, such that whenever we have a Markov kernel we are able to disintegrate it.

\begin{lemma}[Disintegration existence in standard measurable Markov kernels]\label{lem:disint_exist}
For any Markov kernel $\kernel{K}:X\kto W\times Y$ and $X$, $W$, $Y$ are standard measurable, there exists $\kernel{L}:W\times X\kto Y$ such that
\begin{align}
	\kernel{K} = \tikzfig{disintegration_existence}
\end{align}
$\kernel{L}$ is a \emph{disintegration} of $\kernel{K}$.
\end{lemma}

\begin{proof}
\citet{cho_disintegration_2019} Theorem 3.11
\end{proof}

We define the copy-product $\odot$ as a shorthand for the operation in Equation \ref{eq:disint} that combines a marginal with a disintegration to get the original Markov kernel back.

\begin{definition}[Copy-product]\label{def:copyproduct}
Given two Markov kernels $\prob{K}:X\kto Y$ and $\prob{L}:Y\times X\kto Z$, define the copy-product $\prob{K}\odot\prob{L}:X\to Y\times Z$ as
\begin{align}
	\prob{K}\odot\prob{L}:&= \text{copy}_X(\prob{K}\otimes \text{id}_X)(\text{copy}_Y\otimes\text{id}_X )(\text{id}_Y \otimes \prob{L})\\
							&= \tikzfig{copy_product}\\
							&\iff\\
	(\prob{K}\odot\prob{L})(A\times B|x) &= \int_B \prob{L}(A|y,x)\prob{K}(dy|x)
\end{align}
\end{definition}

\subsection{Probabilistic models for causal inference}

The sample space $(\Omega,\sigalg{F})$ along with our collection of variables is a ``model skeleton'' -- it tells us what kind of data we might see. The process $\proc{S}$ which tells us which part of the world we're interested in is related to the model $\Omega$ and the observable variables by the criterion of \emph{consistency with observation}. The kind of problem we are mainly interested in here is one where we make use of data to help make decisions under uncertainty. Probabilistic models have a long history of being used for this purpose, and our interest here is in constructing probabilistic models that can be attached to our variable ``skeleton''. 

Given a model skeleton, a common approach to attaching a probabilistic model involves defining a base measure $\mu$ on $\Omega$ which yields a probability space $(\Omega,\sigalg{F},\mu)$. For causal inference, we need a to generalise this approach, because we need to handle \emph{gaps} in our model. \citet{hajek_what_2003} defines \emph{probability gaps} as propositions that do not have a probability assigned to them. Our view of probability gaps is slightly different -- in this work, a model with probability gaps as one that is missing some key parts or ``inserts''. If we complete such a model with an appropriate insert, we get a standard probability model.

Probability gap models are particularly useful in for decision making. When I have a number of different options I could choose, I need a model that tells me what is likely to happen for each choice I could make. Thus I need a model that can take a provisional choice as an argument and return a probability model representing the results of this choice; in other words, the choices I may make are \emph{probability gaps}.

We formalize probability gap models in terms of probability sets. A probability set on $(\Omega,\sigalg{F})$ is a subset of probability measures on this space. A probability gap model consists of a particular probability set on $(\Omega,\sigalg{F})$, which we call the gap model, along with a collection of probability sets also on $(\Omega,\sigalg{F})$ which we call the domain. We can choose a particular proposition from the domain, and the result is given by the intersection of the gap model and the insert. We will typically consider probability sets represented by collections of conditional probabilities, and in this setting we will describe how the intersection operation corresponds to inserting the proposition into the ``gap'' in the gap model.

This scheme raises the possibility that the intersection of a model and an insert may be empty. We define the notion of \emph{validity} and show that it is sufficient that the probability model and the probability insert are valid for the intersection to be non-empty.

\subsection{Probability sets}

A probability set is a set of probability models.

\begin{definition}[Probability set]
A probability set $\prob{P}_{\{\}}$ on $(\Omega,\sigalg{F})$ is a collection of probability measures on $(\Omega,\sigalg{F})$. In other words it is a subset of $\mathscr{P}(\Delta(\Omega))$, where $\mathscr{P}$ indicates the power set.
\end{definition}

Given a probability set $\prob{P}_{\{\}}$, we define marginal and conditional probabilities as probability measures and Markov kernels that satisfy Definitions \ref{def:pushforward} and \ref{def:disint} respectively for \emph{all} base measures in $\prob{P}_{\{\}}$. There are generally multiple Markov kernels that satisfy the properties of a conditional probability with respect to a probability set, and this definition ensures that marginal and conditional probabilities are ``almost surely'' unique (Definition \ref{def:asequal}) with respect to probability sets.

\begin{definition}[Marginal probability with respect to a probability set]
Given a sample space $(\Omega,\sigalg{F})$, a variable $\RV{X}:\Omega\to X$ and a probability set $\prob{P}_{\{\}}$, the marginal distribution $\prob{P}_{\{\}}^{\RV{X}}=\prob{P}_a^{\RV{X}}$ for any $\prob{P}_a\in\prob{P}_{\{\}}$ if a distribution satisfying this condition exists. Otherwise, it is undefined.
\end{definition}

\begin{definition}[Conditional probability with respect to a probability set]
Given a fundamental probability set $\Omega$ variables $\RV{X}:\Omega\to X$ and $\RV{Y}:\Omega\to Y$ and a probability set $\prob{P}_{\{\}}$, \emph{a version of} $\prob{P}_{\{\}}^{\RV{Y}|\RV{X}}$ is any Markov kernel $X\kto Y$ such that $\prob{P}_{\{\}}^{\RV{Y}|\RV{X}}$ is an $X\kto Y$ disintegration of $\prob{P}_a^{\RV{XY}}$ for all $\prob{P}_a\in \prob{P}_{\{\}}$. If no such Markov kernel exists, $\prob{P}_{\{\}}^{\RV{Y}|\RV{X}}$ is undefined.
\end{definition}

Given a conditional probability with respect to a probability set, we can find other conditional probabilities by ``pushing it forward''.

% \begin{lemma}[Equivalence of pushforward definitions]\label{lem:prod_pushf}
% Given a probability space $\kernel{M}:W\to \Omega$ and $\RV{X}:\Omega\to X$, define $\kernel{K}^{\RV{X}|\RV{W}}:W\kto X$ by $\kernel{K}^{\RV{X}|\RV{W}}(x|w):=\kernel{M}(\RV{X}\yields x|w)$ for any $x\in X$m $w\in W$ and $\kernel{L}^{\RV{X}}:W\kto X$ by
% \begin{align}
% 	\kernel{L}^{\RV{X}|\RV{W}} = \kernel{M}\kernel{F}_{\RV{X}}
% \end{align}
% Then
% \begin{align}
% \kernel{L}^{\RV{X}|\RV{W}} =\kernel{K}^{\RV{X}|\RV{W}}
% \end{align}
% \end{lemma}

% \begin{proof}
% For any $x\in X$, $w\in W$
% \begin{align}
% 	\kernel{L}^{\RV{X}|\RV{W}}(x|w) &= \sum_{\omega\in \Omega} \llbracket x=\RV{X}(\omega)\rrbracket \kernel{M}(\omega|w)\\
% 									&= \sum_{\omega\in \RV{X}^{-1}(x)} \kernel{M}(\omega|w)\\
% 									&= \kernel{M}(\RV{X}\yields x|w)\\
% 									&= \kernel{K}^{\RV{X}|\RV{W}}(x|w)
% \end{align}
% \end{proof}

\begin{theorem}[Recursive pushforward]\label{th:recurs_pushf}
Suppose we have a sample space $\Omega$ variables $\RV{X}:\Omega\to X$ and $\RV{Y}:\Omega\to Y$, $\RV{Z}:\Omega\to Z$ and a probability set $\prob{P}_{\{\}}$ such that $\prob{P}_{\{\}}^{\RV{X}|\RV{Y}}$ is a $\RV{Y}|\RV{X}$ conditional probability of $\prob{P}_{\{\}}$ and $\RV{Z}=f\circ \RV{Y}$ for some $f:Y\to Z$. Then there exists a $\RV{Z}|\RV{X}$ conditional probability of $\prob{P}_{\{\}}$ given by $\prob{P}_{\{\}}^{\RV{Z}|\RV{X}}=\prob{P}_{\{\}}^{\RV{Y}|\RV{X}}\kernel{F}_{f}$.
\end{theorem}

\begin{proof}
For any $\prob{P}_a\in \prob{P}_{\{\}}$, $x,z$
\begin{align}
\prob{P}_a^{\RV{X}}(x)\prob{P}_a^{\RV{Z}|\RV{X}}(z|x) &= \prob{P}_a(\RV{X}^{-1}(x)\cap\RV{Z}^{-1}(z))\\
					   &= \prob{P}_a(\RV{X}^{-1}(x)\cap\RV{Y}^{-1}(f^{-1}(z)))\\
					   &= \prob{P}_a^{\RV{X},\RV{Y}}(\{x\}\times f^{-1}(z))\\
					   &= \prob{P}_a^{\RV{X}}(x)\prob{P}_a^{\RV{Y}|\RV{X}}(f^{-1}(z)|x)
\end{align}
\end{proof}

We introduce a notion of ``almost sure equality'' for Markov kernels with respect to a probability set. In particular, two Markov kernels of the same are almost surely equal with respect to a probability set and a variable $\RV{X}$ if the copy product $\odot$ of all marginal probabilities of $\RV{X}$ with each Markov kernel is identical.

\begin{definition}[Almost sure equality]\label{def:asequal}
Two Markov kernels $\kernel{K}:X\kto Y$ and $\kernel{L}:X\kto Y$ are almost surely equal $\overset{\prob{P}_{\{\}}}{\cong}$ with respect to a probability set $\prob{P}_{\{\}}$ and variable $\RV{X}:\Omega\to X$ if for all $\prob{P}_a \in \prob{P}_{\{\}}$,
\begin{align}
	\prob{P}^{\RV{X}}_a\odot \kernel{K}=\prob{P}^{\RV{X}}_a\odot \kernel{L}
\end{align}
\end{definition}

\begin{lemma}[Conditional probabilities are almost surely equal]
Given $\kernel{K}^{\RV{Y}|\RV{X}}$ and $\kernel{L}^{\RV{Y}|\RV{X}}$ both of which are versions of $\prob{P}_{\{\}}^{\RV{Y}|\RV{X}}$, $\kernel{K}^{\RV{Y}|\RV{X}}\overset{\prob{P}_{\{\}}}{\cong}\kernel{L}^{\RV{Y}|\RV{X}}$
\end{lemma}

\begin{proof}
For all $\prob{P}_a \in \prob{P}_{\{\}}$
\begin{align}
	\prob{P}^{\RV{X}}_a\odot \kernel{K}^{\RV{Y}|\RV{X}} &= \prob{P}^{\RV{X}}_a\\
	&= \prob{P}^{\RV{X}}_a\odot \kernel{L}^{\RV{Y}|\RV{X}}
\end{align}
\end{proof}


% \begin{theorem}[Disintegrations are conditional probabilities]
% Suppose we have a fundamental probability set $\Omega$ variables $\RV{W}:\Omega\to W$, $\RV{X}:\Omega\to X$, $\RV{Y}:\Omega\to Y$ and $\RV{Z}:\Omega\to Z$ and a probability set $\prob{P}_{\{\}}$ such that $\prob{P}_{\{\}}^{\RV{X}|\RV{Y}}$ is a $\RV{Y}|\RV{X}$ conditional probability and there is some $\kernel{K}^{\RV{$
% \end{theorem}

% Given a conditional probability with respect to a probability gap model, we can also find additional conditional probabilities by disintegrating the original conditional probability.

% \begin{lemma}[Recursive disintegration]
% Suppose we have a fundamental probability set $\Omega$, variables $\RV{W}:\Omega\to W$, $\RV{X}:\Omega\to X$ and $\RV{Y}:\Omega\to Y$, $\RV{Z}:\Omega\to Z$ and a probability set $\prob{P}_{\{\}}$ such that $\prob{P}_{\{\}}^{\RV{X}|\RV{Y}}$ is a $\RV{Y}|\RV{X}$ conditional probability. Define $\prob{Q}_{\{\}}$ as the largest probability set such that $\prob{Q}_{\{\}}^{\RV{Y}|\RV{X}}=\prob{P}_{\{\}}^{\RV{Y}|\RV{X}}$. Then if $\prob{Q}_{\{\}}^{\RV{Z}|\RV{W}}$ is a $\RV{Z}|\RV{W}$ conditional probability of $\prob{Q}_{\{\}}$, it is also a $\RV{Z}|\RV{W}$ conditional probability of $\prob{P}_{\{\}}$.
% \end{lemma}

% \begin{proof}
% $\prob{Q}_{\{\}}\supset \prob{P}_{\{\}}$, so any conditional probability of $\prob{Q}_{\{\}}$ is also a conditional probability of $\prob{P}_{\{\}}$.
% \end{proof}

\subsection{Probability sets defined by marginal and conditional probabilities}

In the previous section we defined marginal and conditional probabilities for probability sets. Here we will go in the other direction: define probability sets by specifying key marginal or conditional probabilities. There is an issue to be careful of here: not all probability measures $\prob{Q}^{\RV{X}}$ on $X$ define nonempty sets of probability measures on $\Omega$ with respect to the variable $\RV{X}$. Consider, for example, $\RV{X}=(\RV{Y},\RV{Y})$ for some $\RV{Y}:\Omega\to \{0,1\}$ and any measure $\prob{Q}^{\RV{YY}}$ that assigns nonzero probability to the event $(\RV{Y},\RV{Y})\yields (1,0)$. There is no base measure that pushes forward to such a $\prob{P}^{\RV{YY}}$, because two copies of the same variable must always be deterministically equal. A \emph{valid distribution} is a distribution associated with a particular variable that defines a nonempty set of base measures on $\Omega$ (Theorem \ref{th:completion}).

\begin{definition}[Valid distribution]\label{def:valid_dist}
A valid $\RV{X}$ probability distribution $\prob{P}^{\RV{X}}$ is any probability mesure on $\Delta(X)$ such that $\RV{X}^{-1}(A)=\emptyset\implies \prob{P}^{\RV{X}}(A) = 0$ for all $A\in\sigalg{X}$.
\end{definition}

\emph{Valid conditionals} not only define a nonempty set of base measures on $\Omega$, but the $\odot$ product of two appropriately typed valid conditionals itself defines a nonempty set of base measures on $\Omega$ (Lemma \ref{lem:valid_extendability}). 

\begin{definition}[Valid conditional]\label{def:valid_conditional_prob}
Given $(\Omega,\sigalg{F})$, $\RV{X}:\Omega\to X$, $\RV{Y}:\Omega\to Y$ a \emph{valid $\RV{Y}|\RV{X}$ conditional probability} $\prob{P}^{\RV{Y}|\RV{X}}$ is a Markov kernel $X\kto Y$ such that it assigns probability 0 to contradictions:
\begin{align}
    \forall B\in \sigalg{Y}, x\in \sigalg{X}: (\RV{X},\RV{Y})\yields \{x\}\times B = \emptyset \implies \left(\prob{P}^{\RV{Y}|\RV{X}}(A|x) = 0\right) \lor \left(\RV{X}\yields \{x\} = \emptyset\right)
\end{align}
\end{definition}

This forms the basis of our theory of probability gaps. If we have a collection of valid conditional probabilities $\{\prob{P}_i^{\RV{X}_i|\RV{X}_{[i-1]}}|i\in [n]\}$, we can take the $\odot$ product of each in turn $\prob{P}_1^{\RV{X}_1|\RV{X}_0}\odot\prob{P}_2^{\RV{X}_2|\RV{X}_{\{0,1\}}}\odot ...$ which will yield a conditional probability $\prob{Q}_^{\RV{X}_{[n]}|\RV{X}_0}$ which, by Lemma \ref{lem:valid_extendability} defines a non-empty probability set. We get a probability gap model when some of these conditional probabilities are ``missing'', and we have an insert set $R$ of alternative conditional probabilities we can use to fill the gap.

We focus in particular on \emph{probability *comb}. These are probability gap models defined by a collection of valid conditional probabilities with every even conditional ``missing''; the conditionals can be visualised as ``teeth of a *comb'', with the gaps between the teeth corresponding to the missing conditionals. The inserts are probability sets defined by different choices of missing conditionals. Given an insert, we slot the conditionals it provides into the gaps in the *comb and ``join'' them all using the $\odot$ product. Theorem \ref{th:intersection} shows that this operation corresponds to an intersection between probability sets and Lemma \ref{lem:valid_extendability} once again guarantees that the result is non-empty.

\todo[inline]{If they weren't already called *combs, I'd call them zips}

\subsection{Probability *combs}

% \textbf{A note on terminology:} Probability gap models are written with blackboard letters $\prob{P}_\square$. The same base letter with different superscripts $\prob{P}_\square^{\RV{A}|\RV{B}}$ indicates a conditional probability with respect to $\prob{P}_\square$. We use subscripts to indicate both the model obtained by applying particular choices of insert $\prob{P}_\alpha:=\prob{P}_\square \cap \alpha$. $\RV{A}\CI_{\prob{P}_\square} \RV{B}$ is a statement of independence with respect to the model $\prob{P}_\square$.

\begin{definition}[probability *comb]\label{def:order1_bycond}
Given sample space $(\Omega,\sigalg{F})$ and variables $\RV{X}_{i}:\Omega\to (X_i,\sigalg{X}_i)$ for $i\in D$ where $D$ is either equal to $\{0\}\cup\mathbb{N}$ or $\{0\}\cup[n]$ for $n\in \mathbb{N}$, let $B=\{2i|i\leq |D|/2\}$ and $C=B+1=\{2i+1|i\leq |D|/2\}$. Then a $\RV{X}_C\comb\RV{X}_B$ probability *comb, written $(\prob{P}_\square^{\RV{X}_C\comb \RV{X}_B},A)$ is a probability set $\prob{P}_\square$ defined by a collection of valid conditional probabilities 
\begin{align}
	\{\prob{P}_\square^{\RV{X}_{2i+1}|\RV{X}_{[2i]}}|i\in D\}
\end{align}
along with a set of \emph{inserts} $A$, where each element $\alpha\in A$ is defined by a collection of conditional probabilities
\begin{align}
	&\{\alpha^{\RV{X}_{2i}|\RV{X}_{[2i-1]}}|i\in D\}
\end{align}
\end{definition}

We overload the notation $\prob{P}_\square$ to also stand for a map from $A$ to probability sets on $\Omega$ defined as follows.

\begin{definition}[Engaging the *combs]\labe{def:insert}
Given a probability gap model $(\prob{P}_\square,\RV{X}_B,\RV{X}_C,A)$, define the operation $\prob{P}_\square:A\to \mathscr{P}(\Omega)$ such that for $1\leq i\leq |D|/2$ 
\begin{align}
	\prob{P}_\alpha^{\RV{X}_{[2i+1]}|\RV{X}_1} &= \prob{P}_\alpha^{\RV{X}_{[2i-1]}|\RV{X}_1}\odot \alpha^{\RV{X}_{2i}|\RV{X}_{[2i-1]}}\odot\prob{P}_{\square}^{\RV{X}_{2i+1}|\RV{X}_{[2i]}}
\end{align}
with
\begin{align}
	\prob{P}_\alpha^{\RV{X}_{2}|\RV{X}_1} &= \alpha^{\RV{X}_{2}|\RV{X}_{1}}
\end{align}
\end{definition}

\subsubsection{Examples}

History-based models of reinforcement found in \citet{hutter_universal_2004} are examples of probability zips. Such models posit sequences of variables $(\RV{A}_i,\RV{O}_i,\RV{R}_i)_{i\in \mathbb{N}}$ representing the $i$th action, observation and reward respectively. They also posit an \emph{agent} that implements a \emph{policy}, which is a collection of maps $\pi_i:O^i\times R^i\times A^i \kto A$ for all $i\in \mathbb{N}$ (mapping the history of actions, rewards and observations to the next action), and an \emph{environment} which is a collection of maps $e_i:O^i\times R^i\times A^{i+1}\kto O\times R$, mapping the history of actions, rewards, observations and the next action to the next observation and reward.

We identify the envirnoment $e_i$ with a probability gap model $\{\prob{P}_\square^{\RV{O}_{i}\RV{R}_{i}|\RV{O}_{<i}\RV{R}_{<i}\RV{A}_{i}}|i\in\mathbb{N}\}$ and the policy set with a probability gap $\{\prob{P}_\alpha^{\RV{A}_{i}|\RV{O}_{<i}\RV{R}_{<i}\RV{A}_{<i}}|i\in\mathbb{N},\alpha\in A\}$; the model of observations, actions and rewards arising from a particular choice $\alpha$ of policy is determined by the composition of environment and policy conditional probabilities as in Equation \ref{eq:intersperse_pgap}.

A pair of conditional probabilities with a gap between them induces a \emph{probability 2-combs} \citep{chiribella_quantum_2008,jacobs_causal_2019}. We can depict the map associated with a conditonal gap model graphically in an informal way as ``inserting'' $\prob{P}_\alpha^{\RV{Y}|\RV{X}}$ into $\prob{P}^{\RV{X}\square\RV{Z}|\RV{Y}}$:

\begin{align}
	\text{Insert}\left(\tikzfig{insert_opn2}\right)\\= \tikzfig{2comb_inserted}\label{eq:insert_op}
\end{align}


We will define a shorthand for models of this type. A probability $n$-comb is a Markov kernel $\kernel{K}_n:\prod_{i\in [n-1]} A_i\kto \prod_{i\in [n-1]} B_i$ such that there is an $n-1$ comb $\kernel{K}_{n-1}:\prod_{i\in [n-1]} A_i\kto \prod_{i\in [n-1]} B_i$ satisfying

\begin{align}
	\tikzfig{n_comb_def} \label{eq:n_comb_def}
\end{align}

And a $1-comb$ is simply an arbitrary Markov kernel $\kernel{K}_1:A_1\kto B_1$. A probability $\mathbb{N}$-comb is a collection of $n$-combs $\kernel{K}_n$ for each $n\in \mathbb{N}$ such that $\kernel{K}_{n-1}$ is related to $\kernel{K}_n$ by Equation \ref{eq:n_comb_def}.

We can specify a probability gap model $\prob{P}_\square$ associated with an infinite collection of conditionals $\{\prob{P}_\square^{\RV{X}_{2i+1}|\RV{X}_{i}}|i\in\mathbb{N}\}$ with an $\mathbb{N}$-comb $\{\kernel{K}_{i}:X^i\kto X^i|i\in\mathbb{N}\}$. Crucially, we can specify the \emph{signature} of this $\mathbb{N}$-comb in terms of which variables it takes as inputs and which it maps as outputs; in particular, it takes $(\RV{X}_{2i})_{i\in \mathbb{N}}$ as input and $(\RV{X}_{2i+1})_{\i\in\mathbb{N}}$ as output. Thus we can specify a \emph{probability sequence gap model} with an $\mathbb{N}$-comb, which we will write as $\prob{P}_\square^{\RV{X}_{2\mathbb{N}+1}\|\RV{X}_{2\mathbb{N}}}$, noting the double bar $\|$ to indicate that this is a comb rather than a conditional probability. This is in turn associated with probability gaps that are subsets of the set of $\mathbb{N}$-combs $\{\prob{P}_\alpha^{\RV{X}_{2\mathbb{N}}|\RV{X}_{2\mathbb{N}+1}}|\alpha\in A\}$.

\subsection{Combs and conditionals}

We can combine the collection of conditionals defining a probability *comb into a \emph{probability comb}. This is a single Markov kernel from which we can recover the original collection of conditionals, up to almost sure equality. Furthermore, probability combs are conditional probabilities for all \emph{blind} inserts. Thus, under some conditions, probability *combs can be fully specified by specifying the conditional probability of all of the blind inserts.

It's an open question whether we can represent the comb $\prob{P}_\square^{\RV{X}_{2\mathbb{N}+1}\|\RV{X}_{2\mathbb{N}}}$ with a single Markov kernel $\mathbb{N}\kto \mathbb{N}$ in the vein of Kolmogorov's representation theorem for probability measures.

\subsubsection{Conditional independence}\label{ssec:cond_indep}

Conditional independence has a familiar definition in probability models. We define conditional independence with respect to a probability gap model to be equivalent to conditional independence with resepect to every base measure in the range of the model. This definition is closely related to the idea of \emph{extended conditional independence} proposed by \citet{constantinou_extended_2017}.

\begin{definition}[Conditional independence with respect to a probability model]
For a \emph{probability model} $\model{P}$ and variables $\RV{A},\RV{B},\RV{Z}$, we say $\RV{B}$ is conditionally independent of $\RV{A}$ given $\RV{C}$, written $\RV{B}\CI_{\model{P}_\square}\RV{A}|\RV{C}$, if
\begin{align}
	\kernel{P}_\square^{\RV{ABC}} &= \tikzfig{cond_indep1}
\end{align}
\end{definition}

For any $\prob{P}_\square^{\RV{B}|\RV{C}}$ and $\prob{P}_\square^{\RV{A}|\RV{C}}$. \citet{cho_disintegration_2019} have shown that this definition coincides with the standard notion of conditional independence. In particular, it satisfies the \emph{semi-graphoid axioms}

\begin{enumerate}
	\item Symmetry: $\RV{A}\CI_{\prob{P}_\square} \RV{B}|\RV{C}$ iff $\RV{B}\CI_{\prob{P}_\square} \RV{A}|\RV{C}$
	\item Decomposition: $\RV{A}\CI_{\prob{P}_\square} (\RV{B},\RV{C})|\RV{W}$ implies $\RV{A}\CI_{\prob{P}_\square}\RV{Y}|\RV{W}$ and $\RV{A}\CI_{\prob{P}_\square}\RV{C}|\RV{W}$
	\item Weak union: $\RV{A}\CI_{\prob{P}_\square}(\RV{B},\RV{C})|\RV{W}$ implies $\RV{A}\CI_{\prob{P}_\square}\RV{B}|(\RV{C},\RV{W})$
	\item Contraction: $\RV{A}\CI_{\prob{P}_\square}\RV{C}|\RV{W}$ and $\RV{A}\CI_{\prob{P}_\square}\RV{B}|(\RV{C},\RV{W})$ implies $\RV{A}\CI_{\prob{P}_\square}(\RV{B},\RV{C})|\RV{W}$
\end{enumerate}

\begin{theorem}\label{th:cho_ci_equiv}
Given discrete $\Omega$ and a probability model $\prob{P}_\square$ and variables $\RV{W}:\Omega\to W$, $\RV{X}:\Omega\to X$, $\RV{Y}:\Omega\to Y$, $\RV{Y}\CI_{\prob{P}_\square}\RV{X}|\RV{W}$ if and only if there exists some version of $\prob{P}_\square^{\RV{Y}|\RV{WX}}$ and $\prob{P}_\square^{\RV{Y}|\RV{W}}$ such that
\begin{align}
	\prob{P}_\square^{\RV{Y}|\RV{WX}} &= \tikzfig{cond_indep_erase}\\
	\iff
	\prob{P}_\square^{\RV{Y}|\RV{WX}}(y|w,x) &= \prob{P}^{\RV{Y}|\RV{W}}(y|w)
\end{align}
\end{theorem}

\begin{proof}
See \citet{cho_disintegration_2019}.
\end{proof}


\begin{definition}[Conditional independence with respect to a probability gap model]
Conditional independence $\RV{A}\CI_{\prob{P}_\square}\RV{B}|\RV{C}$ holds for an arbitrary probability gap model $\model{P}_\square:A\to \mathscr{P}(\Delta(\Omega))$ if $\RV{A}\CI_{\prob{P}_\alpha}\RV{B}|\RV{C}$ holds for all probability models $\prob{P}_\alpha$, $\alpha\in A$.
\end{definition}

One case where we can deduce conditional independences in probability gap models is when conditional probabilities exist and they are \emph{unresponsive} to some input variables.

\begin{definition}[Unresponsiveness]
Given discrete $\Omega$, a probability gap model $\prob{P}_\square:A\to \Delta(\Omega)$, variables $\RV{W}:\Omega\to W$, $\RV{X}:\Omega\to X$, $\RV{Y}:\Omega\to Y$, if there is some version of the conditional probability $\prob{P}^{\RV{Y}|\RV{WX}}$ and $\prob{P}_\square^{\RV{Y}|\RV{W}}$ such that
\begin{align}
	\prob{P}_\square^{\RV{Y}|\RV{WX}} &= \tikzfig{cond_indep_erase} \label{eq:higherorder_ci_erase}
\end{align}
then $\prob{P}_\square^{\RV{Y}|\RV{WX}}$ is \emph{unresponsive} to $\RV{X}$.
\end{definition}

\begin{definition}[Domination]
Given a probability gap model $\prob{P}_\square:A\to \Delta(\Omega)$, $\alpha\in A$ dominates $A$ if $\prob{P}_\beta(B)>0\implies \prob{P}_\alpha(B)>0$ for all $\beta in A$, $B\in \sigalg{F}$.
\end{definition}

\begin{theorem}[Conditional independence from kernel unresponsiveness]\label{th:cons_ci}
Given discrete $\Omega$, variables $\RV{W}:\Omega\to W$, $\RV{X}:\Omega\to X$, $\RV{Y}:\Omega\to Y$ and a probability gap model $\prob{P}_\square:A\to \Delta(\Omega)$ with conditional probability $\prob{P}_\square^{\RV{Y}|\RV{WX}}$ and such that there is $\alpha\in A$ dominating $A$, $\RV{Y}\CI_{\prob{P}_\square}\RV{X}|\RV{W}$ if and only if $\prob{P}_\square^{\RV{Y}|\RV{WX}}$ is unresponsive to $\RV{W}$. 
\end{theorem}

\begin{proof}
If:
For every $\alpha\in A$ we can write
\begin{align}
	\prob{P}_\alpha^{\RV{Y}|\RV{WX}} &= \tikzfig{cond_indep_erase_alpha}
\end{align}
And so, by Theorem \ref{th:cho_ci_equiv}, $\RV{Y}\CI_{\prob{P}_\alpha}\RV{X}|\RV{W}$ for all $\alpha\in A$, and so $\RV{Y}\CI_{\prob{P}_\square}\RV{X}|\RV{W}$.
Only if:
For $\alpha$ dominating $A$, by Theorem \ref{th:cho_ci_equiv}, there exists a version of $\prob{P}_\alpha^{\RV{Y}|\RV{WX}}$ unresponsive to $\RV{W}$. Because $\alpha$ dominates $A$, every version of $\prob{P}_\alpha^{\RV{Y}|\RV{WX}}$ is a version of $\prob{P}_\beta^{\RV{Y}|\RV{WX}}$ for all $\beta in A$, thus it is a version of $\prob{P}_\square^{\RV{Y}|\RV{WX}}$ also.
\end{proof}

Note that $\RV{Y}\CI_{\prob{P}_\square}\RV{X}|\RV{W}$ does \emph{not} imply the existence of $\prob{P}_\square^{\RV{Y}|\RV{WX}}$. If we have, for example, $A=\{\alpha,\beta\}$ and $\prob{P}_\alpha^{\RV{AB}}$ is two flips of a fair coin while $\prob{P}_\beta^{\RV{AB}}$ is a flip of a biased coin followed by a flip of a fair coin, then $\RV{A}\CI_{\prob{P}}\RV{B}$ but $\prob{P}^{\RV{AB}}$ does not exist.

We also need the domination condition. Consider $A$ a collection of inserts that all deterministically set a variable $\RV{X}$; then for any variable $\RV{Y}$ $\RV{Y}\CI_{\prob{P}_\square} \RV{X}$ because $\RV{X}$ is deterministic for any $\alpha\in A$. But $\prob{P}_\square^{\RV{Y}|\RV{X}}$ is not necessarily unresponsive to $\RV{X}$.
