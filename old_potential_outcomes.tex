
Recall our distinction between latent variables and observables: observable variables are related to other variables in the model and to the real world, while latent variables are related only to other variables in the model. In the previous section, we considered a sequence of $(\RV{W}_i, \RV{X}_i, \RV{Y}_i)$ variables. Typically the $\RV{X}_i$'s and $\RV{Y}_i$'s are observable variables. The $\RV{W}_i$'s are trickier. They 

Potential outcomes is a widely used approach to causal modelling characterised by its use of ``potential outcome'' random variables. Potential outcome random variables are typically noted for being given counterfactual interpretations. For example, suppose have something we want to model, call it TYT (``The $\RV{Y}$ Thing''), which we represent with a variable $\RV{Y}$. Suppose we want to know how TYT behaves under different regimes 0 and 1 under which we want to know about TYT, and we use a variable $\RV{W}$ to indicate which regime holds at a given point in time. A potential outcomes model will introduce the two additional ``potential outcome'' variables $(\RV{Y}(0), \RV{Y}(1))$. What these variables represent can be given a counterfactual interpretation like ``$\RV{Y}(0)$ represents what TYT would be under regime $0$, whether or not regime $0$ is the actual regime'' and similarly ``$\RV{Y}(1)$ represents what TYT would be under regime $1$, whether or not regime $1$ is the actual regime''. Note that we say ``what TYT would be'' rather that ``what $\RV{Y}$ would be'' as ``what would $\RV{Y}$ be if $\RV{W}$ was 0 if $\RV{W}$ was actually 1'' is not a question we can ask of random variables, but it is one that might make sense for the things we use random variables to model.

This is a key point, so it is worth restating: the assumption that potential outcome variables agree with ``the value TYT would take'' under fixed regimes regardless of the ``actual'' value of the regime seems to be a critical assumption that distinguishes potential outcome variables from arbitrary random variables that happen to take values in the same space as $\RV{Y}$. However, this assumption can only be stated by making reference to the informally defined ``TYT'' and the informal distinction between the supposed and the actual value of the regime.

The potential outcomes framework features other critical assumptions that relate potential outcome variables to things that are only informally defined. For example, \citet{rubin_causal_2005} defines the \emph{Stable Unit Treatment Value Assumption} (SUTVA) as:

\begin{quote}
SUTVA (stable unit treatment value assumption) [...] comprises two subassumptions. First, it assumes that there is no interference between units (Cox 1958); that is, neither $Y_i(1)$ nor $Y_i(0)$ is affected by what action any other unit received. Second, it assumes that there are no hidden versions of treatments; no matter how unit $i$ received treatment $1$, the outcome that would be observed would be $Y_i(1)$ and similarly for treatment $0$
\end{quote}

``Versions of treatments'' do not appear within typical potential outcomes models, so this is also an assumption about how ``the thing we are trying to model'' behaves rather than an assumption stated within the model.

Given informal assumptions like this, one may be motivated to ``formalize'' them. More specifically, one might be motivated to ask whether there is some larger class of models that, under conditions corresponding to the informal conditions above yield regular potential outcome models?

\todo[inline]{I have a vague intuition here that you always need some kind of assumption like ``my model is faithful to the real thing'', but if you are stating fairly specific conditions in English you should also be able to state them mathematically. Among other reasons, this is useful because it's easier for other people to know what you mean when you state them.}

The approach we have introduced here, motivated by decision problems, has in the past been considered a means of avoiding counterfactual statements, which has been considered a positive by some \citep{dawid_causal_2000} and a negative by others:

\begin{quote}
[...] Dawid, in our opinion, incorrectly concludes that an approach to causal inference based on ``decision analysis'' and free of counterfactuals is completely satisfactory for addressing the problem of inference about the effects of causes.\citep{robins_causal_2000}
\end{quote}

It may be surprising to some, then, that we can use see-do models to formally state these key assumptions associated with potential outcomes models. Furthermore, we will argue that potential outcomes are typically a strategy to motivate inductive assumptions in see-do models, and we will show that the counterfactual interpretation is unnecessary for this purpose.

\subsection{Potential outcomes in see-do models}

A basic property of potential outcomes models is the relation between variables representing actual outcomes and variables representing potential outcomes, which was stated informally in the opening paragraph of this section.

In the following definition, $\RV{Y}(W)=(\RV{Y}(w))_{w\in W}$.

\begin{definition}[Potential outcomes]\label{def:potential_outcomes}
Given a Markov kernel space $(\kernel{K},E,F)$, a collection of variables $\{\RV{Y}, \RV{Y}(W), \RV{W}\}$ where $\RV{Y}$ and $\RV{Y}(W)$ are random variables and $\RV{W}$ could be either a state or a random variable is a \emph{potential outcome submodel} if $\kernel{K}[\RV{Y}|\RV{W}\RV{Y}(W)]$ exists and $\kernel{K}[\RV{Y}|\RV{W}\RV{Y}(W)]_{i j_1 j_2 ... j_{|W|}} = \delta[j_i]$. 
\end{definition}

\todo[inline]{How this will change: a potential outcomes model is a comb $\model{K}[\RV{Y}(W)|\RV{H}]\rightrightarrows \model{K}[\RV{Y}|\RV{W}\RV{Y}(W)]$.}

We allow $\RV{X}$ to be a state or a random variable to cover the cases where potential outcomes models feature as submodels of observation models (in which case $\RV{X}$ is a random variable) or as submodels of consequence models (in which case $\RV{X}$ may be a state variable).

As an aside that we could define stochastic potential outcomes if we allow the variables $\RV{Y}(x)$ to take values in $\Delta(Y)$ rather than in $Y$, and then require $\kernel{K}[\RV{Y}|\RV{X}\RV{Y}(X)]_{i j_1 j_2 ... j_{|X|}} = j_i$ (where $j_i$ is an element of $\Delta(Y)$). This is more complex to work with and rarely seen in practice, but it is worth noting that Definition \ref{def:potential_outcomes} can be generalised to cover models where $\RV{Y}(x)$ describes the value $\RV{Y}$ would take if $\RV{X}$ were $x$ \emph{with uncertainty}.

An arbitrary see-do model featuring potential outcome submodels does not necessarily allow for the formal statement of the counterfactual interpretation of potential outcomes. Here we use TYT (``the actual thing'') and ``regime'' to refer to the things we are actually trying to model. We require that $\RV{Y}\overset{a.s.}{=}\RV{Y}(w)$ conditioned on $\RV{W}=w$. If we add an interpretation to this model saying $\RV{Y}$ represents TYT and $\RV{W}$ represents the regime, then we have ``for all $w$, $\RV{Y}(w)$ is equal to $\RV{Y}$ which represents TYT under the regime $w$''. However, this does not guarantee that our model has anything that reasonably represents ``what TYT would be equal to under supposed regime $w$ if the regime is actually $w'$''.

We propose \emph{parallel potential outcome submodels} as a means of formalising statements about what how TYT behaves under ``supposed'' and ``actual'' regimes:
