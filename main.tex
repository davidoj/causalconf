\documentclass{article}

% If your paper is accepted, change the options for the package
% aistats2020 as follows:
%
% \usepackage[accepted]{aistats2020}
%
% This option will print headings for the title of your paper and
% headings for the authors names, plus a copyright note at the end of
% the first column of the first page.

% If you set papersize explicitly, activate the following three lines:
%\special{papersize = 8.5in, 11in}
%\setlength{\pdfpageheight}{11in}
%\setlength{\pdfpagewidth}{8.5in}

% If you use natbib package, activate the following three lines:
\usepackage[round]{natbib}
\renewcommand{\bibname}{References}
\renewcommand{\bibsection}{\subsubsection*{\bibname}}

% If you use BibTeX in apalike style, activate the following line:
%\bibliographystyle{apalike}

\usepackage[T1]{fontenc}    % use 8-bit T1 fonts
\usepackage{hyperref}       % hyperlinks
\usepackage{url}            % simple URL typesetting
\usepackage{booktabs}       % professional-quality tables
\usepackage{amsfonts}       % blackboard math symbols
\usepackage{nicefrac}       % compact symbols for 1/2, etc.
\usepackage{microtype}      % microtypography

% My packages

\usepackage[mathscr]{euscript}
\usepackage{graphicx}
\usepackage {tikz}
\usetikzlibrary {positioning}
\usetikzlibrary{shapes.misc}
\usetikzlibrary{shapes.geometric}
\usetikzlibrary{calc}
\usetikzlibrary{arrows.meta}
\usetikzlibrary{intersections}
\usepackage{amsthm}
\usepackage{amsmath}
\usepackage{amssymb}
\usepackage{dsfont}
\usepackage{stmaryrd }
\usepackage{csquotes}
\usepackage{wasysym}
\usepackage[]{todonotes}
\usepackage[shortlabels]{enumitem}
\usepackage{bm}
\usepackage{isomath}
\usepackage{mathtools}

\makeatletter
\newcommand{\newreptheorem}[2]
  {\newtheorem*{rep@#1}{\rep@title}\newenvironment{rep#1}[1]
  {\def\rep@title{#2 \ref*{##1}}\begin{rep@#1}}{\end{rep@#1}}}
\makeatother

\theoremstyle{plain}
\newtheorem{theorem}{Theorem}[section]
\newtheorem{corollary}[theorem]{Corollary}
\newtheorem{lemma}[theorem]{Lemma}
\newtheorem{proposition}[theorem]{Proposition}
\newreptheorem{theorem}{Theorem}

\newtheorem{innercustomthm}{Theorem}
\newenvironment{customthm}[1]
  {\renewcommand\theinnercustomthm{#1}\innercustomthm}
  {\endinnercustomthm}

\theoremstyle{definition}
\newtheorem{definition}[theorem]{Definition}
\newtheorem{example}[theorem]{Example}

\DeclareMathAlphabet{\mathsfit}{T1}{\sfdefault}{\mddefault}{\sldefault}

\newcommand{\CI}{\mathrel{\text{\scalebox{1.07}{$\perp\mkern-10mu\perp$}}}}
\newcommand{\CII}{\mathrel{\text{\scalebox{1.07}{$\perp\mkern-10mu\perp\mkern-10mu\perp$}}}}
\newcommand{\RV}[1]{\ensuremath{\mathsf{#1}}}
\newcommand{\URV}[1]{\ensuremath{\underline{\RV{#1}}}}
\newcommand{\PA}[2]{\ensuremath{\text{Pa}_{#1}(#2)}}
\newcommand{\ND}[2]{\ensuremath{\text{ND}_{#1}(#2)}}
\newcommand{\CH}[2]{\ensuremath{\text{Ch}_{#1}(#2)}}
\newcommand{\DE}[2]{\ensuremath{\text{De}_{#1}(#2)}}
\newcommand{\ID}[1]{\ensuremath{\text{Id}_{#1}}}
\newcommand{\utimes}{\ensuremath{\underline{\otimes}}}
\newcommand{\prob}[1]{\ensuremath{\mathbb{#1}}}
\newcommand{\kernel}[1]{\ensuremath{\mathbf{#1}}}
\newcommand{\model}[1]{\ensuremath{\mathbb{#1}}}
\newcommand{\seedo}{\ensuremath{\mathbb{T}}}
\newcommand{\diagram}[1]{\ensuremath{\mathscr{#1}}}
\newcommand{\sigalg}[1]{\ensuremath{\mathcal{#1}}}
\newcommand{\vecRV}[1]{\ensuremath{\mathsfbfit{#1}}}
\newcommand{\vecVal}[1]{\ensuremath{\mathbf{#1}}}
\newcommand{\prodSet}[1]{\ensuremath{\mathbf{#1}}}
\newcommand{\indx}[1]{\ensuremath{\mathcal{#1}}}
\newcommand{\nod}[1]{\ensuremath{\mathsfit{#1}}}

\makeatletter
\newcommand*\bigcdot{\mathpalette\bigcdot@{.5}}
\newcommand*\bigcdot@[2]{\mathbin{\vcenter{\hbox{\scalebox{#2}{$\m@th#1\bullet$}}}}}
\makeatother

\tikzset{
    triangle/.style = {regular polygon, regular polygon sides=3 },
    node rotated/.style = {rotate=90},
    border rotated/.style = {shape border rotate=90},
    dist/.style = {triangle,draw,border rotated, inner sep=0pt},
    smalldist/.style = {triangle,draw,border rotated},
    kernel/.style={rectangle,draw,inner sep = 2pt},
    expectation/.style = {triangle,draw,inner sep=0pt,shape border rotate=270},
    copymap/.style = {circle,fill,inner sep=1pt}}

\newcommand\DCI{
    \begin{tikzpicture}[scale=0.35]
    \draw[->] (1,0) -- (0,0);
    \draw (0.6,0) -- (0.6,0.75);
    \draw (0.4,0) -- (0.4,0.75);
    \end{tikzpicture}
}

\newcommand\splitter[1]{%
\begin{tikzpicture}[scale=#1]
\draw (0,-1) -- (0,0);
\draw (0,0) to [bend right] (1,1);
\draw (0,0) to [bend left] (-1,1);
\end{tikzpicture}
}

\newcommand\stopper[1]{%
\begin{tikzpicture}[scale=#1]
\draw[-{Rays [n=8]}] (0,-1) -- (0,0);
\end{tikzpicture}
}

\newcommand\source[1]{%
\begin{tikzpicture}[scale=#1]
\path (0,0) node[prob,fill=gray] (P) {};
\draw (P) -- ($(P.east) + (1,0)$);
\end{tikzpicture}
}

\DeclareMathOperator*{\argmax}{arg\,max}
\DeclareMathOperator*{\argmin}{arg\,min}
\DeclareMathOperator*{\arginf}{arg\,inf}
\DeclareMathOperator*{\argsup}{arg\,sup}

\newcommand{\cheng}[1]{ {\color{purple}[{\bf Cheng:~{#1}}]} }

\title{Causal questions are questions that are answered by a function}
\date{\today}

\author{ David Johnston }

\begin{document}

\maketitle


% \begin{abstract}
The typical way to construct a probability model is to define a sample space and an ambient probability measure, which together form a ``probability space''. This is unweildly for causal models where we are often interested in pulling probability models apart and reassembling them in different ways. We introduce a different approach to building probability models which we call ``modular probability'' that uses a \emph{modelling context} instead of a sample space, in which random variables are akin to types in a computer program and probability measures and Markov kernels are akin to functions. We illustrate the use of modular probability with examples of decision theoretic causal models, Causal Bayesian Networks and Potential Outcomes models.




% \end{abstract}
\tableofcontents


%!TEX root = main.tex


\section{Probability}\label{sec:vague_variables}

\subsection{Section outline}

This section introduces the mathematical foundations used throughout the rest of the paper. The first subsection briefly introduces probability theory, which is likely to be familiar to many readers, as well as how string diagrams can be used to represent probabilistic functions (or \emph{Markov kernels}), which may be less familiar. We use string diagrams for probabilstic reasoning in a number of places, and this section is intended to help interpret mathematical statements in this form.

The second subsection discusses the interpretation of probabilistic variables. Our formalisation of probabilistic variables is standard -- we define them as measurable functions on a fundamental probability set $\Omega$. We discusses how this formalisation can be connected to statements about the real world via \emph{measurement processes}, and distinguishes observed variables (which are associated with measurement processes) from unobserved variables (which are not associated with measurement processes). This section is not part of the mathematical theory of probability gap models, but it is relevant when one wants to apply this theory to real problems or to understand how the theory of probability gap models relates to other theories of causal inference.

Finally, we introduce \emph{probability gap models}. Probability gap models are a generalisation of probability models, and to understand the rest of this paper a reader needs to understand what a probability gap model is, how we define the common kinds of probability gap models used in this paper and what conditional probabilities and conditional independence statements mean for probability gap models.

\subsubsection{Brief outline of probability gap models}

We consider a probability model to be a probability space $(\Omega,\sigalg{F},\mu)$ along with a collection of random variables. However, if I want to use probabilistic models to support decision making, then I need function from options to probability models. For example, suppose I have two options $A=\{0,1\}$, and I want to compare these options based on what I expect to happen if I choose them. If I choose option $0$, then I can (perhaps) represent my expectations about the consequences with a probability model, and if I choose option $1$ I can represent my expectations about the consequences with a different probability model. I can compare the two consequences, then decide which option seems to be better. To make this comparison, I have used a function from elements of $A$ to probability models. A function that takes elements of some set as inputs (which may or may not be decisions) and returns probability models is a \emph{probability gap model}, and the set of inputs it accepts is a \emph{probability gap}.

We are particularly interested in probability gap models where the consequences of all inputs share some marginal or conditional probabilities. The simplest example of a model like this can be represented by a probability distribution $\prob{P}^{\RV{X}}$ for some variable $\RV{X}:\Omega\to X$. Such a probability distribution is consistent with many base measures on the fundamental probability set $\Omega$, and so we can consider the choice of base measure to be a probability gap. Not every probability distribution over $X$ can define a probability gap model in this way. In particular, we need $\prob{P}^{\RV{X}}$ to assign probability 0 to outcomes that are mathematically impossible according to the definition of $\RV{X}$ to ensure that there is some base measure that features $\prob{P}^{\RV{X}}$ as a marginal. We call probability gap models represented by probability distributions \emph{order 0 probability gap models}.

Higher order probability gap models can be represented by conditional probabilities $\prob{P}^{\RV{Y}|\RV{X}}$ or pairs of conditional probabilities $\{\prob{P}^{\RV{X}|\RV{W}},\prob{P}^{\RV{Z}|\RV{WXY}}\}$, which we call \emph{order 1} and \emph{order 2} models respectively. Decision functions in data-driven decision problems correspond to probability gaps in order 2 models, as we discuss in Section \ref{sec:seedo_models}, which makes this type of model particularly interesting for our purposes. We also require these to be valid, and we define conditions for validity and prove that they are sufficient to ensure that models represented by conditional probabilities can in fact be mapped to base measures on the fundamental probability set.

A conditional independence statement in a probability gap model means that the corresponding conditional independence statement holds for all base measures in the range of the function defined by the model. It is possible to deduce conditional independences from ``independences'' in the conditional probabilities that we use to represent these models, and conditional independences can imply the existence of conditional probabilities with certain independence properties.

We can consider causal Bayesian networks to represent order 2 probability gap models. That is, a causal Bayesian network represents a function $\prob{P}$ that take inserts from some set $A$ of conditional probabilities and returns a probability model, and it does so in such a way that there are a pair of conditional probabilities $\{\prob{P}^{\RV{X}|\RV{W}},\prob{P}^{\RV{Z}|\RV{WXY}}\}$ shared by all models in the codomain of $\prob{P}$. The observational distribution is the value of $\prob{P}(\text{obs})$ for some \emph{observational insert} $\text{obs}\in A$, and other choices of inserts yield interventional distributions. Defining causal Bayesian networks in this manner resolves two areas of difficulty with causal Bayesian networks. First, under the standard definition of causal Bayesian networks interventional probabilities may fail to exist; with our perspective we can see that this arises due to misunderstanding the domain of $\prob{P}$. Secondly, there may be multiple distributions that differ in important ways that all satisfy the standard definition of ``interventional distributions''. The one-to-many relationship between observations and interventions is a basic challenge of causal inference, the problem arises when this relationship is obscured by calling multiple different things ``the interventional distribution''. If we consider causal Bayesian networks to represent order 2 probability gap models, we avoid doing this. 


\subsection{Standard probability theory}

\begin{definition}[Probability measure]
Given a measure space $(X,\sigalg{X})$, a probability measure is a $\sigma$-additive function $\mu:\sigalg{X}\to [0,1]$ such that $\mu(\emptyset)=0$ and $\mu(X)=1$. We write $\Delta(X)$ for the set of all probability measures on $(X,\sigalg{X})$.
\end{definition}

\begin{definition}[Markov kernel]
Given measure spaces $(X,\sigalg{X})$, $(Y,\sigalg{Y})$ $\RV{Y}:\Omega\to Y$, a Markov kernel $\prob{Q}:X\kto Y$ is a map $Y\times \sigalg{X}\to [0,1]$ such that
\begin{enumerate}
	\item $y\mapsto \prob{Q}(A|y)$ is $\sigalg{B}$-measurable for all $A\in \sigalg{X}$
	\item $A\mapsto \prob{Q}(A|y)$ is a probability measure on $(X,\sigalg{X})$ for all $y\in Y$
\end{enumerate}
\end{definition}

\begin{definition}[Delta measure]
Given a measureable space $(X,\sigalg{X})$ and $x\in X$, $\delta_x\in \Delta(X)$ is the measure defined by $\delta_x(A)=\llbracket x\in A \rrbracket$.
\end{definition}

\begin{definition}[Probability space]
A probability space is a triple $(\mu,\Omega,\sigalg{F})$, where $\mu$ is a base measure on $\sigalg{F}$.
\end{definition}

\begin{definition}[Variable]
Given a measureable space $(\Omega,\sigalg{F})$ and a set of values $(X,\sigalg{X})$, an \emph{$X$-valued variable} is a measurable function $\RV{X}:\Omega\to X$.
\end{definition}

\begin{definition}[Sequence of variables]
Given a measureable space $(\Omega,\sigalg{F})$ and two variables $\RV{X}:\Omega\to X$, $\RV{Y}:\Omega\to Y$, $(\RV{X},\RV{Y}):\Omega\to X\times Y$ is the variable $\omega\mapsto (\RV{X}(\omega),\RV{Y}(\omega))$.
\end{definition}

\begin{definition}[Marginal distribution with respect to a probability space]\label{def:pushforward}
Given a probability space $(\mu,\Omega,\sigalg{F})$ and a variable $\RV{X}:\Omega\to (X,\sigalg{X})$, we can define the \emph{marginal distribution} of $\RV{X}$ with respect to $\mu$, $\mu^{\RV{X}}:\sigalg{X}\to [0,1]$ by $\mu^{\RV{X}}(A):=\mu(\RV{X}\yields A)$ for any $A\in \sigalg{X}$.
\end{definition}

\begin{lemma}[Marginal distribution as a kernel product]\lemma{lem:pushf_kprod}
Given a probability space $(\mu,\Omega,\sigalg{F})$ and a variable $\RV{X}:\Omega\to (X,\sigalg{X})$, define $\kernel{F}_{\RV{X}}:\Omega\kto X$ by $\kernel{F}_{\RV{X}}(A|\omega)=\delta_{\RV{X}(\omega)}(A)$, then
\begin{align}
	\mu^{\RV{X}} = \mu\kernel{F}_{\RV{X}}
\end{align}
\end{lemma}

\begin{proof}
Consider any $A\in \sigalg{X}$.
\begin{align}
	\mu \kernel{F}_{\RV{X}}(A) &= \int_\Omega \delta_{\RV{X}(\omega)}(A) \mathrm{d}\mu(\omega)\\
	&= \int_{\RV{X}^{-1}(\omega)} \mathrm{d}\mu(\omega)\\
	&= \mu^{\RV{X}}(A)
\end{align}
\end{proof}

\subsection{Not quite standard probability theory}

Instead of having probability distributions and Markov kernels as two different kinds of thing, we can identify probability distributions with Markov kernels whose domain is a one element set $\{*\}$.

\begin{definition}[Probability measures as Markov kernels]
Given $(X,\sigalg{X})$ and $\mu\in \Delta(X)$, the Markov kernel $\kernel{K}:\{*\}\kto X$ given by $\kernel{K}(A|*)=\mu(A)$ for all $A\in \sigalg{X}$ is the Markov kernel associated with the probability measure $\mu$. We will use probability measures and their associated Markov kernels interchangeably, as it is transparent how to get from one to another.
\end{definition}

\begin{definition}[Regular conditional distribution]\label{def:disint}
Given a probability space $(\mu,\Omega)$ and variables $\RV{X}:\Omega\to X$, $\RV{Y}:\Omega\to Y$, the probability of $\RV{Y}$ given $\RV{X}$ is any Markov kernel $\mu^{\RV{Y}|\RV{X}}:X\kto Y$ such that
\begin{align}
	\mu^{\RV{XY}}(A\times B)&=\int_{A} \mu^{\RV{Y}|\RV{X}}(B|x) \mathrm{d}\mu^{\RV{X}}(x) &\forall A\in \sigalg{X}, B\in \sigalg{Y}\\
	&\iff\\
	\mu^{\RV{XY}}&= \tikzfig{disint_def}\label{eq:conditional} 
\end{align}
\end{definition}

We define higher order conditionals as ``conditionals of conditionals''

\begin{definition}[Regular higher order conditionals]
Given a probability space $(\mu,\Omega)$ and variables $\RV{X}:\Omega\to X$, $\RV{Y}:\Omega\to Y$ and $\RV{Z}:\Omega\to Z$, a higher order conditional $\mu^{\RV{Z}|(\RV{Y}|\RV{X})}:X\times Y\to Z$ is any Markov kernel such that, for some $\mu^{\RV{Y}|\RV{X}}$, 
\begin{align}
	\mu^{\RV{ZY}|\RV{X}}(B\times C|x) &=\int_B \mu^{\RV{Z}|(\RV{Y}|\RV{X})}(C|x,y)\mu^{\RV{Y}|\RV{X}}(dy|x)\\ 
	&\iff
	\mu^{\RV{ZY}|\RV{X}} &= \tikzfig{disintegration_existence}\label{eq:disint_def}
\end{align}
\end{definition}

Higher order conditionals are useful because $\mu^{\RV{Z}|(\RV{Y}|\RV{X})}$ is a version of $\mu^{\RV{X}|\RV{YX}}$, so if we're given $\mu^{\RV{ZY}|\RV{X}}$ and we can find some $\mu^{\RV{Z}|(\RV{Y}|\RV{X})}$ then we have a version of $\mu^{\RV{X}|\RV{YX}}$. This also hold for conditional with respect to probability sets, which we will introduce later (Theorem \ref{th:higher_order_conditionals}).

Furthermore, given regular $\mu^{\RV{XY}|\RV{Z}}$ and $\RV{X}$, $\RV{Y}$ standard measurable, it has recently been proven that a regular higher order conditional $\mu^{\RV{Z}|(\RV{Y}|\RV{X})}$ exists \citet{bogachev_kantorovich_2020}, Theorem 3.5. See also Theorem \ref{th:ho_cond_psets} for the extension of this theorem to probability sets.

\subsection{Probabilistic models for causal inference}

The sample space $(\Omega,\sigalg{F})$ along with our collection of variables is a ``model skeleton'' -- it tells us what kind of data we might see. The process $\proc{S}$ which tells us which part of the world we're interested in is related to the model $\Omega$ and the observable variables by the criterion of \emph{consistency with observation}. The kind of problem we are mainly interested in here is one where we make use of data to help make decisions under uncertainty. Probabilistic models have a long history of being used for this purpose, and our interest here is in constructing probabilistic models that can be attached to our variable ``skeleton''. 

Given a model skeleton, a common approach to attaching a probabilistic model involves defining a base measure $\mu$ on $(\Omega,\sigalg{F})$ which yields a probability space $(\Omega,\sigalg{F},\mu)$. For causal inference, we need a to generalise this approach, because we need to handle \emph{choices}. If I have different options I can choose, and I want to use a model to compare the options according to some criteria, then I need a model that can accept a choice and output the expected result of that choice. According to this model, anything that we consider a ``consequence of a choice'' doesn't have a definite probability, because it depends on the choice we make.

In general, we might have arbitrary sets of choices that map to probability models in an arbitrary way. However, we are here interested in a simpler case: we suppose that there are a number of points at which we can act, and prior to acting we can observe some variables, and we are able to choose probabilistic maps from observations to acts. We also assume that, given the same observation and the same act, the same consequence is expected. That is, the consequences do not depend directly way on the choice of map from observations to acts.

These assumptions together imply that our model should contain a number of fixed conditional probabilities -- the probabilities of consequences given observations and acts -- and a number of ``choosable'' conditional probabilities -- the probabilities of acts given observations. The fixed conditional probabilities form a probability model with \emph{gaps}, and those gaps correspond to choices we can make. When we combine the fixed conditional probabilities and a choice of a conditional probability for each gap, we get a regular probability model. The terminology of ``probability gaps'' comes from \citet{hajek_what_2003}. 

To restate our general approach: we model decision problems with a collection of fixed conditional probabilities and a collection of choosable conditional probabilities, and combine the fixed conditionals with particular choices to get a probability measure. Two issues present themselves here: firstly, what \emph{is} a collection of conditional probabilities without a fixed underlying probability measure? Secondly, we need to ensure that our chosen collection of conditional probabilities actually does induce a probability model. We address these questions with \emph{probability sets}. A probability set is a collection of probability measures on $(\Omega,\sigalg{F})$, and we identify a collection of conditional probabilities with the set of probability measures that induce those conditional probabilities. We then define an operation $\odot$ for combining conditional probabilities, and a criterion of \emph{validity} such that a collection of valid conditional probabilities recursively combined using $\odot$ is guaranteed to corresponds to a non-empty probability set.

\subsection{Probability sets}

A probability set is a set of probability measures. This section establishes a number of useful properties of conditional probability with respect to probability sets. Unlike conditional probability with repsect to a probability space, conditional probabilities don't always exist for probability sets. Where they do, however, they are almost surely unique and we can marginalise and disintegrate them to obtain other conditional probabilities with respect to the same probability set.

\begin{definition}[Probability set]
A probability set $\prob{P}_{\{\}}$ on $(\Omega,\sigalg{F})$ is a collection of probability measures on $(\Omega,\sigalg{F})$. In other words it is a subset of $\mathscr{P}(\Delta(\Omega))$, where $\mathscr{P}$ indicates the power set.
\end{definition}

Given a probability set $\prob{P}_{\{\}}$, we define marginal and conditional probabilities as probability measures and Markov kernels that satisfy Definitions \ref{def:pushforward} and \ref{def:disint} respectively for \emph{all} base measures in $\prob{P}_{\{\}}$. There are generally multiple Markov kernels that satisfy the properties of a conditional probability with respect to a probability set, and this definition ensures that marginal and conditional probabilities are ``almost surely'' unique (Definition \ref{def:asequal}) with respect to probability sets.

\begin{definition}[Marginal probability with respect to a probability set]
Given a sample space $(\Omega,\sigalg{F})$, a variable $\RV{X}:\Omega\to X$ and a probability set $\prob{P}_{\{\}}$, the marginal distribution $\prob{P}_{\{\}}^{\RV{X}}=\prob{P}_\alpha^{\RV{X}}$ for any $\prob{P}_\alpha\in\prob{P}_{\{\}}$ if a distribution satisfying this condition exists. Otherwise, it is undefined.
\end{definition}

\begin{definition}[Regular conditional distribution with respect to a probability set]\label{def:cprob_pset}
Given a fundamental probability set $\Omega$ variables $\RV{X}:\Omega\to X$ and $\RV{Y}:\Omega\to Y$ and a probability set $\prob{P}_{\{\}}$, a conditional $\prob{P}_{\{\}}^{\RV{Y}|\RV{X}}$ is any Markov kernel $X\kto Y$ such that $\prob{P}_{\{\}}^{\RV{Y}|\RV{X}}$ is an $X\kto Y$ disintegration of $\prob{P}_\alpha^{\RV{XY}}$ for all $\prob{P}_\alpha\in \prob{P}_{\{\}}$. If no such Markov kernel exists, $\prob{P}_{\{\}}^{\RV{Y}|\RV{X}}$ is undefined.
\end{definition}

\begin{definition}[Regular higher order conditional with respect to a probability set]\label{def:cprob_pset}
Given a fundamental probability set $\Omega$, variables $\RV{X}:\Omega\to X$, $\RV{Y}:\Omega\to Y$ and $\RV{Z}:\Omega\to Z$ and a probability set $\prob{P}_{\{\}}$, if $\prob{P}_{\{\}}^{\RV{ZY}|\RV{X}}$ exists then a higher order conditional $\prob{P}_{\{\}}^{\RV{Z}|(\RV{Y}|\RV{X})}$ is any Markov kenrel $X\times Y\kto Z$ that is a higher order conditional of some version of $\prob{P}_{\{\}}^{\RV{ZY}|\RV{X}}$. If no $\prob{P}_{\{\}}^{\RV{ZY}|\RV{X}}$ exists, $\prob{P}_{\{\}}^{\RV{Z}|(\RV{Y}|\RV{X})}$ is undefined.
\end{definition}

Under the assumption of standard measurable spaces, the existence of a conditional probability $\prob{P}_{\{\}}^{\RV{ZY}|\RV{X}}$ implies the existence of a higher order conditional $\prob{P}_{\{\}}^{\RV{Z}|(\RV{Y}|\RV{X})}$ with respect to the same probability set (Theorem \ref{th:ho_cond_psets}). $\prob{P}_{\{\}}^{\RV{Z}|(\RV{Y}|\RV{X})}$ is in turn a version of the conditional $\prob{P}_{\{\}}^{\RV{Z}|\RV{YX}}$ (Theorem \ref{th:higher_order_conditionals}). Thus, from the existence of $\prob{P}_{\{\}}^{\RV{ZY}|\RV{X}}$ we can derive the existence of $\prob{P}_{\{\}}^{\RV{Z}|\RV{YX}}$.

% \begin{lemma}[Equivalence of pushforward definitions]\label{lem:prod_pushf}
% Given a probability space $\kernel{M}:W\to \Omega$ and $\RV{X}:\Omega\to X$, define $\kernel{K}^{\RV{X}|\RV{W}}:W\kto X$ by $\kernel{K}^{\RV{X}|\RV{W}}(x|w):=\kernel{M}(\RV{X}\yields x|w)$ for any $x\in X$m $w\in W$ and $\kernel{L}^{\RV{X}}:W\kto X$ by
% \begin{align}
% 	\kernel{L}^{\RV{X}|\RV{W}} = \kernel{M}\kernel{F}_{\RV{X}}
% \end{align}
% Then
% \begin{align}
% \kernel{L}^{\RV{X}|\RV{W}} =\kernel{K}^{\RV{X}|\RV{W}}
% \end{align}
% \end{lemma}

% \begin{proof}
% For any $x\in X$, $w\in W$
% \begin{align}
% 	\kernel{L}^{\RV{X}|\RV{W}}(x|w) &= \sum_{\omega\in \Omega} \llbracket x=\RV{X}(\omega)\rrbracket \kernel{M}(\omega|w)\\
% 									&= \sum_{\omega\in \RV{X}^{-1}(x)} \kernel{M}(\omega|w)\\
% 									&= \kernel{M}(\RV{X}\yields x|w)\\
% 									&= \kernel{K}^{\RV{X}|\RV{W}}(x|w)
% \end{align}
% \end{proof}

\subsection{Semidirect product and almost sure equality}

The operation used in Equation \ref{eq:conditional} that combines $\mu^{\RV{X}}$ and $\mu^{\RV{Y}|\RV{X}}$ is something we will use repeatedly, so we call it the \emph{semidirect product} and give it the symbol $\odot$. We also define a notion of almost sure equality with respect to $\odot$: $\kernel{K}\overset{\mu^{\RV{X}}}{\cong} \kernel{L}$ if $\mu^{\RV{X}}\odot \kernel{K}=\mu^{\RV{X}}\odot\kernel{L}$. Thus if two terms are almost surely equal, they are substitutable when they both appear in a semidirect product.

\begin{definition}[Semidirect product]\label{def:copyproduct}
Given $\prob{K}:X\kto Y$ and $\prob{L}:Y\times X\kto Z$, define the copy-product $\prob{K}\odot\prob{L}:X\to Y\times Z$ as
\begin{align}
	\prob{K}\odot\prob{L}:&= \text{copy}_X(\prob{K}\otimes \text{id}_X)(\text{copy}_Y\otimes\text{id}_X )(\text{id}_Y \otimes \prob{L})\\
							&= \tikzfig{copy_product}\\
							&\iff\\
	(\prob{K}\odot\prob{L})(A\times B|x) &= \int_A \prob{L}(B|y,x)\prob{K}(dy|x)&A\in \sigalg{Y},B\in\sigalg{Z}
\end{align}
\end{definition}

\begin{lemma}[Semidirect product is associative]
Given $\prob{K}:X\kto Y$, $\prob{L}:Y\times X\kto Z$ and $\prob{M}:Z\times Y\times X\kto W$
\begin{align}
	(\prob{K}\odot \prob{L})\odot \prob{Z} &= \prob{K}\odot(\prob{L}\odot\prob{Z})\\
\end{align}
\end{lemma}

\begin{proof}
\begin{align}
	(\prob{K}\odot \prob{L})\odot \prob{M} &= \tikzfig{odot_assoc_1}\\
											&=  \tikzfig{odot_assoc_2}\\
											&= \prob{K}\odot (\prob{L}\odot \prob{M})
\end{align}
\end{proof}

Two Markov kernels are almost surely equal with respect to a probability set $\prob{P}_{\{\}}$ if the semidirect product $\odot$ of all marginal probabilities of $\prob{P}_\alpha^\RV{X}$ with each Markov kernel is identical.

\begin{definition}[Almost sure equality]\label{def:asequal}
Two Markov kernels $\kernel{K}:X\kto Y$ and $\kernel{L}:X\kto Y$ are almost surely equal $\overset{\prob{P}_{\{\}}}{\cong}$ with respect to a probability set $\prob{P}_{\{\}}$ and variable $\RV{X}:\Omega\to X$ if for all $\prob{P}_\alpha \in \prob{P}_{\{\}}$,
\begin{align}
	\prob{P}^{\RV{X}}_\alpha\odot \kernel{K}=\prob{P}^{\RV{X}}_\alpha\odot \kernel{L}
\end{align}
\end{definition}

\begin{lemma}[Conditional probabilities are almost surely equal]
If $\kernel{K}:X\kto Y$ and $\kernel{L}:X\kto Y$ are both versions of $\prob{P}_{\{\}}^{\RV{Y}|\RV{X}}$ then $\kernel{K}\overset{\prob{P}_{\{\}}}{\cong}\kernel{L}$
\end{lemma}

\begin{proof}
For all $\prob{P}_\alpha \in \prob{P}_{\{\}}$
\begin{align}
	\prob{P}^{\RV{X}}_\alpha\odot \kernel{K} &= \prob{P}^{\RV{XY}}_\alpha\\
	&= \prob{P}^{\RV{X}}_\alpha\odot \kernel{L}
\end{align}
\end{proof}

\begin{lemma}[Substitution of almost surely equal Markov kernels]
Given $\prob{P}_{\{\}}$, if $\kernel{K}:X\times Y \kto Z$ and $\kernel{L}:X\times Y \kto Z$ are almost surely equal $\kernel{K}\overset{\prob{P}_{\{\}}}{\cong}\kernel{L}$, then for any $\prob{P}_\alpha\in \prob{P}_{\{\}}$
\begin{align}
	\prob{P}_\alpha^{\RV{Y}|\RV{X}}\odot \kernel{K} &\overset{a.s.}{\cong} \prob{P}_\alpha^{\RV{Y}|\RV{X}}\odot \kernel{L}
\end{align}
\end{lemma}

\begin{proof}
For any $\prob{P}_\alpha\in\prob{P}_{\{\}}$
\begin{align}
	\prob{P}_\alpha^{\RV{XY}}\odot \kernel{K} &= (\prob{P}_\alpha^{\RV{X}}\odot \prob{P}_{\{\}}^{\RV{Y}|\RV{X}})\odot \kernel{K}\\
											  &= \prob{P}_\alpha^{\RV{X}}\odot (\prob{P}_{\{\}}^{\RV{Y}|\RV{X}}\odot \kernel{K})\\
											  &= \prob{P}_\alpha^{\RV{X}}\odot (\prob{P}_{\{\}}^{\RV{Y}|\RV{X}}\odot \kernel{L})
\end{align}
\end{proof}

\begin{lemma}[Semidirect product of conditionals is a joint conditional]\label{lem:joint_conditional}
Given a probability set $\prob{P}_{\{\}}$ on $(\Omega,\sigalg{F})$ along with conditional probabilities $\prob{P}_{\{\}}^{\RV{Y}|\RV{X}}$ and $\prob{P}_{\{\}}^{\RV{Z}|\RV{XY}}$, $\prob{P}_{\{\}}^{\RV{YZ}|\RV{X}}$ exists and is equal to
\begin{align}
	\prob{P}_{\{\}}^{\RV{YZ}|\RV{X}} &= \prob{P}_{\{\}}^{\RV{Y}|\RV{X}}\odot \prob{P}_{\{\}}^{\RV{Z}|\RV{XY}}\\
\end{align}
\end{lemma}

\begin{proof}
By definition, for any $\prob{P}_\alpha\in \prob{P}_{\{\}}$
\begin{align}
	\prob{P}_\alpha^{\RV{XYZ}} &= \prob{P}_\alpha^{\RV{X}}\odot \prob{P}_\alpha^{\RV{YZ}|\RV{X}}\\
							   &= \prob{P}_\alpha^{\RV{X}}\odot(\prob{P}_\alpha^{\RV{Y}|\RV{X}}\odot \prob{P}_\alpha^{\RV{Z}|\RV{YX}})\\
							   &= \prob{P}_\alpha^{\RV{X}}\odot(\prob{P}_{\{\}}^{\RV{Y}|\RV{X}}\odot \prob{P}_{\{\}}^{\RV{Z}|\RV{YX}})
\end{align}
\end{proof}



% \begin{theorem}[Disintegrations are conditional probabilities]
% Suppose we have a fundamental probability set $\Omega$ variables $\RV{W}:\Omega\to W$, $\RV{X}:\Omega\to X$, $\RV{Y}:\Omega\to Y$ and $\RV{Z}:\Omega\to Z$ and a probability set $\prob{P}_{\{\}}$ such that $\prob{P}_{\{\}}^{\RV{X}|\RV{Y}}$ is a $\RV{Y}|\RV{X}$ conditional probability and there is some $\kernel{K}^{\RV{$
% \end{theorem}

% Given a conditional probability with respect to a probability gap model, we can also find additional conditional probabilities by disintegrating the original conditional probability.

% \begin{lemma}[Recursive disintegration]
% Suppose we have a fundamental probability set $\Omega$, variables $\RV{W}:\Omega\to W$, $\RV{X}:\Omega\to X$ and $\RV{Y}:\Omega\to Y$, $\RV{Z}:\Omega\to Z$ and a probability set $\prob{P}_{\{\}}$ such that $\prob{P}_{\{\}}^{\RV{X}|\RV{Y}}$ is a $\RV{Y}|\RV{X}$ conditional probability. Define $\prob{Q}_{\{\}}$ as the largest probability set such that $\prob{Q}_{\{\}}^{\RV{Y}|\RV{X}}=\prob{P}_{\{\}}^{\RV{Y}|\RV{X}}$. Then if $\prob{Q}_{\{\}}^{\RV{Z}|\RV{W}}$ is a $\RV{Z}|\RV{W}$ conditional probability of $\prob{Q}_{\{\}}$, it is also a $\RV{Z}|\RV{W}$ conditional probability of $\prob{P}_{\{\}}$.
% \end{lemma}

% \begin{proof}
% $\prob{Q}_{\{\}}\supset \prob{P}_{\{\}}$, so any conditional probability of $\prob{Q}_{\{\}}$ is also a conditional probability of $\prob{P}_{\{\}}$.
% \end{proof}

\subsection{Probability sets defined by marginal and conditional probabilities}

So far we have defined probability sets and conditional probabilities as Markov kernels that can sometimes be derived from a probability set. Actually, we are interested in working in the opposite direction: starting with conditional probabilities and working with probability sets defined by them. We need to be a little bit careful in doing this: we can't take an arbitrary Markov kernel $\kappa:X\kto Y$ and declare it to be a conditional probability $\prob{P}_{\{\}}^{\RV{Y}|\RV{X}}$ for some $\RV{X}:\Omega\to X$ and $\RV{Y}:\Omega\to Y$. The reason for this is that some collections of variables cannot have arbitrary conditional probabilities. 

Consider, for example, $\Omega=\{0,1\}$ with $\RV{X}=(\RV{Z},\RV{Z})$ for $\RV{Z}:=\text{id}_{\Omega}$ and any measure $\kappa\in \Delta(\{0,1\}^2)$ such that $\kappa(\{1\}\times \{0\})>0$. Note that $\RV{X}^{-1}(\{1\}\times \{0\})=\RV{Z}^{-1}(\{1\})\cap \RV{Z}^{-1}(\{0\})=\emptyset$. Thus for any probability measure $\mu\in \Delta(\{0,1\})$, $\mu^{\RV{X}}(\{1\}\times \{0\}) = \mu(\emptyset)=0 $ and so $\kappa$ cannot be the marginal distribution of $\RV{X}$ for any base measure at all. A \emph{valid distribution} is a distribution associated with a particular variable that defines a nonempty set of base measures on $\Omega$ (Theorem \ref{th:completion}), and \emph{valid conditionals} are a set of conditional probabilities closed under $\odot$ and reducing to valid distributions when conditioning on a trivial variable (Lemma \ref{lem:valid_extendability}).

\begin{definition}[Valid distribution]\label{def:valid_dist}
A valid $\RV{X}$ probability distribution $\prob{P}^{\RV{X}}$ is any probability mesure on $\Delta(X)$ such that $\RV{X}^{-1}(A)=\emptyset\implies \prob{P}^{\RV{X}}(A) = 0$ for all $A\in\sigalg{X}$.
\end{definition}

\begin{definition}[Valid conditional]\label{def:valid_conditional_prob}
Given $(\Omega,\sigalg{F})$, $\RV{X}:\Omega\to X$, $\RV{Y}:\Omega\to Y$ a \emph{valid $\RV{Y}|\RV{X}$ conditional probability} $\prob{P}^{\RV{Y}|\RV{X}}$ is a Markov kernel $X\kto Y$ such that it assigns probability 0 to contradictions:
\begin{align}
    \forall B\in \sigalg{Y}, x\in X: (\RV{X},\RV{Y})\yields \{x\}\times B = \emptyset \implies \left(\prob{P}^{\RV{Y}|\RV{X}}(B|x) = 0\right) \lor \left(\RV{X}\yields \{x\} = \emptyset\right)
\end{align}
\end{definition}

\begin{definition}[Probability set defined by a valid conditional]
If $\prob{P}_{\{\}}$ is a probability set such that there is a valid conditional probability $\prob{P}_{\{\}}^{\RV{Y}|\RV{X}}:X\kto Y$ and for every $\mu\in \Delta(\Omega)$ such that $\mu^{\RV{Y}|\RV{X}}\overset{\mu}{\cong}\prob{P}_{\{\}}^{\RV{Y}|\RV{X}}$, we say $\prob{P}_{\{\}}^{\overline{\RV{Y}|\RV{X}}}:=\prob{P}_{\{\}}$ is the probability set defined by $\prob{P}_{\{\}}^{\RV{Y}|\RV{X}}$.
\end{definition}

Suppose we have some collection of Markov kernels that we want to interpret as conditional probabilities $\{\prob{P}_i^{\RV{X}_i|\RV{X}_{[i-1]}}|i\in [n]\}$ which we want to define a probability set by recursively taking the semidirect product $\prob{P}_1^{\RV{X}_1}\odot (\prob{P}_2^{\RV{X}_2|\RV{X}_{1}}\odot ...)$. It is sufficient that each $\prob{P}_i^{\RV{X}_i|\RV{X}_{[i-1]}}$ is valid for the resulting probability set to be nonempty (Lemma \ref{lem:valid_extendability}).

Collections of recursive conditional probabilities often arise in causal modelling -- in particular, they are the foundation of the structural equation modelling approach \citet{richardson2013single,pearl_causality:_2009}.

Note that validity is not a necessary condition for a conditional to define a non-empty probability set. The intuition for this is: if we have some $\kernel{K}:X\kto Y$, $\kernel{K}$ might be an invalid $\RV{Y}|\RV{X}$ conditional on all of $X$, but might be valid on some subset of $X$, and so we might have some probability model $\prob{P}$ that assigns measure 0 to the bad parts of $X$ such that $\kernel{K}$ is a version of $\prob{P}^{\RV{Y}|\RV{X}}$. On the other hand, if we want to take the product of $\kernel{K}$ with arbitary valid $\RV{X}$ probabilities, then the validity of $\kernel{K}$ is necessary (Theorem \ref{th:valid_conditional_probability}).


\subsection{Probability gap models}

For reasoning about decisions, we don't just want a set of models that could explain what is going on. What we want is a function that maps choices to ``outcome'' probability models. In many cases, there might be features that all of the outcome models share -- for example, there might be some variables that are not affected by any choice, and so their marginal distribution is the same in every outcome model. There are also some other features that are entirely determined by the choice we make. For example, if there is a variable $\RV{D}:\Omega\to D$ representing the choice we make, then if we choose option $d\in D$ we must have $\prob{P}^{\RV{D}}(\{d\})=1$.

Here, by ``property'', we mean marginal or conditional probabilities. We suppose that we are able to construct models such that ``properties common to all outcome models'' and ``properties determined by choices'' together represent everything we can say about the appropriate model for our decision problem. That is, we can define a probability set corresponding to all outcome models and, for each choice, a probability set corresponding to all models consistent with the things known to be fixed by that choice, and then the result of picking a certain choice is the intersection of the probability set for all outcome models and the probability set associated with that choice.

We call this kind of map a \emph{probability gap model} (the terminology is from \citet{hajek_interpretations_2019}, though our meaning is a little different). The set of all outcome models represents most of our knowledge relevant to the outcome, but there's a gap -- it doesn't say which choice we will eventually make. The set of choices is the collection of different ways that the gap could be filled.

We don't have an axiomatic justification for using probability gap models to reason about making decisions. There are two considerations motivating this choice: first, they allows us to recover standard representations of decision problems and standard kinds of causal models, and secondly the use of probability sets means that probability gap models are in many ways similar to ordinary probability models.

\begin{itemize}
	\item A fixed probability set $\prob{P}_{\{\}}\subset \Delta(\Omega)$ which we call the \emph{model}
	\item A collection of probability sets $A\subset \Delta(\Omega)$ that we call \emph{choices}
	\item A map $\prob{P}_\square: A\to \mathscr{P}(\Delta(\Omega))$ defined by $\prob{P}_\alpha:=\prob{P}_\square(\alpha)=\prob{P}_{\{\}}\cap\alpha$
\end{itemize}

We require that the choices are compatible with the model in the sense that $\prob{P}_{\{\}}\cap\alpha\neq \emptyset$ for all $\alpha\in A$. Here, we will limit our attention to a a particular type of probability gap model, where we define the probability set $\prob{P}_{\{\}}$ is defined by a conditional probability and each choice is defined by a marginal probability relative to the same variable.

\begin{definition}[Conditional probability model]
A \emph{conditional probability model} $\prob{P}_\square$ is a probability gap model $(\prob{P}_{\{\}}^{\overline{\RV{Y}|\RV{X}}},A)$ such that each $\alpha\in A$ is some probability set  defined by an $\RV{X}$-valid marginal probability $\alpha^{\overline{\RV{X}}}$.
\end{definition}

We will compute the intersection $\prob{P}_\alpha$ between the model $\prob{P}_{\{\}}$ and a choice $\alpha\in A$ as the probability set $\prob{P}_{\alpha}^{\overline{\RV{XY}}}$ such that:
\begin{align}
	\prob{P}_\alpha^{\RV{XY}} &= \alpha^{\RV{X}}\odot \prob{P}_{\{\}}^{\RV{Y}|\RV{X}}\label{eq:semidirect_intersection}
\end{align}

This is justified by Lemma \ref{th:intersection}, which says that the probability set defined by Equation \ref{eq:semidirect_intersection} is equivalent to the intersection of $\alpha$ and $\prob{P}_{\{\}}$.

If the conditional probability $\prob{P}_{\{\}}^{\RV{Y}|\RV{X}}$ and all the marginal probabilities $\alpha^{\overline{\RV{X}}}$ are valid, then by Lemma \ref{lem:valid_extendability} $\prob{P}_{\{\}}\cap\alpha \neq \emptyset$ for all $\alpha\in A$. Thus validity of all the individual parts is enough to ensure compatibility.

We can define more complex probability gap models with a similar approach where, for example, the model is specified by an incomplete collection of conditional probabilities and the choices are each a complementary collection of conditional probabilities; we call such models \emph{probability comb models} after \citet{chiribella_quantum_2008,jacobs_causal_2019}, but we will not address them in this paper.

\subsection{Example: invalidity}

Body mass index is defined as a person's weight divided by the square of their height. Suppose we have a measurment process $\proc{S}=(\proc{W},\proc{H})$ and $\proc{B}=\frac{\proc{W}}{\proc{H}^2}$ - i.e. we figure out someone's body mass index first by measuring both their height and weight, and then passing the result through a function that divides the second by the square of the first. Thus, given the random variables $\RV{W},\RV{H}$ modelling $\proc{W},\proc{H}$, $\proc{B}$ is the function given by $\RV{B}=\frac{\RV{W}}{\RV{H}^2}$. Given $x\in \mathbb{R}$, consider the conditional probability
\begin{align}
	\nu^{\RV{B}|\RV{WH}} &= \tikzfig{invalid_BMI_model} \label{eq:bmi_example}
\end{align}
Then pick some $w,h\in\mathbb{R}$ such that $\frac{w}{h^2}\neq x$ and $(\RV{W},\RV{H})\yields (w,h)\neq \emptyset$ (our measurement procedure could possibly yield $(w,h)$ for a person's height and weight). We have $\nu^{\RV{B}|\RV{WH}}(x|w,h)=1$, but 
\begin{align}
	(\RV{B},\RV{W},\RV{H})\yields \{(x,w,h)\} &= \{\omega|(\RV{W},\RV{H})(\omega) = (w,h),\RV{B}(\omega) = \frac{w}{h^2}\}\\
	&=\emptyset
\end{align}
so $\nu^{\RV{B}|\RV{WH}}$ is invalid, and there is some valid $\mu^{\RV{X}}$ such that the probability set $\prob{P}_{\{\}}$ with $\prob{P}_{\{\}}^{\RV{XY}} = \mu^{\RV{X}}\odot \nu^{\RV{Y}|\RV{X}}$ is empty.

Validity rules out conditional probabilities like \ref{eq:bmi_example}. We guess that in many cases this condition may either be trivial or unconsiously taken into account when constructing conditional probabilities. However, if we are not cognizant of the conditional our model depends on, we may inadvertently propose a model that depends on invalid conditional probabilities. For example, the conditional probability \ref{eq:bmi_example} would be used to evaluate the causal effect of body mass index in the causal diagram found in \citet{shahar_association_2009}, presuming the author used the term ``causal effect'' to depend somehow on the function $x\mapsto P(\cdot|do(\RV{B}=x))$ as is the usual convention when discussing causal Bayesian networks.

\subsubsection{Conditional independence}\label{ssec:cond_indep}

Conditional independence has a familiar definition in probability models. We define conditional independence with respect to a probability gap model to be equivalent to conditional independence with resepect to every base measure in the range of the model. This definition is closely related to the idea of \emph{extended conditional independence} proposed by \citet{constantinou_extended_2017}, see Appendix \ref{ap:eci}.

\begin{definition}[Conditional independence with respect to a probability model]
For a \emph{probability modle} $\model{P}_{\alpha}$ and variables $\RV{A},\RV{B},\RV{Z}$, we say $\RV{B}$ is conditionally independent of $\RV{A}$ given $\RV{C}$, written $\RV{B}\CI_{\model{P}_{\{\}}}\RV{A}|\RV{C}$, if
\begin{align}
	\kernel{P}_{\alpha}^{\RV{ABC}} &= \tikzfig{cond_indep1} \label{eq:cond_indep}
\end{align}
\end{definition}

\citet{cho_disintegration_2019} have shown that this definition coincides with the standard notion of conditional independence for a particular probability model (Theorem \ref{th:cho_ci_equiv}). 

Conditional independence satisfies the \emph{semi-graphoid axioms}. For all standard measurable spaces $(\Omega,\sigalg{F})$ and all probability measures $\prob{P}\in \Delta(\Omega)$:

\begin{enumerate}
	\item Symmetry: $\RV{A}\CI_{\prob{P}} \RV{B}|\RV{C}$ iff $\RV{B}\CI_{\prob{P}} \RV{A}|\RV{C}$
	\item Decomposition: $\RV{A}\CI_{\prob{P}} (\RV{B},\RV{C})|\RV{W}$ implies $\RV{A}\CI_{\prob{P}}\RV{B}|\RV{W}$ and $\RV{A}\CI_{\prob{P}_\square}\RV{C}|\RV{W}$
	\item Weak union: $\RV{A}\CI_{\prob{P}}(\RV{B},\RV{C})|\RV{W}$ implies $\RV{A}\CI_{\prob{P}}\RV{B}|(\RV{C},\RV{W})$
	\item Contraction: $\RV{A}\CI_{\prob{P}}\RV{C}|\RV{W}$ and $\RV{A}\CI_{\prob{P}}\RV{B}|(\RV{C},\RV{W})$ implies $\RV{A}\CI_{\prob{P}_\square}(\RV{B},\RV{C})|\RV{W}$
\end{enumerate}

We define \emph{universal conditional independence} with respect to a probability set as conditional independence for every probability model in the set.

\begin{definition}[Universal conditional independence]
For a \emph{probability set} $\model{P}_{\{\}}$ and variables $\RV{A},\RV{B},\RV{Z}$, we say $\RV{B}$ is universally conditionally independent of $\RV{A}$ given $\RV{C}$, written $\RV{B}\CI_{\model{P}_{\{\}}}\RV{A}|\RV{C}$, if for all $\prob{P}_{\alpha}\in \prob{P}_{\{\}}$ $\RV{A}\CI_{\prob{P}_\alpha} \RV{C}|\RV{B}$.
\end{definition}

It is very straightforward to show that universal conditional independence satisfies the semi-graphoid axioms.

\begin{lemma}
$[\forall x: (f(x)\implies g(x))]\implies[(\forall x: f(x))\implies(\forall x: g(x))]$
\end{lemma}

\begin{proof}
\begin{align}
	\forall x: f(x) \implies g(x)&&\text{premise}\label{eq:premise1}\\
	\forall x: f(x)&& \text{premise}\label{eq:premise2}\\
	f(a) && \text{universal instantiation on }\ref{eq:premise2}\text{ substitute }a/x\label{eq:ui1}\\
	f(a)\implies g(a) && \text{universal instantiation on }\ref{eq:premise1}\text{ substitute }a/x\label{eq:ui2}\\
	g(a)&&\text{ modus ponens }\ref{eq:ui1}\text{ and }\ref{eq:ui2}\label{eq:mp1}\\
	\forall x: g(x)&&\text{universal generalisation on }\ref{eq:mp1}\label{eq:ug1}\\
	(\forall x: f(x))\implies(\forall x: g(x))&& \text{conditional proof }\ref{eq:premise2}-\ref{eq:ug1}\label{eq:cp1}\\
	[\forall x: (f(x)\implies g(x))]\implies[(\forall x: f(x))\implies(\forall x: g(x))]&& \text{conditional proof }\ref{eq:premise1}\text{--}\ref{eq:cp1}
\end{align}

With thanks to \citet{1377555} for the proof.
\end{proof}

\begin{lemma}
Given a standard measurable space $(\Omega,\sigalg{F})$ and $\prob{P}_{\{\}}$ on $\Omega$, universal conditional independence with respect to $\prob{P}_{\{\}}$ satisfies the semi-graphoid axioms.
\end{lemma}

\begin{proof}
For a particular probability $\prob{P}_\alpha$, each of the semi-graphoid axioms consists of a statement of the form $f(\prob{P}_\alpha)\implies g(\prob{P}_\alpha)$ (in the case of the first axiom, it corresponds to two such statements).

As the axioms hold for conditional independence with respect to any probability model, we have in particular $\forall \prob{P}_\alpha\in \prob{P}_{\{\}}: f(\prob{P}_{\alpha}) \implies g(\prob{P}_\alpha)$. 



This implies $(\forall \prob{P}_\alpha: f(\prob{P}_\alpha)) \implies (\forall \prob{P}_\alpha: g(\prob{P}_\alpha))$
\end{proof}


The semi-graphoid axioms hold for all probability measures $\prob{P}$, so in particualr they hold for all $\prob{P}_\alpha\in \prob{P}_{\{\}}$. Thus conditional independence with respect to a probability set also satisfies the semi-graphoid axioms.

% \begin{definition}[Conditional independence with respect to a probability comb]
% Conditional independence $\RV{A}\CI_{\prob{P}_\square}\RV{B}|\RV{C}$ holds for an arbitrary probability comb $\model{P}_\square:A\to \mathscr{P}(\Delta(\Omega))$ if $\RV{A}\CI_{\prob{P}_\alpha}\RV{B}|\RV{C}$ holds for all probability models $\prob{P}_\alpha$, $\alpha\in A$.
% \end{definition}

\subsection{Curried Markov kernels}\label{sec:curry}

Given a function $f:X\times Y\to Z$, we can obtain a curried version $\lambda f:Y\to Z^X$. In particular, if $Y=\{*\}$ then $\lambda f:\{*\}\to Y^X$. At least for countable $X$, we can apply this construction to Markov kernels: given a kernel $\kernel{K}:X\kto Y$, define $\lambda \kernel{K}: \{*\}\kto Y^X$ by 
\begin{align}
	\lambda \kernel{K} ((y_i)_{i\in X}) &= \prod_{i\in X} \kernel{K}(y_i|i)
\end{align}

We can then define an evaluation map $\text{ev}:Y^X\times X\to Y$ by $\text{ev}((y_i)_{i\in X},x)=y_x$. Then

\begin{align}
	\kernel{K} = (\lambda \kernel{K}\otimes \text{id}_X) \kernel{F}_{\text{ev}} \label{eq:curry_identity}
\end{align}

Unlike the case of function currying, $\lambda \kernel{K}$ is not the unique Markov kernel for which \ref{eq:curry_identity} holds. In fact, we can substitute any $\kernel{L}$ such that, for any $i\in X$

\begin{align}
	\sum_{y_{\{i\}^C\in Y^{|X|-1}}} \kernel{L}((y_i)_{i\in X}) = \kernel{K}(y_i|i)
\end{align}

Evaulation of a curried Markov kernel $\lambda \kernel{K}$ resembles the definition of \emph{potential outcomes}; for outcomes $\RV{Y}:\Omega\to Y$ and treatments $X:\Omega\to X$, potential outcomes are described by a probability distribution $\prob{P}^{\RV{Y}^X}$ on $Y^X$ and we have the relation

\begin{align}
	\RV{Y} \overset{a.s.}{=} \text{ev}(\RV{Y}^X,\RV{X})
\end{align}

Then
\begin{align}
	(\prob{P}^{\RV{Y}^X}\otimes \text{id}_X)\kernel{F}_{\text{ev}}
\end{align}

is some Markov kernel $\kernel{K}:X\kto Y$, which is equal to $\prob{P}^{\RV{Y}|\RV{X}}$ if $\RV{Y}^X\CI |\RV{X}$. However, potential outcomes models typically do not explain what the kernel $\kernel{K}$ represents, and instead offer a definition of the variable $\RV{Y}^X$. For $x\in X$, the component $\RV{Y}^x$ of $\RV{Y}^X$ is usually said to express ``the outcomes that would have been observed, if $\RV{X}$ was $x$''.

Our original motivating question was ``when are potential outcomes well-defined?''. We're not actually going to try to answer this question, because our aim is not to tell people using potential outcomes how to do it. Furthermore, that question invites controversy we are not particularly interested in joining; \citet{dawid_causal_2000} and \citet{richardson2013single} have both argued that it is better to use equivalence classes of potential outcomes models induced by a criterion of distinguishability by experiment, while \citet{pearl_causality:_2009} advocates for models that can make finer distinctions than this.

However, given a probability gap model $\prob{P}_\square$, we do have a natural notion of the well-definedness of a conditional probability $\prob{P}^{\RV{Y}|\RV{X}}_\square$ -- it is well-defined when $\prob{P}_\alpha^{\RV{Y}|\RV{X}}$ is equal for all $\alpha$ (Definition \ref{def:cprob_pset}). Furthermore, the formal conditions that guarantee the existence of such a conditional probability very closely resemble the \emph{stable unit treatment value assumption} (SUTVA), which is said to be necessary for the existence of potential outcomes \citet{rubin_causal_2005}:

\begin{blockquote}
(SUTVA) comprises two subassumptions. First, it assumes that \emph{there is no interference between units (Cox 1958)}; that is, neither $Y_i(1)$ nor $Y_i(0)$ is affected by what action any other unit received. Second, it assumes that \emph{there are no hidden versions of treatments}; no matter how unit $i$ received treatment $1$, the outcome that would be observed would be $Y_i(1)$ and similarly for treatment $0$.
\end{blockquote}

The added emphasis is ours. In the next section, we offer formal criteria that correspond to these two statements.
%!TEX root = main.tex

\section{See-do models}\label{sec:seedo_models}

We will first introduce \emph{see-do models} as a type of model that functions as the basic kind of thing which we will use to examine questions in the decision theoretic, potential outcomes and graphical models appraoch.

See-do models can be understood as generalisations of statistical models. Statistical models are a ubiquitous type of model in statistics and machine learning that consist of a set of \emph{states} $S$, and for each state the model prescribes a single probability distribution on a given set of \emph{outcomes} $O$.

\begin{definition}[Statistical model]\label{def:statistical model}
A statistical model is a set of states $S$, a set of outcomes $O$ and a Markov kernel $\kernel{T}:S\to \Delta(O)$.
\end{definition}

For example, a potentially biased coin can be modelled with a statistical model. Suppose the coin has some rate of heads $\theta\in [0,1]$, and we furthermore suppose that for each $\theta$ the result of flipping the coin can be modeled (in some sense) by the probability distribution $\text{Bernoulli}(\theta)$. The statistical model here is the set of states $S=[0,1]$ (corresponding to \emph{rates of heads}), the observation space $O=\{0,1\}^n$ with the discrete sigma-algebra (where $n$ is the number of flips observed) and the stochastic map $\kernel{B}:[0,1]\to \Delta(\mathscr{P}(0,1))$ which is given by $\kernel{B}:\theta\to \text{Bernoulli}(\theta)$.

This example actually goes beyond our formal definitions here in that $\theta$ is real-valued between $0$ and $1$. Extending probability theory to real-valued spaces is well understood, see for example \citet{cinlar_probability_2011}, but in that setting the existence of disintegrations on kernel spaces (section \ref{ssec:disintegration}) is a problem to which we presently only have a partial solution. Discrete sets allow us to discuss see-do models without going into this difficulty. The price we pay is that to properly model the above problem we require $\theta$ to take on discrete values, for example restricting it to the rationals.

A see-do model adds the following structure to a statistical model:

\begin{itemize}
    \item The state is a pair consisting of a \emph{hypothesis} $h\in H$ and a \emph{decision} $d\in D$; $S=H\times D$
    \item The outcome is a pair consisting of an \emph{observation} $x\in X$ and a consequence $y\in Y$
    \item The observation is conditionally independent of the decision given the hypothesis
\end{itemize}

We can use see-do models to model situations where we have some hypotheses and we have the opportunity to make an observation that takes values in $X$. Depending on what we see, we can select a decision from a set of possibilities $D$, and the ultimate consequence depends probabilistically on the decision we selected as well as whichever hypothesis turns out to best describe the world.

\begin{definition}
A \emph{see-do model} $(\kernel{T},\RV{H},\RV{D},\RV{X},\RV{Y})$ is a Markov kernel space $(\kernel{T},H\times D, O)$ along with four variables: the \emph{hypothesis} $\RV{H}:H\times D\times O\to H$, the \emph{decision} $\RV{D}:H\times D\times O\to D$, the \emph{observation} $\RV{X}: H\times D\times O\to X$ and the \emph{consequence} $\RV{Y}:H\times D\times O\to Y$, all given by the projections onto the respective spaces. In addition, a see-do model must observe the conditional independence:
\begin{align}
\RV{X}\CI_\kernel{T} \RV{D}|\RV{H} \label{eq:see_do_independence_requirement}
\end{align}
\end{definition}

See-do models feature variables $\RV{D}$ and $\RV{H}$ that act like Dawid's ``non-stochastic regime indicators'' described by \citet{dawid_influence_2002,dawid_decision-theoretic_2012,dawid_decision-theoretic_2020}. In particular, see-do models induce a collection of probability measures indexed by the elements of $H\times D$, just as regime indicators induce collections of indexed probability measures. Dawid's regime indicators seem to typically do a similar job to a decision variable $\RV{D}$ rather than a decision-hypothesis pair.

The hypothesis set is similar to the parameter set described by \citet{lattimore_replacing_2019} that relates pre- and post-interventional distributions. Lattimore and Rohde consider models with a prior distribution over this parameter set. A similar type of model can be created by taking the prodcut of a prior over the hypothesis set and a see-do model. 

Finally, see-do models are somewhat similar to the models proposed by \citet{savage_foundations_1954} for decision problems if we identify \emph{states} with \emph{hypotheses} and \emph{acts} with \emph{decisions}. Savage's models consider deterministic rather than stochastic functions from acts to outcomes, and did not explicitly distinguish observations from consequences.

\subsection{See-do models are motivated by data-driven decision problems} 

\todo[inline]{I might not have room for this!}

Suppose we have a decision problem that provides us with an observation $x\in X$, and in response to this we can select any decision or stochastic mixture of decisions from a set $D$; that is we can choose a ``strategy'' as any Markov kernel $\kernel{S}:X\to \Delta(D)$. We have a utility function $u:Y\to \mathbb{R}$ that models our preferences over the consequences of our choice. Furthermore, suppose that we maintain a set of hypotheses $H$, and under each hypothesis $h\in H$ we model the result of choosing some strategy $\kernel{S}$ is a joint probability over observations, decisions and consequences $\prob{P}_{h,\kernel{S}}\in \Delta(X\times D\times Y)$.

Holding the hypothesis $h$ fixed we model the observations as having the same distribution under any strategy: $\prob{P}_{h,\kernel{S}}^{\RV{X}}=\prob{P}_{h,\kernel{S}''}^{\RV{X}}$ for all $h,\kernel{S},\kernel{S}'$.

The conditional probability of decisions given observations is equal to the chosen strategy: $\prob{P}_{h,\kernel{S}}^{\RV{D}|\RV{X}}=\kernel{S}$.

Finally, for each $h\in H$ there exists some strategy $\kernel{S}_{(h)}$ such that $\prob{P}_{h,\kernel{S}_{(h)})}^{\RV{D}}$ is strictly positive.

Then there exists a see-do model $(\kernel{T},\RV{H},\RV{D},\RV{X},\RV{Y})$ such that $\prob{P}_{h,\kernel{S}} = \kernel{T}^{\RV{X}}_{h} \kernel{S} \kernel{T}^{\RV{Y}|\RV{DH}}_{h}$.

%!TEX root = main.tex

\section{Repeatable experiments}

While there are types of measurement processes we could consider, statistical inference usually proceeds from repeatable measurement processes. A common precise notion of repeatability is the assumption of \emph{exchangeability}. The term ``exchangeability'', like the term random variable, is used to refer to assumptions about \emph{measurement processes} as well as properties of \emph{probability models}. If I say a measurement process $\proc{S}$ taking values in $S^n$ is exchangeable, I might mean:
\begin{itemize}
    \item I believe that there is some probabilistic model $(\prob{P},\Omega,\sigalg{F})$ and random variable $\RV{S}$ appropriate for modelling $\proc{S}$ and
    \begin{enumerate}
        \item The same model is appropriate for any measurement process that first peforms $\proc{S}$ and subsequently shuffles the results according to any permutation $\text{swap}_a:S^n\to S^n$ or
        \item The same model is appropriate for any measurement process related to $\proc{S}$ by interchanging experimental units or subjects in the real world
    \end{enumerate} 
\end{itemize}

On the other hand, if I say a probability model $(\prob{P},S^{|A|},\sigalg{S}^{|A|})$ is exchangeable, I mean

\begin{itemize}
    \item For any finite permutation $\text{swap}_A:S^{|A|}\kto S^{|A|}$, $\prob{P}^{\RV{S}}\text{swap}_{a} = \prob{P}^{\RV{S}}$
\end{itemize}

If I believe a measurement process is exchangeable in the first sense, then this implies that the same probability model is appropriate to model $\proc{S}$ as to model $\text{swap}_a\circ \proc{S}$, which implies that $\prob{P}^{\RV{S}}$ should be an exchangeable probability model. Measurement process exchangeability in the second sense requires us to make explicit the mathematical implications of ``interchanging experimental units'', as our semantics of random variables do not say anything about swapping things in the real world. However, the second kind of measurement process exchangeability is more interesting in the context of causal modelling. When we are \emph{acting} on the world, our future actions will often depend on what we have observed in the past, which will often rule out exchangeability in the first sense. Furthermore, our actions have consequences and so permuting the \emph{labels} associated with actions while not actually changing the actions we take is not a particularly interesting operation. Rather, we are interested in how a model might or might not change if we swap the \emph{actual actions} we take. Swapping experimental units while holding actions constant is one way to achieve this, as it changes the identity of which unit receives which action. See \citet{dawid_decision-theoretic_2020} and \citet{greenland_identifiability_1986} for further discussions of exchangeability in the context of causal modelling, and note that both authors consider exchanging to be an operation that alters which person receives which treatment.

De Finetti's well-known representation theorem shows that exchangeable probability models feature a ``hypothesis'' $\RV{H}$ such that the sequence $\RV{S}$ is independent and identically distributed conditional on $\RV{H}$. That is: a measurement process that is exchangeable in the first sense should be modelled by a conditionally indpendent and identically distributed sequence of random variables. The question we want to address here is whether measurement processes that are exchangeable in the second sense imply causal models with particular structure. The answer is yes, although as we discuss the key assumption is \emph{causal contractibility} rather than exchangeability.

In this section, we will at first consider \emph{blind} decision functions -- that is, decision functions that pay no attention to the data already available. We are interested in repetitive symmetries, and these symmetries are typically broken if our decisions are based on past observations. Furthermore, we can define the entire see-do model by its behaviour on blind decision functions. We also assume that the hypotheses are trivial $\RV{H}=*$; once the decision is chosen, we are left with a single probability model. This also substantially simplifies the arguments to be made.

We will consider two different notions of ``repeatable experiments''. Both require a sequence of ``decisions'' to be made and a sequence of consequences, and we assume that each decision corresponds to a single consequence. One could think about these paired sqeuences as a series of experiments each with different setting choices available; the decisions are the setting choices and the consequences are the results of each experiment. The first notion we consider will be \emph{commutativity of exchange} -- we consider the same model appropriate if we alter our experiment by swapping the experimental settings, or if we make analogous swaps to the experimental results. This assumption could be considered a version of the assumption that experimental units can be interchanged. Consider an experiment involving handing out money or not to person A or person B. Commutativity of exchange says that we should use the same probability model to represent the following two predictions: 
\begin{itemize}
    \item Applying choice 1 to A and choice 2 to B and predicting the vector (consequences for A, consequences for B)
    \item Applying choice 2 to A and choice 1 to B and predicting the vector (consequences for B, consequences for A)
\end{itemize}

Under the assumption of commutativity of exchange, consequences of decisions for one ``experimental unit'' may still depend on decisions made for other ``experimental units''. Consider again the experiment above, except instead of two people we are considering giving money to everyone in a particular country. Supposing we don't otherwise know much about the people we are giving money to, it might be reasonable to posit that a model of the consequences should observe commutativity of exchange. However, giving money to A as well as everyone else will have different consequences for A than giving money to A and no-one else; in the former case, we will create more inflation than in the latter.

The second notion of ``repeatable experiments'' is \emph{causal contractibility}, a strictly stronger assumption than commutativity of exchange. Causal contractibility is the assumption that, given two different sequences of decisions, the marginal model of consequences corresponding to matching subsequences of decisions will be equal. A causally contractibly model says that, if I make the same choice for any subcollection of experiments, I expect the same results from those experiments regardless of whatever choices I make elsewhere.


% Another way to see where we are going is to consider graphical statements of our and De Finetti's result.

% Take $S=\{0,1\}$ and identify the space $\Delta(S)$ of probability measures on $S$ with the interval $[0,1]$. De Finetti showed that any infinite exchangeable probability measure $\prob{P}_\alpha$ on $\{0,1\}^\mathbb{N}$ can be represented by a prior $\prob{P}_\alpha^{\RV{H}}\in [0,1]$ for some $\RV{H}:\Omega\to H$ and a conditional probability $\prob{P}^{\RV{S}_0|\RV{H}}:[0,1]\kto \{0,1\}$ such that

% \begin{align}
%     \prob{P}_\alpha &= \tikzfig{de_finetti_rep0}\label{eq:definettirep}
% \end{align}

% Here $\prob{P}^{\RV{S}_0|\RV{H}}$ can be defined concretely by $\prob{P}^{\RV{S}_0|\RV{H}}(1|h)=h$. Equivalently, the probability gap model on $S^\mathbb{N}$ defined by the assumption of exchangeability is equivalent to the probability gap model defined by the conditional probability

% \begin{align}
%     \prob{P}^{\RV{S}|\RV{H}} = \tikzfig{de_finetti_conditional}
% \end{align}

% That is, there is some hypothesis $\RV{H}$ and conditional on $\RV{H}$ the measurements are independent and identically distributed. The proof of this is constructive -- $\RV{H}$ is a function of $\RV{S}$.



% \begin{align}
%     \prob{P}^{\RV{Y}|\RV{HD}} = \tikzfig{do_model_representation}
% \end{align}

% We will further argue that the class of see-do models considered in CBN and potential outcomes literature is equivalent to the family of causally contractible and exchangeable do-models where the decision rule for the first $n$ places is fixed to an unknown value, and may be freely chosen thereafter.

\subsection{Assumptions of repeatability applicable to models of decisions and consequences}

In this section we formalise the notion of commutativity of exchange and causal contractibility. We will then go on to prove two representation theorems for causally contractible models -- firstly, that they can be represented with a tabular probability model and a lookup function, a construction that is very similar to the kinds of causal models employed by the potential outcomes framework (although they do not necessarily share the semantics of potential outcomes). Secondly, we will show that contractible causal models can also be represented by jointly independent repetitions of a ``unit-level consequence map'', indexed by a hypothesis $\RV{H}$.

To begin with, we will define do models, which are see-do models with nothing to see.

\begin{definition}[Do model]\label{def:domodel}
A \emph{do model} is an infinite sequential probability gap model $(\prob{P}_{\square}^{\RV{Y}\|\RV{D}},R)$ where $R$ is a subset of the \emph{blind} decision functions: for all $n$, $\prob{P}_\alpha^{\RV{D}_{[n]}\|\RV{Y}_{[n-1]}}=\text{erase}_{Y^{n-1}}\otimes \prob{P}_\alpha^{\RV{D}_{[n]}}$ for some $\prob{P}_\alpha^{\RV{D}_{[n]}}\in\Delta(D^n)$. That is, we don't permit any decision $\RV{D}_i$ to depend on prior observations $\RV{Y}_i$s.
\end{definition}

Do models are useful because the $n-comb$ $\prob{P}_{\square}^{\RV{Y}_{[n]}\|\RV{D}_{[n]}}$ is also the conditional $\prob{P}_{\square}^{\RV{Y}|\RV{D}}$ (which otherwise may not exist).

\begin{theorem}[Existence of conditional in do models]
Given a do model $(\prob{P}_{\square}^{\RV{Y}\|\RV{D}},R)$, for all $\alpha\in R$, $n\in\mathbb{N}$
\begin{align}
    \prob{P}_\alpha^{\RV{Y}_{[n]}\RV{D}_i} = \prob{P}_\alpha^{\RV{D}_{[n]}}\odot \prob{P}_\square^{\RV{Y}_{[n]}\|\RV{D}_{[n]}}
\end{align}
That is, $\prob{P}_\square^{\RV{Y}_{[n]}\|\RV{D}_{[n]}}\cong \prob{P}_\square^{\RV{Y}_{[n]}|\RV{D}_{[n]}}$
\end{theorem}

\begin{proof}
For any $n>1\in \mathbb{N}$, $\alpha\in R$

\begin{align}
    \prob{P}_\alpha^{\RV{Y}_{[n]}\RV{D}_{[n]}} &= \tikzfig{do_model_1}\\
    &= \tikzfig{do_model_2}\\
    &= \tikzfig{do_model_3}\\
    &= \tikzfig{do_model_4}\\
    \implies \prob{P}_\alpha^{\RV{Y}_{[n]}|\RV{D}_{[n]}} &= \tikzfig{do_model_5}\\
    &= \prob{P}_\alpha^{\RV{Y}_{[n-1]}|\RV{D}_{[n-1]}}\combprod \prob{P}_\square^{\RV{Y}_n|\RV{Y}_{[n-1]}\RV{D}_n}
\end{align}

Applying this recursively with $\prob{P}_\alpha^{\RV{Y}_{[1]}|\RV{D}_{[1]}}=\prob{P}_\square^{\RV{Y}_{[1]}|\RV{D}_{[1]}}$ yields

\begin{align}
    \prob{P}_\alpha^{\RV{Y}_{[n]}|\RV{D}_{[n]}} = \prob{P}_\square^{\RV{Y}_{[n]}\|\RV{D}_{[n]}}
\end{align}
as desired.
\end{proof}

A do model ``commutes with exchange'' if exchanging decisions or exchanging consequences yields the same model for any finite permuation. The term \emph{commute} comes from the notion that we can apply the exchange before the conditional $\prob{P}_{\square}^{\RV{Y}|\RV{D}}$ or after it and get the same result.

\begin{definition}[Commutativity of exchange]\label{def:caus_exch}
Suppose we have a fundamental probability set $\Omega$ and a do model $(\prob{P}_{\square}^{\RV{Y}|\RV{D}},R)$ such that $\RV{D}:=(\RV{D}_i)_{i\in \mathbb{N}}$ and $\RV{Y}:=(\RV{Y}_i)_{i\in\mathbb{N}}$. For a finite permutation $\rho:\mathbb{N}\to\mathbb{N}$, define $\text{swap}_{\rho(D)}:D\kto D$ by $(d_i)_{i\in\mathbb{N}}\mapsto \delta_{(d_{\rho(i)})_{i\in\mathbb{N}}}$ and $\text{swap}_{\rho(D\times Y)}:D\times Y\kto D\times Y$ by $(x_i)_{i\in\mathbb{N}}\mapsto \delta_{(x_{\rho(i)})_{i\in\mathbb{N}}}$. If, for any two decision rules $\alpha,\beta \in R$,
\begin{align}
    \prob{P}_\alpha^{\RV{D}}\text{swap}_{\rho(D)} &= \prob{P}_{\beta}^{\RV{D}}\\
    \implies  \prob{P}_\alpha\text{swap}_{\rho(D\times Y)}&=\prob{P}_\beta
\end{align}
Then $\prob{P}$ \emph{commutes with exchanges}.
\end{definition}

A do model is causally contractible if it gives identical results for any identical subsequences of two decisions when we limit our attention to the corresponding subsequences of consequences. For example, if we have $\RV{D}=(\RV{D}_1,\RV{D}_2,\RV{D}_3)$ and $\RV{Y}=(\RV{Y}_1,\RV{Y}_2,\RV{Y}_3)$ and $\prob{P}_\alpha^{\RV{D}_1\RV{D}_3}=\prob{P}_\beta^{\RV{D}_3\RV{D}_2}$ then $\prob{P}_{\alpha}^{\RV{Y}_1\RV{Y}_3}=\prob{P}_\beta^{\RV{Y}_3\RV{Y}_2}$.

\begin{definition}[Causal contractibility]\label{def:caus_cont}
Suppose we have a fundamental probability set $\Omega$ and a do model $(\prob{P},\RV{D},\RV{Y},R)$ such that $\RV{D}:=(\RV{D}_i)_{i\in \mathbb{N}}$ and $\RV{Y}:=(\RV{Y}_i)_{i\in\mathbb{N}}$. For any $A=(s_i)_{i\in A}$, $T=(t_i)_{i\in A}$, $A\subset\mathbb{N}$ and $i<j\implies p_i<p_j \And q_i<q_j$, let $\RV{D}_S:=(\RV{D}_i)_{i\in S}$ and $\RV{D}_T:=(\RV{D}_i)_{i\in T}$. If for any $\alpha,\beta\in R$
\begin{align}
    \prob{P}_\alpha^{\RV{D}_{S}}=\prob{P}_\beta^{\RV{D}_{T}}\implies \prob{P}_\alpha^{(\RV{D_i,Y_i})_{i\in S}}=\prob{P}_\beta^{(\RV{D_i,Y_i})_{i\in T}}
\end{align}
then $\prob{P}$ is \emph{causally contractible}.
\end{definition}

Commutativity of exchange does not imply causal contractibility. For example, suppose $|D|=2$, $D=Y=\{0,1\}$ and we have a do-model $\prob{P}$ such that for all $\alpha\in R$

\begin{align}
    \prob{P}_\alpha^{\RV{Y}_1\RV{Y}_2|\RV{D}_1\RV{D}_2}(y_1,y_2|d_1,d_2) &= \llbracket (y_1,y_2)= (d_1+d_2,d_1+d_2) \rrbracket
\end{align}

Then $\prob{P}_{00}^{\RV{Y}_1}(y_1) = \llbracket y_1=0\rrbracket$ while $\prob{P}_{01}^{\RV{Y}_1} = \llbracket y_1=1 \rrbracket$, so $\prob{P}$ is not piecewise replicable. However, taking $(d_i,d_j)$ to be the decision function that deterministically chooses $(d_i,d_j)$,

\begin{align}
    \prob{P}_{d_2,d_1}^{\RV{Y}_1\RV{Y}_2|\RV{D}_1\RV{D}_2}(y_1,y_2) &= \llbracket (y_1,y_2)= (d_2+d_1,d_2+d_1) \rrbracket\\
    &= \llbracket (y_2,y_1)= (d_1+d_2,d_1+d_2) \rrbracket\\
    &= \prob{P}_{d_1,d_2}^{\RV{Y}_1\RV{Y}_2|\RV{D}_1\RV{D}_2}(y_2,y_1)
\end{align}

so $\prob{P}$ commutes with exchange.

There is a representation theorem for models that commute with exchange which implies that for $\prob{P}$ that commutes with exchange, $\RV{Y}_i\CI_{\prob{P}}(\RV{D}_j,\RV{Y}_j)_{j\in \mathbb{N}}\setminus \{i\}|\RV{H}\RV{D}_i$, where $\RV{H}$ is a symmetric function of $(\RV{Y}_i,\RV{D}_i)_{i\in\mathbb{N}}$.

% \begin{proposition}[Representation of do-models that commute with exchange]
% Suppose we have a fundamental probability set $\Omega$ and a do model $(\prob{P},\RV{D},\RV{Y},R)$ such that $\RV{D}:=(\RV{D}_i)_{i\in \mathbb{N}}$ and $\RV{Y}:=(\RV{Y}_i)_{i\in\mathbb{N}}$ where $\prob{P}$ commutes with exchange and there is some $\alpha^*\in R$ such that $\prob{P}^{\alpha^*}\gg\prob{P}_\beta$ for all $\beta in R$. Then there exists a symmetric function $\RV{H}:(Y\times D)^\mathbb{N}\to H$ such that  $\prob{P}^{\RV{Y}|\RV{DH}}$ exists and $\RV{Y}_i\CI_{\prob{P}}(\RV{D}_j,\RV{Y}_j)_{j\in \mathbb{N}}\setminus \{i\}|\RV{H}\RV{D}_i$, or equivalently 
% \begin{align}
%     \prob{P}^{\RV{Y}} &= \tikzfig{do_model_representation}
% \end{align}
% \end{proposition}

% % \begin{lemma}[Contraction and independence]
% % Let $\RV{J}$, $\RV{K}$ and $\RV{L}$ be variables on $\Omega$ and $\prob{Q}\in \Delta(\Omega)$ a base measure such that $\prob{Q}^{\RV{JK}}=\prob{Q}^{\RV{JL}}$ and $\sigma{K}\subset \sigma{L}$. Then $\RV{J}\CI\RV{L}|\RV{K}$. 
% % \end{lemma}

% % \begin{proof}
% % From Lemma 1.3 in \citet{kallenberg_basic_2005}.
% % \end{proof}

% \begin{proof}
% If $\prob{P}$ commutes with exchange, then for any $\alpha\in R$ such that $\prob{P}_\alpha^{\RV{D}}$ is exchangeable then $\prob{P}_\alpha$ is also exchangeable. Then there exists $\RV{H}$ a symmetric function of $(\RV{Y}_i,\RV{D}_i)_{i\in\mathbb{N}}$ such that $\RV{Y}_i\CI_{\prob{P}}(\RV{D}_j,\RV{Y}_j)_{j\in \mathbb{N}}\setminus \{i\}|\RV{H}\RV{D}_i$. This is De Finetti's representation theorem, and many proofs exists, see for example \citep{kallenberg_basic_2005}.

% In particular, let 

% \begin{align}
%     \RV{H}:=A\times B\mapsto \lim_{n\to\infty} \frac{1}{n}\sum_{i\in n} \mathds{1}_{A\times B}((\RV{Y}_i, \RV{D}_i))
% \end{align}

% Then for all $\alpha\in R$,
% \begin{align}
%     \prob{P}_\alpha^{(\RV{Y}_i,\RV{D}_i)_{i\in\mathbb{N}}|\RV{H}}(A\times B|h) \overset{a.s.}{=} h(A\times B)\label{eq:given_h}
% \end{align}

% The proof that the limit exists and the above equality holds can again be found int \citep{kallenberg_basic_2005}.
% \end{proof}

\subsection{Representations of contractible probability models}

We prove two representation theorems for causally contractible do models. Theorem \ref{th:table_rep} shows that a do model is contractible if and only if it can be represented with a contractible probability distribution over a ``table of variables''\todo{matrix of variables?} and a lookup function. This is interesting in its own right, as tabular probability distributions and lookup functions are core elements of the potential outcomes approach. However, as we will point out, this lookup table may or may not support an interpretation as a table of potential outcomes. Furthermore, we make use of this theorem in proving Theorem \ref{th:iid_rep}, which shows a do model is contractible if and only if it can be represented by independent copies of a unit level consequence map jointly parametrised by a hypothesis. We will argue in the next section that jointly parametrised consequence maps are fundamental to all approaches to causal inference.

\begin{definition}[Contractible probability distribution]
Given a fundamental probability set $\Omega$, variable $\RV{X}:=(\RV{X}_i)_{i\in \mathbb{N}}$ and a probability distribution $\prob{P}^{\RV{X}}\in\Delta(X^{\mathbb{N}})$, any $S=(s_i)_{i\in A}$, $T=(t_i)_{i\in A}$ with $A\subset\mathbb{N}$ and $i<j\implies s_i<s_j \land t_i<t_j$, let $\RV{X}_S:=(\RV{X}_i)_{i\in S}$ and $\RV{X}_T:=(\RV{X}_i)_{i\in T}$. If
\begin{align}
    \prob{P}^{\RV{X}_S} &= \prob{P}^{\RV{X}_T}
\end{align}
 $\prob{P}$ is contractible.
\end{definition}

If we have a do model $\prob{P}$ that is causally contractible, we can represent it as an exchangeable probability distribution and a lookup function.

\todo[inline]{The following can be deduced from the theorems after it, but I thought it might be helpful to have the explanation.}

That is, we can define a variable $\RV{Y}^D:\Omega\to Y^{D\times\mathbb{N}}$ which can be represented as a matrix of variables $\RV{Y}_{ij}$

\begin{align}
    \RV{Y}^D &= \tikzfig{Y_table_representation}
\end{align}

and, given any deterministic decision function $\delta_d$, $d=(d_i)_{i\in\mathbb{N}}\in D^{\mathbb{N}}$, we can find $\prob{P}^{\RV{Y}|\RV{D}}$ by ``looking up'' $d$ in the table. For example, if $d=(1,2,3,2,...)$, Equation \ref{eq:table_lookup_example} illustrates the idea of ``looking up'' the relevant elements of $\RV{Y}^D$ and Equation \ref{eq:table_lookup_cons} illustrates the resulting value of $\prob{P}^{\RV{Y}|\RV{D}}$.

\begin{align}
    \tikzfig{Y_table_lookup}\label{eq:table_lookup_example}\\
    \prob{P}^{\RV{Y}|\RV{D}}(y|(1,2,3,2,...)) = \prob{P}^{\RV{Y}_11\RV{Y}_22\RV{Y}_{33}\RV{Y}_{24}...}(y)\label{eq:table_lookup_cons}
\end{align}

The contractibility of $\prob{P}^{\RV{Y}^D}$ means that any two subcollections of columns of the same size are equal in distribution, and the exchangeability of $\prob{P}^{\RV{Y}^D}$ means that the random variable obtained by permuting its columns is also equal in distribution to $\RV{Y}^D$.

This representation is very similar to the potential outcomes representation of causal models, with two points of friction. Firstly, we used the assumption of contractibility to derive the contractible table representation, and so we make no claims about what kind of do-model is represented by a non-contractible table lookup. Secondly, we do not yet include any notion of observations, which is a key element of potential outcomes models.

\begin{theorem}[Table representation of causally contractible do models]\label{th:table_rep}
Suppose we have a fundamental probability set $\Omega$ and a do model $(\prob{P},\RV{D},\RV{Y},R)$ such that $\RV{D}:=(\RV{D}_i)_{i\in \mathbb{N}}$ and $\RV{Y}:=(\RV{Y}_i)_{i\in\mathbb{N}}$. $\prob{P}$ is causally contractible if and only if 
\begin{align}
    \prob{P}^{\RV{Y}|\RV{D}} &= \tikzfig{lookup_representation}\\
    &\iff\\
    \prob{P}^{\RV{Y}|\RV{D}}(y|d) &= \prob{P}^{(\RV{Y}^D_{d_i i})_{\mathbb{N}}}(y)
\end{align}
Where $\prob{P}^{\RV{Y}^D}$ is a contractible probability measure on $Y^{D\times\mathbb{N}}$ with respect to the sequence $\RV{Y}^D:=(\RV{Y}_{ij}^D)_{i\in D,j\in \mathbb{N}}$ and $\prob{L}^{\RV{D},\RV{Y}^D}$ is the Markov kernel associated with the lookup function
\begin{align}
    l:D^\mathbb{N}\times Y^{D\times \mathbb{N}}&\to Y\\
    ((d_i)_\mathbb{N},(y_{ij})_{i\in D,j\in \mathbb{N}})&\mapsto y_{d_i i}
\end{align}
\end{theorem}

\begin{proof}
Only if:
Choose $e:=(e_i)_{i\in\mathbb{N}}$ such that $e_{|D|i+j}$ is the $i$th element of $D$ for all $i,j\in \mathbb{N}$. Abusing notation, write $e$ also for the decision function that chooses $e$ deterministically.

Define
\begin{align}
    \prob{P}^{\RV{Y}^D}((y_{ij})_{D\times \mathbb{N}}):=\prob{P}_e^{\RV{Y}}((y_{|D|i+j})_{i\in D, j\in \mathbb{N}})
\end{align}

Now consider any $d:=(d_i)_{i\in \mathbb{N}}\in D^{\mathbb{N}}$. By definition of $e$, $e_{|D|d_i + i}=d_i$ for any $i,j\in \mathbb{N}$.

\begin{align}
    \prob{Q}:D\kto Y\\
    \prob{Q}:= \tikzfig{lookup_representation}
\end{align}

and consider some ordered sequence $A\subset \mathbb{N}$ and $B:= ((|D|d_i+i))_{i\in A}$. Note that $e_B:=(e_{|D|d_i +i})_{i\in B}=d_A=(d_i)_{i\in A}$. Then 

\begin{align}
    \sum_{y\in \RV{Y}^{-1}(y_A)} \prob{Q}(y|d) &= \sum_{y\in \RV{Y}^{-1}(y_A)} \prob{P}^{(\RV{Y}^{D}_{d_ii})_{A}}(y) \\
    &= \sum_{y\in \RV{Y}^{-1}(y_A)} \prob{P}_e^{(\RV{Y}_{|D|d_i+i})_{A}}(y)\\
    &= \prob{P}_e^{\RV{Y}_{B}}(y_A)\\
    &= \prob{P}_{d}^{\RV{Y}_A}(y_A)&\text{by causal contractibility}
\end{align}

Because this holds for all $A\subset\mathbb{N}$, by the Kolmogorov extension theorem

\begin{align}
    \prob{Q}(y|d) &= \prob{P}_d^{\RV{Y}}(y)
\end{align}

Because $d$ is the decision function that deterministically chooses $d$, for all $d\in D$

\begin{align}
    \prob{Q}(y|d) &= \prob{P}_d^{\RV{Y}|\RV{D}}(y|d)
\end{align}

And because $\prob{P}_d^{\RV{Y}|\RV{D}}(y|d)$ is unique for all $d\in D^{\mathbb{N}}$ and $\prob{P}^{\RV{Y}|\RV{D}}$ exists by assumption

\begin{align}
    \prob{P}^{\RV{Y}|\RV{D}}=\prob{Q}
\end{align}

Next we will show $\prob{P}^{\RV{Y}^D}$ is contractible. Consider any subsequences $\RV{Y}^D_S$ and $\RV{Y}^D_T$ of $\RV{Y}^D$ with $|S|=|T|$. Let $\rho(S)$ be the ``expansion'' of the indices $S$, i.e. $\rho(S)=(|D|i+j)_{i\in S,j\in D}$. Then by construction of $e$, $e_{\rho(S)}=e_{\rho(T)}$ and therefore

\begin{align}
    \prob{P}^{\RV{Y}^D_S}&= \prob{P}_e^{\RV{Y}_{\rho(S)}})\\
    &= \prob{P}_e^{\RV{Y}_{\rho(T)}})&\text{by contractibility of }\prob{P}\text{ and the equality } e_{\rho(S)}=e_{\rho(T)}\\
    &= \prob{P}^{\RV{Y}^D_T}
\end{align}


If:
Suppose 
\begin{align}
    \prob{P}^{\RV{Y}|\RV{D}} &= \tikzfig{lookup_representation}
\end{align}

and consider any two deterministic decision functions $d,d'\in D^{\mathbb{N}}$ such that some subsequences are equal $d_S=d'_T$.

Let $\RV{Y}^{d_S}=(\RV{Y}_{d_i i})_{i\in S}$.

By definition,

\begin{align}
    \prob{P}^{\RV{Y}_S|\RV{D}}(y_S|d) &= \sum_{y^D_S\in Y^{|D|\times |S|}}\prob{P}^{\RV{Y}^D_S}(y^D_S)\prob{L}^{\RV{D}_S,\RV{Y}^S}(y_S|d,y^D_S)\\
    &= \sum_{y^D_S\in Y^{|D|\times |T|}}\prob{P}^{\RV{Y}^D_T}(y^D_S)\prob{L}^{\RV{D}_S,\RV{Y}^S}(y_S|d,y^D_S)&\text{ by contractibility of }\prob{P}^{\RV{Y}^D_T}\\
    &= \prob{P}^{\RV{Y}_T|\RV{D}}(y_S|d)
\end{align}
\end{proof}

Note that in some versions of potential outcomes, for example \citet{rubin_causal_2005}, potential outcomes are defined as table-and-lookup models, except without the assumption that the probability distribution over the table is contractible. Our argument for a potential outcomes representation does not go through in this case, because it hinges on the fact that we can ``wrap'' the outcomes under a particular blind decision into a table, and then use contractibility to choose one outcome from each column, however using contractibility also gives us exchangeability of the columns.

It is also worth noting that the lookup table does not need to have an interpretation as a collection of potential outcomes. For example, consider a series of bets on fair coinflips -- in this case, the consequence $\RV{Y}_i$ is uniform on $\{0,1\}$ for any decision $\RV{D}_i$. Tha $D=Y=\{0,1\}$ and $\prob{P}_\alpha^{\RV{Y}_n}(y)=\prod_{i\in [n]} 0.5$ for all $n$, $y\in Y^n$, $\alpha\in R$. Then the construction in Theorem \ref{th:table_rep} yields $\prob{P}^{Y^D_i}(y^D_i)=\prod_{j\in D} 0.5$ for all $y^D_i\in Y^D$. That is, $\RV{Y}^0_i$ and $\RV{Y}^1_i$ are independent and uniformly distributed. However, if we wanted $\RV{Y}^0_i$ to represent ``what would happen if I bet 0 on turn $i$'' and $\RV{Y}^1$ to represent ``what would happen if I bet 1 on turn $i$'', then we actually want $\RV{Y}^0_i = 1-\RV{Y}^1_i$. Thus the measurement table lookup is formally similar to the potential outcomes setup, but potential outcomes attributes additional semantics to the entries in the lookup table which can impose extra requirements on their distribution.

Theorem \ref{th:contractibility_commutativity} establishes a claim made earlier: that contractibility is strictly stronger than commutativity of exchange.

\begin{theorem}\label{th:contractibility_commutativity}
Causal contractibility implies commutativity of exchange.
\end{theorem}

\begin{proof}
Given a finite permutation $\rho:\mathbb{N}\to\mathbb{N}$ and any sequence $x:=(x_i)_{i\in \mathbb{N}}$ let $\rho(x)=(x_{\rho(i)})_{i\in\mathbb{N}}$ or equivalently $(x_{i})_{i\in\rho(\mathbb{N})}$. Then for any $d=(d_{i})_{i\in\mathbb{N}}$ and $y^D:=(y_{ij})_{i\in D,j\in \mathbb{N}}$:

\begin{align}
    l(\rho(d),y^D) &= (y_{d_{\rho(i)} i})_{i\in\mathbb{N}}\\
                 &= (y_{d_i \rho^{-1}(i)})_{i\in \rho(\mathbb{N})}\\
                 &= \rho(l(d,\rho^{-1}(y^D)))
\end{align}

Suppose we have a fundamental probability set $\Omega$ and a do model $(\prob{P},\RV{D},\RV{Y},R)$ with $\RV{D}:=(\RV{D}_i)_{i\in \mathbb{N}}$ and $\RV{Y}:=(\RV{Y}_i)_{i\in\mathbb{N}}$ and $\prob{P}$ causally contractible. Then
\begin{align}
    \prob{P}^{\RV{Y}|\RV{D}} &= \tikzfig{lookup_representation}
\end{align}
For contractible $\prob{P}^{\RV{Y}^D}$. Therefore $\prob{P}^{\RV{Y}^D}$ is also exchangeable \citet{kallenberg_basic_2005}. But then, given a decision function $d$ and a finite permutation $\rho:\mathbb{N}\to \mathbb{N}$
\begin{align}
    \prob{P}_{\rho(d)}^{\RV{Y}}(y) &= \sum_{y^{\prime D}\in Y^{D\times \mathbb{N}}} \llbracket l_{DY}(\rho(d),y^{\prime D}) = y \rrbracket \prob{P}^{\RV{Y}^D}(y^{\prime D})\\
                                &= \sum_{y^{\prime D}\in Y^{D\times \mathbb{N}}} \llbracket l_{DY}(d,\rho^{-1}(y^{\prime D})) = \rho^{-1}(y) \rrbracket \prob{P}^{\RV{Y}^D}(y^{\prime D})\\
                                &= \sum_{y^{\prime D}\in Y^{D\times \mathbb{N}}} \llbracket l_{DY}(d,\rho^{-1}(y^{\prime D})) = \rho^{-1}(y) \rrbracket \prob{P}^{\RV{Y}^D}(\rho^{-1}(y^{\prime D}))\\
                                &= \prob{P}_{\rho(d)}^{\RV{Y}}(\rho^{-1}(y))
\end{align}
\end{proof}

We can also represent contractible do-models as a Markov kernels that map from decisions to probability distributions over consequences copied $\mathbb{N}$ times and jointly parametrised by a hypothesis $\RV{H}$. 

\begin{theorem}\label{th:iid_rep}
Suppose we have a fundamental probability set $\Omega$ and a do model $(\prob{P},\RV{D},\RV{Y},R)$ such that $\RV{D}:=(\RV{D}_i)_{i\in \mathbb{N}}$ and $\RV{Y}:=(\RV{Y}_i)_{i\in\mathbb{N}}$. $\prob{P}$ is causally contractible if and only if there exists some $\RV{H}:\Omega\to H$ such that $\prob{P}^{\RV{Y}_i|\RV{H}\RV{D}_i}$ exists for all $i\in \mathbb{N}$ and
\begin{align}
    \prob{P}^{\RV{Y}|\RV{H}\RV{D}} &= \tikzfig{do_model_representation}\\
    &\iff\\
    \RV{Y}_i&\CI_{\prob{P}} \RV{Y}_{\mathbb{N}\setminus i},\RV{D}_{\mathbb{N}\setminus i}|\RV{H}\RV{D}_i&\forall i\in \mathbb{N}\\
    \land \prob{P}^{\RV{Y}_i|\RV{H}\RV{D}_i} &= \prob{P}^{\RV{Y}_0|\RV{H}\RV{D}_0} & \forall i\in \mathbb{N}
\end{align}
\end{theorem}

\begin{proof}
If:
By the assumptions of independence and identical conditionals, for any deterministic decision functions $d,d'\in D$ with equal subsequences $d_S=d'_T$
\begin{align}
    \prob{P}_d^{\RV{Y}_S|\RV{H}\RV{D}}(y|d) &= \int_H\prod_{i\in S}\prob{P}^{\RV{Y}_0|\RV{H}\RV{D}_0}(y_i|h,d_i)d\prob{P}^{\RV{H}}(h)\\
                                          &= \int_{H}\prod_{i\in T}\prob{P}^{\RV{Y}_0|\RV{H}\RV{D}_0}(y_i|h,d'_i)d\prob{P}^{\RV{H}}(h) & \text{by equality of subsequences}\\
                                          &= \prob{P}_{d'}^{\RV{Y}_T|\RV{H}\RV{D}}(y|d)
\end{align}

Only if:
We have
\begin{align}
    \prob{P}^{\RV{Y}|\RV{D}} &= \tikzfig{lookup_representation}
\end{align}

Also, by contractibility of $\prob{P}^{\RV{Y}^D}$ and De Finetti's theorem, there is some $\RV{H}$ such that

\begin{align}
    \prob{P}^{\RV{Y}^\RV{D}} &= \tikzfig{de_finetti_potential_outcomes}
\end{align}

In particular, let $\RV{Y}^D_{\cdot i}:=(\RV{Y}^D_{ji})_{j\in D}$ and $\RV{Y}^D_{\cdot \{i\}^C} = (\RV{Y}^D_{jk})_{j\in D, k\in \mathbb{N}\setminus \{i\}}$, and

\begin{align}
    &\RV{Y}^D_{\cdot i} \CI_{\prob{P}} \RV{Y}^D_{\cdot \{i\}^C} |\RV{H} & \text{ representation theorem}\label{eq:pci_1}\\
    &\RV{Y}^D\RV{H} \CI_{\prob{P}} \RV{D} &\text{ by Theorem \ref{th:cons_ci} and existence of }\prob{P}^{\RV{Y}^D\RV{H}}\label{eq:pci_2}\\
    &\RV{Y}^D_{\cdot i}\CI_{\prob{P}} \RV{D} |\RV{Y}^D_{\cdot \{i\}^C}\RV{H}&\text{ weak union on Eq. }\ref{eq:pci_2}\\
    &\RV{Y}^D_{\cdot i}\CI_{\prob{P}} \RV{D}\RV{Y}^D_{\cdot \{i\}^C} |\RV{H}&\text{ contraction on Eqs. \ref{eq:pci_1} and \ref{eq:pci_2}}\label{eq:pci_4}\\
    &\RV{Y}^D_{\cdot i}\CI_{\prob{P}} \RV{D}_{\{i\}^C}\RV{Y}^D_{\cdot \{i\}^C} |\RV{H}\RV{D}_i&\text{ weak union on Eq. \ref{eq:pci_4}}\label{eq:pci_5}\\
    &\RV{D}_{i} \CI_{\prob{P}}\RV{Y}^D_{\cdot \{i\}^C} \RV{D}_{\{i\}^C} |\RV{H}\RV{D}_i \RV{Y}^D_{\cdot i}&\text{ due to conditioning on }\RV{D}_i\label{eq:pci_6}\\
    &\RV{Y}^D_{i}\RV{D}_i \CI_{\prob{P}} \RV{D}_{\{i\}^C}\RV{Y}^D_{\cdot \{i\}^C} |\RV{H}\RV{D}_i&\text{ contraction on Eqs. \ref{eq:pci_5} and \ref{eq:pci_6}}\label{eq:pci_7}\\
\end{align}

Now, note that $(\RV{Y}_i,\RV{D}_i)$ is a deterministic function of $(\RV{Y}^D_{i},\RV{D}_i)$ and $(\RV{Y}_{\{i\}^C},\RV{D}_{\{i\}^C})$ is a deterministic function of $(\RV{Y}^D_{\{i\}^C},\RV{D}_{\{i\}^C})$. Therefore

\begin{align}
    &\RV{Y}_i \CI_{\prob{P}} \RV{D}_{\{i\}^C}\RV{Y}_{\{i\}^C} |\RV{H}\RV{D}_i&
\end{align}

So, by Theorem \ref{th:cons_ci}, $\prob{P}^{\RV{Y}_i|\RV{HD}}$ exists and by contractibility of $\prob{P}^{\RV{Y}^D}$, for any $i,j\in\mathbb{N}$

\begin{align}
    \prob{P}^{\RV{Y}_i|\RV{HD_i}}(y_i|h,d_i) &= \prob{P}^{\RV{Y}^D_{d_i i}|\RV{H}}(y_i|h) \\
    &= \prob{P}^{\RV{Y}^D_{d_i j}|\RV{H}}(y_i|h)\\
    &= \prob{P}^{\RV{Y}_j|\RV{HD}_j}(y_i|h,d_i)
\end{align}
\end{proof}

\subsection{Potential outcomes}


%!TEX root = main.tex


\section{Potential outcomes models}

Potential outcomes is an approach to causal modelling that employs ``potential outcome'' random variables. We have something, call it TAT (``The Actual Thing'') that we want to model, and we have different regimes 0 and 1 under which we want to know something about TAT. In the potential outcomes approach we model it with a set of variables $(\RV{Y}, \RV{Y}(0), \RV{Y}(1))$, with $\RV{Y}(0)$ and $\RV{Y}(1)$ being called ``potential outcomes''. These variables are given the interpretations ``$\RV{Y}$ represents TAT'', ``$\RV{Y}(0)$ represents what TAT would be under regime $0$, whether or not regime $0$ is the actual regime'' and similarly ``$\RV{Y}(1)$ represents what TAT would be under regime $1$, whether or not regime $1$ is the actual regime''. Such interpretations are sometimes called \emph{counterfactual}. To explain why: ``$\RV{Y}(0)$ represents what TTYR would be under regime $0$, whether or not regime $0$ is the actual regime'' is a conjunction of claims - first, if the regime is actually $0$ then $\RV{Y}(0)$ represents TAT; this is reasonably easy to understand. Also, if the regime is actually $1$ then $\RV{Y}(1)$ represents what TAT would be if the regime were $0$. This claim seems to be predicated on two mutually exclusive conditions -- ``if the regime is actually $1$'' and ``if the regime were $0$'', and this is what makes it a counterfactual claim.

It is not our aim to add anything to the discussion of how to understand counterfactual claims, or whether they should or should not play a part in models built for the purpose of causal inference. Rather, we want to clarify two points:

\begin{enumerate}
    \item Even though they are motivated by the need to answer decision problems rather than a desire to model counterfactuals, see-do models can nonetheless do anything potential outcomes models can do
    \item What a model builder's model represents is the choice of the model builder; whether or not this includes counterfactual propositions is a matter of choice and is not forced upon them by any approach to causal inference
\end{enumerate}

The decision problem discussed in subsection \ref{ssec:data_driven_decision} involves hypotheses, observations, decisions and consequences. When describing such problems we could say that $\kernel{T}[\RV{Y}|\RV{HD}]_{ij}$ describes ``the value that $\RV{Y}$ would take if $\RV{H}=i$ and $\RV{D}=j$'', but there does not appear to be an additional need for variables describing ``the value that $\RV{Y}$ would take if $\RV{H}=i$ and $\RV{D}=j$ even if $\RV{H}=i$ and $\RV{D}=j$ does not actually hold''. A number of other authors of decision theoretic approaches have noted that we do not seem to need models of counterfactuals in order to solve data-driven decision problems. \citet{dawid_causal_2000} is actually titled ``Causal Inference without Counterfactuals'', although more recent work by the same author discusses counterfactual assumptions \citep{dawid_decision-theoretic_2020}.

\begin{quote}
[...] Dawid, in our opinion, incorrectly concludes that an approach to causal inference based on ``decision analysis'' and free of counterfactuals is completely satisfactory for addressing the problem of inference about the effects of causes.
\end{quote}
%!TEX root = main.tex

\section{Appendix:see-do model representation}\label{sec:see-do-rep}

\todo[inline]{Modularise the treatment of probability}

\begin{theorem}[See-do model representation]\label{th:see_do_rep}
Suppose we have a decision problem that provides us with an observation $x\in X$, and in response to this we can select any decision or stochastic mixture of decisions from a set $D$; that is we can choose a ``strategy'' as any Markov kernel $\kernel{S}:X\to \Delta(D)$. We have a utility function $u:Y\to \mathbb{R}$ that models preferences over the potential consequences of our choice. Furthermore, suppose that we maintain a denumerable set of hypotheses $H$, and under each hypothesis $h\in H$ we model the result of choosing some strategy $\kernel{S}$ as a joint probability over observations, decisions and consequences $\prob{P}_{h,\kernel{S}}\in \Delta(X\times D\times Y)$.

Define $\RV{X},\RV{Y}$ and $\RV{D}$ such that $\RV{X}_{xdy} = x$, $\RV{Y}_{xdy}=y$ and $\RV{D}_{xdy} = d$. Then making the following additional assumptions:
\begin{enumerate}
    \item Holding the hypothesis $h$ fixed the observations as have the same distribution under any strategy: $\prob{P}_{h,\kernel{S}}[\RV{X}]=\prob{P}_{h,\kernel{S}''}[\RV{X}]$ for all $h,\kernel{S},\kernel{S}'$ (observations are given ``before'' our strategy has any effect)
    \item The chosen strategy is a version of the conditional probability of decisions given observations: $\kernel{S}=\prob{P}_{h,\kernel{S}}[\RV{D}|\RV{X}]$
    \item There exists some strategy $\kernel{S}$ that is strictly positive
    \item For any $h\in H$ and any two strategies $\kernel{Q}$ and $\kernel{S}$, we can find versions of each disintegration such that $\prob{P}_{h,\kernel{Q}}[\RV{Y}|\RV{D}\RV{X}]=\prob{P}_{h,\kernel{S}}[\RV{Y}|\RV{D}\RV{X}]$ (our strategy tells us nothing about the consequences that we don't already know from the observations and decisions)
\end{enumerate}

Then there exists a unique see-do model $(\kernel{T},\RV{H}',\RV{D}',\RV{X}',\RV{Y}')$ such that $\prob{P}_{h,\kernel{S}}[\RV{XDY}]^{ijk} = \kernel{T}[\RV{X'}|\RV{H'}]_{h}^{i} \kernel{S}_i^j  \kernel{T}[\RV{Y'}|\RV{X'H'D'}]_{ijk}^k$.
\end{theorem}

\begin{proof}
Consider some probability $\prob{P}\in \Delta(X\times D\times Y)$. By the definition of disintegration (section \ref{ssec:disintegration}), we can write

\begin{align}
 \prob{P}[\RV{XDY}]^{ijk} = \prob{P}[\RV{X}]^i\prob{P}[\RV{D}|\RV{X}]_i^{j} \prob{P}[\RV{Y}|\RV{XD}]_{ij}^{k} \label{eq:disint}
\end{align}

Fix some $h\in H$ and some strictly positive strategy $\kernel{S}$ and define $\kernel{T}:H\times D\to \Delta(X\times Y)$ by
\begin{align}
    \kernel{T}_{hj}^{kl} &= \prob{P}_{h,\kernel{S}}[\RV{X}]^k \prob{P}_{h,\kernel{S}}[\RV{Y}|\RV{X}\RV{D}]^l_{kj} \label{eq:comb_disint}
\end{align}

Note that because $\kernel{S}$ is strictly positive and by assumption $\kernel{S}=\prob{P}_{h,\kernel{S}}[\RV{D}|\RV{X}]$, $\prob{P}_{h,\kernel{S}}[\RV{D}]$ is also strictly positive. Therefore $\prob{P}_{h,\kernel{S}}[\RV{Y}|\RV{D}]$ is unique and therefore $\kernel{T}$ is also unique.

Define $\RV{X}'$ and $\RV{Y}'$ by $\RV{X}'_{xy}=x$ and $\RV{Y}'_{xy}=y$. Define $\RV{H}'$ and $\RV{D}'$ by $\RV{H}'_{hd} = h$ and $\RV{D}'_{hd} = d$.

We then have
\begin{align}
    \kernel{T}[\RV{X'}|\RV{H'D'}]_{hj}^{k} &= \kernel{T}\underline{\RV{X}'}_{hj}^k\\
                                           &= \sum_l \kernel{T}_{hj}^{kl} \\
                                           &= \prob{P}_{h,\kernel{S}}[\RV{X}]^k\\
                                           &= \kernel{T}[\RV{X'}|\RV{H'D'}]_{hj'}^{k}
\end{align}

Thus $\RV{X}'\CI_{\kernel{T}} \RV{D}'|\RV{H}'$ and so $\kernel{T}[\RV{X}'|\RV{H}']$ exists (section \ref{ssec:cond_indep}) and $(\kernel{T},\RV{H}',\RV{D}',\RV{X}',\RV{Y}')$ is a see-do model.

Applying Equation \ref{eq:disint} to $\prob{P}_{h,\kernel{S}}$:

\begin{align}
    \prob{P}_{h,\kernel{S}}[\RV{XDY}]^{ijk} &= \prob{P}_{h,\kernel{S}}[\RV{X}]^i\prob{P}_{h,\kernel{S}}[\RV{D}|\RV{X}]_i^{j} \prob{P}_{h,\kernel{S}}[\RV{Y}|\RV{XD}]_{ij}^{k}\label{eq:t_is_comb_disint_start}\\
     &=  \prob{P}_{h,\kernel{S}}[\RV{X}]^i\prob{P}_{h,\kernel{S}}[\RV{Y}|\RV{XD}]_{ij}^{k}\\
     &= \prob{P}_{h,\kernel{S}}[\RV{D}|\RV{X}]_i^{j} \kernel{T}[\RV{X'Y'}|\RV{H'D'}]_{hj}^{ik}\\
     &= \kernel{S}_i^j \kernel{T}[\RV{X'Y'}|\RV{H'D'}]_{hj}^{ik}\\
     &= \kernel{S}_i^j \kernel{T}[\RV{X'}|\RV{H'D'}]_{hj}^{i} \kernel{T}[\RV{Y'}|\RV{X'H'D'}]_{ihj}^k\\
     &= \kernel{T}[\RV{X'}|\RV{H'}]_{h}^{i} \kernel{S}_i^j  \kernel{T}[\RV{Y'}|\RV{X'H'D'}]_{ihj}^k\label{eq:t_is_comb_disint_end}
\end{align}

Consider some arbitrary alternative strategy $\kernel{Q}$. By assumption

\begin{align}
    \prob{P}_{h,\kernel{S}}[\RV{X}]^{i} &= \prob{P}_{h,\kernel{Q}}[\RV{X}]^{i}\\
    \prob{P}_{h,\kernel{S}}[\RV{Y}|\RV{XD}]_{ij}^k &= \prob{P}_{h,\kernel{Q}}[\RV{Y}|\RV{XD}]_{ij}^k\text{ for some version of }\prob{P}_{h,\kernel{Q}}[\RV{Y}|\RV{XD}]
\end{align}

It follows that, for some version of $\prob{P}_{h,\kernel{Q}}[\RV{Y}|\RV{XD}]$,
\begin{align}
    \kernel{T}_{hj}^{kl} &= \prob{P}_{h,\kernel{Q}}[\RV{X}]^k \prob{P}_{h,\kernel{Q}}[\RV{Y}|\RV{X}\RV{D}]^l_{kj} \label{eq:comb_disint_nonuniq}
\end{align}

Then by substitution of $\kernel{Q}$ for $\kernel{S}$ in Equation \ref{eq:t_is_comb_disint_start} and working through the same steps

\begin{align}
    \prob{P}_{h,\kernel{S}}[\RV{XDY}]^{ijk} &= \kernel{T}[\RV{X'}|\RV{H'}]_{h}^{i} \kernel{Q}_i^j  \kernel{T}[\RV{Y'}|\RV{X'H'D'}]_{ihj}^k
\end{align}

As $\kernel{Q}$ was arbitrary, this holds for all strategies.
\end{proof}
% \input{chapter_3_twoplayer_statistical_models}
% \input{chapter_4_statistical_decision_theory}
% \input{chapter_5_interventions_counterfactuals}
% \input{chapter_6_imitability_inference}
% \input{chapter_7_godscomputer}

\bibliographystyle{plainnat}
\bibliography{references}

\appendix
\newpage
\section*{Appendix:}

% \input{appendix_AIstats}

\end{document}
