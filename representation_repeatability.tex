%!TEX root = main.tex



\section{Syntax and semantics of ``causal effects''}






If we have a conditional probability model $\prob{P}_\square^{\overline{\RV{Y}|\RV{D}}}$, then by definition there is a conditional probability $\prob{P}_\square^{\RV{Y}|\RV{D}}$ which has a curried representation. Our aim is to show when certain conditional probabilities exist with respect to a probability gap model, which in this case is a triviality.

The question becomes more interesting when we propose conditional probability model $\prob{P}_\square^{\overline{\RV{Y}|\RV{D}}}$ of a sequential experiment. That is, $\RV{Y}:=\RV{Y}_M=(\RV{Y}_i)_{i\in M}$ and $\RV{D}:=\RV{D}_M=(\RV{D}_i)_{i\in M}$ and we say that $\RV{Y}_i$ is the consequence corresponding to the decision $\RV{D}_i$ for all $i\in M$. We will call a $(\RV{D}_i,\RV{Y}_i)$ pair an experimental unit. In this case, the conditional probability $\prob{P}_\square^{\RV{Y}_i|\RV{D}_i}$ does not generally exist. We might suppose, however, it would exist if the experiment was, in some sense, suitably regular or repeatable.

\subsection{Repeatable experiments}

We have a conditional probability model $\prob{P}_\square^{\overline{\RV{Y}_A|\RV{D}_A}}$ with choices $A$ that represents a sequential experiment. What might we mean when we say this experiment is repeatable? We're going to propose two conditions. The first condition is \emph{commutativity of exchange}, which is the assumption that swapping the choices that we apply at each step and then applying the corresponding inverse swap to consequences leaves the model unchanged. The second condition is \emph{commutativity of marginalisation} -- if we perform the whole experiment multiple times, making the same choice $\RV{D}_i$ at any point $i$ gets the same results, regardless of what other choices are made.

Commutativity of exchange is similar to the condition of \emph{post-treatment exchangeability} found in \citet{dawid_decision-theoretic_2020}, and commutativity of marginalisation is similar to the stable unit treatment distribution assumption (SUTDA) in the same, as well as the ``no interference'' part of the stable unit treatment value assumption (SUTVA) with which it shares a name. Commutativity of exchange is also very similar to the exchangeability assumption of \citet{greenland_identifiability_1986} for further discussions of exchangeability in the context of causal modelling, and note that both authors consider exchanging to be an operation that alters which person receives which treatment. The assumption of exchangeability found in \citet{banerjee_chapter_2017} can also be regarded as similar to commutativity of exchange.

\todo[inline]{I think the useful part is not that these ideas are conceptually new, but they have sharp definitions instead of }

\todo[inline]{Not sure if or where I want to put this, I just think it helps to illustrate the difference}

Commutativity of exchange is not equivalent to exchangeability in the sense of De Finetti's well-known theorem \citet{de_finetti_foresight_1992}. The latter can be understood as expressing an indifference between conducting the experiment as normal, or conducting the experiment and then swapping some labels. However, swapping \emph{choices} will (usually) lead to different experimental units receive different treatment, which is something that can't be achieved by swapping labels after the experiment has concluded.

The difference is illustrated by the following pair of diagrams.

Exchangeability (swapping labels):

\begin{align}
    \tikzfig{exchangeability}
\end{align}

Commutativity of exchange (swapping choices $\sim$ swapping labels):

\begin{align}
    \tikzfig{commutativity of exchange}
\end{align}

Commutativity of exchange is a property of probability gap models, not a property of fixed probability model for which there is no analogue of ``attaching a different choice'' in that case.

\todo[inline]{----end not sure where to put------}


% Another way to see where we are going is to consider graphical statements of our and De Finetti's result.

% Take $S=\{0,1\}$ and identify the space $\Delta(S)$ of probability measures on $S$ with the interval $[0,1]$. De Finetti showed that any infinite exchangeable probability measure $\prob{P}_\alpha$ on $\{0,1\}^\mathbb{N}$ can be represented by a prior $\prob{P}_\alpha^{\RV{H}}\in [0,1]$ for some $\RV{H}:\Omega\to H$ and a conditional probability $\prob{P}^{\RV{S}_0|\RV{H}}:[0,1]\kto \{0,1\}$ such that

% \begin{align}
%     \prob{P}_\alpha &= \tikzfig{de_finetti_rep0}\label{eq:definettirep}
% \end{align}

% Here $\prob{P}^{\RV{S}_0|\RV{H}}$ can be defined concretely by $\prob{P}^{\RV{S}_0|\RV{H}}(1|h)=h$. Equivalently, the probability gap model on $S^\mathbb{N}$ defined by the assumption of exchangeability is equivalent to the probability gap model defined by the conditional probability

% \begin{align}
%     \prob{P}^{\RV{S}|\RV{H}} = \tikzfig{de_finetti_conditional}
% \end{align}

% That is, there is some hypothesis $\RV{H}$ and conditional on $\RV{H}$ the measurements are independent and identically distributed. The proof of this is constructive -- $\RV{H}$ is a function of $\RV{S}$.



% \begin{align}
%     \prob{P}^{\RV{Y}|\RV{HD}} = \tikzfig{do_model_representation}
% \end{align}

% We will further argue that the class of see-do models considered in CBN and potential outcomes literature is equivalent to the family of causally contractible and exchangeable do-models where the decision rule for the first $n$ places is fixed to an unknown value, and may be freely chosen thereafter.

% \begin{theorem}[Existence of conditional in do models]
% Given a do model $(\prob{P}_{\square}^{\RV{Y}\|\RV{D}},R)$, for all $\alpha\in R$, $n\in\mathbb{N}$
% \begin{align}
%     \prob{P}_\alpha^{\RV{Y}_{[n]}\RV{D}_i} = \prob{P}_\alpha^{\RV{D}_{[n]}}\odot \prob{P}_\square^{\RV{Y}_{[n]}\|\RV{D}_{[n]}}
% \end{align}
% That is, $\prob{P}_\square^{\RV{Y}_{[n]}\|\RV{D}_{[n]}}\cong \prob{P}_\square^{\RV{Y}_{[n]}|\RV{D}_{[n]}}$
% \end{theorem}

% \begin{proof}
% For any $n>1\in \mathbb{N}$, $\alpha\in R$

% \begin{align}
%     \prob{P}_\alpha^{\RV{Y}_{[n]}\RV{D}_{[n]}} &= \tikzfig{do_model_1}\\
%     &= \tikzfig{do_model_2}\\
%     &= \tikzfig{do_model_3}\\
%     &= \tikzfig{do_model_4}\\
%     \implies \prob{P}_\alpha^{\RV{Y}_{[n]}|\RV{D}_{[n]}} &= \tikzfig{do_model_5}\\
%     &= \prob{P}_\alpha^{\RV{Y}_{[n-1]}|\RV{D}_{[n-1]}}\combprod \prob{P}_\square^{\RV{Y}_n|\RV{Y}_{[n-1]}\RV{D}_n}
% \end{align}

% Applying this recursively with $\prob{P}_\alpha^{\RV{Y}_{[1]}|\RV{D}_{[1]}}=\prob{P}_\square^{\RV{Y}_{[1]}|\RV{D}_{[1]}}$ yields

% \begin{align}
%     \prob{P}_\alpha^{\RV{Y}_{[n]}|\RV{D}_{[n]}} = \prob{P}_\square^{\RV{Y}_{[n]}\|\RV{D}_{[n]}}
% \end{align}
% as desired.
% \end{proof}
More precisely, a conditional probability model ``commutes with exchange'' if applying any finite permutation to blind choices or separately applying the corresponding permuation to consequences each yields the same result. We can apply the exchange ``before'' multiplying by the conditional $\prob{P}_{\square}^{\RV{Y}|\RV{D}}$ or after it and we get the same result.

\begin{definition}[Swap map]
Given $M\subset \mathbb{N}$ a finite permutation $\rho:M\to M$ and a variable $\RV{X}:\Omega\to X^M$ such that $\RV{X}=(\RV{X}_i)_{i\in M}$, define the Markov kernel $\text{swap}_{\rho(\RV{X})}:X^M\kto X^M$ by $(d_i)_{i\in\mathbb{N}}\mapsto \delta_{(d_{\rho(i)})_{i\in\mathbb{N}}}$.
\end{definition}

\begin{definition}[Commutativity of exchange]\label{def:caus_exch}
Suppose we have a sample space $(\Omega,\sigalg{F})$ and a conditional probability model $(\prob{P}_{\square}^{\overline{\RV{Y}|\RV{D}}},A)$ with $\RV{Y}=\RV{Y}_M$, $\RV{D}=\RV{D}_M$, $M\subseteq \mathbb{N}$. If, for any two decision rules $\alpha^{\overline{\RV{D}}},\beta^{\overline{\RV{D}}} \in A$,
\begin{align}
    \alpha^{\RV{D}}\odot \text{swap}_{\rho(\RV{D})} \prob{P}_{\square}^{\RV{Y}|\RV{D}} &= \alpha^{\RV{D}}\odot \prob{P}_{\square}^{\RV{Y}|\RV{D}}\text{swap}_{\rho(\RV{D}\times \RV{Y})}
\end{align}
Then $\prob{P}_\square$ \emph{commutes with exchanges}.
\end{definition}

A do model is non interfering if it gives identical results for identical subsequences of different choices when we limit our attention to the corresponding subsequences of consequences. For example, if we have $\RV{D}=(\RV{D}_1,\RV{D}_2,\RV{D}_3)$ and $\RV{Y}=(\RV{Y}_1,\RV{Y}_2,\RV{Y}_3)$ and $\alpha^{\RV{D}_1\RV{D}_3}=\prob{P}_\beta^{\RV{D}_1\RV{D}_3}$ then $\prob{P}_{\alpha}^{\RV{Y}_1\RV{D}_1\RV{Y}_3\RV{D}_3}=\prob{P}_\beta^{\RV{Y}_1\RV{D}_1\RV{Y}_3\RV{D}_3}$.

\begin{definition}[Commutativity of marginalisation]\label{def:caus_cont}
Suppose we have a sample space $(\Omega,\sigalg{F})$ and a conditional probability model $(\prob{P}_{\square}^{\overline{\RV{Y}|\RV{D}}},A)$ with $\RV{Y}=\RV{Y}_M$, $\RV{D}=\RV{D}_M$, $M\subseteq \mathbb{N}$. For any $S=(s_i)_{i\in Q}$, $Q\subset M$, and $i<j\implies p_i<p_j \And q_i<q_j$, let $\RV{D}_S:=(\RV{D}_i)_{i\in S}$ and $\RV{D}_T:=(\RV{D}_i)_{i\in T}$. If for any $\alpha,\beta\in R$
\begin{align}
    \prob{P}_\alpha^{\RV{D}_{S}}&=\prob{P}_\beta^{\RV{D}_{S}}\\
    \implies \prob{P}_\alpha^{(\RV{D_i,Y_i})_{i\in S}}&=\prob{P}_\beta^{(\RV{D_i,Y_i})_{i\in S}}
\end{align}
then $\prob{P}_\square$ \emph{commutes with marginalisation}.
\end{definition}

Neither condition implies the other. 
\begin{lemma}
Commutativity of exchange does not imply commutativity or vise versa.
\end{lemma}

\begin{proof}
Suppose $D=Y=\{0,1\}$ and we have a conditional probability model $(\prob{P}_\square^{\overline{\RV{Y}|\RV{D}}},A)$ where $\RV{D}=(\RV{D}_1,\RV{D}_2)$, $\RV{Y}=(\RV{Y}_1,\RV{Y}_2)$ and A contains all deterministic probability measures in $\Delta(D^2)$. If

\begin{align}
    \prob{P}_\square^{\RV{Y}_1\RV{Y}_2|\RV{D}_1\RV{D}_2}(y_1,y_2|d_1,d_2) &= \llbracket (y_1,y_2)= (d_1+d_2,d_1+d_2) \rrbracket
\end{align}

Then $\prob{P}_{\delta_{00}}^{\RV{Y}_1\RV{D}_1}(y_1) = \llbracket y_1=0\rrbracket$ while $\prob{P}_{\delta_{01}}^{\RV{Y}_1} = \llbracket y_1=1 \rrbracket$. However, $\delta_00^{\RV{D}_1}=\delta_{01}^{\RV{D}_1}=\delta_0^{\RV{D}_1}$ so $\prob{P}_\square$ does not commute with marginalisation. However, taking $(d_i,d_j):=\delta_{d_i d_j}\in A$,

\begin{align}
    \prob{P}_{d_2,d_1}^{\RV{Y}_1\RV{D}_1\RV{Y}_2\RV{D}_2}(y_1,d_1,y_2,d_2) &= \llbracket (y_1,y_2)= (d_2+d_1,d_2+d_1) \rrbracket\\
    &= \llbracket (y_2,y_1)= (d_1+d_2,d_1+d_2) \rrbracket\\
    &= \prob{P}_{d_1,d_2}^{\RV{Y}_1\RV{D}_1\RV{Y}_2\RV{D}_2}(y_2,d_2,y_1,d_1)
\end{align}

so $\prob{P}_\square$ commutes with exchange.

Alternatively, suppose the same setup, but define $\prob{P}_\square$ instead by, for all $\alpha\in A$

\begin{align}
    \prob{P}_\square{\RV{Y}_1\RV{Y}_2|\RV{D}_1\RV{D}_2}(y_1,y_2|d_1,d_2) &= \llbracket (y_1,y_2)= (0,1) \rrbracket
\end{align}

Then $\prob{P}_\square$ commutes with marginalisation. If $\prob{P}_\alpha^{\RV{D}_S}=\prob{P}_\beta^{\RV{D}_S}$ for $S\subset\{0,1\}$ then

\begin{align}
    \prob{P}_{\alpha}^{\RV{Y}_S\RV{D}_S}(y_s,d_s) &= \sum_{y'_2\in \{0,1\}^{S^C}} \llbracket (y_1,y_2)= (0,1) \rrbracket\prob{P}_\alpha^{\RV{D}_S}(d_s) \\
                                                  &= \prob{P}_{\beta}^{\RV{Y}_S\RV{D}_S}(y_s,d_s)
\end{align}
but not exchange. For all $\alpha,\beta \in A$:

\begin{align}
    \prob{P}_\alpha{\RV{Y}_1\RV{Y}_2}(y_1,y_2) &= \llbracket (y_1,y_2)= (0,1) \rrbracket\\
    &\neq \prob{P}_\beta{\RV{Y}_1\RV{Y}_2}(y_2,y_1)
\end{align}
\end{proof}

Although commutativity of marginalisation seems to be a bit like non-interference -- the marginal distribution I get for $\RV{Y}_i$ depends only on the decision $\RV{D}_i$ -- it still allows for some models in which we seem to have interference of a kind. For example: in the first experiment I flip a coin and decide either to pass the results to the second experiment ($\RV{D}_1=0$) or flip another coin and pass those results second experiment ($\RV{D}_1=1$). In the second I either copy the results I have been given ($\RV{D}_2=0$) or invert them ($\RV{D}_2=1$). Then
\begin{itemize}
    \item The marginal distribution of both experiments is $\text{Bernoulli}(0.5)$ no matter what choices I make, so it satisfies Definition \ref{def:caus_cont}
    \item Nevertheless, the choice for the first experiment seems to ``affect'' the result of the second experiment (affect in quotes because it is an intuitive judgement, not a formal property)
\end{itemize}

Here we are most interested in the conjunction of these assumptions, a condition we call \emph{causal contractibility}

\begin{definition}[Causal contractibility]
A conditional probability model $(\prob{P}_{\square}^{\overline{\RV{Y}|\RV{D}}},A)$ is causally contractible if it is both commutative with exchange and commutative with marginalisation.
\end{definition}

% \begin{proposition}[Representation of do-models that commute with exchange]
% Suppose we have a fundamental probability set $\Omega$ and a do model $(\prob{P},\RV{D},\RV{Y},R)$ such that $\RV{D}:=(\RV{D}_i)_{i\in \mathbb{N}}$ and $\RV{Y}:=(\RV{Y}_i)_{i\in\mathbb{N}}$ where $\prob{P}$ commutes with exchange and there is some $\alpha^*\in R$ such that $\prob{P}^{\alpha^*}\gg\prob{P}_\beta$ for all $\beta in R$. Then there exists a symmetric function $\RV{H}:(Y\times D)^\mathbb{N}\to H$ such that  $\prob{P}^{\RV{Y}|\RV{DH}}$ exists and $\RV{Y}_i\CI_{\prob{P}}(\RV{D}_j,\RV{Y}_j)_{j\in \mathbb{N}}\setminus \{i\}|\RV{H}\RV{D}_i$, or equivalently 
% \begin{align}
%     \prob{P}^{\RV{Y}} &= \tikzfig{do_model_representation}
% \end{align}
% \end{proposition}

% % \begin{lemma}[Contraction and independence]
% % Let $\RV{J}$, $\RV{K}$ and $\RV{L}$ be variables on $\Omega$ and $\prob{Q}\in \Delta(\Omega)$ a base measure such that $\prob{Q}^{\RV{JK}}=\prob{Q}^{\RV{JL}}$ and $\sigma{K}\subset \sigma{L}$. Then $\RV{J}\CI\RV{L}|\RV{K}$. 
% % \end{lemma}

% % \begin{proof}
% % From Lemma 1.3 in \citet{kallenberg_basic_2005}.
% % \end{proof}

% \begin{proof}
% If $\prob{P}$ commutes with exchange, then for any $\alpha\in R$ such that $\prob{P}_\alpha^{\RV{D}}$ is exchangeable then $\prob{P}_\alpha$ is also exchangeable. Then there exists $\RV{H}$ a symmetric function of $(\RV{Y}_i,\RV{D}_i)_{i\in\mathbb{N}}$ such that $\RV{Y}_i\CI_{\prob{P}}(\RV{D}_j,\RV{Y}_j)_{j\in \mathbb{N}}\setminus \{i\}|\RV{H}\RV{D}_i$. This is De Finetti's representation theorem, and many proofs exists, see for example \citep{kallenberg_basic_2005}.

% In particular, let 

% \begin{align}
%     \RV{H}:=A\times B\mapsto \lim_{n\to\infty} \frac{1}{n}\sum_{i\in n} \mathds{1}_{A\times B}((\RV{Y}_i, \RV{D}_i))
% \end{align}

% Then for all $\alpha\in R$,
% \begin{align}
%     \prob{P}_\alpha^{(\RV{Y}_i,\RV{D}_i)_{i\in\mathbb{N}}|\RV{H}}(A\times B|h) \overset{a.s.}{=} h(A\times B)\label{eq:given_h}
% \end{align}

% The proof that the limit exists and the above equality holds can again be found int \citep{kallenberg_basic_2005}.
% \end{proof}

\subsection{When does a canonical ``effect of a decision'' exist?}

The main result in this section is Theorem \ref{th:iid_rep} which shows that a conditional probability model $\prob{P}_\square$ is causally contractible if and only if it can be represented as the product of a distribution over hypotheses $\prob{P}_\square^{\RV{H}}$ and a collection of identical conditional probabilities $\prob{P}_\square^{\RV{Y}_1|\RV{D}_1\RV{H}}$. This can be interpreted as expressing the idea that all experimental units $(\RV{Y}_i,\RV{D}_i)$ share a canonical but unknown ``consequence function'' $D\kto Y$. As discussed already in Section \ref{sec:curry}, the existence of such a conditional probability implies the existence of a common unknown \emph{curried} conditional probability for all experimental units, which resembles a potential outcomes model. In fact, we prove the existence of a curried representation first, in Lemma \ref{th:table_rep}.

\begin{Lemma}[Exchangeable curried representation]\label{th:table_rep}
A conditional probability model $(\prob{P}^{\RV{Y}|\RV{D}}_\square,A)$ such that $\RV{D}:=(\RV{D}_i)_{i\in \mathbb{N}}$ and $\RV{Y}:=(\RV{Y}_i)_{i\in \mathbb{N}}$. $\prob{P}_\square$ is causally contractible if and only if
\begin{align}
    \prob{P}_\square^{\RV{Y}|\RV{D}} &= \tikzfig{lookup_representation}\\
    &\iff\\
    \prob{P}_\square^{\RV{Y}|\RV{D}}(y|d) &= \prob{P}^{(\RV{Y}^D_{d_i i})_{\mathbb{N}}}(y)
\end{align}
Where $\prob{P}^{\RV{Y}^D}$ is an exchangeable probability measure on $Y^{D\times\mathbb{N}}$, for convenience we extend the sample space with the random variable $\RV{Y}^D:=(\RV{Y}_{ij}^D)_{i\in D,j\in \mathbb{N}}$ and $\prob{L}^{\RV{D},\RV{Y}^D}$ is the Markov kernel associated with the lookup function
\begin{align}
    l:D^\mathbb{N}\times Y^{D\times \mathbb{N}}&\to Y\\
    ((d_i)_\mathbb{N},(y_{ij})_{i\in D,j\in \mathbb{N}})&\mapsto y_{d_i i}
\end{align}
\end{Lemma}

\begin{proof}
Only if:
Choose $e:=(e_i)_{i\in\mathbb{N}}$ such that $e_{|D|i+j}$ is the $i$th element of $D$ for all $i,j\in \mathbb{N}$. Abusing notation, write $e$ also for the decision function that chooses $e$ deterministically.

Define
\begin{align}
    \prob{P}^{\RV{Y}^D}((y_{ij})_{D\times \mathbb{N}}):=\prob{P}_e^{\RV{Y}}((y_{|D|i+j})_{i\in D, j\in \mathbb{N}})
\end{align}

Now consider any $d:=(d_i)_{i\in \mathbb{N}}\in D^{\mathbb{N}}$. By definition of $e$, $e_{|D|d_i + i}=d_i$ for any $i,j\in \mathbb{N}$.

\begin{align}
    \prob{Q}:D\kto Y\\
    \prob{Q}:= \tikzfig{lookup_representation}
\end{align}

and consider some ordered sequence $A\subset \mathbb{N}$ and $B:= ((|D|d_i+i))_{i\in A}$. Note that $e_B:=(e_{|D|d_i +i})_{i\in B}=d_A=(d_i)_{i\in A}$. Then 

\begin{align}
    \sum_{y\in \RV{Y}^{-1}(y_A)} \prob{Q}(y|d) &= \sum_{y\in \RV{Y}^{-1}(y_A)} \prob{P}^{(\RV{Y}^{D}_{d_ii})_{A}}(y) \\
    &= \sum_{y\in \RV{Y}^{-1}(y_A)} \prob{P}_e^{(\RV{Y}_{|D|d_i+i})_{A}}(y)\\
    &= \prob{P}_e^{\RV{Y}_{B}}(y_A)\\
    &= \prob{P}_{d}^{\RV{Y}_A}(y_A)&\text{by causal contractibility}
\end{align}

Because this holds for all $A\subset\mathbb{N}$, by the Kolmogorov extension theorem

\begin{align}
    \prob{Q}(y|d) &= \prob{P}_d^{\RV{Y}}(y)
\end{align}

Because $d$ is the decision function that deterministically chooses $d$, for all $d\in D$

\begin{align}
    \prob{Q}(y|d) &= \prob{P}_d^{\RV{Y}|\RV{D}}(y|d)
\end{align}

And because $\prob{P}_d^{\RV{Y}|\RV{D}}(y|d)$ is unique for all $d\in D^{\mathbb{N}}$ and $\prob{P}^{\RV{Y}|\RV{D}}$ exists by assumption

\begin{align}
    \prob{P}^{\RV{Y}|\RV{D}}=\prob{Q}
\end{align}

Next we will show $\prob{P}^{\RV{Y}^D}$ is contractible. Consider any subsequences $\RV{Y}^D_S$ and $\RV{Y}^D_T$ of $\RV{Y}^D$ with $|S|=|T|$. Let $\rho(S)$ be the ``expansion'' of the indices $S$, i.e. $\rho(S)=(|D|i+j)_{i\in S,j\in D}$. Then by construction of $e$, $e_{\rho(S)}=e_{\rho(T)}$ and therefore

\begin{align}
    \prob{P}^{\RV{Y}^D_S}&= \prob{P}_e^{\RV{Y}_{\rho(S)}})\\
    &= \prob{P}_e^{\RV{Y}_{\rho(T)}})&\text{by contractibility of }\prob{P}\text{ and the equality } e_{\rho(S)}=e_{\rho(T)}\\
    &= \prob{P}^{\RV{Y}^D_T}
\end{align}


If:
Suppose 
\begin{align}
    \prob{P}^{\RV{Y}|\RV{D}} &= \tikzfig{lookup_representation}
\end{align}

and consider any two deterministic decision functions $d,d'\in D^{\mathbb{N}}$ such that some subsequences are equal $d_S=d'_T$.

Let $\RV{Y}^{d_S}=(\RV{Y}_{d_i i})_{i\in S}$.

By definition,

\begin{align}
    \prob{P}^{\RV{Y}_S|\RV{D}}(y_S|d) &= \sum_{y^D_S\in Y^{|D|\times |S|}}\prob{P}^{\RV{Y}^D_S}(y^D_S)\prob{L}^{\RV{D}_S,\RV{Y}^S}(y_S|d,y^D_S)\\
    &= \sum_{y^D_S\in Y^{|D|\times |T|}}\prob{P}^{\RV{Y}^D_T}(y^D_S)\prob{L}^{\RV{D}_S,\RV{Y}^S}(y_S|d,y^D_S)&\text{ by contractibility of }\prob{P}^{\RV{Y}^D_T}\\
    &= \prob{P}^{\RV{Y}_T|\RV{D}}(y_S|d)
\end{align}
\end{proof}

The curried representation of Lemma \ref{th:table_rep} does not need to support an interpretation as a distribution of potential outcomes. For example, consider a series of bets on fair coinflips -- in this case, the consequence $\RV{Y}_i$ is uniform on $\{0,1\}$ for any decision $\RV{D}_i$. Tha $D=Y=\{0,1\}$ and $\prob{P}_\alpha^{\RV{Y}_n}(y)=\prod_{i\in [n]} 0.5$ for all $n$, $y\in Y^n$, $\alpha\in R$. Then the construction in Lemma \ref{th:table_rep} yields $\prob{P}^{Y^D_i}(y^D_i)=\prod_{j\in D} 0.5$ for all $y^D_i\in Y^D$. That is, $\RV{Y}^0_i$ and $\RV{Y}^1_i$ are independent and uniformly distributed. However, if we wanted $\RV{Y}^0_i$ to represent ``what would happen if I bet on outcome 0 on turn $i$'' and $\RV{Y}^1$ to represent ``what would happen if I bet on outcome 1 on turn $i$'', then it seems that we ought to have $\RV{Y}^0_i = 1-\RV{Y}^1_i$. 

We could suppose that Lemma \ref{th:table_rep} provides necessary but not sufficient conditions for the existence of a potential outcomes representation of a conditional probability model. However, it doesn't seem to succeed at that either. We note, for example, that \citet{rubin_causal_2005} does not assume that the distribution of potential outcomes is exchangeable. A non-exchangeable $\prob{P}^{\RV{Y}^D}$ does not induce a causally contractible conditional probability model, and at the same time commutativity with marginalisation is not sufficient for a conditional probability model to support a curried representation in the sense of Lemma \ref{th:table_rep}. What seems to be missing is an additional assumption that consequences are mutually independent of one another given the associated decision. 

We can also represent contractible conditional probability models repeated copies of an unknown ``consequence function'', a Markov kernel that maps from decisions to probability distributions over consequences, coupled by a common hypothesis $\RV{H}$. 

\begin{theorem}\label{th:iid_rep}
Suppose we have a fundamental probability set $\Omega$ and a do model $(\prob{P},\RV{D},\RV{Y},R)$ such that $\RV{D}:=(\RV{D}_i)_{i\in \mathbb{N}}$ and $\RV{Y}:=(\RV{Y}_i)_{i\in\mathbb{N}}$. $\prob{P}$ is causally contractible if and only if there exists some $\RV{H}:\Omega\to H$ such that $\prob{P}^{\RV{Y}_i|\RV{H}\RV{D}_i}$ exists for all $i\in \mathbb{N}$ and
\begin{align}
    \prob{P}^{\RV{Y}|\RV{H}\RV{D}} &= \tikzfig{do_model_representation}\\
    &\iff\\
    \RV{Y}_i&\CI_{\prob{P}} \RV{Y}_{\mathbb{N}\setminus i},\RV{D}_{\mathbb{N}\setminus i}|\RV{H}\RV{D}_i&\forall i\in \mathbb{N}\\
    \land \prob{P}^{\RV{Y}_i|\RV{H}\RV{D}_i} &= \prob{P}^{\RV{Y}_0|\RV{H}\RV{D}_0} & \forall i\in \mathbb{N}
\end{align}
\end{theorem}

\begin{proof}
We make use of Lemma \ref{th:table_rep} to show that we can represent the conditional probability as an exchangeable tabular probability distribution. We then use the property of exchangeability of the columns of that distribution in conjunction with De Finetti's theorem to derive the result.
\end{proof}

\subsection{In decison problems, policies are extreme}

Theorem \ref{th:iid_rep} shows that a formal property of conditional probability models -- causal contractibility -- implies the existence of a map $D\times H\kto Y$ that, when coupled with a distribution over $H$, yields the original model. We have also suggested that causal contractibility might be a property of a model of an experiment where we expect that experimental outcomes will be the same if we shuffle the planned experimental actions and apply the inverse shuffle to results, and also if we believe that outcomes for units of interest will be unchanged if we substitute the experimental plan for any other plan that prescribes the same actions for the same units.

One could be inclined to leave things there and say that the question of whether causal contractibility holds is up to the judgement of the practitioner. However, we believe that making this judgement is a subtle issue that deserves further commentary.

The subtlety is illustrated by comparing the following three situations:

\begin{enumerate}
    \item I am considering different policies for treating 100 patients for lower back pain, and want to predict the outcomes of each policy
    \item I am treating 50 patients for lower back pain, not necessarily according to a fixed policy, and subsequently want to predict the outcomes of 50 additional patients treated according to a fixed policy
\end{enumerate}

The two situations seem similar - in all cases, I treat 100 patients for lower back pain and I want to predict some outcomes. The only difference is the presence or absence of a fixed policy.

If we accept for the sake of argument that we are going to use probability models to predict these outcomes, it seems reasonable that a conditional probability model for the first situation should be causally contractible. I have no means of distinguishing one ordering of patients from the other, and so it seems that I should use the same probability to predict outcomes for any ordering. Furthermore, the treatment of each patient seems like it is a separate matter -- there's no reason to expect patient $i$'s outcome to differ if I give different treatments to the rest of the patients.

Causal contractibility implies that, given enough patients and any policy with full support, I will converge to a single Markov kernel representing the response of any patient to a treatment, which will be equal to the conditional probability of recovery given treatment. On the other hand, it well understood that I should not always assume that the conditional probability of recovery given treatment is a good guide to my selection of policy. In particular, it would be unwise to estimate the probability of recovery given treatment for the first 50 patients in the second example and assume this is the same as the conditional probability of recovery given treatment for the second 50 patients.

The question is: what is being accomplished by the use of a policy that distinguishes situation 1 from 2? The answer, we think, is that the word ``policy'' implies more than just ``a probability distribution associated with a variable something that we happen to call a decision''. In particular, suppose policy $\alpha_1$ is to always choose decision $1$ and policy $\alpha_2$ is to always choose decision $2$. Choosing a mixed policy $0.5\alpha_1 + 0.5\alpha_2$ does not mean ``I choose to be uncertain by degree $a$ over which decision I will choose'', it means ``I will consult a random number generator that I know to yield $1$ half of the time and $2$ the other half, then choose $\alpha_i$ according to the result''. 



We can use probability gap models for many things. Maybe we want $\prob{P}_\alpha$ on Monday we do some work to figure out $\prob{P}_\square^{\RV{Y}|\RV{X}}$ and on Tuesday we were going to do a bit more to work out $\prob{P}_\alpha^{\RV{X}}$, but actually we found the answer to our question would be the same in any case. Whether or not we explicitly set it up, we've made use of a conditional probability model to come to this conclusion.

However, we actually want to model decision problems. For a decision problem we compare different policies $\alpha$ and choose the best one according to some criterion. We choose decisions according to the policy we arrive at. Decisions can be represented by variables, but they have a very important property: they must be deterministic for any policy choice $\alpha$.

\begin{definition}[Decision variable]
Given a probability gap model $(\prob{P}_\square, A)$ on $\Omega$, a variable $\RV{D}$ is a \emph{decision variable} if there is some variable $\RV{R}$ which we call \emph{purely random} such that, for $\alpha\in A$, $\prob{P}_\alpha^{\RV{D}|\RV{R}}$ is deterministic.
\end{definition}

We don't attempt to define what a ``purely random'' variable is here.

Because $\RV{D}$ is deterministic given $\RV{R}$, we have $\RV{Y}\CI_{\prob{P}_\square}\RV{R}|\RV{D}$, and so we can marginalise over $\RV{R}$ and talk about policies as if they address marginal probabilities over $\RV{D}$: $\prob{P}_\alpha^{\RV{D}}$.

The reason we include this definition is that it constrains the measurement procedures that are allowed to be associated with decision variables. In particular, all such measurement procedures must, with probability 1, yield a function of the policy choice $\alpha$ and a purely random measurement procedure $\proc{R}$. In particular, this means that if I talk about a ``mixed policy'' $a\alpha+(1-a)\beta$ I mean: consult a purely random process and, if the result is consistent with $a$ act according to $\alpha$, while if the result is consistent with $(1-a)$ act according to $\beta$. This is what people usually mean when they talk about mixing policies, and we think this is important enough to bake into the theory rather than leave it as an implicit side constraint.

This is quite important, because it helps us to distinguish between these two apparently similar situations:

\begin{enumerate}
    \item Some doctor is going to treat 49 (otherwise unknown) patients for lower back pain, then I will treat one patient for the same
    \item I am preparing to treat 50 (otherwise unknown) patients for lower back pain
\end{enumerate}

In the second case, because decisions are deterministic, I don't learn anything about the patient when I treat them or not. In the first case, because the other doctor's decisions are uncertain from my point of view, their treatment decisions can inform me about the patient \emph{and} determine whether the patient takes the medicine or not.

If we go ahead and set this up formally, we could use probability gap models for both; 

In the first case, we know it is unwise to assume that the result of giving patient 50 treatment $x$ will be probabilistically the same as the results experienced by all previous patients who received $x$. On the other hand, in the second case, it seems reasonable to expect the result of giving treatment $x$ to patient 50 is, a priori, the same as giving treatment $x$ to any other patient. That is, causal contractibility seems plausible if we imagine ourselves in a situation of planning a sequence of treatments, but a rather similar assumption seems unwise if we imagine the treatments having already taken place.

There is a subtle asymmetry between these situations: while ``patient $i$ receives treatment $x$'' is a sensible proposition for both situations, \emph{it corresponds to a different measurement procedure}.  