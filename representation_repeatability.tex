%!TEX root = main.tex

\section{When do response functions exist?}\label{sec:response_functions}

We model decision problems with probability sets $\prob{P}_C$ for some set of choices $C$. If we have a pair of variables $\RV{X}$ and $\RV{Y}$ such that the uniform conditional $\prob{P}_C^{\RV{Y}|\RV{X}}$ exists (Definition \ref{def:cprob_pset}), then the joint outcome $\prob{P}_\alpha^{\RV{XY}}$ of any choice $\alpha\in C$ can be computed from the marginal distribution $\prob{P}_\alpha^{\RV{X}}$ alone.

We're interested in models that feature a particular kind of ``causal effect'' that we call a \emph{conditionally independent and identical response functions}, or just ``response functions'' for short\todo{is that OK?}. They are a causal analogue of conditionally independent and identical sequences of random variables. Concretely, a model with response functions is a probability set $\prob{P}_C$ with variables $\RV{Y}:=(\RV{Y}_i)_{i\in M}$, $(\RV{X}_i)_{i\in M}$ for some index set $M$ and some $\RV{H}$ such that $\RV{Y} \CI_{\prob{P}_C}^e C|\RV{X}_i\RV{H}$ and $\RV{H}\CI_{\prob{P}_C}^e \RV{X}_i C$ (see Section \ref{sec:eci} for extended conditional independence) and $\prob{P}_C^{\RV{Y}_i|\RV{X}_i\RV{H}}=\prob{P}_C^{\RV{Y}_j|\RV{X}_j\RV{H}}$ for all $i,j\in M$ (the identical response conditional requirement).

We will focus on the case where $\prob{P}_\alpha^{\RV{H}|\RV{Y}_A\RV{X}_A}$ approaches a deterministic distribution as $|A|\to \infty$, for appropriate $\alpha\in C$. We could say this is the case of ``identifiable'' response conditionals.

Put together, these conditions say: in the limit of infinite samples under an appropriate sampling regime $\alpha\in C$, the model converges to a probabilistic function $X\kto Y$ that represents ``the probability of $\RV{Y}_i$ given $\RV{X}_i$'' for any unobserved $(\RV{X}_i,\RV{Y}_i)$. For arbitrary $\alpha'\in C$, we may instead converge to a set containing this limiting distribution. We think -- although it usually isn't stated in these terms -- that given a causal Bayesian network, if the function ``$x\mapsto \prob{P}(\RV{Y}|\mathrm{do}(\RV{X}=x))$'' is identifiable, then it is an instance of the kind of function that we described in the previous sentences.

We prove our result under the simplifying assumption that $\RV{X}_i\CI_{\prob{P}_C}^e \RV{Y}_{<i} C|\RV{X}_{<i}$. This is a limiting assumption -- for example, it excludes cases where $\RV{X}_i$ depends on $(\RV{X}_{<i},\RV{Y}_{<i})$. In the more general case where this does not hold, the conditions we provide must still hold for any $C'\subset C$ such that $\RV{X}_i\CI_{\prob{P}_C}^e \RV{Y}_{<i>} C|\RV{X}_{<i}$, but providing sufficient conditions in this case is the topic of further work.

Under this assumption, we show that \emph{causal contractibility} is necessary and sufficient for the existence of response conditionals. Causal contractibility can be broken down into two sub-assumptions: \emph{exchange commutativity} and \emph{consequence locality}. The first is the assumption that the uniform conditional probability $\prob{P}_C^{\RV{Y}|\RV{X}}$ ``commutes'' with the permutation operation, and the second is the assumption that $\RV{X}_i$ ``has no effect'' on any $\RV{Y}_j$ for $j\neq i$.

\subsection{Relevance to previous work}

Both sub-assumptions have precedent in existing literature, but these precedents tend to have been stated at somewhat informally.

\emph{Post-treatment exchangeability} found in \citet{dawid_decision-theoretic_2020} is implied by exchange commutativity, but not the reverse. ``Causal exchangeability'' notions are also found in \citet{greenland_identifiability_1986} and \citet{banerjee_chapter_2017}; a subtle difference between these notions and exchange commutativity is that these latter notions are given as symmetries of \emph{decision procedures} -- they involve actually swapping actions taken or individuals in an experiment -- while exchange commutativity is a symmetry of probability sets.

Consequence locality is similar to the stable unit treatment distribution assumption (SUTDA) in \citet{dawid_decision-theoretic_2020}, although consequence locality is distinguished by being a concrete extended conditional independence (Definition \ref{def:caus_cont}) while SUTDA is given as the assumption that $\RV{Y}_i$ ``depends only on'' $\RV{X}_i$.  Consequence locality is also similar to stable unit treatment value assumption (SUTVA). The stable unit treatment value assumption (SUTVA) is given as \citep{rubin_causal_2005}:

\begin{blockquote}
(SUTVA) comprises two sub-assumptions. First, it assumes that \emph{there is no interference between units (Cox 1958)}; that is, neither $Y_i(1)$ nor $Y_i(0)$ is affected by what action any other unit received. Second, it assumes that \emph{there are no hidden versions of treatments}; no matter how unit $i$ received treatment $1$, the outcome that would be observed would be $Y_i(1)$ and similarly for treatment $0$.
\end{blockquote}

String diagram statements of both sub-assumptions can give some intuition about what they mean. For clarity, these diagrams illustrate the assumptions with exactly two inputs and two outputs, while the general definitions are for any countable number of inputs and outputs.

Exchange commutativity for two inputs and outputs is given by the following equality:

\begin{align}
    \tikzfig{commutativity_of_exchange}
\end{align}

While consequence locality for two inputs and outputs is given by the following pair of equalities:

\begin{align}
    \tikzfig{cons_locality_1}\\
    \tikzfig{cons_locality_2}
\end{align}

% Another way to see where we are going is to consider graphical statements of our and De Finetti's result.

% Take $S=\{0,1\}$ and identify the space $\Delta(S)$ of probability measures on $S$ with the interval $[0,1]$. De Finetti showed that any infinite exchangeable probability measure $\prob{P}_\alpha$ on $\{0,1\}^\mathbb{N}$ can be represented by a prior $\prob{P}_\alpha^{\RV{H}}\in [0,1]$ for some $\RV{H}:\Omega\to H$ and a conditional probability $\prob{P}^{\RV{S}_0|\RV{H}}:[0,1]\kto \{0,1\}$ such that

% \begin{align}
%     \prob{P}_\alpha &= \tikzfig{de_finetti_rep0}\label{eq:definettirep}
% \end{align}

% Here $\prob{P}^{\RV{S}_0|\RV{H}}$ can be defined concretely by $\prob{P}^{\RV{S}_0|\RV{H}}(1|h)=h$. Equivalently, the probability gap model on $S^\mathbb{N}$ defined by the assumption of exchangeability is equivalent to the probability gap model defined by the conditional probability

% \begin{align}
%     \prob{P}^{\RV{S}|\RV{H}} = \tikzfig{de_finetti_conditional}
% \end{align}

% That is, there is some hypothesis $\RV{H}$ and conditional on $\RV{H}$ the measurements are independent and identically distributed. The proof of this is constructive -- $\RV{H}$ is a function of $\RV{S}$.



% \begin{align}
%     \prob{P}^{\RV{Y}|\RV{HD}} = \tikzfig{do_model_representation}
% \end{align}

% We will further argue that the class of see-do models considered in CBN and potential outcomes literature is equivalent to the family of causally contractible and exchangeable do-models where the decision rule for the first $n$ places is fixed to an unknown value, and may be freely chosen thereafter.

% \begin{theorem}[Existence of conditional in do models]
% Given a do model $(\prob{P}_{\square}^{\RV{Y}\|\RV{D}},R)$, for all $\alpha\in R$, $n\in\mathbb{N}$
% \begin{align}
%     \prob{P}_\alpha^{\RV{Y}_{[n]}\RV{D}_i} = \prob{P}_\alpha^{\RV{D}_{[n]}}\odot \prob{P}_\square^{\RV{Y}_{[n]}\|\RV{D}_{[n]}}
% \end{align}
% That is, $\prob{P}_\square^{\RV{Y}_{[n]}\|\RV{D}_{[n]}}\cong \prob{P}_\square^{\RV{Y}_{[n]}|\RV{D}_{[n]}}$
% \end{theorem}

% \begin{proof}
% For any $n>1\in \mathbb{N}$, $\alpha\in R$

% \begin{align}
%     \prob{P}_\alpha^{\RV{Y}_{[n]}\RV{D}_{[n]}} &= \tikzfig{do_model_1}\\
%     &= \tikzfig{do_model_2}\\
%     &= \tikzfig{do_model_3}\\
%     &= \tikzfig{do_model_4}\\
%     \implies \prob{P}_\alpha^{\RV{Y}_{[n]}|\RV{D}_{[n]}} &= \tikzfig{do_model_5}\\
%     &= \prob{P}_\alpha^{\RV{Y}_{[n-1]}|\RV{D}_{[n-1]}}\combprod \prob{P}_\square^{\RV{Y}_n|\RV{Y}_{[n-1]}\RV{D}_n}
% \end{align}

% Applying this recursively with $\prob{P}_\alpha^{\RV{Y}_{[1]}|\RV{D}_{[1]}}=\prob{P}_\square^{\RV{Y}_{[1]}|\RV{D}_{[1]}}$ yields

% \begin{align}
%     \prob{P}_\alpha^{\RV{Y}_{[n]}|\RV{D}_{[n]}} = \prob{P}_\square^{\RV{Y}_{[n]}\|\RV{D}_{[n]}}
% \end{align}
% as desired.
% \end{proof}
\subsection{Causal contractibility}\label{sec:ccontracibility}

Here we set out formal definitions of exchange commutativity and locality of consequences, as well as ``consequence contractibility'', which is the conjunction of both conditions.

\begin{definition}[Locality of consequences]\label{def:caus_cont}
Suppose we have a sample space $(\Omega,\sigalg{F})$ and a probability set $\prob{P}_C$  where $\RV{Y}:=\RV{Y}:=(\RV{Y}_i)_{M}$, $\RV{D}:=\RV{D}_M:=(\RV{D}_i)_M$, $M\subseteq \mathbb{N}$. If for any $A\subset M$, $\RV{Y}_A\CI^e_{\prob{P}_C} \RV{D}_{A^C} C |\RV{D}_A$ then $\prob{P}_C$ exhibits $(\RV{D};\RV{Y})$-\emph{local consequences}.
\end{definition}

If $\prob{P}_C$ exhibits $(\RV{D};\RV{Y})$-local consequences then, given two different choices $\alpha$ and $\alpha'$ such that $\prob{P}_\alpha^{\RV{D}_A}=\prob{P}_{\alpha'}^{\RV{D}_A}$ then $\prob{P}_\alpha^{\RV{Y}_A}=\prob{P}_{\alpha'}^{\RV{Y}_A}$. However, $\prob{P}_C$ may exhibit consequence locality even if no such pair of choices exists.

Note that consequence locality implies $\RV{Y}_M \CI^e_{\prob{P}_C} C |\RV{D}_M$, and hence we have the uniform conditional $\prob{P}_C^{\RV{Y}_M|\RV{D}_M}$. We assume the existence of such a conditional for the next definition.

\begin{definition}[Swap map]
Given $M\subset \mathbb{N}$ a finite permutation $\rho:M\to M$ and a variable $\RV{X}:\Omega\to X^M$ such that $\RV{X}=(\RV{X}_i)_{i\in M}$, define the Markov kernel $\text{swap}_{\rho(\RV{X})}:X^M\kto X^M$ by $(d_i)_{i\in\mathbb{N}}\mapsto \delta_{(d_{\rho(i)})_{i\in\mathbb{N}}}$.
\end{definition}

\begin{definition}[Exchange commutativity]\label{def:caus_exch}
Suppose we have a sample space $(\Omega,\sigalg{F})$ and a probability set $\prob{P}_C$ with uniform conditional probability $\prob{P}_C^{\RV{Y}|\RV{D}}$ where $\RV{Y}:=\RV{Y}:=(\RV{Y}_i)_{M}$, $\RV{D}:=\RV{D}_M:=(\RV{D}_i)_M$, $M\subseteq \mathbb{N}$. If for any finite permutation $\rho:M\to M$
\begin{align}
    \text{swap}_{\rho(\RV{D})} \prob{P}_{C}^{\RV{Y}|\RV{D}} &\overset{\prob{P}_C}{\cong} \prob{P}_{C}^{\RV{Y}|\RV{D}}\text{swap}_{\rho(\RV{Y})}
\end{align}
Then $\prob{P}_C^{\RV{Y}|\RV{D}}$ is $(\RV{D};\RV{Y})$-\emph{exchange commutative}.
\end{definition}

If $\prob{P}_C$ is $(\RV{D};\RV{Y})$-exchange commutative and we have $\alpha,\alpha'\in C$ such that $\prob{P}_\alpha^{\RV{C}} = \prob{P}_{\alpha'}^{\RV{C}}\text{swap}_{\rho(\RV{D})}$, then $\prob{P}_\alpha^{\RV{Y}} = \prob{P}_{\alpha'}^{\RV{Y}}\text{swap}_{\rho(\RV{Y})}$. However, $\prob{P}_C$ may commute with exchange even if there are no such $\alpha$ and $\alpha'\in C$.

Theorem \ref{th:no_implication} shows that neither condition implies the other. 

\begin{theorem}\label{th:no_implication}
Exchange commutativity does not imply locality of consequences or vise versa.
\end{theorem}

\begin{proof}
Appendix \ref{sec:ccontract_appendix}.
\end{proof}

If we are modelling the treatment of several patients whom who have already been examined, we might assume consequence locality -- patient B's treatment does not affect patient A -- but not exchange commutativity -- we don't expect the same results from giving patient A's treatment to patient B as we would from giving patient A's treatment to patient A. 

A model of stimulus payments might exhibit exchange commutativity but not consequence locality. If exactly $n$ payments of \$10 000 are made, we might suppose that it doesn't matter much exactly who receives the payments, but the amount of inflation induced depends on the number of payments made; making 100 such payments will have a negligible effect on inflation, while making payments to everyone in the country will have a substantial effect. \citet{dawid_causal_2000} offers the example of herd immunity in vaccination campaigns as a situation where post-treatment exchangeability holds but locality of consequences does not.

Although locality of consequences seems to intuitively encompass an assumption of non-interference, it still allows for some models in which exhibit certain kinds of interference between actions and outcomes of different indices. For example: I have an experiment where I first flip a coin and record the results of this flip as the outcome of the first step of the experiment, but I can choose either to record this same outcome as the provisional result of the second step (this is the choice $\RV{D}_1=0$), or choose to flip a second coin and record the result of that as the provisional result of the second step of the experiment (this is the choice $\RV{D}_1=1$). At the second step, I may further choose to copy the provisional results ($\RV{D}_2=0$) or invert them ($\RV{D}_2=1$). Then

\begin{align}
    \prob{P}_S^{\RV{Y}_1|\RV{D}}(y_1|d_1,d_2) &= 0.5\\
    \prob{P}_S^{\RV{Y}_2|\RV{D}}(y_2|d_1,d_2) &= 0.5
\end{align}
\begin{itemize}
    \item The marginal distribution of both experiments in isolation is $\text{Bernoulli}(0.5)$ no matter what choices I make, so a model of this experiment would satisfies Definition \ref{def:caus_cont}
    \item Nevertheless, the choice for the first experiment affects the result of the second experiment
\end{itemize}

We call the conjunction of exchange commutativity and consequence locality \emph{causal contractibility}.

\begin{definition}[Causal contractibility]
A probability set $\prob{P}_C$ is $(\RV{D};\RV{Y})$-\emph{causally contractible} if it is both exchange commutative and exhibits consequence locality.
\end{definition}

\begin{theorem}[Equality of reduced conditionals]\label{th:equal_of_condits}
A probability set $\prob{P}_C$ that is $(\RV{D};\RV{Y})$-causally contractible has, for any $A,B\subset M$ with $|A|=|B|$
\begin{align}
    \prob{P}_C^{\RV{Y}_A|\RV{D}_A} \overset{\prob{P}_C}{\cong} \prob{P}_C^{\RV{Y}_B|\RV{D}_B}
\end{align}
\end{theorem}

\begin{proof}
Only if:
For any $A,B\subset M$, let $\text{s}_{BA}:D^M\kto D^M$ be the swap map that sends the $B$ indices to $A$ indices and $\text{s}_{AB}:Y^M\kto Y^M$ be the swap map that sends $A$ indices to $B$ indices.
\begin{align}
    \tikzfig{consequence_locality} &= \tikzfig{contractibility_from_exchange_1}\\
    &= \tikzfig{contractibility_from_exchange_2}\\
    &= \tikzfig{contractibility_from_exchange_3}
\end{align}
Thus
\begin{align}
    \prob{P}_C^{\RV{Y}_A|\RV{D}_A\RV{D}_{M\setminus A}} &\overset{\prob{P}_C}{\cong} \prob{P}_C^{\RV{Y}_B|\RV{D}_B\RV{D}_{M\setminus B}}\\
    &\overset{\prob{P}_C}{\cong} \tikzfig{consequence_locality}\label{eq:cons_local}\\
    \implies \prob{P}_C^{\RV{Y}_A|\RV{D}_A} \overset{\prob{P}_C}{\cong} \prob{P}_C^{\RV{Y}_B|\RV{D}_B}
\end{align}
\end{proof}



% \begin{proposition}[Representation of do-models that commute with exchange]
% Suppose we have a fundamental probability set $\Omega$ and a do model $(\prob{P},\RV{D},\RV{Y},R)$ such that $\RV{D}:=(\RV{D}_i)_{i\in \mathbb{N}}$ and $\RV{Y}:=(\RV{Y}_i)_{i\in\mathbb{N}}$ where $\prob{P}$ commutes with exchange and there is some $\alpha^*\in R$ such that $\prob{P}^{\alpha^*}\gg\prob{P}_\beta$ for all $\beta in R$. Then there exists a symmetric function $\RV{H}:(Y\times D)^\mathbb{N}\to H$ such that  $\prob{P}^{\RV{Y}|\RV{DH}}$ exists and $\RV{Y}_i\CI_{\prob{P}}(\RV{D}_j,\RV{Y}_j)_{j\in \mathbb{N}}\setminus \{i\}|\RV{H}\RV{D}_i$, or equivalently 
% \begin{align}
%     \prob{P}^{\RV{Y}} &= \tikzfig{do_model_representation}
% \end{align}
% \end{proposition}

% % \begin{lemma}[Contraction and independence]
% % Let $\RV{J}$, $\RV{K}$ and $\RV{L}$ be variables on $\Omega$ and $\prob{Q}\in \Delta(\Omega)$ a base measure such that $\prob{Q}^{\RV{JK}}=\prob{Q}^{\RV{JL}}$ and $\sigma{K}\subset \sigma{L}$. Then $\RV{J}\CI\RV{L}|\RV{K}$. 
% % \end{lemma}

% % \begin{proof}
% % From Lemma 1.3 in \citet{kallenberg_basic_2005}.
% % \end{proof}

% \begin{proof}
% If $\prob{P}$ commutes with exchange, then for any $\alpha\in R$ such that $\prob{P}_\alpha^{\RV{D}}$ is exchangeable then $\prob{P}_\alpha$ is also exchangeable. Then there exists $\RV{H}$ a symmetric function of $(\RV{Y}_i,\RV{D}_i)_{i\in\mathbb{N}}$ such that $\RV{Y}_i\CI_{\prob{P}}(\RV{D}_j,\RV{Y}_j)_{j\in \mathbb{N}}\setminus \{i\}|\RV{H}\RV{D}_i$. This is De Finetti's representation theorem, and many proofs exists, see for example \citep{kallenberg_basic_2005}.

% In particular, let 

% \begin{align}
%     \RV{H}:=A\times B\mapsto \lim_{n\to\infty} \frac{1}{n}\sum_{i\in n} \mathds{1}_{A\times B}((\RV{Y}_i, \RV{D}_i))
% \end{align}

% Then for all $\alpha\in R$,
% \begin{align}
%     \prob{P}_\alpha^{(\RV{Y}_i,\RV{D}_i)_{i\in\mathbb{N}}|\RV{H}}(A\times B|h) \overset{a.s.}{=} h(A\times B)\label{eq:given_h}
% \end{align}

% The proof that the limit exists and the above equality holds can again be found int \citep{kallenberg_basic_2005}.
% \end{proof}

\subsection{Existence of response conditionals}

The main result in this section is Theorem \ref{th:iid_rep} which shows that a probability set $\prob{P}_C$ is causally contractible if and only if it can be represented as the product of a distribution over hypotheses $\prob{P}_\square^{\RV{H}}$ and a collection of identical uniform conditionals $\prob{P}_C^{\RV{Y}_1|\RV{D}_1\RV{H}}$. Note the hypothesis $\RV{H}$ that appears in this conditional; it can be given the interpretation of a random variable that expresses the ``true but initially unknown'' $\RV{Y}_1|\RV{D}_1$ conditional probability.

\begin{theorem}\label{th:table_rep}
Given a probability set $\prob{P}_C$ and variables $\RV{D}:=(\RV{D}_i)_{i\in \mathbb{N}}$ and $\RV{Y}:=(\RV{Y}_i)_{i\in \mathbb{N}}$, $\prob{P}_C$ is  $(\RV{D};\RV{Y})$-causally contractible if and only if there exists a column exchangeable probability distribution $\mu^{\RV{Y}^D}\in \Delta(Y^{|D|\times \mathbb{N}})$ such that
\begin{align}
    \prob{P}_C^{\RV{Y}|\RV{D}} &= \tikzfig{lookup_representation}\label{eq:lup_rep}\\
    &\iff\\
    \prob{P}_C^{\RV{Y}|\RV{D}}(y|(d_i)_{i\in \mathbb{N}}) &= \mu^{\RV{Y}^D} \Pi_{(d_i i)_{i\in\mathbb{N}}}(y)
\end{align}
Where $\Pi_{(d_i i)_{i\in\mathbb{N}}}:Y^{|D|\times \mathbb{N}}\to Y^{\mathbb{N}}$ is the function that projects the $(d_i,i)$ indices for all $i\in \mathbb{N}$ and $\prob{F}_{\text{ev}}$ is the Markov kernel associated with the evaluation map
\begin{align}
    \text{ev}:D^\mathbb{N}\times Y^{D\times \mathbb{N}}&\to Y\\
    ((d_i)_\mathbb{N},(y_{ij})_{i\in D,j\in \mathbb{N}})&\mapsto (y_{d_i i})_{i\in \mathbb{N}}
\end{align}
\end{theorem}

\begin{proof}
Appendix \ref{sec:ccontract_appendix}.
\end{proof}

We would prefer to talk about $\RV{Y}^D$ as a latent variable, rather than needing to refer to the the factorisation of a model in terms of $\mu^{\RV{Y}^D}$ in Equation \ref{eq:lup_rep}. This motivates the definition of an \emph{augmented} causally contractible model.

\begin{lemma}[Augmented causally contractible model]\label{lem:aug_cc}
Given a $(\RV{D};\RV{Y})$-causally contractible model $\prob{P}_C'$ on $(\Omega',\sigalg{F}')$, there exists an \emph{augmented} model $\prob{P}_C$ on $(\Omega,\sigalg{F}):=((\Omega'\times Y^D,\sigalg{F}'\otimes\sigalg{Y}^D)$ such that $\prob{P}_C\Pi_{\Omega'}=\prob{P}_C'$ and, defining $\RV{Y}^D:\Omega\times Y^D\to Y^D$ as the projection onto $Y^D$
\begin{align}
    \prob{P}_C^{\RV{Y}|\RV{D}} &= \tikzfig{lookup_representation_variablised}\label{eq:lup_rep_varb}
\end{align}
\end{lemma}

\begin{proof}
Appendix \ref{sec:ccontract_appendix}.
\end{proof}

An augmented causally contractible model looks in some respects similar to a potential outcomes model - both have a distribution over an unobserved ``tabular'' variable $\RV{Y}^D$, and the value of $\RV{Y}_i$ given $\RV{D}$ is deterministically equal to the $\RV{Y}_i^\RV{D}$ (abusing notation). However, the $\RV{Y}^D$ in an augmented causally contractible model usually can't be interpreted as potential outcomes. For example, consider a series of bets on fair coin flips. Model the consequence $\RV{Y}_i$ as uniform on $\{0,1\}$ for any decision $\RV{D}_i$, for all $i$. Specifically, $D=Y=\{0,1\}$ and $\prob{P}_\alpha^{\RV{Y}_n}(y)=\prod_{i\in [n]} 0.5$ for all $n$, $y\in Y^n$, $\alpha\in R$. Then the construction of $\prob{P}^{\RV{Y}^D}$ following the method in Lemma \ref{th:table_rep} yields $\prob{P}^{Y^D_i}(y^D_i)=\prod_{j\in D} 0.5$ for all $y^D_i\in Y^D$. In this model $\RV{Y}^0_i$ and $\RV{Y}^1_i$ are independent and uniformly distributed. However, if we wanted $\RV{Y}^0_i$ to be interpretable as ``what would happen if I bet on outcome 0 on turn $i$'' and $\RV{Y}^1$ to represent ``what would happen if I bet on outcome 1 on turn $i$'', then we ought to have $\RV{Y}^0_i = 1-\RV{Y}^1_i$.

The following is the main theorem of this section, that establishes the equivalence between causal contractibility and the existence of response conditionals. The argument in outline is: because $\prob{P}_C^{\RV{Y}^D}$ is a column exchangeable probability distribution we can apply De Finetti's theorem to show $\prob{P}_C^{\RV{Y}^D}$ is representable as a product of identical parallel copies of $\prob{P}_C^{\RV{Y}_1^D|\RV{H}}$ and a common prior $\prob{P}_C^{\RV{H}}$. This in turn can be used to show that $\prob{P}_C^{\RV{Y}|\RV{D}}$ can be represented as a product of identical parallel copies of $\prob{P}_C^{\RV{Y}_1|\RV{D}_1\RV{H}}$ and the same common prior $\prob{P}_C^{\RV{H}}$.

\begin{theorem}\label{th:iid_rep}
Suppose we have a sample space $(\Omega,\sigalg{F})$ and a probability set $\prob{P}_C$ and variables $\RV{D}:=(\RV{D}_i)_{i\in \mathbb{N}}$ and $\RV{Y}:=(\RV{Y}_i)_{i\in \mathbb{N}}$. Suppose also that  $\prob{P}_C$ is $(\RV{D};\RV{Y})$-causally contractible if and only if there exists some $\RV{H}:\Omega\to H$ such that $\prob{P}^{\RV{H}}_C$ and $\prob{P}_C^{\RV{Y}_i|\RV{H}\RV{D}_i}$ exist for all $i\in \mathbb{N}$ and
\begin{align}
    \prob{P}_C^{\RV{Y}|\RV{D}} &= \tikzfig{do_model_representation}\\
    &\iff\\
    \RV{Y}_i&\CI^e_{\prob{P}_C} \RV{Y}_{\mathbb{N}\setminus i},\RV{D}_{\mathbb{N}\setminus i}C|\RV{H}\RV{D}_i&\forall i\in \mathbb{N}\\
    \land \RV{H} &\CI^e_{\prob{P}_C} \RV{D} C\\
    \land \RV{D}_i&\CI_{\prob{P}_C}^e \RV{Y}_{<i>} C|\RV{D}_{<i}\\
    \land \prob{P}_C^{\RV{Y}_i|\RV{H}\RV{D}_i} &= \prob{P}^{\RV{Y}_0|\RV{H}\RV{D}_0} & \forall i\in \mathbb{N}
\end{align}
Where $\Pi_{D,i}:D^\mathbb{N}\kto D$ is the $i$th projection map.
\end{theorem}

\begin{proof}
Appendix \ref{sec:ccontract_appendix}.
\end{proof}

\subsection{Elaborations and examples}

Theorem \ref{th:iid_rep} requires an infinite sequence of causally contractible pairs. In practice we only want to model finite sequences of variables, but this theorem applies as long as it is possible to extend the finite model to an infinite model maintaining causal contractibility.


Theorem \ref{th:iid_rep} applies whatever procedure we use to obtain the $(\RV{D}_i,\RV{Y}_i)$ pairs -- the $\RV{D}_i$s may be randomised, passive observations or active choices. Purely passive observations can be modeled with a probability set of size 1, and in this case an exchangeable sequence of $(\RV{D}_i,\RV{Y}_i)$ will also be causally contractible.

If we are modelling $M$ passive observations followed by $N$ active choices, then we will have a model $\prob{P}_C$ with $\RV{D}_{[M]}\CI_{\prob{P}_C}^e C$ (because these are passive observations). If this model is $(\RV{D};\RV{Y})$-causally contractible, then one consequence of this is an ``observational imitation'' condition: any choice $\alpha$ that makes $\prob{P}_\alpha^{\RV{D}_{[M+N]}}$ exchangeable also makes $\prob{P}_\alpha^{\RV{Y}_{[M+N]}}$ exchangeable. That is, if for some permutation $\mathrm{swap}_\rho$
\begin{align}
    \prob{P}_\alpha^{\RV{D}_{[M+N]}}\mathrm{swap}_\rho &= \prob{P}_\alpha^{\RV{D}_{[M+N]}}
\end{align}
then by commutativity of exchange
\begin{align}
    \prob{P}_\alpha^{\RV{Y}_{[M+N]}} &= \prob{P}_\alpha^{\RV{D}_{[M+N]}} \prob{P}_C^{\RV{Y}_{[M+N]}|\RV{D}_{[M+N]}}\\
    &=  \prob{P}_\alpha^{\RV{D}_{[M+N]}}\mathrm{swap}_\rho \prob{P}_C^{\RV{Y}_{[M+N]}|\RV{D}_{[M+N]}}\\
    &= \prob{P}_\alpha^{\RV{D}_{[M+N]}} \prob{P}_C^{\RV{Y}_{[M+N]}|\RV{D}_{[M+N]}}\mathrm{swap}_\rho\\
    &= \prob{P}_\alpha^{\RV{Y}_{[M+N]}}\mathrm{swap}_\rho
\end{align}

If we assume a probability set $\prob{P}_C$ is $(\RV{D},\RV{X};\RV{Y})$-causally contractible and $\RV{X}_{i}\CI^e_{\prob{P}_C}\RV{D}_{i}C|\RV{H}$ -- that is, $\RV{D}_i$ is must be independent of $\RV{X}_i$ conditional on $\RV{H}$ -- then we get a version of the ``backdoor adjustment'' formula. Specifically
\begin{align}
    \prob{P}_\alpha^{\RV{Y}_{i}|\RV{D}_{i}\RV{H}}(A|d,h) &= \int_X \prob{P}_\alpha^{\RV{Y}_{i}|\RV{X}_{i}\RV{D}_{i}\RV{H}}(A|d,x,h)\prob{P}_\alpha^{\RV{X}_{i}|\RV{D}_{i}\RV{H}}(\mathrm{d}x|d,h)\\
    &= \int_X \prob{P}_C^{\RV{Y}_{1}|\RV{X}_{1}\RV{D}_{1}\RV{H}}(A|d,x,h)\prob{P}_C^{\RV{X}_{i}|\RV{H}}(\mathrm{d}x|h)
\end{align}

If we additionally assume $\prob{P}_C^{\RV{X}_{i}|\RV{H}}\cong \prob{P}_C^{\RV{X}_{1}|\RV{H}}$ then 
\begin{align}
    \prob{P}_\alpha^{\RV{Y}_{i}|\RV{D}_{i}\RV{H}}(A|d,h) &= \int_X \prob{P}_C^{\RV{Y}_{1}|\RV{X}_{1}\RV{D}_{1}\RV{H}}(A|d,x,h)\prob{P}_C^{\RV{X}_{1}|\RV{H}}(\mathrm{d}x|h)\label{eq:backdoor}
\end{align}

Equation \ref{eq:backdoor} is identical to the backdoor adjustment formula for an intervention on $\RV{D}_1$ targeting $\RV{Y}_1$ where $\RV{X}_1$ is a common cause of both.

While it is formally possible to use a causally contractible model for a decision procedure that involves both passive observations and active choices, causal contractibility is a very strong assumption. Suppose we have a decision procedure in which $M$ passive observations are made $(\RV{X}_M,\RV{Y}_M)$, followed by $M$ active choices $(\RV{X}_{(M,2M]},\RV{Y}_{(M,2M]})$. If a model $\prob{P}_C$ of this procedure is $(\RV{X}_{2M},\RV{Y}_{2M})$-causally contractible model then the following holds (see corollary \ref{th:equal_of_condits}):
\begin{align}
    \prob{P}^{\RV{Y}_{[2,M+1]}|\RV{X}_{[2,M+1]}}_C &= \prob{P}^{\RV{Y}_{(M,2M]}|\RV{X}_{(M,2M]}}\\
    \implies \prob{P}^{\RV{Y}_{M+1}|\RV{X}_{[2,M+1]}\RV{Y}_{[2,M]}}_C &= \prob{P}^{\RV{Y}_{M+1}|\RV{X}_{(M,2M]}\RV{Y}_{(M+1,2M]}}
\end{align}
That is, causal contractibility implies that there is no difference between conditioning on observational results or on the results of active choices; active choices are as good for predicting observations as vise-versa. Normally one might consider randomised experimental results to be ``better'' than passive observations, but this is not compatible with the assumption of causal contractibility.

\subsection{Assessing causal contractibility}\label{ssec:assessing}

We have a formal condition -- causal contractibility -- equivalent to the existence of response conditionals. A challenge we still need to face is to evaluate when a decision procedure is appropriately modeled with a probability set causally contractible with respect to some pair of variables.

Similar questions have been examined before, and a variety of answers have been given. These answers all say something like: the experiment consists of a sequence of similar or interchangeable ``individuals'' or ``units'', and that ``treatment assignments'' must be independent of the individual's identities, or may be swapped without swapping the individuals. For example, \citet{greenland_identifiability_1986} explain

\begin{quote}
    Equivalence of response type may be thought of in terms of exchangeability of individuals: if the exposure states of the two individuals had been exchanged, the same data distribution would have resulted.
\end{quote}

\citet{dawid_decision-theoretic_2020}:

\begin{quote}
    A group of individuals whom I can regard, in an intuitive sense, as similar to myself, with headaches similar to my own. [...] Whether or not the exchangeability assumption can be regarded as reasonable will be highly dependent on the background information informing my personal probability assessments
\end{quote}

\citet{rubin_causal_2005}:

\begin{quote}
    indexing of the units is, by definition, a random permutation of $1,..., N$, and thus any distribution on the science must be row-exchangeable [...] The second critical fact is that if the treatment assignment mechanism is ignorable (e.g., randomized), then when the expression for the assignment mechanism (2) is evaluated at the observed data, it is free of dependence on $Y_{mis}$
\end{quote}

Theorem \ref{th:cc_ind_treat} formalises these ideas. As an example of its application, consider an experiment where $N$ patients are treated, each with an individual identifier $\RV{I}_i$, who receives treatment $\RV{X}_i$ and experiences outcome $\RV{Y}_i$. We assume a $(\RV{X},\RV{I};\RV{Y})$-causally contractible model is appropriate. This reflects two judgments; firstly, that treatments and identifiers only affect their local consequences (Definition \ref{def:caus_cont}), and secondly we are indifferent the order in which the individuals appear and the treatments are received. We also assume that at the outset we are ignorant about any differences between each individual; specifically, permuting the individuals involved in the experiment leaves us with the same model.

Under these conditions, additionally assuming that treatments are independent of identifiers conditional on the hypothesis $\RV{H}$ implies that the model is also $(\RV{X};\RV{Y})$-causally contractible. To understand the meaning of this last assumption, consider that each identifier $a\in I$ is associated, in the large sample limit $h\in H$, with a stochastic ``individual response function'' $\kernel{K}: X\kto Y$ defined by

\begin{align}
    \kernel{K}_{a,h}(A|x) := \prob{P}_C^{\RV{Y}_i|\RV{X}_i\RV{I}_i\RV{H}}(A|x,a,h)
\end{align}

Then the assumption $\RV{I}_i\CI^e_{\prob{P}_C} \RV{X}_iC|\RV{H}$ means that $\RV{X}_i$ is uninformative about the individual response function $\kernel{K}_{\RV{I}_i,\RV{H}}$ (abusing notation). This is strongly reminiscent of the ``ignorability'' assumption found in the Potential Outcomes literature \citep{rubin_causal_2005}.

\begin{definition}[Reindexability]\label{def:ident_permutability}
Suppose we have a probability set $\prob{P}_C$, and variables $\RV{X}:=(\RV{X}_i)_{i\in \mathbb{N}}$, $\RV{I}:=(\RV{I}_i)_{i\in \mathbb{N}}$ and $\RV{Y}:=(\RV{Y}_i)_{i\in\mathbb{N}}$. For an arbitrary finite permutation $\rho:I\to I$, let $\kernel{F}_{\rho_I}:X^\mathbb{N}\times I^{\mathbb{N}}\kto X^\mathbb{N}\times I^{\mathbb{N}}$ be the Markov kernel associated with the function 
\begin{align}
    \rho_I:(x_i,n_i)_{i\in\mathbb{N}}\mapsto (x_i,\rho(n_i))_{i\in \mathbb{N}}
\end{align}
If for any $\rho$
\begin{align}
    \prob{P}_C^{\RV{Y}|\RV{XI}}&\overset{\prob{P}_C}{\cong} \kernel{F}_{\rho_I}\prob{P}_C^{\RV{Y}|\RV{XI}}
\end{align}
then $\prob{P}_C$ is $\RV{I}$-reindexable.
\end{definition}

\begin{theorem}[Causal contractibility with independent treatments]\label{th:cc_ind_treat}
Suppose we have a probability set $\prob{P}_C$, $(\RV{X},\RV{I};\RV{Y})$-causally contractible for $\RV{X}:=(\RV{X}_i)_{i\in \mathbb{N}}$, $\RV{I}:=(\RV{I}_i)_{i\in \mathbb{N}}$ and $\RV{Y}:=(\RV{Y}_i)_{i\in\mathbb{N}}$, $\RV{I}_i:\Omega\to I$ countable and $\RV{X}_i:\Omega\to X$, $\RV{Y}_i:\Omega\to Y$ standard measurable.

Suppose $\prob{P}_C$ is $\RV{I}$-reindexable and for all $i\in \mathbb{N}$, $\RV{I}_i\CI^e_{\prob{P}_C} \RV{X}_i C|\RV{H}$. 

Then $\prob{P}_C$ is $(\RV{X};\RV{Y})$-causally contractible.
\end{theorem}

\begin{proof}
By causal contractibility, we have for all $i\in |I|$
\begin{align}
    \prob{P}_C^{\RV{Y}_i|\RV{X}_i\RV{H}\RV{I}_{|I|}} &\overset{\prob{P}_C}{\cong} \prob{P}_C^{\RV{Y}_1|\RV{X}_1\RV{H}\RV{I}_1}\otimes \text{erase}_{I^{|I|-1}}
\end{align}

Furthermore, by assumption, for arbitrary permutation $\rho:I\to I$
\begin{align}
    \prob{P}_C^{\RV{Y}_i|\RV{X}_i\RV{H}\RV{I}_i}(A|x,h,n) &\overset{\prob{P}_C}{\cong} \prob{P}_C^{\RV{Y}_i|\RV{X}_i\RV{H}\RV{I}}(A|x,h,n,m) & \forall m\in I\\
     &\overset{\prob{P}_C}{\cong} (\kernel{F}_{\rho_I}\prob{P}_C^{\RV{Y}_i|\RV{X}_i\RV{H}\RV{I}_{|I|}})(A|x,h,n,m)\\
    &\overset{\prob{P}_C}{\cong} \prob{P}_C^{\RV{Y}_i|\RV{X}_i\RV{H}\RV{I}_{|I|}}(A|x,h,\rho(n,m))\\
    &\overset{\prob{P}_C}{\cong} \prob{P}_C^{\RV{Y}_1|\RV{X}_1\RV{H}\RV{I}_1}(A|x,h,\Pi_i(\rho(n,m)))
\end{align}

Where $\Pi_i$ projects the $i$-th coordinate of $\rho(n,m)$. Because $m$ and $\rho$ are arbitrary, this implies for any $p\in I$
\begin{align}
    \prob{P}_C^{\RV{Y}_i|\RV{X}_i\RV{H}\RV{I}_i}(A|x,h,n) &\overset{\prob{P}_C}{\cong} \prob{P}_C^{\RV{Y}_1|\RV{X}_1\RV{H}\RV{I}_1}(A|x,h,p))\\
    \implies \prob{P}_C^{\RV{Y}_i|\RV{X}_i\RV{H}\RV{I}_i} &\overset{\prob{P}_C}{\cong} \kernel{K}\otimes \text{erase}_I
\end{align}

where $\kernel{K}:(A|x,h)\mapsto \prob{P}_C^{\RV{Y}_i|\RV{X}_i\RV{H}\RV{I}_i}(A|x,h,q)$ for some $q\in I$. Therefore $\RV{Y}_i\CI^e_{\prob{P}_C}\RV{I}_iC|\RV{X}_i\RV{H}$.

Furthermore, by assumption, $\RV{I}_i\CI^e_{\prob{P}_C} \RV{X}_i C|\RV{H}$, and so by weak union $\RV{Y}_i\CI_{\prob{P}_C}^e \RV{I}_i C|\RV{X}_i\RV{H}$ and $\prob{P}_C$ is therefore $(\RV{X};\RV{Y})$-causally contractible by Lemma \ref{lem:eci_cc_extend}.
\end{proof}

It is possible that we can accept $\RV{I}_i\CI^e_{\prob{P}_C} \RV{X}_i C|\RV{H}$ after examining the data, but in some cases we may have exactly one observation for each possible value of $I$. In this case, there are situations in which we might be inclined to accept $\RV{I}_i\CI^e_{\prob{P}_C} \RV{X}_i C|\RV{H}$ a priori:
\begin{itemize}
    \item For any choice $\alpha\in C$, $\RV{X}_i$ is deterministic
    \item For any choice $\alpha\in C$, $\RV{X}_i$ is determined by some function of a known random source fixed ahead of time
\end{itemize}

\subsection{Body mass index revisited}

If we have a probability set $\prob{P}_C$ with $\RV{B}:=(\RV{B}_i)_{i\in M}$ representing body mass index and $\RV{Y}:=(\RV{Y}_i)_{i\in M}$ representing health outcomes of interest, we probably should assume causal contractibility a priori, not matter what kind of experimental setup is available:

\begin{itemize}
    \item There are multiple different actions that can affect body mass index in roughly the same way, so $\RV{Y}\CI_{\prob{P}_C}^e C|\RV{B}$ is doubtful
    \item Body mass index can neither be set deterministically, nor determined by a function of a random source, so Theorem \ref{th:cc_ind_treat} does not apply
\end{itemize} 

However, it is still possible that $\prob{P}_C$ could be $(\RV{B};\RV{Y})$-causally contractible. The key is Theorem \ref{th:cons_ci} -- if we start with $\prob{P}_C$ that is $(\RV{D};\RV{Y})$-causally contractible, and we find for a suitable $\alpha$ that $\RV{Y}_i\CI_{\prob{P}_\alpha} \RV{D}_i |\RV{H}\RV{B}_i$ for all $i$, then we can conclude that $\RV{Y}_i\CI_{\prob{P}_C}^e \RV{D}_i C|\RV{H}\RV{B}_i$ for all $i$, and that $\prob{P}_C$ is $(\RV{B};\RV{Y})$-causally contractible (Theorem \ref{lem:proxy_control}). There are two important things to bear in mind:

\begin{itemize}
    \item Causal contractibility is always relative to a set of choices $C$; our theory provides no way to assess the existence of ``causal effects'' independent of such a set
    \item Whether or not $\RV{Y}_i\CI_{\prob{P}_\alpha} \RV{D}_i |\RV{H}\RV{B}_i$ holds for a suitable choice $\alpha$ is a testable question, so it may be inappropriate to assume it does by asking at the outset about ``the causal effect of $\RV{B}$''
\end{itemize}

Thus it is possible in principle to have a ``causal effect of body mass index'', but our analysis suggests that it should be demonstrated rather than assumed.

\begin{theorem}\label{lem:proxy_control}
Suppose we have a probability set $\prob{P}_C$ that is $(\RV{D};\RV{X},\RV{Y})$-causally contractible, where $\RV{D}:=(\RV{D}_i)_{i\in M}$ and likewise for $\RV{X}$ and $\RV{Y}$. If there exists $\alpha\in C$ such that $\prob{P}_\alpha^{\RV{D}}\gg \{\prob{P}_\beta^{\RV{D}}|\beta\in C\}$ and $\RV{Y}_i\CI_{\prob{P}_\alpha} \RV{D}_i|\RV{HX}_i$ for all $i\in M$, then $\prob{P}_C$ is also $(\RV{Y};\RV{X})$-causally contractible.
\end{theorem}

\begin{proof}
See Appendix \ref{sec:bmi_revis}
\end{proof}

% \subsection{Weakening causal contractibility}

% We have pointed out that causal contractibility is a very strong assumption, and will usually be unacceptable. We proposed three assumptions that could justify causal contractibility for some decision procedures:

% \begin{enumerate}
%     \item Two procedures $\proc{S}_\alpha$ and $\proc{S}_{\alpha'}$ are indistinguishable if the description of one can be obtained by permuting patients in the description of the other (\emph{patient indistinguishability})
%     \item Two procedures $\proc{S}_\alpha$ and $\proc{S}_{\alpha'}$ are indistinguishable if the description of one can be obtained by permuting the order of treatment administration in the description of the other (\emph{order indistinguishability})
%     \item $c:A\to D^2$ is an invertible function
% \end{enumerate}

% We can imagine that many decision procedures involving patient treatment might satisfy the first two properties, but not the third. A question to explore is: is there a representation theorem relevant to a decision procedure in which the third assumption is removed or weakened? A common assumption in other causal frameworks for procedures like this is to assume that there is some variable $(\RV{X},\RV{Z})$ that is ``causally sufficent'' for $\RV{Y}$, but $\RV{Z}$ is not observed. In our framework, this would amount to $\prob{P}_C$ being $((\RV{X},\RV{Z});\RV{Y})$-causally contractible, with $\RV{Z}$ unobserved. However, at this point we do not know of a relevant representation theorem for unobservable causal contractibility like this, nor of an argument that connects the representation with a particular family of decision procedures.