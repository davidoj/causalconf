%!TEX root = main.tex



\section{Syntax and semantics of causal consequences}

Causal Bayesian networks and potential outcomes employ different naming conventions to distinguish ``causal effects'' from ``simple correlations''. Causal Bayesian networks write $P(\RV{Y}|do(\RV{X}))$ and $P(\RV{Y}|\RV{X})$, while potential outcomes distinguishes $P(\RV{Y}|\RV{X})$ from $x\mapsto P(\RV{Y}^x)$. If we are not going to worry too much about details of interpretation, we can interpret the expression $P(\RV{Y}|\RV{X})$ as expressing something like this: there is an objective probability $P(\RV{Y},\RV{X})$ that describes a sequence of independent and identically distributed observations, and $P(\RV{Y}|\RV{X})$ is a disintegration of this probability. The existence of an objective probability $P(\RV{Y},\RV{X})$ can be justified by an assumption that the sequence of observations should be modeled exchangeably.

We pursue a similar line of thinking with respect to understanding causal consequences like $P(\RV{Y}|do(\RV{X}))$ or $x\mapsto P(\RV{Y}^x)$. We assume that ``causal consequences'' are conditional probabilities of the form $\prob{P}_\square^{\RV{Y}|\RV{D}\RV{H}}$ where $\RV{Y}$ is an outcome, $\RV{D}$ is some decision, $\RV{H}$ is a hypothesis and $\prob{P}_\square$ is a probability gap model. Our interest is in understanding what causal consequences are \emph{from the point of view of someone choosing a decision function}. We do not address the question of how they may be inferred from observed data.

 We show that conditional probability models that are \emph{causally contractible} with respect to a sequence of decisions and a corresponding sequence of outcomes are representible by mixtures of ``objective but unknown'' conditional probabilities. This is analogous to De Finetti's theorem that shows exchangeable probability distributions are representable by mixtures of ``objective but unknown'' independent and identically distributed probability distributions. A similar argument to ours is found in \citet{dawid_decision-theoretic_2020}.

We also consider the question of when causal contractibility could be supposed to hold. This is a subtle question, as the answer appears to differ for situations that are quite similar. For example, consider:
\begin{enumerate}
    \item Dr Alice is going to see two patients who are both complaining of lower back pain and are otherwise unknown to Alice. Prior to seeing them, she considers the available research and formulates a general sense of whether or not she'll treat each one, which she quantifies with $\prob{P}_\alpha^{\RV{D}_1\RV{D}_2}$
    \item As before, but prior to seeing the patients she considers the available research and decides to treat on the basis of applying a function to a random number generator with known characteristics. The choice of function and random number generator allow her to quantity probability of treatment with $\prob{P}_\alpha^{\RV{D}_1\RV{D}_2}$
\end{enumerate}

\todo[inline]{I removed the discussion of probability combs for simplicity, so I have not considered policies for treatment that depend on earlier experiments in the examples above}

We will argue that Alice could reasonably assume causal contractibility in the second case but not the first. While we are unable to offer a general theory of when causal contractibility holds, we note that an apparently key difference between the two situations is that in the first case the ``decision'' $\RV{D}_1$ is indeterministic for some $\alpha$, though $\RV{D}_2$ is deterministic, while in the second case both $\RV{D}_1$ and $\RV{D}_2$ are determinstic functions.

\subsection{Repeatable experiments}

A conditional probability model $(\prob{P}_\square^{\overline{\RV{Y}|\RV{D}}},A)$ is a model of a sequential experiment if $\RV{Y}:=\RV{Y}_M=(\RV{Y}_i)_{i\in M}$ and $\RV{D}:=\RV{D}_M=(\RV{D}_i)_{i\in M}$ for some index set $M$. We say that $\RV{Y}_i$ is the consequence corresponding to the decision $\RV{D}_i$ for all $i\in M$. A $(\RV{D}_i,\RV{Y}_i)$ pair is an \emph{experimental unit}. We identify a ``causal consequence'' with a conditional probability of the form $\prob{P}_\square^{\RV{Y}_i|\RV{H}\RV{D}_i}$, where $\RV{H}$ is a hypothesis that is identical for every experimental unit. Causal consequences do not generally exist, see Definition \ref{def:cprob_pset}.

If $(\prob{P}_\square^{\overline{\RV{Y}|\RV{D}}},A)$ represents a sequential experiment, we might guess that causal consequences exist if the experiment is in some sense ``repeatable''. We consider two precise notions of repeatability. The first condition is \emph{commutativity of exchange}, which is the assumption that swapping the choices that we apply at each step and then applying the corresponding inverse swap to consequences leaves the model unchanged. The second condition is \emph{commutativity of marginalisation} -- if we perform the whole experiment multiple times, making the same choice $\RV{D}_i$ at any point $i$ gets the same results, regardless of what other choices are made.

Commutativity of exchange is similar to the condition of \emph{post-treatment exchangeability} found in \citet{dawid_decision-theoretic_2020}, and commutativity of marginalisation is similar to the stable unit treatment distribution assumption (SUTDA) in the same, as well as the ``no interference'' part of the stable unit treatment value assumption (SUTVA) with which it shares a name. Commutativity of exchange is also very similar to the exchangeability assumption of \citet{greenland_identifiability_1986} for further discussions of exchangeability in the context of causal modelling, and note that both authors consider exchanging to be an operation that alters which person receives which treatment. The assumption of exchangeability found in \citet{banerjee_chapter_2017} can also be regarded as similar to commutativity of exchange.

\todo[inline]{I think the useful part is not that these ideas are conceptually new, but they have sharp definitions instead of }

\todo[inline]{Not sure if or where I want to put this, I just think it helps to illustrate the difference}

Commutativity of exchange is not equivalent to exchangeability in the sense of De Finetti's well-known theorem \citet{de_finetti_foresight_1992}. The latter can be understood as expressing an indifference between conducting the experiment as normal, or conducting the experiment and then swapping some labels. However, swapping \emph{choices} will (usually) lead to different experimental units receive different treatment, which is something that can't be achieved by swapping labels after the experiment has concluded.

The difference is illustrated by the following pair of diagrams.

Exchangeability (swapping labels):

\begin{align}
    \tikzfig{exchangeability}
\end{align}

Commutativity of exchange (swapping choices $\sim$ swapping labels):

\begin{align}
    \tikzfig{commutativity of exchange}
\end{align}

Commutativity of exchange is a property of probability gap models, not a property of fixed probability model for which there is no analogue of ``attaching a different choice'' in that case.

\todo[inline]{----end not sure where to put------}


% Another way to see where we are going is to consider graphical statements of our and De Finetti's result.

% Take $S=\{0,1\}$ and identify the space $\Delta(S)$ of probability measures on $S$ with the interval $[0,1]$. De Finetti showed that any infinite exchangeable probability measure $\prob{P}_\alpha$ on $\{0,1\}^\mathbb{N}$ can be represented by a prior $\prob{P}_\alpha^{\RV{H}}\in [0,1]$ for some $\RV{H}:\Omega\to H$ and a conditional probability $\prob{P}^{\RV{S}_0|\RV{H}}:[0,1]\kto \{0,1\}$ such that

% \begin{align}
%     \prob{P}_\alpha &= \tikzfig{de_finetti_rep0}\label{eq:definettirep}
% \end{align}

% Here $\prob{P}^{\RV{S}_0|\RV{H}}$ can be defined concretely by $\prob{P}^{\RV{S}_0|\RV{H}}(1|h)=h$. Equivalently, the probability gap model on $S^\mathbb{N}$ defined by the assumption of exchangeability is equivalent to the probability gap model defined by the conditional probability

% \begin{align}
%     \prob{P}^{\RV{S}|\RV{H}} = \tikzfig{de_finetti_conditional}
% \end{align}

% That is, there is some hypothesis $\RV{H}$ and conditional on $\RV{H}$ the measurements are independent and identically distributed. The proof of this is constructive -- $\RV{H}$ is a function of $\RV{S}$.



% \begin{align}
%     \prob{P}^{\RV{Y}|\RV{HD}} = \tikzfig{do_model_representation}
% \end{align}

% We will further argue that the class of see-do models considered in CBN and potential outcomes literature is equivalent to the family of causally contractible and exchangeable do-models where the decision rule for the first $n$ places is fixed to an unknown value, and may be freely chosen thereafter.

% \begin{theorem}[Existence of conditional in do models]
% Given a do model $(\prob{P}_{\square}^{\RV{Y}\|\RV{D}},R)$, for all $\alpha\in R$, $n\in\mathbb{N}$
% \begin{align}
%     \prob{P}_\alpha^{\RV{Y}_{[n]}\RV{D}_i} = \prob{P}_\alpha^{\RV{D}_{[n]}}\odot \prob{P}_\square^{\RV{Y}_{[n]}\|\RV{D}_{[n]}}
% \end{align}
% That is, $\prob{P}_\square^{\RV{Y}_{[n]}\|\RV{D}_{[n]}}\cong \prob{P}_\square^{\RV{Y}_{[n]}|\RV{D}_{[n]}}$
% \end{theorem}

% \begin{proof}
% For any $n>1\in \mathbb{N}$, $\alpha\in R$

% \begin{align}
%     \prob{P}_\alpha^{\RV{Y}_{[n]}\RV{D}_{[n]}} &= \tikzfig{do_model_1}\\
%     &= \tikzfig{do_model_2}\\
%     &= \tikzfig{do_model_3}\\
%     &= \tikzfig{do_model_4}\\
%     \implies \prob{P}_\alpha^{\RV{Y}_{[n]}|\RV{D}_{[n]}} &= \tikzfig{do_model_5}\\
%     &= \prob{P}_\alpha^{\RV{Y}_{[n-1]}|\RV{D}_{[n-1]}}\combprod \prob{P}_\square^{\RV{Y}_n|\RV{Y}_{[n-1]}\RV{D}_n}
% \end{align}

% Applying this recursively with $\prob{P}_\alpha^{\RV{Y}_{[1]}|\RV{D}_{[1]}}=\prob{P}_\square^{\RV{Y}_{[1]}|\RV{D}_{[1]}}$ yields

% \begin{align}
%     \prob{P}_\alpha^{\RV{Y}_{[n]}|\RV{D}_{[n]}} = \prob{P}_\square^{\RV{Y}_{[n]}\|\RV{D}_{[n]}}
% \end{align}
% as desired.
% \end{proof}
More precisely, a conditional probability model ``commutes with exchange'' if applying any finite permutation to blind choices or separately applying the corresponding permuation to consequences each yields the same result. We can apply the exchange ``before'' multiplying by the conditional $\prob{P}_{\square}^{\RV{Y}|\RV{D}}$ or after it and we get the same result.

\begin{definition}[Swap map]
Given $M\subset \mathbb{N}$ a finite permutation $\rho:M\to M$ and a variable $\RV{X}:\Omega\to X^M$ such that $\RV{X}=(\RV{X}_i)_{i\in M}$, define the Markov kernel $\text{swap}_{\rho(\RV{X})}:X^M\kto X^M$ by $(d_i)_{i\in\mathbb{N}}\mapsto \delta_{(d_{\rho(i)})_{i\in\mathbb{N}}}$.
\end{definition}

\begin{definition}[Commutativity of exchange]\label{def:caus_exch}
Suppose we have a sample space $(\Omega,\sigalg{F})$ and a conditional probability model $(\prob{P}_{\square}^{\overline{\RV{Y}|\RV{D}}},A)$ with $\RV{Y}=\RV{Y}_M$, $\RV{D}=\RV{D}_M$, $M\subseteq \mathbb{N}$. If, for any two decision rules $\alpha^{\overline{\RV{D}}},\beta^{\overline{\RV{D}}} \in A$,
\begin{align}
    \alpha^{\RV{D}}\odot \text{swap}_{\rho(\RV{D})} \prob{P}_{\square}^{\RV{Y}|\RV{D}} &= \alpha^{\RV{D}}\odot \prob{P}_{\square}^{\RV{Y}|\RV{D}}\text{swap}_{\rho(\RV{D}\times \RV{Y})}
\end{align}
Then $\prob{P}_\square$ \emph{commutes with exchanges}.
\end{definition}

A do model is non interfering if it gives identical results for identical subsequences of different choices when we limit our attention to the corresponding subsequences of consequences. For example, if we have $\RV{D}=(\RV{D}_1,\RV{D}_2,\RV{D}_3)$ and $\RV{Y}=(\RV{Y}_1,\RV{Y}_2,\RV{Y}_3)$ and $\alpha^{\RV{D}_1\RV{D}_3}=\prob{P}_\beta^{\RV{D}_1\RV{D}_3}$ then $\prob{P}_{\alpha}^{\RV{Y}_1\RV{D}_1\RV{Y}_3\RV{D}_3}=\prob{P}_\beta^{\RV{Y}_1\RV{D}_1\RV{Y}_3\RV{D}_3}$.

\begin{definition}[Commutativity of marginalisation]\label{def:caus_cont}
Suppose we have a sample space $(\Omega,\sigalg{F})$ and a conditional probability model $(\prob{P}_{\square}^{\overline{\RV{Y}|\RV{D}}},A)$ with $\RV{Y}=\RV{Y}_M$, $\RV{D}=\RV{D}_M$, $M\subseteq \mathbb{N}$. For any $S=(s_i)_{i\in Q}$, $Q\subset M$, and $i<j\implies p_i<p_j \And q_i<q_j$, let $\RV{D}_S:=(\RV{D}_i)_{i\in S}$ and $\RV{D}_T:=(\RV{D}_i)_{i\in T}$. If for any $\alpha,\beta\in R$
\begin{align}
    \prob{P}_\alpha^{\RV{D}_{S}}&=\prob{P}_\beta^{\RV{D}_{S}}\\
    \implies \prob{P}_\alpha^{(\RV{D_i,Y_i})_{i\in S}}&=\prob{P}_\beta^{(\RV{D_i,Y_i})_{i\in S}}
\end{align}
then $\prob{P}_\square$ \emph{commutes with marginalisation}.
\end{definition}

Neither condition implies the other. 
\begin{lemma}
Commutativity of exchange does not imply commutativity or vise versa.
\end{lemma}

\begin{proof}
Suppose $D=Y=\{0,1\}$ and we have a conditional probability model $(\prob{P}_\square^{\overline{\RV{Y}|\RV{D}}},A)$ where $\RV{D}=(\RV{D}_1,\RV{D}_2)$, $\RV{Y}=(\RV{Y}_1,\RV{Y}_2)$ and A contains all deterministic probability measures in $\Delta(D^2)$. If

\begin{align}
    \prob{P}_\square^{\RV{Y}_1\RV{Y}_2|\RV{D}_1\RV{D}_2}(y_1,y_2|d_1,d_2) &= \llbracket (y_1,y_2)= (d_1+d_2,d_1+d_2) \rrbracket
\end{align}

Then $\prob{P}_{\delta_{00}}^{\RV{Y}_1\RV{D}_1}(y_1) = \llbracket y_1=0\rrbracket$ while $\prob{P}_{\delta_{01}}^{\RV{Y}_1} = \llbracket y_1=1 \rrbracket$. However, $\delta_00^{\RV{D}_1}=\delta_{01}^{\RV{D}_1}=\delta_0^{\RV{D}_1}$ so $\prob{P}_\square$ does not commute with marginalisation. However, taking $(d_i,d_j):=\delta_{d_i d_j}\in A$,

\begin{align}
    \prob{P}_{d_2,d_1}^{\RV{Y}_1\RV{D}_1\RV{Y}_2\RV{D}_2}(y_1,d_1,y_2,d_2) &= \llbracket (y_1,y_2)= (d_2+d_1,d_2+d_1) \rrbracket\\
    &= \llbracket (y_2,y_1)= (d_1+d_2,d_1+d_2) \rrbracket\\
    &= \prob{P}_{d_1,d_2}^{\RV{Y}_1\RV{D}_1\RV{Y}_2\RV{D}_2}(y_2,d_2,y_1,d_1)
\end{align}

so $\prob{P}_\square$ commutes with exchange.

Alternatively, suppose the same setup, but define $\prob{P}_\square$ instead by, for all $\alpha\in A$

\begin{align}
    \prob{P}_\square{\RV{Y}_1\RV{Y}_2|\RV{D}_1\RV{D}_2}(y_1,y_2|d_1,d_2) &= \llbracket (y_1,y_2)= (0,1) \rrbracket
\end{align}

Then $\prob{P}_\square$ commutes with marginalisation. If $\prob{P}_\alpha^{\RV{D}_S}=\prob{P}_\beta^{\RV{D}_S}$ for $S\subset\{0,1\}$ then

\begin{align}
    \prob{P}_{\alpha}^{\RV{Y}_S\RV{D}_S}(y_s,d_s) &= \sum_{y'_2\in \{0,1\}^{S^C}} \llbracket (y_1,y_2)= (0,1) \rrbracket\prob{P}_\alpha^{\RV{D}_S}(d_s) \\
                                                  &= \prob{P}_{\beta}^{\RV{Y}_S\RV{D}_S}(y_s,d_s)
\end{align}
but not exchange. For all $\alpha,\beta \in A$:

\begin{align}
    \prob{P}_\alpha{\RV{Y}_1\RV{Y}_2}(y_1,y_2) &= \llbracket (y_1,y_2)= (0,1) \rrbracket\\
    &\neq \prob{P}_\beta{\RV{Y}_1\RV{Y}_2}(y_2,y_1)
\end{align}
\end{proof}

Although commutativity of marginalisation seems to be a bit like non-interference -- the marginal distribution I get for $\RV{Y}_i$ depends only on the decision $\RV{D}_i$ -- it still allows for some models in which we seem to have interference of a kind. For example: in the first experiment I flip a coin and decide either to pass the results to the second experiment ($\RV{D}_1=0$) or flip another coin and pass those results second experiment ($\RV{D}_1=1$). In the second I either copy the results I have been given ($\RV{D}_2=0$) or invert them ($\RV{D}_2=1$). Then
\begin{itemize}
    \item The marginal distribution of both experiments is $\text{Bernoulli}(0.5)$ no matter what choices I make, so it satisfies Definition \ref{def:caus_cont}
    \item Nevertheless, the choice for the first experiment seems to ``affect'' the result of the second experiment (affect in quotes because it is an intuitive judgement, not a formal property)
\end{itemize}

Here we are most interested in the conjunction of these assumptions, a condition we call \emph{causal contractibility}

\begin{definition}[Causal contractibility]
A conditional probability model $(\prob{P}_{\square}^{\overline{\RV{Y}|\RV{D}}},A)$ is causally contractible if it is both commutative with exchange and commutative with marginalisation.
\end{definition}

% \begin{proposition}[Representation of do-models that commute with exchange]
% Suppose we have a fundamental probability set $\Omega$ and a do model $(\prob{P},\RV{D},\RV{Y},R)$ such that $\RV{D}:=(\RV{D}_i)_{i\in \mathbb{N}}$ and $\RV{Y}:=(\RV{Y}_i)_{i\in\mathbb{N}}$ where $\prob{P}$ commutes with exchange and there is some $\alpha^*\in R$ such that $\prob{P}^{\alpha^*}\gg\prob{P}_\beta$ for all $\beta in R$. Then there exists a symmetric function $\RV{H}:(Y\times D)^\mathbb{N}\to H$ such that  $\prob{P}^{\RV{Y}|\RV{DH}}$ exists and $\RV{Y}_i\CI_{\prob{P}}(\RV{D}_j,\RV{Y}_j)_{j\in \mathbb{N}}\setminus \{i\}|\RV{H}\RV{D}_i$, or equivalently 
% \begin{align}
%     \prob{P}^{\RV{Y}} &= \tikzfig{do_model_representation}
% \end{align}
% \end{proposition}

% % \begin{lemma}[Contraction and independence]
% % Let $\RV{J}$, $\RV{K}$ and $\RV{L}$ be variables on $\Omega$ and $\prob{Q}\in \Delta(\Omega)$ a base measure such that $\prob{Q}^{\RV{JK}}=\prob{Q}^{\RV{JL}}$ and $\sigma{K}\subset \sigma{L}$. Then $\RV{J}\CI\RV{L}|\RV{K}$. 
% % \end{lemma}

% % \begin{proof}
% % From Lemma 1.3 in \citet{kallenberg_basic_2005}.
% % \end{proof}

% \begin{proof}
% If $\prob{P}$ commutes with exchange, then for any $\alpha\in R$ such that $\prob{P}_\alpha^{\RV{D}}$ is exchangeable then $\prob{P}_\alpha$ is also exchangeable. Then there exists $\RV{H}$ a symmetric function of $(\RV{Y}_i,\RV{D}_i)_{i\in\mathbb{N}}$ such that $\RV{Y}_i\CI_{\prob{P}}(\RV{D}_j,\RV{Y}_j)_{j\in \mathbb{N}}\setminus \{i\}|\RV{H}\RV{D}_i$. This is De Finetti's representation theorem, and many proofs exists, see for example \citep{kallenberg_basic_2005}.

% In particular, let 

% \begin{align}
%     \RV{H}:=A\times B\mapsto \lim_{n\to\infty} \frac{1}{n}\sum_{i\in n} \mathds{1}_{A\times B}((\RV{Y}_i, \RV{D}_i))
% \end{align}

% Then for all $\alpha\in R$,
% \begin{align}
%     \prob{P}_\alpha^{(\RV{Y}_i,\RV{D}_i)_{i\in\mathbb{N}}|\RV{H}}(A\times B|h) \overset{a.s.}{=} h(A\times B)\label{eq:given_h}
% \end{align}

% The proof that the limit exists and the above equality holds can again be found int \citep{kallenberg_basic_2005}.
% \end{proof}

\subsection{Causal consequences exist if the model is causally contractible}

The main result in this section is Theorem \ref{th:iid_rep} which shows that a conditional probability model $\prob{P}_\square$ is causally contractible if and only if it can be represented as the product of a distribution over hypotheses $\prob{P}_\square^{\RV{H}}$ and a collection of identical conditional probabilities $\prob{P}_\square^{\RV{Y}_1|\RV{D}_1\RV{H}}$. This can be interpreted as expressing the idea that all experimental units $(\RV{Y}_i,\RV{D}_i)$ share a canonical but unknown ``consequence function'' $D\kto Y$. As discussed already in Section \ref{sec:curry}, the existence of such a conditional probability implies the existence of a common unknown \emph{curried} conditional probability for all experimental units, which resembles a potential outcomes model. In fact, we prove the existence of a curried representation first, in Lemma \ref{th:table_rep}.

\begin{Lemma}[Exchangeable curried representation]\label{th:table_rep}
A conditional probability model $(\prob{P}^{\RV{Y}|\RV{D}}_\square,A)$ such that $\RV{D}:=(\RV{D}_i)_{i\in \mathbb{N}}$ and $\RV{Y}:=(\RV{Y}_i)_{i\in \mathbb{N}}$. $\prob{P}_\square$ is causally contractible if and only if
\begin{align}
    \prob{P}_\square^{\RV{Y}|\RV{D}} &= \tikzfig{lookup_representation}\\
    &\iff\\
    \prob{P}_\square^{\RV{Y}|\RV{D}}(y|d) &= \prob{P}^{(\RV{Y}^D_{d_i i})_{\mathbb{N}}}(y)
\end{align}
Where $\prob{P}^{\RV{Y}^D}$ is an exchangeable probability measure on $Y^{D\times\mathbb{N}}$, for convenience we extend the sample space with the random variable $\RV{Y}^D:=(\RV{Y}_{ij}^D)_{i\in D,j\in \mathbb{N}}$ and $\prob{L}^{\RV{D},\RV{Y}^D}$ is the Markov kernel associated with the lookup function
\begin{align}
    l:D^\mathbb{N}\times Y^{D\times \mathbb{N}}&\to Y\\
    ((d_i)_\mathbb{N},(y_{ij})_{i\in D,j\in \mathbb{N}})&\mapsto y_{d_i i}
\end{align}
\end{Lemma}

\begin{proof}
Only if:
Choose $e:=(e_i)_{i\in\mathbb{N}}$ such that $e_{|D|i+j}$ is the $i$th element of $D$ for all $i,j\in \mathbb{N}$. Abusing notation, write $e$ also for the decision function that chooses $e$ deterministically.

Define
\begin{align}
    \prob{P}^{\RV{Y}^D}((y_{ij})_{D\times \mathbb{N}}):=\prob{P}_e^{\RV{Y}}((y_{|D|i+j})_{i\in D, j\in \mathbb{N}})
\end{align}

Now consider any $d:=(d_i)_{i\in \mathbb{N}}\in D^{\mathbb{N}}$. By definition of $e$, $e_{|D|d_i + i}=d_i$ for any $i,j\in \mathbb{N}$.

\begin{align}
    \prob{Q}:D\kto Y\\
    \prob{Q}:= \tikzfig{lookup_representation}
\end{align}

and consider some ordered sequence $A\subset \mathbb{N}$ and $B:= ((|D|d_i+i))_{i\in A}$. Note that $e_B:=(e_{|D|d_i +i})_{i\in B}=d_A=(d_i)_{i\in A}$. Then 

\begin{align}
    \sum_{y\in \RV{Y}^{-1}(y_A)} \prob{Q}(y|d) &= \sum_{y\in \RV{Y}^{-1}(y_A)} \prob{P}^{(\RV{Y}^{D}_{d_ii})_{A}}(y) \\
    &= \sum_{y\in \RV{Y}^{-1}(y_A)} \prob{P}_e^{(\RV{Y}_{|D|d_i+i})_{A}}(y)\\
    &= \prob{P}_e^{\RV{Y}_{B}}(y_A)\\
    &= \prob{P}_{d}^{\RV{Y}_A}(y_A)&\text{by causal contractibility}
\end{align}

Because this holds for all $A\subset\mathbb{N}$, by the Kolmogorov extension theorem

\begin{align}
    \prob{Q}(y|d) &= \prob{P}_d^{\RV{Y}}(y)
\end{align}

Because $d$ is the decision function that deterministically chooses $d$, for all $d\in D$

\begin{align}
    \prob{Q}(y|d) &= \prob{P}_d^{\RV{Y}|\RV{D}}(y|d)
\end{align}

And because $\prob{P}_d^{\RV{Y}|\RV{D}}(y|d)$ is unique for all $d\in D^{\mathbb{N}}$ and $\prob{P}^{\RV{Y}|\RV{D}}$ exists by assumption

\begin{align}
    \prob{P}^{\RV{Y}|\RV{D}}=\prob{Q}
\end{align}

Next we will show $\prob{P}^{\RV{Y}^D}$ is contractible. Consider any subsequences $\RV{Y}^D_S$ and $\RV{Y}^D_T$ of $\RV{Y}^D$ with $|S|=|T|$. Let $\rho(S)$ be the ``expansion'' of the indices $S$, i.e. $\rho(S)=(|D|i+j)_{i\in S,j\in D}$. Then by construction of $e$, $e_{\rho(S)}=e_{\rho(T)}$ and therefore

\begin{align}
    \prob{P}^{\RV{Y}^D_S}&= \prob{P}_e^{\RV{Y}_{\rho(S)}})\\
    &= \prob{P}_e^{\RV{Y}_{\rho(T)}})&\text{by contractibility of }\prob{P}\text{ and the equality } e_{\rho(S)}=e_{\rho(T)}\\
    &= \prob{P}^{\RV{Y}^D_T}
\end{align}


If:
Suppose 
\begin{align}
    \prob{P}^{\RV{Y}|\RV{D}} &= \tikzfig{lookup_representation}
\end{align}

and consider any two deterministic decision functions $d,d'\in D^{\mathbb{N}}$ such that some subsequences are equal $d_S=d'_T$.

Let $\RV{Y}^{d_S}=(\RV{Y}_{d_i i})_{i\in S}$.

By definition,

\begin{align}
    \prob{P}^{\RV{Y}_S|\RV{D}}(y_S|d) &= \sum_{y^D_S\in Y^{|D|\times |S|}}\prob{P}^{\RV{Y}^D_S}(y^D_S)\prob{L}^{\RV{D}_S,\RV{Y}^S}(y_S|d,y^D_S)\\
    &= \sum_{y^D_S\in Y^{|D|\times |T|}}\prob{P}^{\RV{Y}^D_T}(y^D_S)\prob{L}^{\RV{D}_S,\RV{Y}^S}(y_S|d,y^D_S)&\text{ by contractibility of }\prob{P}^{\RV{Y}^D_T}\\
    &= \prob{P}^{\RV{Y}_T|\RV{D}}(y_S|d)
\end{align}
\end{proof}

The curried representation of Lemma \ref{th:table_rep} does not need to support an interpretation as a distribution of potential outcomes. For example, consider a series of bets on fair coinflips -- in this case, the consequence $\RV{Y}_i$ is uniform on $\{0,1\}$ for any decision $\RV{D}_i$. Tha $D=Y=\{0,1\}$ and $\prob{P}_\alpha^{\RV{Y}_n}(y)=\prod_{i\in [n]} 0.5$ for all $n$, $y\in Y^n$, $\alpha\in R$. Then the construction in Lemma \ref{th:table_rep} yields $\prob{P}^{Y^D_i}(y^D_i)=\prod_{j\in D} 0.5$ for all $y^D_i\in Y^D$. That is, $\RV{Y}^0_i$ and $\RV{Y}^1_i$ are independent and uniformly distributed. However, if we wanted $\RV{Y}^0_i$ to represent ``what would happen if I bet on outcome 0 on turn $i$'' and $\RV{Y}^1$ to represent ``what would happen if I bet on outcome 1 on turn $i$'', then it seems that we ought to have $\RV{Y}^0_i = 1-\RV{Y}^1_i$. 

We could suppose that Lemma \ref{th:table_rep} provides necessary but not sufficient conditions for the existence of a potential outcomes representation of a conditional probability model. However, it doesn't seem to succeed at that either. We note, for example, that \citet{rubin_causal_2005} does not assume that the distribution of potential outcomes is exchangeable. A non-exchangeable $\prob{P}^{\RV{Y}^D}$ does not induce a causally contractible conditional probability model, and at the same time commutativity with marginalisation is not sufficient for a conditional probability model to support a curried representation in the sense of Lemma \ref{th:table_rep}. What seems to be missing is an additional assumption that consequences are mutually independent of one another given the associated decision. 

We can also represent contractible conditional probability models repeated copies of an unknown ``consequence function'', a Markov kernel that maps from decisions to probability distributions over consequences, coupled by a common hypothesis $\RV{H}$. 

\begin{theorem}\label{th:iid_rep}
Suppose we have a fundamental probability set $\Omega$ and a do model $(\prob{P},\RV{D},\RV{Y},R)$ such that $\RV{D}:=(\RV{D}_i)_{i\in \mathbb{N}}$ and $\RV{Y}:=(\RV{Y}_i)_{i\in\mathbb{N}}$. $\prob{P}$ is causally contractible if and only if there exists some $\RV{H}:\Omega\to H$ such that $\prob{P}^{\RV{Y}_i|\RV{H}\RV{D}_i}$ exists for all $i\in \mathbb{N}$ and
\begin{align}
    \prob{P}^{\RV{Y}|\RV{H}\RV{D}} &= \tikzfig{do_model_representation}\\
    &\iff\\
    \RV{Y}_i&\CI_{\prob{P}} \RV{Y}_{\mathbb{N}\setminus i},\RV{D}_{\mathbb{N}\setminus i}|\RV{H}\RV{D}_i&\forall i\in \mathbb{N}\\
    \land \prob{P}^{\RV{Y}_i|\RV{H}\RV{D}_i} &= \prob{P}^{\RV{Y}_0|\RV{H}\RV{D}_0} & \forall i\in \mathbb{N}
\end{align}
\end{theorem}

\begin{proof}
We make use of Lemma \ref{th:table_rep} to show that we can represent the conditional probability as an exchangeable tabular probability distribution. We then use the property of exchangeability of the columns of that distribution in conjunction with De Finetti's theorem to derive the result.
\end{proof}

\subsection{Modelling different measurement procedures}

An important question is: when is it reasonable to assume causal contractibility? To justify such an assumption, we might argue as follows:
\begin{itemize}
    \item Two measurement procedures, related to one another by swaps and erasures, should be considered equivalent
    \item This implies that the model of each measurement procedure should be symmetric with respect to the corresponding swap and erase operations
\end{itemize}

First, we want to spell out in detail how one model can apply to two different measurement procedures. Recall that we assume that a single master measurement procedure $\proc{S}$ taking values in $\Psi$, and observables are all functions of $\proc{S}$. Given a model $(\prob{P}_\square,A)$ associated with $\proc{S}$, the model does not in general apply to an alternative measurement procedure $\proc{S}'$.

However, it is also a principle of measurement procedures that a measurement procedure followed by the application of a function is itself a measurement procedure. Thus a model $(\prob{P}_\square,A)$ associated with $\proc{S}$ may also be informative about a procedure $f\circ \proc{S}$ for any $f:\Psi\to X$.

In particular, consider measurement procedures related by \emph{swaps}. For example, suppose we have $(\proc{D}_1,\proc{D}_2)$ and $(\proc{D}^{\text{swap}}_1,\proc{D}^{\text{swap}}_2):=(\proc{D}_2,\proc{D}_1)$. Then, given any probability model $\prob{P}_\alpha^{\RV{D}_1\RV{D}_2}$ we have $\prob{P}_\alpha^{\RV{D}_1^{\text{swap}}\RV{D}_2^{\text{swap}}} = \prob{P}_\alpha^{\RV{D}_1\RV{D}_2}$. In this way, $\prob{P}_\alpha^{\RV{D}_1\RV{D}_2}$ is a model of $(\proc{D}_1,\proc{D}_2)$ and induces a unique model of $(\proc{D}^{\text{swap}}_1,\proc{D}^{\text{swap}}_2)$.

Technically, this requires an assumption: if $\RV{X}$ is associated with $\proc{X}$ then $f\circ \RV{X}$ is associated with $f\circ \proc{X}$ (roughly: the abstract mathematical idea of composing a function with something and the actual process of applying a function to something and obtaining a result are treated as the same thing)

\subsection{Example: commutativity of exchange and treatment choices}

Concretely, commutativity of exchange can be justified if we suppose that the same model $(\prob{P}_\square^{\RV{Y}_M|\RV{D})_M},A)$ should describe
\begin{itemize}
    \item A measurement procedure $\proc{S}$ that yields $|M|$ outcomes $\proc{Y}_M$ and and $|M|$ decisions $\proc{D}_M$
    \item Any other $|M|$ outcomes $\proc{Y}^{\text{swap}}_M$ and $|M|$ decisions $\proc{D}^{\text{swap}}_M$, related to the originals by a swap.
\end{itemize}

For a concrete example, consider the following two scenarios:

\begin{enumerate}
    \item Dr Alice is going to see two patients who are both complaining of lower back pain and are otherwise unknown to Alice. Prior to seeing them, she settles on a decision function $\alpha$ which deterministically sets her treatment choices according to a function $\text{decisions}(\alpha)$
    \item As before, but $\alpha$ is a ``decision inclination'' and $\prob{P}_\alpha^{\RV{D}_1\RV{D}_2}$ nondeterministic
\end{enumerate}

Alice could model both situations with a sequential conditional probability model $(\prob{P}_\square^{\RV{Y}_1\RV{Y}_2|\RV{D}_1\RV{D}_2},A)$ with the elements of $A$ identified with probability models of the form $\prob{P}_\alpha^{\RV{D}_1\RV{D}_2}$. Might she, in one or both situations, consider this condiitonal probability model to be causally contractible?

We will assume that both satisfy commutativity of marginalisation -- that is, the first patient's outcomes are expected to be the same no matter what is planned for the second patient and vise versa. We want to know if they satisfy commutativity of exchange.

The argument we want to make (if it can be supported) is:
\begin{itemize}
    \item We can describe two measurement procedures that should share the same model
    \item The first is a measurement procedure for $(\RV{D}_1,\RV{D}_2,\RV{Y}_1,\RV{Y}_2)$
    \item The second is a measurement procedure for $(\RV{D}^{\text{swap}}_1,\RV{D}^{\text{swap}}_2,\RV{Y}^{\text{swap}^{-1}}_1,\RV{Y}^{\text{swap}^{-1}}_2)$
\end{itemize}

At the outset, Alice does not know any features that might distinguish the two patients, so it is reasonable to think that she should adopt the same model for a) the original experiment and b) the same experiment, except with the patients interchanged. Note that interchanging \emph{patients} does not correspond directly to any operation on the model $(\prob{P}_\square^{\RV{Y}_1\RV{Y}_2|\RV{D}_1\RV{D}_2},A)$ which describes decisions and, not patients.

We will define measurement procedures using pseudocode because the argument requires precision.

Suppose the first scenario corresponds to the following procedure $\proc{S}$ which yields values in $A\times D^2\times Y^2$. $\RV{D}_i$ is the projection $(\alpha,d_1,d_2,y_1,y_2)\mapsto d_i$ composed with $\proc{S}$ and $\RV{Y}_i$ is the projection $(\alpha,d_1,d_2,y_1,y_2)\mapsto y_i$ composed with $\proc{S}$.

\todo[inline]{I feel like I've put myself in a weird position of doing pseudo-mathematical derivations with English words. I'm not sure that there's a better alternative, though.}

\begin{algorithmic}
    \proc{S}:
        \State $\alpha \gets $choose_alpha()
        \State $(\proc{D}_1,\proc{D}_2) \gets $decisions($\alpha$)
        \State $\proc{Y}_1)\gets \text{apply}(\proc{D}_1,\text{patient A})$
        \State $\proc{Y}_2)\gets \text{apply}(\proc{D}_2,\text{patient B})$
    \Return $(\alpha,\proc{D}_1,\proc{D}_2,\proc{Y}_1,\proc{Y}_2)$
\end{algorithmic}

Make the assumption that, on the basis that the patients are indistinguishable to Alice at the time of model construction, the measurement procedure with swapped patients is equivalent. Assume also that swapping the order of treatment and swapping the order in which outcomes are recorded yields an equivalent measurment procedure. Putting these two assumptions together, the following procedure $\proc{S}'$ is equivalent to the original:

\begin{algorithmic}
    \proc{S}':
        \State $\alpha \gets $choose_alpha()
        \State $(\proc{D}_1,\proc{D}_2) \gets $decisions($\alpha$)
        \State $\proc{Y}_2\gets \text{apply}(\proc{D}_2,\text{patient A})$
        \State $\proc{Y}_1\gets \text{apply}(\proc{D}_1,\text{patient B})$
    \Return $(\alpha,\proc{D}_1,\proc{D}_2,\proc{Y}_1,\proc{Y}_2)$
\end{algorithmic}

Now, consider a measurement procedure $\proc{S}''$ that involves swapping decisions, then applying the inverse swap to outcomes

\begin{itemize}
    \proc{S}'':
        \State $\alpha \gets $choose_alpha()
        \State $(\proc{D}_1,\proc{D}_2) \gets $decisions($\alpha$)
        \State $(\proc{D}_1^{\text{swap}},\proc{D}_2^{\text{swap}}) \gets (\proc{D}_2,\proc{D}_1)$
        \State $\proc{Y}^{\text{swap}}_1\gets \text{apply}(\proc{D}_1^{\text{swap}},\text{patient A})$
        \State $\proc{Y}_2^{\text{swap}}\gets \text{apply}(\proc{D}_2^{\text{swap}},\text{patient B})$
        \State $(\proc{Y}_1,\proc{Y}_2)\gets (\proc{Y}^{\text{swap}}_2,\proc{Y}^{\text{swap}}_1)$
    \Return $(\alpha,\proc{D}_1,\proc{D}_2,\proc{Y}_1,\proc{Y}_2)$
\end{itemize}

(Note that we have implicitly reversed the swap we used to get $(\proc{D}_1^{\text{swap}},\proc{D}_2^{\text{swap}})$ so we have the original $(\proc{D}_1,\proc{D}_2)$ in the output)

If we substitute the swapped variables in $\proc{S}''$ we get

\begin{itemize}
    \proc{S}'':
        \State $\alpha \gets $choose_alpha()
        \State $(\proc{D}_1,\proc{D}_2) \gets $decisions($\alpha$)
        \State $\proc{Y}_2\gets \text{apply}(\proc{D}_2,\text{patient A})$
        \State $\proc{Y}_1\gets \text{apply}(\proc{D}_1,\text{patient B})$
    \Return $(\alpha,\proc{D}_1,\proc{D}_2,\proc{Y}_1,\proc{Y}_2)$
\end{itemize}

Thus $\proc{S}''$ is exactly the same as $\proc{S}'$, which by assumption is equivalent to the original $\proc{S}$. Thus the assumptions of interchangeable patients and reversible order of treatment application imply the model should commute with exchange.

This argument does \emph{not} hold for scenario 2. The catch is, in the absence of a function $decisions(\alpha)$ which \emph{defines} the procedure for obtaining $\proc{D}_1$ and $\proc{D}_2$, we need to invoke some other measurment procedures for each of these variables.


Assume furthermore that it 

On the other hand, interchanging actions and then interchanging results means we define procedures $. Then we reason using the swapped procedures

\begin{itemize}
    \item Decide on some $\alpha$, which corresponds to a particular $\prob{P}_\alpha^{\RV{D}_1\RV{D}_2}$ and, via determinism, to a pair of decisions $(\proc{D}_1,\proc{D}_2)$
     \item Treat the patient A according to $\proc{D}^{\text{swap}}_1$, record the outcome as $\proc{Y}_1^{\text{swap}}$
    \item Treat the patient B according to $\proc{D}^{\text{swap}}_2$, record the outcome as $\proc{Y}_2^{\text{swap}}$
\end{itemize}

Using $\proc{D}_1'=\proc{D}_2$ and $\proc{D}_2'=\proc{D}_1$ and $\proc{Y}_1^{\text{swap}}=\proc{Y}_2^{\text{swap}}$

\begin{itemize}
    \item Decide on some $\alpha$, which corresponds to a particular $\prob{P}_\alpha^{\RV{D}_1\RV{D}_2}$ and, via determinism, to a pair of decisions $(\proc{D}_1,\proc{D}_2)$
    \item Treat the patient A according to $\proc{D}_2$, record the outcome as $\proc{Y}_2$
    \item Treat the patient B according to $\proc{D}_1$, record the outcome as $\proc{Y}_1$
\end{itemize}

If we also assume that the order of treating patients also doesn't matter, then we can invert the order of the final two steps and obtain a procedure that is equivalent to the original procedure with patients interchanged.

In the first scenario, however, this argument does not go through. Because the procedure is vague as to the precise nature of $\proc{D}_1$ and $\proc{D}_2$ once $\alpha$ is chosen, it is possible that the following procedure is used:

\begin{itemize}
    \item Decide on some ``decision inclination'' $\alpha$
    \item Inspect patient A, then somehow (using $\alpha$) find $\proc{D}_1$
    \item Inspect patient B, then somehow (using $\alpha$) find $\proc{D}_2$
    \item Treat the patient A according to $\proc{D}_1$, record the outcome as $\proc{Y}_1$
    \item Treat the patient B according to $\proc{D}_2$, record the outcome as $\proc{Y}_2$
\end{itemize}

Interchangeability of \emph{patients} suggests the following procedure is equivalent

\begin{itemize}
    \item Decide on some ``decision inclination'' $\alpha$
    \item Inspect patient B, then somehow that depends on $\alpha$ find $\proc{D}_1$
    \item Inspect patient A, then somehow that depends on $\alpha$ find $\proc{D}_2$
    \item Treat the patient B according to $\proc{D}_1$, record the outcome as $\proc{Y}_1$
    \item Treat the patient A according to $\proc{D}_2$, record the outcome as $\proc{Y}_2$
\end{itemize}

Interchaning decisions followed by interchanging outcomes yields the following procedure

\begin{itemize}
    \item Decide on some ``decision inclination'' $\alpha$
    \item Inspect patient A, then somehow (using $\alpha$) find $\proc{D}_1$
    \item Inspect patient B, then somehow (using $\alpha$) find $\proc{D}_2$
    \item Treat the patient A according to $\proc{D}_1^{\text{swap}}$, record the outcome as $\proc{Y}_2^{\text{swap}}$
    \item Treat the patient B according to $\proc{D}_2^{\text{swap}}$, record the outcome as $\proc{Y}_1^{\text{swap}}$
\end{itemize}

Substituting equivalent procedures

\begin{itemize}
    \item Decide on some ``decision inclination'' $\alpha$
    \item Inspect patient A, then somehow (using $\alpha$) find $\proc{D}_1$
    \item Inspect patient B, then somehow (using $\alpha$) find $\proc{D}_2$
    \item Treat the patient A according to $\proc{D}_2$, record the outcome as $\proc{Y}_2$
    \item Treat the patient B according to $\proc{D}_1$, record the outcome as $\proc{Y}_1$
\end{itemize}

Then applying the assumption of reordering

\begin{itemize}
    \item Decide on some ``decision inclination'' $\alpha$
    \item Inspect patient A, then somehow (using $\alpha$) find $\proc{D}_1$
    \item Inspect patient B, then somehow (using $\alpha$) find $\proc{D}_2$
    \item Treat the patient B according to $\proc{D}_1$, record the outcome as $\proc{Y}_1$
    \item Treat the patient A according to $\proc{D}_2$, record the outcome as $\proc{Y}_2$
\end{itemize}

This is not the original procedure with an interchange of patients; in particular, the patients have not been interchanged with regard to their role in selecting the procedures $\proc{D}_1$ and $\proc{D}_2$. Thus the first scenario combined with the assumption of interchangeable patients and indifference to treatment order does \emph{not} imply commutativity of exchange.

A simlar argument that may be more familiar goes: in the first scenario, Alice's decisions $(\proc{D}_1,\proc{D}_2)$ could depend on some unobserved patient characteristics, which could also influence the outcomes $(\proc{Y}_1,\proc{Y}_2)$, so the observed conditional probability $\prob{P}_\alpha^{\RV{Y}_1\RV{Y}_2|\RV{D}_1\RV{D}_2}$ might not correspond to the causal consequence $\prob{P}^{\RV{Y}_1\RV{Y}_2|do(\RV{D}_1,\RV{D}_2)}$. The problem here is that we want to know when we can establish the existence of causal consequences, so an argument that refers to causal consequences does not do us any good for this purpose. Furthermore, as Judea Pearl has often reminded us, it is very difficult to explain the meaning of terms like ``influence'' without invoking causal consequences at some point.

