%!TEX root = main.tex


\section{Potential outcomes}\label{sec:potential_outcomes}

Like causal Bayesian networks, causal models in the potential outcomes framework typically do not include any variables representing what we call ``consequences''. A potential outcomes model features a sequence of observable variables $(\RV{Y}_i,\RV{X}_i,\RV{Z}_i)_{i\in [n]}$ and a collection of potential outcomes $(\RV{Y}^x_i)_{x\in X, i\in[n]}$. Also like causal Bayeisan networks, we think that introducing the idea of consequences clarifies the meaning of potential outcomes models.

We begin with a formal definition of potential outcomes, but as we will discuss this formal definition is not enough on its own to tell us what potential outcomes are. Formally, potential outcomes of $\RV{Y}$ taking values in $Y$ with respect to $\RV{X}$ taking values in $X$ are a variable $\RV{Y}^X$ taking values in $Y^X$ such that $\RV{Y}$ is related to $\RV{Y}^X$ and $\RV{X}$ via a $\emph{selector}$.

\begin{definition}[Selector]
Given variables $\RV{X}:\Omega\to X$ and $\{\RV{Y}^x:\Omega\to Y|x\in X\}$, define $\RV{Y}^X:(\RV{Y}^x)_{x\in X}$. The selector $\pi:X\times Y^X\to Y$ is the function that sends $(x,y^1,...,y^{|X|})\to y^x$.
\end{definition}

\begin{definition}[Potential outcomes: formal requirement]\label{def:potential_outcomes}
Given variables $\RV{Y}:\Omega\to Y$ and $\RV{X}:\Omega\to X$, we introduce a collection of latent variables called \emph{potential outcomes} $\RV{Y}^X:=(\RV{Y}^x)_{x\in X}$ such that $\RV{Y} = \pi\circ(\RV{X},\RV{Y}^X)$. A \emph{potential outcomes model} is any consistent model of $\RV{Y}$, $\RV{X}$ and $\RV{Y}^X$.
\end{definition}

Lemma \ref{lem:po_triviality} shows we can always define trivial potential outcomes of $\RV{Y}$ with respect to $\RV{X}$ by taking the product of $|X|$ copies of $\RV{Y}$. We need some other constraint on the values of potential outcomes besides the formal definition \ref{def:potential_outcomes} if we want them to be informative.

\begin{lemma}[Trivial formal potential outcomes]\label{lem:po_triviality}
For any variables $\RV{Y}:\Omega\to Y$, $\RV{X}:\Omega\to X$ and $\RV{W}:\Omega\to W$, we can always define potential outcomes $\RV{Y}_X$ such that any consistent model $\model{K}^{\RV{YX}|\RV{W}}$ can be extended to a consistent model of $\model{K}^{\RV{YXY}^X|\RV{W}}$.
\end{lemma}

\begin{proof}
Define $\RV{Y}^X:=(\RV{Y})_{x\in X}$. Then we can consistently extend $\model{K}^{\RV{YX}|\RV{W}}$ to $\model{K}^{\RV{YXY}^X|\RV{W}}$ by repeated application of Lemma \ref{lem:nocopy1}.
\end{proof}

The trivial potential outcomes of Lemma \ref{lem:po_triviality} are in many cases unsatisfactory for what we want potential outcomes to represent. Thus Definition \ref{def:potential_outcomes} is incomplete. In common with observable variables, the definition of potential outcomes involves both the formal requirement of Definition \ref{def:potential_outcomes}, and an indication of the parts of the real world that they model. Unlike observable variables, the ``part of the world'' that potential outcomes model will not at any point resolve to a canonical value. We say the potential outcome $\RV{Y}^x:=\pi(x,\RV{Y})$ is ``the value that $\RV{Y}$ would take if $\RV{X}$ were $x$, whether or not $\RV{X}$ actually takes the value $x$''. We will call this additional element of the definition of potential outcomes the \emph{counterfactual extension}.

\begin{definition}[What potential outcomes model: counterfactual extension]\label{def:po_whatmodel}
Given observables $\RV{X}$, $\RV{Y}$ and $\RV{Y}^X$, $\RV{Y}^X$ are potential outcomes if they satisfy Definition \ref{def:potential_outcomes} and for all $x\in X$, the individual potential outcome $\RV{Y}^x:=\pi(x,\RV{Y})$ models the value $\RV{Y}$ would take if $\RV{X}$ took the value $x$.
\end{definition}

Because observables resolve to a single canonical value, the conditional in Definition \ref{def:po_whatmodel} is eventually satisfied for exactly one $x\in X$, at which point $\RV{Y}^{x'}$ for all $x'\neq x$ are guaranteed not to resolve. Nevertheless, we can maybe draw some conclusions about $\RV{Y}^X$ from Definition \ref{def:po_whatmodel}. For example, it seems unreasonable in light of this definition to assert that $\RV{Y}^x$ is \emph{necessarily} identical to $\RV{Y}$ for all $x\in X$, which rules out the strictly trivial potential outcomes of Lemma \ref{lem:po_triviality}.

We will note at this point that if $\RV{X}$ refers to a person's body mass index and $\RV{Y}$ to an indicator of whether or not they experience heart disease, it is metaphysically subtle to say whether $\RV{Y}^X$ is well-defined with regard to Definitions \ref{def:potential_outcomes} and \ref{def:po_whatmodel} together. Recall that there are multiple ways that a given level of body mass index ($\RV{X}$) could be achieved. One might say that, when there are multiple possibile paths, there is no unique way to choose a path. However, a very similar argument can be made that whenever there are multiple possible values of $\RV{Y}^x$ (which is whenever $\RV{X}$ does not take the value $x$), then there is no unique choice of $\RV{Y}^x$, which implies that the full set of potential outcomes $\RV{Y}^X$ is \emph{almost never well-defined}. Alternatively, if there is some method of making a canonical choice of $\RV{Y}^x$, then perhaps this same method can also make a canonical choice of which path was taken to achieve this value of $\RV{X}$.

We will set Definition \ref{def:po_whatmodel} aside and propose an alternative decision-theoretic extension of the definition of potential outcomes. To motivate this proposal, we first note that, if we are using potential outcomes $\RV{Y}^X$ to model an observation of $\RV{X}$ and $\RV{Y}$ only conditional on some hypothesis (or parameter) $\RV{H}$, then by repeated application of Lemma \ref{lem:subm_exist}, we can represent the model $\prob{P}^{\RV{X}\RV{Y}\RV{Y}^X|\RV{H}}$ of these variables as

\begin{align}
    \prob{P}^{\RV{X}\RV{Y}\RV{Y}^X|\RV{H}} &= \tikzfig{po_probmodel} \label{eq:po_probmodel}
\end{align}

For any collection of representative kernels $\kernel{T}^{\RV{Y}^X|\RV{H}}$, $\kernel{T}^{\RV{X}|\RV{Y}^X\RV{H}}$ and $\kernel{T}^{\RV{Y}|\RV{H}\RV{Y}^X\RV{X}}$. We can simplify Equation \ref{eq:po_probmodel} somewhat. Firstly, $\model{P}^{\RV{Y}|\RV{H}\RV{Y}^X\RV{X}}$ must always be represented a \emph{selector kernel} $\Pi:X\times Y^{|X|}\kto Y$, as shown by Lemma \ref{lem:selector}.

\begin{lemma}[Selector kernel]\label{lem:selector}
Let the selector kernel $\Pi:X\times Y^X\kto Y$ be defined by $\Pi_{(x,y^X)}^y=\llbracket \pi(x,y^X)=y \rrbracket$. Given $\RV{X}$, $\RV{Y}$, potential outcomes $\RV{Y}^X$ and arbitrary $\RV{W}$, defining $\kernel{Q}:X\times Y^X\times W\kto Y$ by
\begin{align}
    \kernel{Q} &:= \tikzfig{multiplexer}&\label{eq:multiplex2}\\
    &\iff&\\
    \kernel{Q}^y_{(y^X,x,w)} &= \Pi_{(x,y^X)}^y &\forall y,y^X,x,w\label{eq:multiplex1}
\end{align}
Then any potential outcomes model $\model{T}^{\RV{Y}\RV{Y}^X\RV{X}|\RV{W}}$ must have the property that, for all $x,w,y^X$ and $y$, $\kernel{Q}$ is a representative of $\model{T}^{\RV{Y}|\RV{Y}^X\RV{X}\RV{W}}$.
\end{lemma}

\begin{proof}
Recall $\RV{Y} = \pi\circ(\RV{X},\RV{Y}^X)$. Thus consistency implies that $\RV{Y} \overset{a.s.}{=} \pi\circ(\RV{X},\RV{Y}^X)$ for all $(x, y^X,w)\in \text{Range}(\RV{X})\times \text{Range}(\RV{Y})\times \text{Range}(\RV{W})$ such that $\RV{X}^{-1}(x)\cap (\RV{Y}^X)^{-1}(y^X)\cap \RV{W}^{-1}(w)\neq \emptyset$. However, wherever $\RV{X}^{-1}(x)\cap (\RV{Y}^X)^{-1}(y^X)\cap \RV{W}^{-1}(w)= \emptyset$, consistency implies $\model{T}^{\RV{Y}\RV{Y}^X\RV{X}|\RV{W}}(y,y^X,x|w)=0$ and so $\model{T}^{\RV{Y}|\RV{Y}^X\RV{X}\RV{W}}$ is arbitrary on this collection of values. Equations \ref{eq:multiplex2} and \ref{eq:multiplex1} are equivalent to the statement $\RV{Y} \overset{a.s.}{=} \pi\circ(\RV{X},\RV{Y}^X)$.
\end{proof}

Thus we can without loss of generality choose $\Pi$ to represent $\model{T}^{\RV{Y}|\RV{Y}^X\RV{X}\RV{W}}$. We observe that when \citet{rubin_causal_2005} describes a potential outcomes model, he calls $\model{T}^{\RV{Y}^X|\RV{H}}$ ``the science'' and $\model{T}^{\RV{X}|\RV{H}\RV{Y}^X}$ the ``selection function''. He goes on to explain that the science ``is not affected by how or whether we try to learn about it''.

We propose a definition of potential outcomes that enshrines the stability of ``the science''.

\begin{definition}{Potential outcomes: decision theoretic extension}\label{def:po_dte}
Given a standard decision problem $\{\model{T}^{\RV{WZ}|\RV{HD}},\{\model{S}_\alpha\}_{\alpha\in A}\}$, $\RV{Y}^X$ is a potential outcome for $\RV{Y}$ with respect to $\RV{X}$ if it satisfies Definition \ref{def:potential_outcomes} and is a prechoice variable; that is, $(\RV{Y}^X,\RV{W})\CI_{\model{T}} \RV{D}|\RV{H}$.
\end{definition}

Owing to the subtlety of interpreting Definition \ref{def:po_whatmodel}, we don't know a straightforward argument to the effect that Definition \ref{def:po_dte} is implied by it. Besides the fact that it seems to formalise the idea that the distribution of potential outcomes is unaffected by our actions, we will point out that a key feature of prechoice variables -- decisions can be chosen so that they are random with respect to all prechoice variables -- is used in practice to justify the assumption of ignorability in randomised experiments.

Definition \ref{def:po_dte} can sometimes (but not always) rule out potential outcomes if there is more than one way to achieve a given value of $\RV{X}$. Recall that \citet{hernan_does_2008} argued potential outcomes are ``ill-defined'' in the presence of multiple treatments.

\begin{example}
Suppose we have a standard decision problem $\{\model{T}^{\RV{WZ}|\RV{HD}},\{\model{S}_\alpha\}_{\alpha\in A}\}$ where observations are $\RV{W}$, consequences $\RV{Z}$, hypotheses $\RV{H}$ and decisions $\RV{D}\in \{0,1,2,3\}$. Suppose we also have some $\RV{X}\in \{0,1\}$, $\RV{Y}$ such that $\model{T}^{\RV{X}|\RV{HWD}}(x|h,w,d) = \llbracket x = d\text{ mod }2 \rrbracket$ for all $h,w$ and, for some $y$
\begin{align}
    \model{T}^{\RV{Y}|\RV{HWXD}}(y|h,w,0,0)\neq \model{T}^{\RV{Y}|\RV{HWXD}}(y|h,w,0,2) \label{eq:different_ys}
\end{align}
Then we can consider strategies $\model{S}_0^{\RV{D}|\RV{W}}:=w\mapsto \delta_0$ and $\model{S}_2^{\RV{D}|\RV{W}}:=w\mapsto \delta_2$. By assumption, 
\begin{align}
    \model{P}_0^{\RV{Y}|\RV{HD}}(y|h,0) &=  \sum_{x\in \{0,1\},w\in W}\model{T}^{\RV{W}|\RV{H}}(w|h) \model{S}_0^{\RV{D}|\RV{W}}(0|w) \model{T}^{\RV{X}|\RV{HWD}}(x|h,w,0) \model{T}^{\RV{Y}|\RV{HWXD}}(y|h,w,x,0)\\
    &= \model{T}^{\RV{Y}|\RV{HWXD}}(y|h,w,0,0)\\
    &\neq \model{P}_2^{\RV{Y}|\RV{HD}} \label{eq:po_exist_con1}
\end{align}
Suppose we had some potential outcomes $\RV{Y}^X$ for $\RV{Y}$ with respect to $\RV{X}$. Then, by assumption
\begin{align}
    \model{P}_0^{\RV{Y}|\RV{HD}}(y|h,0) &= \sum_{y^{X}\in Y^2,x\in\{0,1\}} \model{T}^{\RV{Y}^X|\RV{H}}(y^X|h)\model{T}^{\RV{X}|\RV{HDY}^X}(x|h,0,y^X)\Pi(y|x,y^X)\\
                                        &= \sum_{y^X} \model{T}^{\RV{Y}^X|\RV{H}}(y^X|h)\Pi(y|0,y^X)\\
                                        &= \sum_{y^{X}\in Y^2,x\in\{0,1\}} \model{T}^{\RV{Y}^X|\RV{H}}(y^X|h)\model{T}^{\RV{X}|\RV{HDY}^X}(x|h,2,y^X)\Pi(y|x,y^X)\\
                                        &= \model{P}_2^{\RV{Y}|\RV{HD}} \label{eq:po_exist_con2}
\end{align}
Here we use the property $\RV{Y}^X\CI_{\model{T}}\RV{D}|\RV{H}$, implied by the assumption that $\RV{Y}^X$ is a prechoice variable. Equations \ref{eq:po_exist_con1} and \ref{eq:po_exist_con2} are clearly contradictory, thus there can be no potential outcomes $\RV{Y}^X$ in this example.
\end{example}

\todo[inline]{I think I asked the wrong question here -- should've asked when I can extend a see-do model with additonal pre-choice variables. I think it's possible to always choose some deterministic potential outcomes.}

\begin{theorem}[Existence of potential outcomes]
Suppose we have a standard decision problem $\{\model{T}^{\RV{WZ}|\RV{HD}},\{\model{S}_\alpha\}_{\alpha\in A}\}$, and let $\RV{U}$ be the sequence of all prechoice variables. For some $\RV{Y}$ and $\RV{X}$, there exist potential outcomes $\RV{Y}^X$ in the sense of Definition \ref{def:po_dte} if and only if $\model{T}^{\RV{Y}|\RV{UX}}$ exists and is deterministic.
\end{theorem}

\begin{proof}
If:
If $\model{T}^{\RV{Y}|\RV{UX}}$ exists and is deterministic then there exists some $f:U\times X\to Y$ such that $\RV{Y}\overset{a.s.}{=} f\circ(\RV{U},\RV{X})$. Let $\RV{Y}^X:=(f(\RV{U},x))_{x\in X}$. Then $\pi\circ(\RV{X},\RV{Y}^X)=f(\RV{U},\RV{X})\overset{a.s.}{=}\RV{Y}$.

Only if:
By definition, $\RV{Y}^X=g\circ \RV{U}$. From Lemma \ref{lem:selector}, $\model{T}^{\RV{Y}|\RV{XY}^X}$ exists and is deterministic. Thus $\model{T}^{\RV{Y}|\RV{XW}}$ also exists and is also deterministic.
\end{proof}

\begin{corollary}
Potential outcomes $\RV{Y}^X$ in the sense of Definition \ref{def:po_dte} exist only if 
\begin{align}
    \RV{Y}\CI_{\model{T}}\RV{D}|\RV{WX} \label{eq:po_proxycontrol}
\end{align}
\end{corollary}

\begin{proof}
$\model{T}^{\RV{Y}|\RV{UX}}$ exists only if $\RV{Y}\CI_{\model{T}}\RV{D}|\RV{UX}$.
\end{proof}

Note the similarity between Equation \ref{eq:po_proxycontrol} and the condition for proxy control in the previous section. Indeed, the two are identical if we identify $\RV{U}$ with $(\RV{W}_j,\RV{V}_A,\RV{X}_j)$.