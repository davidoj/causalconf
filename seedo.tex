%!TEX root = main.tex

\section{See-do models}\label{sec:seedo_models}

We will first introduce \emph{see-do models} as a type of model that functions as the basic kind of thing which we will use to examine questions in the decision theoretic, potential outcomes and graphical models appraoch.

See-do models can be understood as generalisations of statistical models. Statistical models are a ubiquitous type of model in statistics and machine learning that consist of a set of \emph{states} $S$, and for each state the model prescribes a single probability distribution on a given set of \emph{outcomes} $O$.

\begin{definition}[Statistical model]\label{def:statistical model}
A statistical model is a set of states $S$, a set of outcomes $O$ and a Markov kernel $\kernel{T}:S\to \Delta(O)$.
\end{definition}

For example, a potentially biased coin can be modelled with a statistical model. Suppose the coin has some rate of heads $\theta\in [0,1]$, and we furthermore suppose that for each $\theta$ the result of flipping the coin can be modeled (in some sense) by the probability distribution $\text{Bernoulli}(\theta)$. The statistical model here is the set of states $S=[0,1]$ (corresponding to \emph{rates of heads}), the observation space $O=\{0,1\}^n$ with the discrete sigma-algebra (where $n$ is the number of flips observed) and the stochastic map $\kernel{B}:[0,1]\to \Delta(\mathscr{P}(0,1))$ which is given by $\kernel{B}:\theta\to \text{Bernoulli}(\theta)$.

This example actually goes beyond our formal definitions here in that $\theta$ is real-valued between $0$ and $1$. Extending probability theory to real-valued spaces is well understood, see for example \citet{cinlar_probability_2011}, but in that setting the existence of disintegrations on kernel spaces (section \ref{ssec:disintegration}) is a problem to which we presently only have a partial solution. Discrete sets allow us to discuss see-do models without going into this difficulty. The price we pay is that to properly model the above problem we require $\theta$ to take on discrete values, for example restricting it to the rationals.

A see-do model adds the following structure to a statistical model:

\begin{itemize}
    \item The state is a pair consisting of a \emph{hypothesis} $h\in H$ and a \emph{decision} $d\in D$; $S=H\times D$
    \item The outcome is a pair consisting of an \emph{observation} $x\in X$ and a consequence $y\in Y$
    \item The observation is conditionally independent of the decision given the hypothesis
\end{itemize}

We can use see-do models to model situations where we have some hypotheses and the opportunity to make an observation that takes values in $X$. Depending on what we see, we can select a decision from a set of possibilities $D$, and the ultimate consequence depends probabilistically on the decision we selected as well as whichever hypothesis turns out to best describe the world.

\begin{definition}
A \emph{see-do model} $(\kernel{T},\RV{H},\RV{D},\RV{X},\RV{Y})$ is a Markov kernel space $(\kernel{T},H\times D, O)$ along with four variables: the \emph{hypothesis} $\RV{H}:H\times D\times O\to H$, the \emph{decision} $\RV{D}:H\times D\times O\to D$, the \emph{observation} $\RV{X}: H\times D\times O\to X$ and the \emph{consequence} $\RV{Y}:H\times D\times O\to Y$, all given by the projections onto the respective spaces. In addition, a see-do model must observe the conditional independence:
\begin{align}
\RV{X}\CI_\kernel{T} \RV{D}|\RV{H} \label{eq:see_do_independence_requirement}
\end{align}
\end{definition}

See-do models feature variables $\RV{D}$ and $\RV{H}$ that act like Dawid's ``non-stochastic regime indicators'' described by \citet{dawid_influence_2002,dawid_decision-theoretic_2012,dawid_decision-theoretic_2020}. In particular, see-do models induce a collection of probability measures indexed by the elements of $H\times D$, just as regime indicators induce collections of indexed probability measures. Dawid's regime indicators seem to typically do a similar job to a decision variable $\RV{D}$ rather than a decision-hypothesis pair.

The hypothesis set is similar to the parameter set described by \citet{lattimore_replacing_2019} that relates pre- and post-interventional distributions. Lattimore and Rohde consider models with a prior distribution over this parameter set. A similar type of model can be created by taking the prodcut of a prior over the hypothesis set and a see-do model. See-do models are somewhat similar to the models proposed by \citet{savage_foundations_1954} for decision problems if we identify \emph{states} with \emph{hypotheses} and \emph{acts} with \emph{decisions}. Savage's models consider deterministic rather than stochastic functions from acts to outcomes, and did not explicitly distinguish observations from consequences. Savage's models themselves form the basis of the ``decision theoretic'' approach to causal inference set out by \citet{heckerman_decision-theoretic_1995} (where I use quotes to indicate that there are several distinct ``decision theoretic'' approaches in existence).

\subsection{See-do models for data-driven decision problems}\label{ssec:data_driven_decision}

We can be more precise about the type of decision problem see-do models are appropriate for. 
\begin{theorem}[See-do model representation]\label{th:see_do_rep}
Suppose we have a decision problem that provides us with an observation $x\in X$, and in response to this we can select any decision or stochastic mixture of decisions from a set $D$; that is we can choose a ``strategy'' as any Markov kernel $\kernel{S}:X\to \Delta(D)$. We have a utility function $u:Y\to \mathbb{R}$ that models preferences over the potential consequences of our choice. Furthermore, suppose that we maintain a denumerable set of hypotheses $H$, and under each hypothesis $h\in H$ we model the result of choosing some strategy $\kernel{S}$ as a joint probability over observations, decisions and consequences $\prob{P}_{h,\kernel{S}}\in \Delta(X\times D\times Y)$.

Define $\RV{X},\RV{Y}$ and $\RV{D}$ such that $\RV{X}_{xdy} = x$, $\RV{Y}_{xdy}=y$ and $\RV{D}_{xdy} = d$. Then making the following additional assumptions:
\begin{enumerate}
    \item Holding the hypothesis $h$ fixed the observations as have the same distribution under any strategy: $\prob{P}_{h,\kernel{S}}[\RV{X}]=\prob{P}_{h,\kernel{S}''}[\RV{X}]$ for all $h,\kernel{S},\kernel{S}'$ (observations are given ``before'' our strategy has any effect)
    \item The chosen strategy is a version of the conditional probability of decisions given observations: $\kernel{S}=\prob{P}_{h,\kernel{S}}[\RV{D}|\RV{X}]$
    \item There exists some strategy $\kernel{S}$ that is strictly positive
    \item For any $h\in H$ and any two strategies $\kernel{Q}$ and $\kernel{S}$, we can find versions of each disintegration such that $\prob{P}_{h,\kernel{Q}}[\RV{Y}|\RV{D}\RV{X}]=\prob{P}_{h,\kernel{S}}[\RV{Y}|\RV{D}\RV{X}]$ (our strategy tells us nothing about the consequences that we don't already know from the observations and decisions)
\end{enumerate}

Then there exists a unique see-do model $(\kernel{T},\RV{H}',\RV{D}',\RV{X}',\RV{Y}')$ such that $\prob{P}_{h,\kernel{S}}[\RV{XDY}]^{ijk} = \kernel{T}[\RV{X'}|\RV{H'}]_{h}^{i} \kernel{S}_i^j  \kernel{T}[\RV{Y'}|\RV{X'H'D'}]_{ijk}^k$.
\end{theorem}

One thing that is worth pointing out here is that the property defining observations -- that they are independent of decisions given the hypothesis -- does \emph{not} imply that observations are probabilistically independent of decisions in the decision problem modeled in the manner described above. Instead, it reflects the set of assumptions made above.

\todo[inline]{This will eventually move to an appendix}

\begin{proof}
Consider some probability $\prob{P}\in \Delta(X\times D\times Y)$. By the definition of disintegration (section \ref{ssec:disintegration}), we can write

\begin{align}
 \prob{P}[\RV{XDY}]^{ijk} = \prob{P}[\RV{X}]^i\prob{P}[\RV{D}|\RV{X}]_i^{j} \prob{P}[\RV{Y}|\RV{XD}]_{ij}^{k} \label{eq:disint}
\end{align}

Fix some $h\in H$ and some strictly positive strategy $\kernel{S}$ and define $\kernel{T}:H\times D\to \Delta(X\times Y)$ by
\begin{align}
    \kernel{T}_{hj}^{kl} &= \prob{P}_{h,\kernel{S}}[\RV{X}]^k \prob{P}_{h,\kernel{S}}[\RV{Y}|\RV{X}\RV{D}]^l_{kj} \label{eq:comb_disint}
\end{align}

Note that because $\kernel{S}$ is strictly positive and by assumption $\kernel{S}=\prob{P}_{h,\kernel{S}}[\RV{D}|\RV{X}]$, $\prob{P}_{h,\kernel{S}}[\RV{D}]$ is also strictly positive. Therefore $\prob{P}_{h,\kernel{S}}[\RV{Y}|\RV{D}]$ is unique and therefore $\kernel{T}$ is also unique.

Define $\RV{X}'$ and $\RV{Y}'$ by $\RV{X}'_{xy}=x$ and $\RV{Y}'_{xy}=y$. Define $\RV{H}'$ and $\RV{D}'$ by $\RV{H}'_{hd} = h$ and $\RV{D}'_{hd} = d$.

We then have
\begin{align}
    \kernel{T}[\RV{X'}|\RV{H'D'}]_{hj}^{k} &= \kernel{T}\underline{\RV{X}'}_{hj}^k\\
                                           &= \sum_l \kernel{T}_{hj}^{kl} \\
                                           &= \prob{P}_{h,\kernel{S}}[\RV{X}]^k\\
                                           &= \kernel{T}[\RV{X'}|\RV{H'D'}]_{hj'}^{k}
\end{align}

Thus $\RV{X}'\CI_{\kernel{T}} \RV{D}'|\RV{H}'$ and so $\kernel{T}[\RV{X}'|\RV{H}']$ exists (section \ref{ssec:cond_indep}) and $(\kernel{T},\RV{H}',\RV{D}',\RV{X}',\RV{Y}')$ is a see-do model.

Applying Equation \ref{eq:disint} to $\prob{P}_{h,\kernel{S}}$:

\begin{align}
    \prob{P}_{h,\kernel{S}}[\RV{XDY}]^{ijk} &= \prob{P}_{h,\kernel{S}}[\RV{X}]^i\prob{P}_{h,\kernel{S}}[\RV{D}|\RV{X}]_i^{j} \prob{P}_{h,\kernel{S}}[\RV{Y}|\RV{XD}]_{ij}^{k}\label{eq:t_is_comb_disint_start}\\
     &=  \prob{P}_{h,\kernel{S}}[\RV{X}]^i\prob{P}_{h,\kernel{S}}[\RV{Y}|\RV{XD}]_{ij}^{k}\\
     &= \prob{P}_{h,\kernel{S}}[\RV{D}|\RV{X}]_i^{j} \kernel{T}[\RV{X'Y'}|\RV{H'D'}]_{hj}^{ik}\\
     &= \kernel{S}_i^j \kernel{T}[\RV{X'Y'}|\RV{H'D'}]_{hj}^{ik}\\
     &= \kernel{S}_i^j \kernel{T}[\RV{X'}|\RV{H'D'}]_{hj}^{i} \kernel{T}[\RV{Y'}|\RV{X'H'D'}]_{ihj}^k\\
     &= \kernel{T}[\RV{X'}|\RV{H'}]_{h}^{i} \kernel{S}_i^j  \kernel{T}[\RV{Y'}|\RV{X'H'D'}]_{ihj}^k\label{eq:t_is_comb_disint_end}
\end{align}

Consider some arbitrary alternative strategy $\kernel{Q}$. By assumption

\begin{align}
    \prob{P}_{h,\kernel{S}}[\RV{X}]^{i} &= \prob{P}_{h,\kernel{Q}}[\RV{X}]^{i}\\
    \prob{P}_{h,\kernel{S}}[\RV{Y}|\RV{XD}]_{ij}^k &= \prob{P}_{h,\kernel{Q}}[\RV{Y}|\RV{XD}]_{ij}^k\text{ for some version of }\prob{P}_{h,\kernel{Q}}[\RV{Y}|\RV{XD}]
\end{align}

It follows that, for some version of $\prob{P}_{h,\kernel{Q}}[\RV{Y}|\RV{XD}]$,
\begin{align}
    \kernel{T}_{hj}^{kl} &= \prob{P}_{h,\kernel{Q}}[\RV{X}]^k \prob{P}_{h,\kernel{Q}}[\RV{Y}|\RV{X}\RV{D}]^l_{kj} \label{eq:comb_disint_nonuniq}
\end{align}

Then by substitution of $\kernel{Q}$ for $\kernel{S}$ in Equation \ref{eq:t_is_comb_disint_start} and working through the same steps

\begin{align}
    \prob{P}_{h,\kernel{S}}[\RV{XDY}]^{ijk} &= \kernel{T}[\RV{X'}|\RV{H'}]_{h}^{i} \kernel{Q}_i^j  \kernel{T}[\RV{Y'}|\RV{X'H'D'}]_{ihj}^k
\end{align}

As $\kernel{Q}$ was arbitrary, this holds for all strategies.
\end{proof}