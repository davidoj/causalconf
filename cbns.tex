%!TEX root = main.tex


\section{Causal Bayesian Networks}\label{sec:CBN}

Like some of the causal modelling frameworks discussed in the previous section, including see-do models, Causal Bayesian Networks (CBNs) represent both ``observations'' and ``consequences of interventions''. It seems reasonable to think that the real-world things that the see-do framework and the CBN framework address are sometimes the same. The question we have here is: if we have a decision problem represented by a see-do model, when can we represent the same problem with a CBN?

In order to answer this question, we have to deal with the fact that neither theory is formally contained by the other, so for example there's no precise way in which decisions correspond to interventions. The correspondence exists in the territory, the world that is inhabited by measurement processes, not the mathematical world that is inhabited by random variables. We therefore have to make some choices about what corresponds to what that seem to be reasonable given our understanding of what these models are used for.

To compare CBNs to see-do models, we will argue that CBNs can be understood as describing probabilistic models of observations and consequences, just like see-do models. Furthermore, CBNs feature an order-1 probability gap and so they describe a probability 2-comb over observations, interventions and consequences. If we suppose that there is some variable describing decisions that does not appear within the CBN, then we can posit a see-do model over observations, decisions and consequences. Finally, we ask: when is the see-do model compatible with the CBN 2-comb, or more precisely, when can we identify each \emph{decision rule} with a \emph{intervention rule} such that the probability model obtained by inserting a decision rule into the see-do model is identical to the probability model obtained by inserting an intervention rule into the CBN 2-comb. We show that see-do models that exhibit a particular type of symmetry are compatible with CBN 2-combs.

\subsection{Probability 2-combs represented by causal Bayesian networks}

Consider a simplified kind of CBN where a single variable may be intervened on. Note that the structure of the previous section -- $\RV{X}\longrightarrowRHD \RV{Y}$ and $\RV{X}\longleftarrowRHD \RV{W} \longrightarrowRHD \RV{Y}$ -- is generically appliccable to such a model if we identify $\RV{W}$ with the variable formed by taking a sequence of all of the ancestors of $\RV{X}$ and $\RV{Y}$ with the variable formed by taking a sequence of all non-ancestors of $\RV{X}$. The existence of an edge from $\RV{X}$ to $\RV{Y}$ in such a case does no harm as if $\RV{Y}$ is not ``actually'' a descendent of $\RV{X}$ then it will be independent conditional on $\RV{Z}$ (see \citet{pearl_book_2018} for a detailed treatment of when two graphs may or may not imply the same underlying model).

We will adopt the ``definitional inversion'' discussed in Section \ref{sec:truncated_fac_again}, where we take the causal Bayesian network to express the assumption that the result of intervention is modeled by the probability gap model $\prob{Q}^{\RV{W}\square \RV{Y}|\RV{X}}$ and the observations are distributed according to $\text{insert}(\disint{Q}_{\text{obs}}^{\RV{X}|\RV{Z}},\prob{Q}^{\RV{W}\square\RV{Y}|\RV{X}})$ for some $\disint{Q}_{\text{obs}}^{\RV{X}|\RV{Z}}$.

We also want to define the fundamental probability set and the variables that we are modelling with the causal Bayesian network. We are actually interested in modelling a sequence of ``observation'' variables $\RV{V}_{[n]}:=(\RV{W}_i,\RV{X}_i,\RV{Y}_i)_{i\in [n]}$ and a sequence of ``consequence'' variables modeled by $\RV{V}_{(n,m]}:=(\RV{W}_i,\RV{X}_i,\RV{Y}_i)_{i\in (n,m]}$ both defined on the probability set $\Omega$ (assume $n<m$). We suppose that a causal Bayesian network defines a probability gap model $\model{Q}$ of some type for these variables. Suppose for all $i$, $\RV{W}_i:\Omega\to W$, $\RV{X}_i:\Omega\to X$, $\RV{Y}_i:\Omega\to Y$.

Beyond the choice of intervention, when we have a causal Bayesian network there is a second gap in our knowledge -- in particular, we do not know what $\disint{Q}_{\text{obs}}^{\RV{X}|\RV{Z}}$ and $\prob{Q}^{\RV{W}\square\RV{Y}|\RV{X}}$ are. We address this ignorance of the ``correct'' probability model as we did in Section \ref{sec:seedo_models} by introducing an unobserved hypothesis $\RV{H}$. We further assume that, conditional on $\RV{H}$, observations are mutually independent; that is, $\RV{V}_i\CI_{\model{Q}} \RV{V}_{[n]\setminus \{i\}}|\RV{H}$ for $i\in [n]$. When we say ``observations are distributed according to $\disint{Q}_{\text{obs}}^{\RV{W_1X_1Y_1}|\RV{H}}$'', we mean that each $\RV{V}_i$ $i\in [n]$ is distributed according to $\model{Q}_{\text{obs}}^{\RV{W_1X_1Y_1}|\RV{H}}$ (we add the subscripts to avoid ambiguity over what the subscript-less variables refer to). Furthermore, with subscripts in place, the assumption relating observations to consequences reads

\begin{align}
    \disint{Q}_{\text{obs}}^{\RV{W_1X_1Y_1}|\RV{H}} &= \text{insert}(\disint{Q}_{\text{obs}}^{\RV{X}_{n+1}|\RV{HW}_{n+1}},\prob{Q}^{\RV{W}_{n+1}\square\RV{Y}_{n+1}|\RV{X}_{n+1}})
\end{align}

We make similar ``mutually independent and identially distributed'' assumptions for consequences, but they are a bit more subtle to state. We assume that, conditional on $\RV{H}$, $\RV{W}_j$ are mutually independent for all $j\in [m]$ and conditional on $\RV{H}$ and $\RV{X}_j$, $\RV{Y}_j$ are mutually independent for all $j\in [m]$. That is, $\RV{W}_j\CI_{\model{Q}} \RV{V}_{[m]\setminus \{j\}}|\RV{H}$ and $\RV{Y}$ and $\RV{Y}_j\CI_{\model{Q}} \RV{V}_{[m]\setminus\{j\}}|(\RV{H},\RV{X}_j)$. Furthermore, when we say ``$\prob{Q}^{\RV{W}_1|\RV{H}\square\RV{Y}_1|\RV{X}_1}$ is a model of interventions'' we mean, given an insert $\disint{Q}_{\alpha}^{\RV{X}_j|\RV{W}_j\RV{H}}$, $\RV{V}_j$ will be distributed according to $\text{insert}(\disint{Q}_{\alpha}^{\RV{X}_{n+1}|\RV{W}_{n+1}\RV{H}},\prob{Q}^{\RV{W}_{n+1}|\RV{H}\square\RV{Y}_{n+1}|\RV{X}_{n+1}})$.

There seems to be one outstanding issue here: in general, we are interested in decision rules influencing $\RV{X}_j$ that may depend on the observations $\RV{V}_{[n]}$ in addition to possible dependence on $\RV{H}$ and $\RV{W}_j$, but we have only defined how consequences will be distributed given a subset of these inserts -- namely, those that have no dependence on $\RV{V}_{[n]}$. 

However, by the mutual independence assumptions we make of consequences, we have for \emph{any} insert $\alpha$

\begin{align}
    \text{insert}(\disint{Q}_{\alpha}^{\RV{X}_{n+1}|\RV{HW}_{n+1}\RV{V}_{[n]}},\prob{Q}^{\RV{W}_{n+1}\RV{V}_{[n]}|\RV{H}\square\RV{Y}_{n+1}|\RV{X}_{n+1}}) &= \disint{Q}^{\RV{W}_{n+1}\RV{V}_{[n]}|\RV{H}_{n+1}}\odot \disint{Q}_{\alpha}^{\RV{X}_{n+1}|\RV{HW}_{n+1}\RV{V}_{[n]}}\odot \model{Q}^{\RV{Y}_{n+1}|\RV{HW}_{n+1}\RV{X}_{n+1}\RV{V}_{[n}}\\
\end{align}

\begin{align}
    \disint{Q}^{\RV{W}_j|\RV{X}_j\RV{H}}(w|a,h) &= \prob{Q}^{\RV{W}_1|\RV{H}}(w|h)\\
    \implies \disint{Q}^{\RV{W}_j|\RV{X}_j\RV{H}} = \prob{Q}^{\RV{W}_1|\RV{H}}\otimes \text{erase}_{X}
\end{align}

Thus there exists $\disint{Q}^{\RV{W}_j|\RV{H}}$ for $j\in (n,m]$ and

\begin{align}
    \disint{Q}^{\RV{W}_j|\RV{X}_j} &= \prob{Q}^{\RV{W}_1|\RV{H}}
\end{align}

And by disintegration of $\disint{Q}^{\RV{W}_j\RV{X}_j\RV{Y}_j|\RV{X}_j\RV{H}}$ (and Theorem \ref{th:recursive-disint})

\begin{align}
    \disint{Q}^{\RV{Y}_j|\RV{X}_j\RV{W}_j\RV{H}} &= \disint{Q}^{\RV{Y}_1|\RV{X_1W_1H}}
\end{align}

Thus the probability gap model $\prob{Q}$ defined by the CBN features a collection of conditional probabilities:

\begin{itemize}
    \item $\disint{Q}^{\RV{V}_{i}|\RV{H}}$, $i\in [n]$
    \item $\disint{Q}^{\RV{W}_i|\RV{H}}$, $i\in [m]$
    \item $\disint{Q}^{\RV{Y}_i|\RV{X}_i\RV{W}_i\RV{H}}$, $i\in [m]$
\end{itemize}

To this, we add the assumption of mutual conditional independence: $\RV{V}_i\CI_{\prob{Q}} \RV{V}_{[m]\setminus\{i\}}|\RV{H}$. Then we have

\todo[inline]{Representation of conditional IID models; can I just quote something here?}

\begin{align}
    \disint{Q}^{\RV{V}_{[n]}\RV{W}_{(n,m]}|\RV{H}} &= \text{copy}^m_H\left((\bigotimes_{i\in[n]} \disint{Q}^{\RV{V}_{i}|\RV{H}}) \otimes (\bigotimes \disint{Q}^{\RV{W}_i|\RV{H}})\right)\\
    \disint{Q}^{\RV{Y}_{(n,m]}|\RV{X}_{(n,m]}\RV{W}_{(n,m]}\RV{V}_{[n]}\RV{H}} &= \text{copy}^{m-n}_H\otimes \text{id}_{XWV}(\disint{Q}^{\RV{Y}_i|\RV{X}_i\RV{W}_i\RV{H}}\otimes \text{erase}_{V^n})
\end{align}

\begin{align}
    \prob{Q}^{\RV{Y}_i|\RV{H}\RV{X}_i\RV{V}_{[n]}\RV{W}_i} &= \tikzfig{cond_indep_disint_Qcbn}
\end{align}

At this point, we note that $\disint{Q}^{\RV{V}_{[n]}\RV{W}_{(n,m]}|\RV{H}}$ and $\disint{Q}^{\RV{Y}_{(n,m]}|\RV{X}_{(n,m]}\RV{W}_{(n,m]}\RV{V}_{[n]}\RV{H}}$ together define a probability 2-comb $\disint{Q}$, which motivates the following definition

\begin{definition}[CBN probability 2-comb]
A CBN probability 2-comb is a probability 2-comb $\prob{Q}^{\RV{V}_{[n]}\RV{W}_{(n,m]}|\RV{H}\square \RV{Y}_{(n,m]}|\RV{X}_{(n,m]}}$ where $\RV{V}_{i} = (\RV{W}_i,\RV{X}_i,\RV{Y}_i)$ for $i\in [m]$, such that $\RV{V}_i\CI^2_{\prob{Q}} \RV{V}_{[m]\setminus\{i\}} |\RV{H}$, for all $i\in [m]$ 
\begin{align}
    \prob{Q}^{\RV{V}_{i}|\RV{H}} = \prob{Q}^{\R{V}_j|\RV{H}}
\end{align}
for $i,j\in [n]$,
\begin{align}
    \prob{Q}^{\RV{W}_i|\RV{H}} = \prob{Q}^{\RV{W}_j|\RV{H}}
\end{align}
for $i,j\in\[m]$ and there exists some $\disint{Q}^{\RV{Y}_j|\RV{X}_j\RV{W}_j\RV{H}}$, $j\in [n]$ such that
\begin{align}
    \disint{Q}^{\RV{Y}_i|\RV{X}_i\RV{W}_i\RV{H}\RV{V}_{<i}} &= \disint{Q}^{\RV{Y}_j|\RV{X}_j\RV{W}_j\RV{H}}\otimes \text{erase}_{V^{i-1}}
\end{align}
For all $i\in (n,m]$, some $\disint{Q}^{\RV{Y}_i|\RV{X}_i\RV{W}_i\RV{H}\RV{V}_{<i}}$.
\end{definition}

Inserts for a CBN 2-comb take the form $\prob{Q}_\alpha^{\RV{X}_{(n,m]}|\RV{V}_{[n]}\RV{W}_{(n,m]}\RV{H}}$. Such inserts are not necessarily decision rules as defined in the previous section -- they don't necessarily admit an interpretation ``if I see $(\RV{V}_{[n]},\RV{W}_{(n,m]},\RV{H})$ then I do this to $\RV{X}_{(n,m]}$'', not the least because $\RV{H}$ is by definition unobserved. The question we will now ask is: under what conditions can we specify a see-do model for which the inserts \emph{do} admit such an interpretation as decision rules and each decision rule corresponds to some insert $\prob{Q}_\alpha$ in our CBN.

\subsection{See-do models compatible with causal Bayesian networks}

When does a see-do model $\prob{T}^{\RV{V}_{[n]}|\RV{H}\square \RV{V}_{(n,m]}|\RV{D}}$ with decision rules $\{\prob{T}_\alpha^{\RV{D}|\RV{V}_{[n]}}\}_{\alpha\in A}$ correspond to a CBN probability 2-comb $\prob{Q}^{\RV{V}_{[n]}\RV{W}_{(n,m]}|\RV{H}\square \RV{Y}_{(n,m]}|\RV{X}_{(n,m]}}$ with inserts $\{\prob{Q}_\alpha^{\RV{X}_{(n,m]}|\RV{V}_{[n]}\RV{W}_{(n,m]}\RV{H}}\}_{\alpha\in A}$? 

By correspondence, we mean that for each $\alpha\in A$, the probabilistic model given by $\text{insert}(\prob{T}^{\RV{V}_{[n]}|\RV{H}\square \RV{V}_{(n,m]}|\RV{D}},\prob{T}_\alpha^{\RV{D}|\RV{V}_{[n]}})$ followed by marginalising over $\RV{D}$ is the same as the model given by $\text{insert}(\prob{Q}^{\RV{V}_{[n]}\RV{W}_{(n,m]}|\RV{H}\square \RV{Y}_{(n,m]}|\RV{X}_{(n,m]}},\prob{Q}_\alpha^{\RV{X}_{(n,m]}|\RV{V}_{[n]}\RV{W}_{(n,m]}\RV{H}})$
\begin{align}
    \model{T}^{\RV{V}_{[m]}|\RV{H}}_\alpha&:=\tikzfig{seedo_equality2}\\
    &=\tikzfig{seedo_equality} \label{eq:consistent}\\
    &=: \model{Q}^{\RV{V}_{[m]}|\RV{H}}_\alpha
\end{align}

\begin{theorem}\label{th:seedo_rep}
Given a see-do model $\prob{T}^{\RV{V}_{[n]}|\RV{H}\square \RV{V}_{(n,m]}|\RV{D}}$ there exists a corresponding CBN probability 2-comb $\prob{Q}^{\RV{V}_{[n]}\RV{W}_{(n,m]}|\RV{H}\square \RV{Y}_{(n,m]}|\RV{X}_{(n,m]}}$ if and only if
\begin{enumerate}
    \item $(\RV{V}_{[n]},\RV{W}_j)\CI^2_{\model{T}} \RV{D}|\RV{H}$
    \item $\model{T}^{\RV{V}_{[n]}\RV{W}_j|\RV{H}} = \model{U}^{\RV{V}_{[n]}\RV{W}_j|\RV{H}}$
    \item $\RV{Y}_j\CI^2_{\model{T}}\RV{D}|\RV{W}_j\RV{HX}_j$
    \item $\model{T}^{\RV{Y}_j|\RV{W}_j\RV{HX}_j} = \model{T}^{\RV{Y}_i|\RV{W}_i\RV{HX}_i}$ for $i\in [n]$, $j\in (n,m]$
\end{enumerate}
\end{theorem}

\begin{proof}
\textbf{If:}
If all assumptions hold, we can write
\begin{align}
    \model{T}^{\RV{V}_{[n]}\RV{V}_j|\RV{HD}} = \tikzfig{t_vs_u}
\end{align}
For each $\model{S}_\alpha^{\RV{D}|\RV{V}_{[n]}}$, define
\begin{align}
    \model{R}_\alpha^{\RV{X}_j|\RV{V}_{[n]}\RV{W}_j\RV{H}}:= \tikzfig{defn_ra}
\end{align}
Then
\begin{align}
    &\tikzfig{seedo_equality2}\\
    &= \tikzfig{seedo_cbn_with_s}\\
    &= \tikzfig{seedo_equality}
\end{align}
\textbf{Only if:}
Suppose assumption 1 does not hold. Then there exists some $d,d'\in D$, $w\in W$, $h\in H$ such that $\model{T}^{\RV{W}_j|\RV{HD}}(w_j|h,d)\neq \model{T}^{\RV{W}_j|\RV{HD}}(w|h,d')$. Then choose $\model{S}_d^{\RV{D}|\RV{V}_{[n]}}:v_A\mapsto \delta_{d}$ and $\model{S}_{d'}^{\RV{D}|\RV{V}_{[n]}}:v\mapsto \delta_{d'}$ for all $v\in V^{|A|}$. Then define
\begin{align}
    \model{P}_d^{\RV{W}_j|\RV{H}}(w|h) &= \model{T}^{\RV{W}_j|\RV{HD}}(w_j|h,d)\\
                                       &\neq \model{T}^{\RV{W}_j|\RV{HD}}(w_j|h,d')\\
                                       &= \model{P}_{d'}^{\RV{W}_j|\RV{H}}(w|h)
\end{align}
But for any $\alpha, \alpha'$, $\model{Q}_\alpha^{\RV{W}_j|\RV{H}}=\model{Q}_{\alpha'}^{\RV{W}_j|\RV{H}}$ as $\RV{W}_j\CI_{\RV{U}} \RV{X}_j|\RV{H}$, so $\model{Q}\neq \model{P}$.
Suppose assumption 1 holds but assumption 2 does not. Then for any $\alpha$
\begin{align}
    \model{P}_\alpha^{\RV{V}_{[n]}\RV{W}_j|\RV{H}}&=\model{T}^{\RV{V}_{[n]}\RV{W}_j|\RV{H}}\\
                                              &\neq \model{U}^{\RV{V}_{[n]}\RV{W}_j|\RV{H}}\\
                                              &= \model{Q}_\alpha^{\RV{V}_{[n]}\RV{W}_j|\RV{H}}
\end{align}
Suppose assumption 3 does not hold. Then there is some $d,d'\in D$, $w\in W$, $h\in H$, $v\in V^{|A|}$, $x\in X$ and $y\in Y$ such that
\begin{align}
    \model{T}^{\RV{Y}_j|\RV{W}_j\RV{V}_{[n]}\RV{HX}_j\RV{D}}(y|w,v,h,x,d) &\neq \model{T}^{\RV{Y}_j|\RV{W}_j\RV{V}_{[n]}\RV{HX}_j\RV{D}}(y|w,v,h,x,d')\label{eq:not_indep}\\
    &\text{and }\model{T}^{\RV{X}_j\RV{W}_j\RV{V}_{[n]}|\RV{HD}}(x,w,v|h,d) >0\\
    &\text{and }\model{T}^{\RV{X}_j\RV{W}_j\RV{V}_{[n]}|\RV{HD}}(x,w,v|h,d') >0\\
\end{align}
The latter conditions hold as if Equation \ref{eq:not_indep} only held on sets of measure 0 then we could choose versions of the conditional probabilities such that the independence held.

Then
\begin{align}
    \model{P}_d^{\RV{Y}_j|\RV{W}_j\RV{V}_{[n]}\RV{HX}_j\RV{D}}(y|w,v,h,x)&= \model{T}^{\RV{Y}_j|\RV{W}_j\RV{V}_{[n]}\RV{HX}_j\RV{D}}(y|w,v,h,x,d)\\
    &\neq \model{T}^{\RV{Y}_j|\RV{W}_j\RV{V}_{[n]}\RV{HX}_j\RV{D}}(y|w,v,h,x,d')\\
    &= \model{P}_{d'}^{\RV{Y}_j|\RV{W}_j\RV{V}_{[n]}\RV{HX}_j\RV{D}}(y|w,v,h,x)\\
    \implies \model{P}_d^{\RV{Y}_j|\RV{W}_j\RV{V}_{[n]}\RV{HX}_j\RV{D}}(y|w,v,h,x) &\neq \model{Q}_d^{\RV{Y}_j|\RV{W}_j\RV{V}_{[n]}\RV{HX}_j\RV{D}}(y|w,v,h,x)\\
    \text{or } \model{P}_{d'}^{\RV{Y}_j|\RV{W}_j\RV{V}_{[n]}\RV{HX}_j\RV{D}}(y|w,v,h,x) &\neq \model{Q}_{d'}^{\RV{Y}_j|\RV{W}_j\RV{V}_{[n]}\RV{HX}_j\RV{D}}(y|w,v,h,x)
\end{align}
As the conditional probabilities disagree on a positive measure set, $\model{P}\neq\model{Q}$.

Suppose assumption 3 holds but assumption 4 does not. Then for some $h\in H$, some $w\in W$, $v\in V^{|A|}$, $x\in X$ with positive measure and some $y\in Y$
\begin{align}
    \model{P}_d^{\RV{Y}_j|\RV{W}_j\RV{V}_{[n]}\RV{HX}_j\RV{D}}(y|w,v,h,x)&= \model{T}^{\RV{Y}_j|\RV{W}_j\RV{V}_{[n]}\RV{HX}_j}(y|w,v,h,x)\\
    &\neq \model{U}^{\RV{Y}_j|\RV{W}_j\RV{V}_{[n]}\RV{HX}_j}(y|w,v,h,x)\\
    &\neq model{Q}_d^{\RV{Y}_j|\RV{W}_j\RV{V}_{[n]}\RV{HX}_j\RV{D}}(y|w,v,h,x)
\end{align}
\end{proof}

Conditional independences like $(\RV{V}_{[n]},\RV{W}_j)\CI_{\model{T}} \RV{D}|\RV{H}$ and $\RV{Y}_j\CI_{\model{T}}\RV{D}|\RV{W}_j\RV{V}_{[n]}\RV{HX}_j$ bear some resemblance to the condition of ``limited unresponsiveness'' proposed by \citet{heckerman_decision-theoretic_1995}. They are conceptually similar in that they indicate that a particular variable does not ``depend on'' a decision $\RV{D}$ in some sense. As Heckerman points out, however, limited unresponsiveness is not equivalent to conditional independence. We tentatively speculate that there may be a relation between our ``pre-choice variables'' $(\RV{W}_j,\RV{V}_{[n]},\RV{H})$ and the ``state'' in Heckerman's work crucial for defining limited unresponsiveness.

\subsection{Proxy control}

We say that $(\RV{V}_{[n]},\RV{W}_j)\CI_{\model{T}} \RV{D}|\RV{H}$ expresses the notion that $\RV{W}_j$ is a \emph{pre-choice variable} and $(\RV{W}_j,\RV{V}_{[n]},\RV{X}_j)$ are \emph{proxies for }$\RV{D}$ with respect to $\RV{Y}$ under conditions of full information. To justify this terminology, we note that under a strong assumption of identifiability $\RV{Y}_j\CI\RV{H}|\RV{W}_j\RV{V}_{[n]}\RV{X}_j$ (i.e. the observed data allow us to identify $\RV{H}$ for the purposes of determining $\RV{T}^{\RV{Y}_j|\RV{W}_j\RV{V}_{[n]}\RV{X}_j\RV{H}}$), then we can write
\begin{align}
    \model{T}^{\RV{V}_{[n]}\RV{V}_{(n,m]}|\RV{H}\RV{D}} &=\tikzfig{strong_identifiability}\\
                                              &=\tikzfig{strong_identifiability2}
                                              &= \model{T}^{\RV{V}_{[n]}\RV{W}_j\RV{X}_j|\RV{H}\RV{D}}\kernel{M}
\end{align}

That is, under conditions of full information, knowing how to control the proxies $(\RV{W}_j,\RV{V}_{[n]},\RV{X}_j)$ is sufficient to control $\RV{Y}$. This echoes \citet{pearl_does_2018}'s view on causal effects representing ``stable characteristics'':
\begin{quote}
Smoking cannot be stopped by any legal or educational means available to us today; cigarette advertising can. That does not stop researchers from aiming to estimate ``the effect of smoking on cancer,'' and doing so from experiments in which they vary the instrument—cigarette advertisement—not smoking. The reason they would be interested in the atomic intervention $P(\text{cancer}|do(\text{smoking}))$ rather than (or in addition to) $P(\text{cancer}|do(\text{advertising}))$ is that the former represents a stable biological characteristic of the population, uncontaminated by social factors that affect susceptibility to advertisement, thus rendering it transportable across cultures and environments. With the help of this stable characteristic, one can assess the effects of a wide variety of practical policies, each employing a different smoking-reduction instrument.
\end{quote}