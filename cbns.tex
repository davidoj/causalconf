%!TEX root = main.tex


\section{Causal Bayesian Networks}\label{sec:CBN}

Like some of the causal modelling frameworks discussed in the previous section, including see-do models, Causal Bayesian Networks (CBNs) represent both ``observations'' and ``consequences of interventions''. It seems reasonable to think that the real-world things that the see-do framework and the CBN framework address are sometimes the same. The question we have here is: if we have a decision problem represented by a see-do model, when can we represent the same problem with a CBN?

In order to answer this question, we have to deal with the fact that neither theory is formally contained by the other, so for example there's no precise way in which decisions correspond to interventions. The correspondence exists in the territory, the world that is inhabited by measurement processes, not the mathematical world that is inhabited by random variables. We therefore have to make some choices about what corresponds to what that seem to be reasonable given our understanding of what these models are used for.

To compare CBNs to see-do models, we will argue that CBNs can be understood as describing probabilistic models of observations and consequences, just like see-do models. Furthermore, CBNs feature an order-1 probability gap and so they describe a probability 2-comb over observations, interventions and consequences. If we suppose that there is some variable describing decisions that does not appear within the CBN, then we can posit a see-do model over observations, decisions and consequences. Finally, we ask: when is the see-do model compatible with the CBN 2-comb, or more precisely, when can we identify each \emph{decision rule} with a \emph{intervention rule} such that the probability model obtained by inserting a decision rule into the see-do model is identical to the probability model obtained by inserting an intervention rule into the CBN 2-comb. We show that see-do models that exhibit a particular type of symmetry are compatible with CBN 2-combs.

\subsection{Probability 2-combs represented by causal Bayesian networks}

Consider a simplified kind of CBN where a single variable may be intervened on. Note that the structure of the previous section -- $\RV{X}\longrightarrowRHD \RV{Y}$ and $\RV{X}\longleftarrowRHD \RV{W} \longrightarrowRHD \RV{Y}$ -- is generically appliccable to such a model if we identify $\RV{W}$ with the variable formed by taking a sequence of all of the ancestors of $\RV{X}$ and $\RV{Y}$ with the variable formed by taking a sequence of all non-ancestors of $\RV{X}$. The existence of an edge from $\RV{X}$ to $\RV{Y}$ in such a case does no harm as if $\RV{Y}$ is not ``actually'' a descendent of $\RV{X}$ then it will be independent conditional on $\RV{Z}$ (see \citet{peters_structural_2015} for a detailed treatment of when two graphs may or may not imply the same underlying model).

We will adopt the definiton discussed in Section \ref{sec:truncated_fac_again}: we take the causal Bayesian network in question to express the assumption that the result of intervention is modeled by some probability 2-comb $\prob{P}^{\RV{W}\square \RV{Y}|\RV{X}}$ and the observations are distributed according to $\text{insert}(\model{P}_{\text{obs}}^{\RV{X}|\RV{Z}},\prob{P}^{\RV{W}\square\RV{Y}|\RV{X}})$ for some $\model{P}_{\text{obs}}^{\RV{X}|\RV{Z}}$.

We are uncertain about the particular 2-comb $\prob{P}^{\RV{W}\square \RV{Y}|\RV{X}}$ that we should use to model interventions, as well as the particular $\model{P}_{\text{obs}}^{\RV{X}|\RV{Z}}$ appropriate for observations, so we will represent this uncertainty with an unobserved variable $\RV{H}$. Furthermore, if we are being precise about what is being modeled, we suppose that we have a sequence of ``observation'' variables $\RV{V}_{[n]}:=(\RV{W}_i,\RV{X}_i,\RV{Y}_i)_{i\in [n]}$ and a sequence of ``consequence'' variables modeled by $\RV{V}_{(n,m]}:=(\RV{W}_i,\RV{X}_i,\RV{Y}_i)_{i\in (n,m]}$ both defined on a fundamental probability set $\Omega$ (assume $n<m$). 

With this additional detail, we interpret Equation \ref{eq:int_to_obs} as saying

\begin{align}
    \model{Q}_{\text{obs}}^{\RV{W_iX_iY_i}|\RV{H}} &= \text{insert}(\model{Q}_{\text{obs}}^{\RV{X}_{j}|\RV{HW}_{j}},\prob{Q}^{\RV{W}_{j}|\RV{H}\square\RV{Y}_{j}|\RV{X}_{j}})
\end{align}

for all $i\in[n]$, $j\in (n,m]$.

Because we are now considering sequences of observations and consequences, we also think it is reasonable to understand a CBN model as coming equipped with assumptions of mutual independence:

\begin{align}
    \RV{V}_i\CI_{\prob{Q}} \RV{V}_{[m]\setminus \{i\}}|\RV{H}&\forall i\in [n]\\
    \RV{W}_i\CI_{\prob{Q}} \RV{V}_{[m]\setminus \{i\}}|\RV{H}&\forall i\in [m]\\
    \RV{Y}_i\CI_{\prob{Q}} \RV{V}_{[m]\setminus \{i\}}|(\RV{H},\RV{X}_i,\RV{W}_i)&\forall i\in [m]\\
\end{align}

The first condition says that the observations $\RV{V}_{[n]}$ are mutually independent, the second that if we ignore the $\RV{X}_i$s and $\RV{Y}_i$s, then the $\RV{W}_i$s are mutually independent for all $[m]$ and the third says that the $\RV{Y}_i$s are independent of the other variables in the sequence conditional on $(\RV{H},\RV{X}_i,\RV{W}_i)$. Note that we exclude $\RV{X}_i$ from these conditional independence assumptions. The reason for this is that we interpret $\RV{X}_i$ as a directly controlled variable and as such it may be chosen to be dependent on other variables in the sequence.

\begin{definition}[Order 2 model associated with the CBN in question]\label{def:cbn_o2}
A CBN order 2 model $\prob{Q}^{\RV{V}_{[n]}\RV{W}_{(n,m]}|\RV{H}\square \RV{Y}_{(n,m]}|\RV{X}_{(n,m]}}$ where $\RV{V}_{i} = (\RV{W}_i,\RV{X}_i,\RV{Y}_i)$ for $i\in [m]$, such that the CBN mutual independences hold:
\begin{align}
    \RV{V}_i\CI_{\prob{Q}} \RV{V}_{[m]\setminus \{i\}}|\RV{H}&\forall i\in [n]\\
    \RV{W}_i\CI_{\prob{Q}} \RV{V}_{[m]\setminus \{i\}}|\RV{H}&\forall i\in [m]\\
    \RV{Y}_i\CI_{\prob{Q}} \RV{V}_{[m]\setminus \{i\}}|\RV{HX}_i&\forall i\in [m]\\
\end{align}
and
\begin{align}
    \model{Q}_{\text{obs}}^{\RV{W_iX_iY_i}|\RV{H}} &= \text{insert}(\model{Q}_{\text{obs}}^{\RV{X}_{j}|\RV{HW}_{j}},\prob{Q}^{\RV{W}_{j}|\RV{H}\square\RV{Y}_{j}|\RV{X}_{j}})\label{eq:cbn_insert}
\end{align}

for all $i\in[n]$, $j\in (n,m]$.
\end{definition}

Equation \ref{eq:cbn_insert} has a number of implications which we will use in the upcoming Theorem \ref{th:seedo_rep}:

For all $i,j\in [n]$
\begin{align}
    \prob{Q}^{\RV{V}_{i}|\RV{H}} = \prob{Q}^{\RV{V}_j|\RV{H}}& \text{and both conditional probabilities exist}
\end{align}
for $i,j\in [m]$,
\begin{align}
    \prob{Q}^{\RV{W}_i|\RV{H}} = \prob{Q}^{\RV{W}_j|\RV{H}}& \text{and both conditional probabilities exist}
\end{align}
for $i,j\in [m]$
\begin{align}
    \model{Q}^{\RV{Y}_i|\RV{X}_i\RV{W}_i\RV{H}} &= \model{Q}^{\RV{Y}_j|\RV{X}_j\RV{W}_j\RV{H}}&\text{ and both conditional probabilities exist}
\end{align}


Inserts for a CBN order 2 model are of the type $\prob{Q}_\alpha^{\RV{X}_{(n,m]}|\RV{V}_{[n]}\RV{W}_{(n,m]}\RV{H}}$. Such inserts are not necessarily decision rules as defined in the previous section. They don't necessarily admit an interpretation ``if I see $(\RV{V}_{[n]},\RV{W}_{(n,m]},\RV{H})$ then I do this to $\RV{X}_{(n,m]}$''. For example, $\RV{H}$ is by definition unobserved and so it would be odd to consider rules that depend on direct knowledge of $\RV{H}$. However, some subset of inserts of this type may correspond to decision rules that may be chosen. Specifically, we may be able to construct a see-do model such that each decision rule in the see-do model corresponds to a particular insert of the type $\prob{Q}_\alpha^{\RV{X}_{(n,m]}|\RV{V}_{[n]}\RV{W}_{(n,m]}\RV{H}}$ in the CBN model. The question we will now ask is: what kinds of see-do models allow for this?

\subsection{See-do models corresponding to causal Bayesian networks}

When does a see-do model $\prob{T}^{\RV{V}_{[n]}|\RV{H}\square \RV{V}_{(n,m]}|\RV{D}}$ with decision rules of type $\{\prob{T}_\alpha^{\RV{D}|\RV{V}_{[n]}}\}_{\alpha\in A}$ correspond to a CBN probability 2-comb $\prob{Q}^{\RV{V}_{[n]}\RV{W}_{(n,m]}|\RV{H}\square \RV{Y}_{(n,m]}|\RV{X}_{(n,m]}}$ with inserts of type $\{\prob{Q}_\alpha^{\RV{X}_{(n,m]}|\RV{V}_{[n]}\RV{W}_{(n,m]}\RV{H}}\}_{\alpha\in A}$? 

By correspondence, we mean that for each $\alpha\in A$, the probabilistic model given by $\text{insert}(\prob{T}^{\RV{V}_{[n]}|\RV{H}\square \RV{V}_{(n,m]}|\RV{D}},\prob{T}_\alpha^{\RV{D}|\RV{V}_{[n]}})$ followed by marginalising over $\RV{D}$ is the same as the model given by $\text{insert}(\prob{Q}^{\RV{V}_{[n]}\RV{W}_{(n,m]}|\RV{H}\square \RV{Y}_{(n,m]}|\RV{X}_{(n,m]}},\prob{Q}_\alpha^{\RV{X}_{(n,m]}|\RV{V}_{[n]}\RV{W}_{(n,m]}\RV{H}})$
\begin{align}
    \model{T}^{\RV{V}_{[m]}|\RV{H}}_\alpha&:=\tikzfig{seedo_equality2}\\
    &=\tikzfig{seedo_equality} \label{eq:consistent}\\
    &=: \model{Q}^{\RV{V}_{[m]}|\RV{H}}_\alpha
\end{align}

Theorem \ref{th:seedo_rep} shows that this correspondence holds exactly when:
\begin{itemize}
    \item The CBN mutual indpendences (Definition \ref{def:cbn_o2}) hold for the $\prob{T}$
    \item For all $i\in [m]$, $\RV{Y}_i\CI_{\prob{T}}\RV{D}|(\RV{W}_i,\RV{X}_i,\RV{H})$; we say that under conditions of perfect information, $(\RV{W}_i,\RV{X}_i)$ control $\RV{Y}_i$ by proxy
\end{itemize}

\begin{theorem}\label{th:seedo_rep}
Given a see-do model $\prob{T}^{\RV{V}_{[n]}|\RV{H}\square \RV{V}_{(n,m]}|\RV{D}}$ there exists a corresponding CBN probability 2-comb $\prob{Q}^{\RV{V}_{[n]}\RV{W}_{(n,m]}|\RV{H}\square \RV{Y}_{(n,m]}|\RV{X}_{(n,m]}}$ if and only if the CBN mutual independences hold
\begin{align}
    \RV{V}_i\CI_{\prob{T}} \RV{V}_{[m]\setminus \{i\}}|\RV{H}&\forall i\in [n]\\
    \RV{W}_i\CI_{\prob{T}} \RV{V}_{[m]\setminus \{i\}}|\RV{H}&\forall i\in [m]\\
    \RV{Y}_i\CI_{\prob{T}} \RV{V}_{[m]\setminus \{i\}}|\RV{HX}_i&\forall i\in [m]\\
\end{align}

as well as the identical conditionals
\begin{align}
    \prob{T}^{\RV{V}_{i}|\RV{H}} = \prob{T}^{\RV{V}_j|\RV{H}}& \text{and both conditional probabilities exist}
\end{align}
for $i,j\in [m]$,
\begin{align}
    \prob{T}^{\RV{W}_i|\RV{H}} = \prob{T}^{\RV{W}_j|\RV{H}}& \text{and both conditional probabilities exist}
\end{align}
for $i,j\in [m]$
\begin{align}
    \model{T}^{\RV{Y}_i|\RV{X}_i\RV{W}_i\RV{H}} &= \model{T}^{\RV{Y}_j|\RV{X}_j\RV{W}_j\RV{H}}&\text{ and both conditional probabilities exist}
\end{align}

And under conditions of perfect information, $(\RV{W}_i,\RV{X}_i,\RV{V}_{[n]})$ control $\RV{Y}_i$ by proxy:
\begin{align}
    \RV{Y}_i\CI_{\prob{T}}\RV{D}|(\RV{W}_i,\RV{X}_i,\RV{H},\RV{V}_{[n]})&\forall i\in [m]
\end{align}
\end{theorem}

\begin{proof}
\textbf{If:}
If all assumptions hold, we can write
\begin{align}
    \model{T}^{\RV{V}_{[n]}\RV{V}_j|\RV{HD}} = \tikzfig{t_vs_u}
\end{align}
For each $\model{S}_\alpha^{\RV{D}|\RV{V}_{[n]}}$, define
\begin{align}
    \model{R}_\alpha^{\RV{X}_j|\RV{V}_{[n]}\RV{W}_j\RV{H}}:= \tikzfig{defn_ra}
\end{align}
Then
\begin{align}
    &\tikzfig{seedo_equality2}\\
    &= \tikzfig{seedo_cbn_with_s}\\
    &= \tikzfig{seedo_equality}
\end{align}
\textbf{Only if:}
Suppose the CBN mutual independences or identical conditionals do not hold for $\prob{T}$. Then there must be some $\alpha$ such that one of these assumptions is violated for $\prob{T}_\alpha$, but these assumptions hold for all $\prob{Q}_\beta$, $\beta\in A$, so there is no corresponding insert for $\prob{T}_\alpha$.

Suppose the assumption of proxy control does not hold for $\prob{T}$. Then there is some $d,d'\in D$, $w\in W$, $h\in H$, $v\in V^{n}$, $x\in X$ and $y\in Y$ such that
\begin{align}
    \model{T}^{\RV{Y}_j|\RV{W}_j\RV{V}_{[n]}\RV{HX}_j\RV{D}}(y|w,v,h,x,d) &\neq \model{T}^{\RV{Y}_j|\RV{W}_j\RV{V}_{[n]}\RV{HX}_j\RV{D}}(y|w,v,h,x,d')\label{eq:not_indep}\\
    &\text{and }\model{T}_\alpha^{\RV{X}_j\RV{W}_j\RV{V}_{[n]}|\RV{HD}}(x,w,v|h,d) >0\\
    &\text{and }\model{T}_\alpha^{\RV{X}_j\RV{W}_j\RV{V}_{[n]}|\RV{HD}}(x,w,v|h,d') >0\\
\end{align}

If Equation \ref{eq:not_indep} only held on sets of measure 0 then we could choose versions of the conditional probabilities such that the independence held.

Then
\begin{align}
    \model{T}_d^{\RV{Y}_j|\RV{W}_j\RV{V}_{[n]}\RV{HX}_j}(y|w,v,h,x)&= \model{T}^{\RV{Y}_j|\RV{W}_j\RV{V}_{[n]}\RV{HX}_j\RV{D}}(y|w,v,h,x,d)\\
    &\neq \model{T}^{\RV{Y}_j|\RV{W}_j\RV{V}_{[n]}\RV{HX}_j\RV{D}}(y|w,v,h,x,d')\\
    &= \model{T}_{d'}^{\RV{Y}_j|\RV{W}_j\RV{V}_{[n]}\RV{HX}_j\RV{D}}(y|w,v,h,x)\\
    \implies \model{T}_d^{\RV{Y}_j|\RV{W}_j\RV{V}_{[n]}\RV{HX}_j\RV{D}}(y|w,v,h,x) &\neq \model{Q}_d^{\RV{Y}_j|\RV{W}_j\RV{V}_{[n]}\RV{HX}_j\RV{D}}(y|w,v,h,x)\\
    \text{or } \model{T}_{d'}^{\RV{Y}_j|\RV{W}_j\RV{V}_{[n]}\RV{HX}_j\RV{D}}(y|w,v,h,x) &\neq \model{Q}_{d'}^{\RV{Y}_j|\RV{W}_j\RV{V}_{[n]}\RV{HX}_j\RV{D}}(y|w,v,h,x)
\end{align}
As the conditional probabilities disagree on a positive measure set, $\model{P}\neq\model{Q}$.

Suppose assumption 3 holds but assumption 4 does not. Then for some $h\in H$, some $w\in W$, $v\in V^{|A|}$, $x\in X$ with positive measure and some $y\in Y$
\begin{align}
    \model{P}_d^{\RV{Y}_j|\RV{W}_j\RV{V}_{[n]}\RV{HX}_j\RV{D}}(y|w,v,h,x)&= \model{T}^{\RV{Y}_j|\RV{W}_j\RV{V}_{[n]}\RV{HX}_j}(y|w,v,h,x)\\
    &\neq \model{U}^{\RV{Y}_j|\RV{W}_j\RV{V}_{[n]}\RV{HX}_j}(y|w,v,h,x)\\
    &\neq model{Q}_d^{\RV{Y}_j|\RV{W}_j\RV{V}_{[n]}\RV{HX}_j\RV{D}}(y|w,v,h,x)
\end{align}
\end{proof}

Conditional independences like $(\RV{V}_{[n]},\RV{W}_j)\CI_{\model{T}} \RV{D}|\RV{H}$ and $\RV{Y}_j\CI_{\model{T}}\RV{D}|\RV{W}_j\RV{V}_{[n]}\RV{HX}_j$ bear some resemblance to the condition of ``limited unresponsiveness'' proposed by \citet{heckerman_decision-theoretic_1995}. They are conceptually similar in that they indicate that a particular variable does not ``depend on'' a decision $\RV{D}$ in some sense. As Heckerman points out, however, limited unresponsiveness is not equivalent to conditional independence. We tentatively speculate that there may be a relation between our ``pre-choice variables'' $(\RV{W}_j,\RV{V}_{[n]},\RV{H})$ and the ``state'' in Heckerman's work crucial for defining limited unresponsiveness.

\subsection{Proxy control}

We say that $(\RV{V}_{[n]},\RV{W}_j)\CI_{\model{T}} \RV{D}|\RV{H}$ expresses the notion that $\RV{W}_j$ is a \emph{pre-choice variable} and $(\RV{W}_j,\RV{V}_{[n]},\RV{X}_j)$ are \emph{proxies for }$\RV{D}$ with respect to $\RV{Y}$ under conditions of full information. To justify this terminology, we note that under a strong assumption of identifiability $\RV{Y}_j\CI\RV{H}|\RV{W}_j\RV{V}_{[n]}\RV{X}_j$ (i.e. the observed data allow us to identify $\RV{H}$ for the purposes of determining $\RV{T}^{\RV{Y}_j|\RV{W}_j\RV{V}_{[n]}\RV{X}_j\RV{H}}$), then we can write
\begin{align}
    \model{T}^{\RV{V}_{[n]}\RV{V}_{(n,m]}|\RV{H}\RV{D}} &=\tikzfig{strong_identifiability}\\
                                              &=\tikzfig{strong_identifiability2}
                                              &= \model{T}^{\RV{V}_{[n]}\RV{W}_j\RV{X}_j|\RV{H}\RV{D}}\kernel{M}
\end{align}

That is, under conditions of full information, knowing how to control the proxies $(\RV{W}_j,\RV{V}_{[n]},\RV{X}_j)$ is sufficient to control $\RV{Y}$. This echoes \citet{pearl_does_2018}'s view on causal effects representing ``stable characteristics'':
\begin{quote}
Smoking cannot be stopped by any legal or educational means available to us today; cigarette advertising can. That does not stop researchers from aiming to estimate ``the effect of smoking on cancer,'' and doing so from experiments in which they vary the instrument—cigarette advertisement—not smoking. The reason they would be interested in the atomic intervention $P(\text{cancer}|do(\text{smoking}))$ rather than (or in addition to) $P(\text{cancer}|do(\text{advertising}))$ is that the former represents a stable biological characteristic of the population, uncontaminated by social factors that affect susceptibility to advertisement, thus rendering it transportable across cultures and environments. With the help of this stable characteristic, one can assess the effects of a wide variety of practical policies, each employing a different smoking-reduction instrument.
\end{quote}