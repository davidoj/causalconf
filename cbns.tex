%!TEX root = main.tex


\section{Causal Bayesian Networks}\label{sec:CBN}

Like some of the causal modelling frameworks discussed in the previous section, including see-do models, Causal Bayesian Networks (CBNs) represent both ``observations'' and ``consequences of interventions''. It seems reasonable to think that the real-world things that the see-do framework and the CBN framework address are sometimes the same. The question we have here is: if we have a decision problem represented by a see-do model, when can we represent the same problem with a CBN?

In order to answer this question, we have to deal with the fact that neither theory is formally contained by the other, so there's no precise way, for example, for decisions to correspond to interventions. The correspondence exists in the territory, the world that is inhabited by measurement processes, not the precise mathematical world that is inhabited by random variables. We therefore have to make some choices about what corresponds to what that seem to be reasonable given our understanding of what these models are used for.

First, we will consider what kind of see-do model is represented by a CBN. We will assume a sample space $\Omega$ and a collection of variables, and we will consider a simplified kind of CBN where a single variable may be intervened on. With this condition, according to \citet{pearl_causality:_2009}, a causal Bayesian network is a probability model $\prob{P}$, a collection of interventional probability models $\{\prob{P}_{\RV{X}=a}|a\in X_i\}$ and a directed acyclic graph $\mathcal{G}$ whose nodes are identified with variables. Given one intervenable variable, we can in general consider just three variables $\{\RV{W},\RV{X},\RV{Y}\}$, where $\RV{W}$ is all of the ancestors of $\RV{X}$ in $\mathcal{G}$, $\RV{X}$ is the ``intervenable'' variable and $\RV{Y}$ are all the other variables associated with nodes of $\mathcal{G}$. The interventional probability models must all obey the truncated factorisation condition with respect to $\mathcal{G}$:
\begin{align}
    \prob{P}_{\RV{X}\set a}^{\RV{W}\RV{X}\RV{Y}}(w,x,y) &= \prob{P}^{\RV{W}}(w)\prob{P}^{\RV{Y}|\RV{XW}}(y|x,w)\llbracket x=a \rrbracket\label{eq:truncated_fac2}
\end{align}

We will interpret observations and interventions as follows: we have a sequence of ``observation'' variables $\RV{V}_[n]:=(\RV{W}_i,\RV{X}_i,\RV{Y}_i)_{i\in [n]}$ mutually independent and identically distributed and a sequence of ``consequence'' variables modeled by $\RV{V}_{[n,m]}:=(\RV{W}_i,\RV{X}_i,\RV{Y}_i)_{i\in [n,m]}$ mutually independent and identically distributed for any choice of intervention $\RV{X}\set a$. We furthermore have an oracle that tells us that each observation is distributed according to $\prob{P}^{\RV{WXY}}\in \Delta(W\times X\times Y)$. Given $\prob{P}^{\RV{WXY}}\in \Delta(W\times X\times Y)$ and some $\RV{X}\set a$, the consequences will each be distributed according to $\prob{P}_{\RV{X}\set a}^{\RV{W}\RV{X}\RV{Y}}\in \Delta(W\times X\times Y)$, related to $\prob{P}^{\RV{WXY}}$ via Equation \ref{eq:truncated_fac2}.

The proposition ``there is an oracle that tells us $\prob{P}^{\RV{WXY}}$'' sounds a bit like we have a measurment process for determining $\prob{P}^{\RV{WXY}}$ -- namely, consult the oracle. However this is not an actual process; instead, the injunction to consult the oracle is actually saying \emph{if we had a measurement process for} $\prob{P}^{\RV{WXY}}$, then here is how we would proceed. In the previous section, we represented a gap in our knowledge of this type with a hypothesis variable $\RV{H}$ along with a model $\prob{T}$ such that $\prob{T}^{\RV{H}}$ is a probability gap, and what is referred to by the CBN setup as $\prob{P}^{\RV{WXY}}$ corresponds to $\prob{T}^{\RV{WXY}|\RV{H}}(\cdot|h)$ -- that is, the probability distribution that we would use for $(\RV{W,X,Y})$ if the oracle told us that $\RV{H}\yields h$.

To represent this as a see-do model $\prob{T}^{\RV{V}_[n]|\RV{H}\square \RV{V}_{[n,m]}|\RV{D}}$, we identify the following:
\begin{itemize}
    \item The hypothesis $\RV{H}$ determines the ``actual'' observational distribution, which by Equation \ref{eq:truncated_fac2} also determines the ``actual'' collection of interventional consequences
    \item The observational distribution $\prob{P}^{\RV{V}_{[n]}}$ is identified with $\disint{T}^{\RV{V}_i|\RV{H}}$, where the uncertainty over the 
\end{itemize}

 represent this relation within the model if we introduce a hypothesis variable $\RV{H}$. Recall that $\RV{H}$ stood in for a gap in our knowledge regarding which probabilistic model we should use to model our problem. Thus we suppose that $\prob{P}^{\RV{V}_n}$ is the probability distribution given by some $\model{T}^{\RV{V}_[n]|\RV{H}}$ for some $h\in H$ and likewise $\prob{P}_{\RV{X}\set a}^{\RV{W}\RV{X}\RV{Y}}$ for some $\model{U}^{\RV{V}_{[n,m]}|\RV{H}\RV{X}_{[n,m]}}$ and some $h\in H$, $a\in X^{m-n}$ like so:
\begin{align}
    \prob{P}^{\RV{V}_i}(v)&:=\model{T}^{\RV{V}_i|\RV{H}}(v|h)\text{ for some }h\in H\text{ and any }i\in [n], v\in W\times X\times Y\label{eq:what_is_p}\\
    \prob{P}_{\RV{X}_j\set a}^{\RV{V}_j}&:=\model{U}^{\RV{V}_j|\RV{HX}_j}(v|h,a)\text{ for some }h\in H\text{ and any }j\in B, v\in W\times X\times Y\label{eq:what_is_do}
\end{align}
For line \ref{eq:what_is_do}, note that $\model{U}^{\RV{V}_i|\RV{HX}_i}$ is a Markov kernel $H\times X\kto W\times X\times Y$, which is the same type as the map $\kernel{Q}:=h,a\mapsto \prob{P}_{\RV{X}=a}$, and in addition Equation \ref{eq:truncated_fac2} ensures that defining $\model{U}^{\RV{V}_i|\RV{HX}_i}:=\kernel{Q}$ is consistent via Lemma \ref{lem:nocopy2}.

Then Equation \ref{eq:truncated_fac2} becomes
\begin{align}
    \model{U}^{\RV{W}_j\RV{X}_j\RV{Y}_j|\RV{HX}_j}(w,x,y|h,a) &= \model{T}^{\RV{W}_i|\RV{H}}(w)\model{U}^{\RV{Y}_i|\RV{X_iW_iH}}(y|x,w,h)\llbracket x=a \rrbracket & i\in A, j\in B \label{eq:truncated_fac3}
\end{align}

In most situations, $A$ will be disjoint from $B$. While we don't necessarily want to rule out considering consequences to be equal to observations, we usually want to consider consequences that may take different values from observations.

For $i\in A$, $j\in B$, we can write $\model{U}^{\RV{V}_{i}\RV{V}_j|\RV{HX}_{j}}$ as follows
\begin{align}
    \model{U}^{\RV{V}_{i}\RV{V}_j|\RV{HX}_{j}} &= \tikzfig{cbn_sd2}\\
    \model{U}^{\RV{V}_j|\RV{HX}_j} &= \tikzfig{truncated_fac_definition}\\
    \implies \model{U}^{\RV{V}_A\RV{V}_j|\RV{HX}_{j}} &= \tikzfig{cbn_sd23}\label{eq:cbn_maybe_comb}
\end{align}

Note that we replace the single observation $\RV{V}_i$ with the full observations $\RV{V}_A$ as we will make use of them subsequently, and we can do this without issue due to the assumption of conditional independence among the $\RV{V}_k$s. It will be sufficient to consider a single consequence $\RV{V}_j$. Equation \ref{eq:cbn_maybe_comb} defines a model $\model{U}^{\cdot|\RV{HX}_j}$ which relates observations to consequences in the manner suggested by Equation \ref{eq:truncated_fac2}. We will call $\model{U}$ a ``CBN model''. We note that the model in Equation \ref{eq:cbn_maybe_comb} looks like a 2-comb:
\begin{align}
    \tikzfig{cbn_sd_2comb} \label{eq:cbn_2comb}
\end{align}

However, we have not at this point assumed that we have a convex set of strategies. Suppose we have some standard see-do model $\mathscr{M}:=\{\model{T}^{\RV{O}\RV{V}_B|\RV{HD}},\{\model{S}_\alpha^{\RV{D}|\RV{O}}|\alpha\in \mathscr{A}\}\}$. The question we want to ask is: when can we posit a see-do model $\{\model{U}^{\RV{V}_A\RV{V}_j|\RV{HX}_j},\{\model{R}_\alpha^{\RV{X}_j|\RV{V}_A\RV{W}_j\RV{H}}|\alpha\in \mathscr{A}\}\}$ consistent with $\mathscr{M}$ in the sense that, for all $\alpha\in \mathscr{A}$:
\begin{align}
    \model{P}^{\RV{V}_B|\RV{H}}_\alpha&:=\tikzfig{seedo_equality2}\\
    &=\tikzfig{seedo_equality} \label{eq:consistent}\\
    &=: \model{Q}^{\RV{V}_B|\RV{H}}_\alpha
\end{align}

\todo[inline]{I think reusing the same $\RV{H}$ between $\RV{U}$ and $\RV{T}$ is a mistake here. Maybe not a big problem, but ideally one would check!}
\begin{theorem}\label{th:seedo_rep}
Given a standard see-do model $\mathscr{M}:=\{\model{T}^{\RV{O}\RV{V}_B|\RV{HD}},\{\model{S}_\alpha^{\RV{D}|\RV{V}_A}|\alpha\in \mathscr{A}\}\}$ and a CBN model $\model{U}^{\RV{V}_A\RV{V}_j|\RV{HX}_j}$ as defined in Equation \ref{eq:cbn_maybe_comb}, then there exists a see-do model $\{\model{U}^{\RV{V}_i\RV{V}_j|\RV{HX}_j},\{\model{R}_\alpha^{\RV{X}_j|\RV{V}_A\RV{W}_j\RV{H}}|\alpha\in \mathscr{A}\}\}$ consistent with $\mathscr{M}$ if and only if
\begin{enumerate}
    \item $\RV{W}_j$ is a pre-choice variable, i.e. $(\RV{V}_A,\RV{W}_j)\CI_{\model{T}} \RV{D}|\RV{H}$
    \item $\model{T}^{\RV{V}_A\RV{W}_j|\RV{H}} = \model{U}^{\RV{V}_A\RV{W}_j|\RV{H}}$
    \item $\RV{Y}_j\CI_{\model{T}}\RV{D}|\RV{W}_j\RV{V}_A\RV{HX}_j$
    \item $\model{T}^{\RV{Y}_j|\RV{W}_j\RV{V}_A\RV{HX}_j} = \model{U}^{\RV{Y}_j|\RV{W}_j\RV{V}_A\RV{HX}_j}$
\end{enumerate}
\end{theorem}

\begin{proof}
\textbf{If:}
If all assumptions hold, we can write
\begin{align}
    \model{T}^{\RV{V}_A\RV{V}_j|\RV{HD}} = \tikzfig{t_vs_u}
\end{align}
For each $\model{S}_\alpha^{\RV{D}|\RV{V}_A}$, define
\begin{align}
    \model{R}_\alpha^{\RV{X}_j|\RV{V}_A\RV{W}_j\RV{H}}:= \tikzfig{defn_ra}
\end{align}
Then
\begin{align}
    &\tikzfig{seedo_equality2}\\
    &= \tikzfig{seedo_cbn_with_s}\\
    &= \tikzfig{seedo_equality}
\end{align}
\textbf{Only if:}
Suppose assumption 1 does not hold. Then there exists some $d,d'\in D$, $w\in W$, $h\in H$ such that $\model{T}^{\RV{W}_j|\RV{HD}}(w_j|h,d)\neq \model{T}^{\RV{W}_j|\RV{HD}}(w|h,d')$. Then choose $\model{S}_d^{\RV{D}|\RV{V}_A}:v_A\mapsto \delta_{d}$ and $\model{S}_{d'}^{\RV{D}|\RV{V}_A}:v\mapsto \delta_{d'}$ for all $v\in V^{|A|}$. Then define
\begin{align}
    \model{P}_d^{\RV{W}_j|\RV{H}}(w|h) &= \model{T}^{\RV{W}_j|\RV{HD}}(w_j|h,d)\\
                                       &\neq \model{T}^{\RV{W}_j|\RV{HD}}(w_j|h,d')\\
                                       &= \model{P}_{d'}^{\RV{W}_j|\RV{H}}(w|h)
\end{align}
But for any $\alpha, \alpha'$, $\model{Q}_\alpha^{\RV{W}_j|\RV{H}}=\model{Q}_{\alpha'}^{\RV{W}_j|\RV{H}}$ as $\RV{W}_j\CI_{\RV{U}} \RV{X}_j|\RV{H}$, so $\model{Q}\neq \model{P}$.
Suppose assumption 1 holds but assumption 2 does not. Then for any $\alpha$
\begin{align}
    \model{P}_\alpha^{\RV{V}_A\RV{W}_j|\RV{H}}&=\model{T}^{\RV{V}_A\RV{W}_j|\RV{H}}\\
                                              &\neq \model{U}^{\RV{V}_A\RV{W}_j|\RV{H}}\\
                                              &= \model{Q}_\alpha^{\RV{V}_A\RV{W}_j|\RV{H}}
\end{align}
Suppose assumption 3 does not hold. Then there is some $d,d'\in D$, $w\in W$, $h\in H$, $v\in V^{|A|}$, $x\in X$ and $y\in Y$ such that
\begin{align}
    \model{T}^{\RV{Y}_j|\RV{W}_j\RV{V}_A\RV{HX}_j\RV{D}}(y|w,v,h,x,d) &\neq \model{T}^{\RV{Y}_j|\RV{W}_j\RV{V}_A\RV{HX}_j\RV{D}}(y|w,v,h,x,d')\label{eq:not_indep}\\
    &\text{and }\model{T}^{\RV{X}_j\RV{W}_j\RV{V}_A|\RV{HD}}(x,w,v|h,d) >0\\
    &\text{and }\model{T}^{\RV{X}_j\RV{W}_j\RV{V}_A|\RV{HD}}(x,w,v|h,d') >0\\
\end{align}
The latter conditions hold as if Equation \ref{eq:not_indep} only held on sets of measure 0 then we could choose versions of the conditional probabilities such that the independence held.

Then
\begin{align}
    \model{P}_d^{\RV{Y}_j|\RV{W}_j\RV{V}_A\RV{HX}_j\RV{D}}(y|w,v,h,x)&= \model{T}^{\RV{Y}_j|\RV{W}_j\RV{V}_A\RV{HX}_j\RV{D}}(y|w,v,h,x,d)\\
    &\neq \model{T}^{\RV{Y}_j|\RV{W}_j\RV{V}_A\RV{HX}_j\RV{D}}(y|w,v,h,x,d')\\
    &= \model{P}_{d'}^{\RV{Y}_j|\RV{W}_j\RV{V}_A\RV{HX}_j\RV{D}}(y|w,v,h,x)\\
    \implies \model{P}_d^{\RV{Y}_j|\RV{W}_j\RV{V}_A\RV{HX}_j\RV{D}}(y|w,v,h,x) &\neq \model{Q}_d^{\RV{Y}_j|\RV{W}_j\RV{V}_A\RV{HX}_j\RV{D}}(y|w,v,h,x)\\
    \text{or } \model{P}_{d'}^{\RV{Y}_j|\RV{W}_j\RV{V}_A\RV{HX}_j\RV{D}}(y|w,v,h,x) &\neq \model{Q}_{d'}^{\RV{Y}_j|\RV{W}_j\RV{V}_A\RV{HX}_j\RV{D}}(y|w,v,h,x)
\end{align}
As the conditional probabilities disagree on a positive measure set, $\model{P}\neq\model{Q}$.

Suppose assumption 3 holds but assumption 4 does not. Then for some $h\in H$, some $w\in W$, $v\in V^{|A|}$, $x\in X$ with positive measure and some $y\in Y$
\begin{align}
    \model{P}_d^{\RV{Y}_j|\RV{W}_j\RV{V}_A\RV{HX}_j\RV{D}}(y|w,v,h,x)&= \model{T}^{\RV{Y}_j|\RV{W}_j\RV{V}_A\RV{HX}_j}(y|w,v,h,x)\\
    &\neq \model{U}^{\RV{Y}_j|\RV{W}_j\RV{V}_A\RV{HX}_j}(y|w,v,h,x)\\
    &\neq model{Q}_d^{\RV{Y}_j|\RV{W}_j\RV{V}_A\RV{HX}_j\RV{D}}(y|w,v,h,x)
\end{align}
\end{proof}

Conditional independences like $(\RV{V}_A,\RV{W}_j)\CI_{\model{T}} \RV{D}|\RV{H}$ and $\RV{Y}_j\CI_{\model{T}}\RV{D}|\RV{W}_j\RV{V}_A\RV{HX}_j$ bear some resemblance to the condition of ``limited unresponsiveness'' proposed by \citet{heckerman_decision-theoretic_1995}. They are conceptually similar in that they indicate that a particular variable does not ``depend on'' a decision $\RV{D}$ in some sense. As Heckerman points out, however, limited unresponsiveness is not equivalent to conditional independence. We tentatively speculate that there may be a relation between our ``pre-choice variables'' $(\RV{W}_j,\RV{V}_A,\RV{H})$ and the ``state'' in Heckerman's work crucial for defining limited unresponsiveness.

\subsection{Proxy control}

We say that $(\RV{V}_A,\RV{W}_j)\CI_{\model{T}} \RV{D}|\RV{H}$ expresses the notion that $\RV{W}_j$ is a \emph{pre-choice variable} and $(\RV{W}_j,\RV{V}_A,\RV{X}_j)$ are \emph{proxies for }$\RV{D}$ with respect to $\RV{Y}$ under conditions of full information. To justify this terminology, we note that under a strong assumption of identifiability $\RV{Y}_j\CI\RV{H}|\RV{W}_j\RV{V}_A\RV{X}_j$ (i.e. the observed data allow us to identify $\RV{H}$ for the purposes of determining $\RV{T}^{\RV{Y}_j|\RV{W}_j\RV{V}_A\RV{X}_j\RV{H}}$), then we can write
\begin{align}
    \model{T}^{\RV{V}_A\RV{V}_B|\RV{H}\RV{D}} &=\tikzfig{strong_identifiability}\\
                                              &=\tikzfig{strong_identifiability2}
                                              &= \model{T}^{\RV{V}_A\RV{W}_j\RV{X}_j|\RV{H}\RV{D}}\kernel{M}
\end{align}

That is, under conditions of full information, knowing how to control the proxies $(\RV{W}_j,\RV{V}_A,\RV{X}_j)$ is sufficient to control $\RV{Y}$. This echoes \citet{pearl_does_2018}'s view on causal effects representing ``stable characteristics'':
\begin{quote}
Smoking cannot be stopped by any legal or educational means available to us today; cigarette advertising can. That does not stop researchers from aiming to estimate ``the effect of smoking on cancer,'' and doing so from experiments in which they vary the instrument—cigarette advertisement—not smoking. The reason they would be interested in the atomic intervention $P(\text{cancer}|do(\text{smoking}))$ rather than (or in addition to) $P(\text{cancer}|do(\text{advertising}))$ is that the former represents a stable biological characteristic of the population, uncontaminated by social factors that affect susceptibility to advertisement, thus rendering it transportable across cultures and environments. With the help of this stable characteristic, one can assess the effects of a wide variety of practical policies, each employing a different smoking-reduction instrument.
\end{quote}