%!TEX root = main.tex

\section{Appendix:see-do model representation}\label{sec:see-do-rep}

\todo[inline]{Modularise the treatment of probability}

\begin{theorem}[See-do model representation]\label{th:see_do_rep}
Suppose we have a decision problem that provides us with an observation $x\in X$, and in response to this we can select any decision or stochastic mixture of decisions from a set $D$; that is we can choose a ``strategy'' as any Markov kernel $\kernel{S}:X\to \Delta(D)$. We have a utility function $u:Y\to \mathbb{R}$ that models preferences over the potential consequences of our choice. Furthermore, suppose that we maintain a denumerable set of hypotheses $H$, and under each hypothesis $h\in H$ we model the result of choosing some strategy $\kernel{S}$ as a joint probability over observations, decisions and consequences $\prob{P}_{h,\kernel{S}}\in \Delta(X\times D\times Y)$.

Define $\RV{X},\RV{Y}$ and $\RV{D}$ such that $\RV{X}_{xdy} = x$, $\RV{Y}_{xdy}=y$ and $\RV{D}_{xdy} = d$. Then making the following additional assumptions:
\begin{enumerate}
    \item Holding the hypothesis $h$ fixed the observations as have the same distribution under any strategy: $\prob{P}_{h,\kernel{S}}[\RV{X}]=\prob{P}_{h,\kernel{S}''}[\RV{X}]$ for all $h,\kernel{S},\kernel{S}'$ (observations are given ``before'' our strategy has any effect)
    \item The chosen strategy is a version of the conditional probability of decisions given observations: $\kernel{S}=\prob{P}_{h,\kernel{S}}[\RV{D}|\RV{X}]$
    \item There exists some strategy $\kernel{S}$ that is strictly positive
    \item For any $h\in H$ and any two strategies $\kernel{Q}$ and $\kernel{S}$, we can find versions of each disintegration such that $\prob{P}_{h,\kernel{Q}}[\RV{Y}|\RV{D}\RV{X}]=\prob{P}_{h,\kernel{S}}[\RV{Y}|\RV{D}\RV{X}]$ (our strategy tells us nothing about the consequences that we don't already know from the observations and decisions)
\end{enumerate}

Then there exists a unique see-do model $(\kernel{T},\RV{H}',\RV{D}',\RV{X}',\RV{Y}')$ such that $\prob{P}_{h,\kernel{S}}[\RV{XDY}]^{ijk} = \kernel{T}[\RV{X'}|\RV{H'}]_{h}^{i} \kernel{S}_i^j  \kernel{T}[\RV{Y'}|\RV{X'H'D'}]_{ijk}^k$.
\end{theorem}

\begin{proof}
Consider some probability $\prob{P}\in \Delta(X\times D\times Y)$. By the definition of disintegration (section \ref{ssec:disintegration}), we can write

\begin{align}
 \prob{P}[\RV{XDY}]^{ijk} = \prob{P}[\RV{X}]^i\prob{P}[\RV{D}|\RV{X}]_i^{j} \prob{P}[\RV{Y}|\RV{XD}]_{ij}^{k} \label{eq:disint}
\end{align}

Fix some $h\in H$ and some strictly positive strategy $\kernel{S}$ and define $\kernel{T}:H\times D\to \Delta(X\times Y)$ by
\begin{align}
    \kernel{T}_{hj}^{kl} &= \prob{P}_{h,\kernel{S}}[\RV{X}]^k \prob{P}_{h,\kernel{S}}[\RV{Y}|\RV{X}\RV{D}]^l_{kj} \label{eq:comb_disint}
\end{align}

Note that because $\kernel{S}$ is strictly positive and by assumption $\kernel{S}=\prob{P}_{h,\kernel{S}}[\RV{D}|\RV{X}]$, $\prob{P}_{h,\kernel{S}}[\RV{D}]$ is also strictly positive. Therefore $\prob{P}_{h,\kernel{S}}[\RV{Y}|\RV{D}]$ is unique and therefore $\kernel{T}$ is also unique.

Define $\RV{X}'$ and $\RV{Y}'$ by $\RV{X}'_{xy}=x$ and $\RV{Y}'_{xy}=y$. Define $\RV{H}'$ and $\RV{D}'$ by $\RV{H}'_{hd} = h$ and $\RV{D}'_{hd} = d$.

We then have
\begin{align}
    \kernel{T}[\RV{X'}|\RV{H'D'}]_{hj}^{k} &= \kernel{T}\underline{\RV{X}'}_{hj}^k\\
                                           &= \sum_l \kernel{T}_{hj}^{kl} \\
                                           &= \prob{P}_{h,\kernel{S}}[\RV{X}]^k\\
                                           &= \kernel{T}[\RV{X'}|\RV{H'D'}]_{hj'}^{k}
\end{align}

Thus $\RV{X}'\CI_{\kernel{T}} \RV{D}'|\RV{H}'$ and so $\kernel{T}[\RV{X}'|\RV{H}']$ exists (section \ref{ssec:cond_indep}) and $(\kernel{T},\RV{H}',\RV{D}',\RV{X}',\RV{Y}')$ is a see-do model.

Applying Equation \ref{eq:disint} to $\prob{P}_{h,\kernel{S}}$:

\begin{align}
    \prob{P}_{h,\kernel{S}}[\RV{XDY}]^{ijk} &= \prob{P}_{h,\kernel{S}}[\RV{X}]^i\prob{P}_{h,\kernel{S}}[\RV{D}|\RV{X}]_i^{j} \prob{P}_{h,\kernel{S}}[\RV{Y}|\RV{XD}]_{ij}^{k}\label{eq:t_is_comb_disint_start}\\
     &=  \prob{P}_{h,\kernel{S}}[\RV{X}]^i\prob{P}_{h,\kernel{S}}[\RV{Y}|\RV{XD}]_{ij}^{k}\\
     &= \prob{P}_{h,\kernel{S}}[\RV{D}|\RV{X}]_i^{j} \kernel{T}[\RV{X'Y'}|\RV{H'D'}]_{hj}^{ik}\\
     &= \kernel{S}_i^j \kernel{T}[\RV{X'Y'}|\RV{H'D'}]_{hj}^{ik}\\
     &= \kernel{S}_i^j \kernel{T}[\RV{X'}|\RV{H'D'}]_{hj}^{i} \kernel{T}[\RV{Y'}|\RV{X'H'D'}]_{ihj}^k\\
     &= \kernel{T}[\RV{X'}|\RV{H'}]_{h}^{i} \kernel{S}_i^j  \kernel{T}[\RV{Y'}|\RV{X'H'D'}]_{ihj}^k\label{eq:t_is_comb_disint_end}
\end{align}

Consider some arbitrary alternative strategy $\kernel{Q}$. By assumption

\begin{align}
    \prob{P}_{h,\kernel{S}}[\RV{X}]^{i} &= \prob{P}_{h,\kernel{Q}}[\RV{X}]^{i}\\
    \prob{P}_{h,\kernel{S}}[\RV{Y}|\RV{XD}]_{ij}^k &= \prob{P}_{h,\kernel{Q}}[\RV{Y}|\RV{XD}]_{ij}^k\text{ for some version of }\prob{P}_{h,\kernel{Q}}[\RV{Y}|\RV{XD}]
\end{align}

It follows that, for some version of $\prob{P}_{h,\kernel{Q}}[\RV{Y}|\RV{XD}]$,
\begin{align}
    \kernel{T}_{hj}^{kl} &= \prob{P}_{h,\kernel{Q}}[\RV{X}]^k \prob{P}_{h,\kernel{Q}}[\RV{Y}|\RV{X}\RV{D}]^l_{kj} \label{eq:comb_disint_nonuniq}
\end{align}

Then by substitution of $\kernel{Q}$ for $\kernel{S}$ in Equation \ref{eq:t_is_comb_disint_start} and working through the same steps

\begin{align}
    \prob{P}_{h,\kernel{S}}[\RV{XDY}]^{ijk} &= \kernel{T}[\RV{X'}|\RV{H'}]_{h}^{i} \kernel{Q}_i^j  \kernel{T}[\RV{Y'}|\RV{X'H'D'}]_{ihj}^k
\end{align}

As $\kernel{Q}$ was arbitrary, this holds for all strategies.
\end{proof}

\section{Appendix: Connection is associative}\label{sec:connect_associative}

This will be proven with string diagrams, and consequently generalises to the operation defined by Equation \ref{eq:extn2} in other Markov kernel categories.

Define

\begin{align}
    \RV{I}_{K\cdot\cdot}&:=\RV{I}_K\setminus\RV{I}_L\setminus\RV{I}_J\\
    \RV{I}_{KL\cdot}&:=\RV{I}_K\cap\RV{I}_L\setminus\RV{I}_J\\
    \RV{I}_{K\cdot J}&:=\RV{I}_K\cap\RV{I}_J\setminus\RV{I}_L\\
    \RV{I}_{KL J}&:=\RV{I}_K\cap \RV{I}_L\cap\RV{I}_J\\
    \RV{I}_{\cdot L \cdot} &:= \RV{I}_L\setminus\RV{I}_K\setminus \RV{I}_J\\
    \RV{I}_{\cdot L J} &:= \RV{I}_L\cap \RV{I}_J\setminus \RV{I}_K\\
    \RV{I}_{\cdot\cdot J} &:= \RV{I}_J\setminus\RV{I}_K\setminus\RV{I}_L\\
    \RV{O}_{K\cdot\cdot} &:= \RV{O}_K\setminus\RV{I}_N\setminus\RV{I}_J\\
    \RV{O}_{KL\cdot} &:= \RV{O}_K\cap\RV{I}_L\setminus\RV{I}_J\\
    \RV{O}_{K\cdot J} &:= \RV{O}_K\cap\RV{I}_J\setminus\RV{I}_L\\
    \RV{O}_{KLJ} &:= \RV{O}_K\cap\RV{I}_L\cap\RV{I}_J\\
    \RV{O}_{L\cdot} &:= \RV{O}_L\setminus\RV{I}_J\\
    \RV{O}_{LJ} &:= \RV{O}_L\cap\RV{I}_J
\end{align}

Also define
\begin{align}
    (\kernel{P},\RV{I}_P,\RV{O}_P)&:=\kernel{K}\rightrightarrows \kernel{L}\\
    (\kernel{Q},\RV{I}_Q,\RV{O}_Q)&:=\kernel{L}\rightrightarrows \kernel{J}
\end{align}

Then

\begin{align}
    (\kernel{K}\rightrightarrows \kernel{L})\rightrightarrows \kernel{J} &= \kernel{P}\rightrightarrows \kernel{J}\\
                                                                         &= \begin{tikzpicture}[baseline={([yshift=-.2ex]current bounding box.center)}]
        \path (0,0) node (Y) {$\RV{I}_{P\cdot}$}
        + (0,-0.3) node (Q) {$\RV{I}_{PJ}$}
        + (0,-0.8) node (R) {$\RV{I}_{\cdot J}$}
        ++ (0.5,-0.3) node[copymap] (copy0) {}
        ++ (0.5,0.15) node[kernel] (K) {$\kernel{P}$}
        ++ (0.5,-0.15) node[copymap] (copy1) {}
        ++ (0.6,-0.5) node[kernel] (L) {$\kernel{J}$}
        ++ (0.6, 0.8) node (Z) {$\RV{O}_{P\cdot}$}
        + (0,-0.3) node (X) {$\RV{O}_{PJ}$}
        + (0,-0.8) node (W) {$\RV{O}_J$};
        \draw (Y) -- ($(K.west) + (0,0.15)$) (Q) -- ($(K.west) + (0,-0.15)$);
        \draw (copy0) to [out=-45,in=180] ($(L.west) + (0,0)$) (copy1) to [out=-60,in=180] ($(L.west) + (0,0.15)$);
        \draw (R) to [out=0,in=180] ($(L.west) + (0,-0.15)$);
        \draw ($(K.east) + (0,-0.15)$) to (copy1);
        \draw ($(K.east) + (0,0.15)$) -- (Z) (copy1) to [out=0,in=180] (X) (L) -- (W);
    \end{tikzpicture}\\
    &= \begin{tikzpicture}[baseline={([yshift=-.2ex]current bounding box.center)}] \path (0,0) node (IKdd) {$\RV{I}_{K\cdot\cdot}$}
        + (0,-0.4) node (IKLd) {$\RV{I}_{KL\cdot}$}
        + (0,-0.8) node (IdLd) {$\RV{I}_{\cdot L \cdot}$}
        + (0,-1.2) node (IKdJ) {$\RV{I}_{K\cdot J}$}
        + (0,-1.6) node (IKLJ) {$\RV{I}_{KLJ}$}
        + (0,-2) node (IdLJ) {$\RV{I}_{\cdot LJ}$}
        + (0,-2.4) node (IddJ) {$\RV{I}_{\cdot\cdot J}$}
        + (0.7,-0.4) node[copymap] (copyKL) {}
        + (0.7,-1.2) node[copymap] (copyKJ) {}
        + (0.7,-1.6) node[copymap] (copyKLJ) {}
        + (0.7,-2) node[copymap] (copyLJ) {}
        ++ (1.5,-0.3) node[kernel,inner sep=5pt] (K) {$\kernel{K}$}
        ++ (1.2,0.15) node[copymap] (copyOKL) {}
        +  (0,-0.3) node[copymap] (copyOKJ) {}
        + (0,-0.45) node[copymap] (copyOKLJ) {}
        ++ (1.2,-0.9) node[kernel,inner sep=6pt] (L) {$\kernel{L}$}
        ++ (1.2,-0.15) node[copymap] (copyOLJ) {}
        ++ (1.2,-0.9) node[kernel,inner sep=6pt] (J) {$\kernel{J}$}
        ++ (2.1, 2.1) node (OKdd) {$\RV{O}_{K\cdot\cdot}$}
        + (0,-0.4) node (OKLd) {$\RV{O}_{KL\cdot}$}
        + (0,-0.8) node (OKdJ) {$\RV{O}_{K\cdot J}$}
        + (0,-1.2) node (OKLJ) {$\RV{O}_{KLJ}$}
        + (0,-1.6) node (OLd) {$\RV{O}_{L\cdot}$}
        + (0,-2) node (OLJ) {$\RV{O}_{LJ}$}
        + (0,-2.4) node (OJ) {$\RV{O}_{J}$};
        \draw (IKdd) to [out=0,in=180] ($(K.west) + (0,0.25)$) ($(K.east) + (0,0.25)$) to [out=0,in=180] (OKdd);
        \draw (IKLd) -- (copyKL) to [out=-45,in=180] ($(L.west) + (0,0.1)$) (copyKL) to [out=55,in=180] ($(K.west) + (0,0.125)$)
        ($(K.east)+(0,0.125)$) to [out=0,in=180] (copyOKL) to [out=-90,in=180] ($(L.west) + (0,0.3)$)
        (copyOKL) to [out=0,in=180] (OKLd);
        \draw (IKLJ) to [out=0,in=180] (copyKLJ) to [out=40,in=180] ($(K.west) + (0,-0.25)$) 
        (copyKLJ) to [out=10,in=180] ($(L.west) + (0,0)$)
        (copyKLJ) to [out=-25,in=180] ($(J.west) + (0,-0.2)$);
        \draw (IKdJ) to [out=0,in=180] (copyKJ) to [out=65,in=180] ($(K.west) + (0,-0.125)$)
        (copyKJ) to [out=-30,in=180] ($(J.west) + (0,-0.1)$);
        \draw (IdLd) to [out=0,in=160] ($(IdLd)+(0.8,-0.1)$) to [out=-20,in=180] ($(L.west)+(0,-0.1)$);
        \draw (IdLJ) to [out=0,in=180] (copyLJ) to[out=15,in=180] ($(L.west)+(0,-0.2)$)
        (copyLJ) to [out=-10,in=180] ($(J.west) + (0,0.)$);
        \draw (IddJ) to [out=0,in=180] ($(IddJ)+ (0.5,0)$) to [out=-15,in=180] ($(J.west) + (0,-0.3)$);
        \draw ($(K.east)+(0,-0.125)$) to (copyOKJ) to [out=-75,in=180] ($(J.west) + (0,0.2)$)
        (copyOKJ) to [out=0,in=180] (OKdJ)
        (copyOKLJ) to [out=-80,in=180] ($(J.west) + (0,0.1)$);
        \draw ($(K.east) + (0,-0.25)$) to [out=0,in=180] (copyOKLJ) to [out=0,in=180] (OKLJ)
        (copyOKLJ) to [out=-35,in=180] ($(L.west) + (0,0.2)$);
        \draw ($(L.east) + (0,0.2)$) to [out=0,in=180] (OLd)
        ($(L.east) + (0,0)$) to [out=0,in=170] (copyOLJ) to [out=-10,in=180] (OLJ)
        (copyOLJ) to [out=-45,in=180] ($(J.west) + (0,0.3)$);
        \draw (J) to [out=0,in=180] (OJ);
    \end{tikzpicture}\\
    &\overset{perm}{=} \begin{tikzpicture}[baseline={([yshift=-.2ex]current bounding box.center)}] \path (0,0) node (IKdd) {$\RV{I}_{K\cdot\cdot}$}
        + (0,-0.4) node (IKLd) {$\RV{I}_{KL\cdot}$}
        + (0,-0.8) node (IKdJ) {$\RV{I}_{K\cdot J}$}
        + (0,-1.2) node (IKLJ) {$\RV{I}_{KLJ}$}
        + (0,-1.6) node (IdLd) {$\RV{I}_{\cdot L \cdot}$}
        + (0,-2) node (IdLJ) {$\RV{I}_{\cdot LJ}$}
        + (0,-2.4) node (IddJ) {$\RV{I}_{\cdot\cdot J}$}
        + (0.7,-0.4) node[copymap] (copyKL) {}
        + (0.7,-.8) node[copymap] (copyKJ) {}
        + (0.7,-1.2) node[copymap] (copyKLJ) {}
        + (0.7,-2) node[copymap] (copyLJ) {}
        ++ (1.5,-0.3) node[kernel,inner sep=5pt] (K) {$\kernel{K}$}
        ++ (1.2,0.15) node[copymap] (copyOKL) {}
        +  (0,-0.3) node[copymap] (copyOKJ) {}
        + (0,-0.45) node[copymap] (copyOKLJ) {}
        ++ (1.2,-0.9) node[kernel,inner sep=6pt] (L) {$\kernel{L}$}
        ++ (1.2,-0.15) node[copymap] (copyOLJ) {}
        ++ (1.2,-0.9) node[kernel,inner sep=6pt] (J) {$\kernel{J}$}
        ++ (2.1, 2.1) node (OKdd) {$\RV{O}_{K\cdot\cdot}$}
        + (0,-0.4) node (OKLd) {$\RV{O}_{KL\cdot}$}
        + (0,-0.8) node (OKdJ) {$\RV{O}_{K\cdot J}$}
        + (0,-1.2) node (OKLJ) {$\RV{O}_{KLJ}$}
        + (0,-1.6) node (OLd) {$\RV{O}_{L\cdot}$}
        + (0,-2) node (OLJ) {$\RV{O}_{LJ}$}
        + (0,-2.4) node (OJ) {$\RV{O}_{J}$};
        \draw (IKdd) to [out=0,in=180] ($(K.west) + (0,0.25)$) ($(K.east) + (0,0.25)$) to [out=0,in=180] (OKdd);
        \draw (IKLd) -- (copyKL) to [out=-45,in=180] ($(L.west) + (0,0.1)$) (copyKL) to [out=55,in=180] ($(K.west) + (0,0.125)$)
        ($(K.east)+(0,0.125)$) to [out=0,in=180] (copyOKL) to [out=-90,in=180] ($(L.west) + (0,0.3)$)
        (copyOKL) to [out=0,in=180] (OKLd);
        \draw (IKLJ) to [out=0,in=180] (copyKLJ) to [out=40,in=180] ($(K.west) + (0,-0.25)$) 
        (copyKLJ) to [out=10,in=180] ($(L.west) + (0,0)$)
        (copyKLJ) to [out=-25,in=180] ($(J.west) + (0,-0.2)$);
        \draw (IKdJ) to [out=0,in=180] (copyKJ) to [out=65,in=180] ($(K.west) + (0,-0.125)$)
        (copyKJ) to [out=-30,in=180] ($(J.west) + (0,-0.1)$);
        \draw (IdLd) to [out=0,in=180] ($(L.west)+(0,-0.1)$);
        \draw (IdLJ) to [out=0,in=180] (copyLJ) to[out=15,in=180] ($(L.west)+(0,-0.2)$)
        (copyLJ) to [out=-10,in=180] ($(J.west) + (0,0.)$);
        \draw (IddJ) to [out=0,in=180] ($(IddJ)+ (0.5,0)$) to [out=-15,in=180] ($(J.west) + (0,-0.3)$);
        \draw ($(K.east)+(0,-0.125)$) to (copyOKJ) to [out=-75,in=180] ($(J.west) + (0,0.2)$)
        (copyOKJ) to [out=0,in=180] (OKdJ)
        (copyOKLJ) to [out=-80,in=180] ($(J.west) + (0,0.1)$);
        \draw ($(K.east) + (0,-0.25)$) to [out=0,in=180] (copyOKLJ) to [out=0,in=180] (OKLJ)
        (copyOKLJ) to [out=-35,in=180] ($(L.west) + (0,0.2)$);
        \draw ($(L.east) + (0,0.2)$) to [out=0,in=180] (OLd)
        ($(L.east) + (0,0)$) to [out=0,in=170] (copyOLJ) to [out=-10,in=180] (OLJ)
        (copyOLJ) to [out=-45,in=180] ($(J.west) + (0,0.3)$);
        \draw (J) to [out=0,in=180] (OJ);
    \end{tikzpicture}\\
    &= \begin{tikzpicture}[baseline={([yshift=-.2ex]current bounding box.center)}]
        \path (0,0) node (Y) {$\RV{I}_{K\cdot}$}
        + (0,-0.3) node (Q) {$\RV{I}_{KQ}$}
        + (0,-0.8) node (R) {$\RV{I}_{\cdot Q}$}
        ++ (0.5,-0.3) node[copymap] (copy0) {}
        ++ (0.5,0.15) node[kernel] (K) {$\kernel{K}$}
        ++ (0.5,-0.15) node[copymap] (copy1) {}
        ++ (0.6,-0.5) node[kernel] (L) {$\kernel{Q}$}
        ++ (0.6, 0.8) node (Z) {$\RV{O}_{K\cdot}$}
        + (0,-0.3) node (X) {$\RV{O}_{KQ}$}
        + (0,-0.8) node (W) {$\RV{O}_Q$};
        \draw (Y) -- ($(K.west) + (0,0.15)$) (Q) -- ($(K.west) + (0,-0.15)$);
        \draw (copy0) to [out=-45,in=180] ($(L.west) + (0,0)$) (copy1) to [out=-60,in=180] ($(L.west) + (0,0.15)$);
        \draw (R) to [out=0,in=180] ($(L.west) + (0,-0.15)$);
        \draw ($(K.east) + (0,-0.15)$) to (copy1);
        \draw ($(K.east) + (0,0.15)$) -- (Z) (copy1) to [out=0,in=180] (X) (L) -- (W);
    \end{tikzpicture}\\
    &= \kernel{K}\rightrightarrows (\kernel{L}\rightrightarrows \kernel{J})
\end{align}

\section{Appendix: String Diagram Examples}

Recall the definition of \emph{connection}:

\begin{align}
    \kernel{K}\rightrightarrows \kernel{L} &:=  \begin{tikzpicture}[baseline={([yshift=-.2ex]current bounding box.center)}]
        \path (0,0) node (Y) {$\RV{I}_{F\cdot}$}
        + (0,-0.3) node (Q) {$\RV{I}_{FS}$}
        + (0,-0.8) node (R) {$\RV{I}_{\cdot S}$}
        ++ (0.5,-0.3) node[copymap] (copy0) {}
        ++ (0.5,0.15) node[kernel] (K) {$\kernel{K}$}
        ++ (0.5,-0.15) node[copymap] (copy1) {}
        ++ (0.6,-0.5) node[kernel] (L) {$\kernel{L}$}
        ++ (0.6, 0.8) node (Z) {$\RV{O}_{F\cdot}$}
        + (0,-0.3) node (X) {$\RV{O}_{FS}$}
        + (0,-0.8) node (W) {$\RV{O}_S$};
        \draw (Y) -- ($(K.west) + (0,0.15)$) (Q) -- ($(K.west) + (0,-0.15)$);
        \draw (copy0) to [out=-45,in=180] ($(L.west) + (0,0)$) (copy1) to [out=-60,in=180] ($(L.west) + (0,0.15)$);
        \draw (R) to [out=0,in=180] ($(L.west) + (0,-0.15)$);
        \draw ($(K.east) + (0,-0.15)$) to (copy1);
        \draw ($(K.east) + (0,0.15)$) -- (Z) (copy1) to [out=0,in=180] (X) (L) -- (W);
    \end{tikzpicture}\label{eq:extn_definition1}\\
    &:= \kernel{J}\\
    \kernel{J}_{yqr}^{zxw} &= \kernel{K}_{yq}^{zx} \kernel{L}_{xqr}^{w}\label{eq:extn_definition2}
\end{align}
\end{definition}

Equation \ref{eq:extn_definition1} can be broken down to the product of four Markov kernels, each of which is itself a tensor product of a number of other Markov kernels:
\begin{align}
    (\kernel{J},(\RV{I}_{F\cdot},\RV{I}_{FS},\RV{I}_{\cdot S}), (\RV{O}_{F\cdot},\RV{O}_{FS},\RV{O}_S)) &= \left[ \begin{tikzpicture}[baseline={([yshift=-.5ex]current bounding box.center)}]
        \path (0,0) node (Y) {$\RV{I}_{F\cdot}$}
        + (0,-0.3) node (Q) {$\RV{I}_{FS}$}
        + (0,-0.75) node (R) {$\RV{I}_{\cdot S}$}
        ++ (0.5,-0.3) node[copymap] (copy1) {}
        ++ (0.5,0.3) node (Z) {}
        ++ (0,-0.15) node (Q1) {}
        ++ (0,-0.3) node (Q2) {}
        ++ (0,-0.3) node (R2) {};
        \draw (Y) -- (Z) (Q) -- (copy1) to [out=45,in=180] (Q1);
        \draw (copy1) to [out=-45,in=180] (Q2);
        \draw (R) -- (R2); \end{tikzpicture}\right]
        \left[\begin{tikzpicture}[baseline={([yshift=-.5ex]current bounding box.center)}]
        \path (0,0)  node (Z) {}
        ++ (0,-0.15) node (Q1) {}
        ++ (0,-0.3) node (Q2) {}
        ++ (0,-0.3) node (R) {}
        ++ (0.5,0.65) node[kernel] (K) {$\model{K}$}
        ++ (0.5,0.1) node (Z1) {}
        +  (0,-0.15) node (W) {}
        + (0,-0.45) node (Q3) {}
        + (0,-0.75) node (R2) {};
        \draw (Z) -- ($(K.west) + (0,0.1)$) (Q1) -- ($(K.west) + (0,-0.05)$);
        \draw (Q2) -- (Q3) (R) -- (R2) ($(K.east) + (0,0.1)$) -- (Z1); 
        \draw ($(K.east) + (0,-0.05)$) -- (W);\end{tikzpicture}\right] 
        \left[ \begin{tikzpicture}[baseline={([yshift=-.5ex]current bounding box.center)}]
        \path (0,0) node (Z) {}
        + (0,-0.15) node (X) {}
        + (0,-0.45) node (Q) {}
        + (0,-0.75) node (R) {}
        ++ (0.5,-0.3) node[copymap] (copy1) {}
        ++ (0.5,0.3) node (Z1) {}
        ++ (0,-0.15) node (X1) {}
        ++ (0,-0.3) node (X2) {}
        ++ (0,-0.15) node (Q2) {}
        ++ (0,-0.15) node (R2) {};
        \draw (Z) -- (Z1) (X) to [out=0,in=180] (copy1) to [out=45,in=180] (X1);
        \draw (copy1) to [out=-45,in=180] (X2);
        \draw (Q) to [out=0,in=180] (Q2);
        \draw (R) -- (R2); \end{tikzpicture}\right]
        \left[\begin{tikzpicture}[baseline={([yshift=-.5ex]current bounding box.center)}]
        \path (0,0) node (Z) {}
        ++ (0,-0.15) node (X1) {}
        ++ (0,-0.3) node (X2) {}
        ++ (0,-0.15) node (Q) {}
        ++ (0,-0.15) node (R) {}
        ++ (0.5,0.15) node[kernel] (L) {$\model{L}$}
        ++ (0.7,0) node (W) {$\RV{O}_{F\cdot}$}
        ++ (0,0.35) node (X3) {$\RV{O}_{FS}$}
        ++ (0,0.25) node (Z1) {$\RV{O}_S$};
        \draw (X1) to [out=0,in=180] (X3) (Z) -- (Z1);
        \draw (X2) to [out=0,in=180] ($(L.west) + (0,0.15)$);
        \draw (Q) to [out=0,in=180] ($(L.west) + (0,0)$);
        \draw (R) to [out=0,in=180] ($(L.west) + (0,-0.15)$);
        \draw (L) -- (W);\end{tikzpicture}\right]\\
\end{align}