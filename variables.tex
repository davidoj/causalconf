
%!TEX root = main.tex

\section{Variables in probabilistic models}\label{sec:variables}

Our main question concerns the existence of causal relationships between \emph{variables}. If we want to offer a clear account of what this means, we need to start with a clear account of what variables are. Both observed an unobserved variables play important roles in causal modelling and we think it is worth clarifying what variables of either type refer to. We will start with observed variables, which we consider to be parts of our model whose role is to ``point to the parts of the world the model is explaining''. Unobserved variables, on the other hand, are parts of the model that do not refer to the external world but may be introduced, for example, for notational convenience.

Our approach in short is: a probabilistic model is associated with a particular experiment or measurement procedure. The measurment procedure yields values in a well-defined set. Observable results are obtained by applying well-defined functions to the result of this procedure. The observable sample space is the set of values that can be obtained from the experiment, and observable variables are the functions associated with particular observable results. We extend the set of values obtained from the observable sample space to a sample space that can contains both obserable and unobservable variables. Unobservable variables, like observable variables, are functions on the sample space, but they do not correspond to any observable results.

As far as we know, distinguishing variables from procedures is somewhat nonstandard, but we feel it is useful to distinguish the formal elements of the theory (variables) from the semi-formal elements (measurement procedures). Both variables and procedures are often discussed in statistical texts. For example, \citet{pearl_causality:_2009} offers the following two, purportedly equivalent, definitions of variables:
\begin{quote}
By a \emph{variable} we will mean an attribute, measurement or inquiry that may take on one of several possible outcomes, or values, from a specified domain. If we have beliefs (i.e., probabilities) attached to the possible values that a variable may attain, we will call that variable a random variable.
\end{quote}

\begin{quote}
This is a minor generalization of the textbook definition, according to which a random variable is a mapping from the fundamental probability set (e.g., the set of elementary events) to the real line. In our definition, the mapping is from the fundamental probability set to any set of objects called ``values,'' which may or may not be ordered.
\end{quote}

Our view is that the first definition is a definition of a procedure, while the second is a definition of a variable. Variables model procedures, but they are not the same thing. We can establish this by noting that, under our definition, every procedure of interest -- that is, all procedures that can be written $f\circ \proc{S}$ for some $f$ -- is modeled by a variable, but there may be variables defined on $\Omega$ that do not factorise through $\proc{S}$, and these variables do not model procedures.


We illustrate this approach with the example of Newton's second law in the form $\RV{F}=\RV{MA}$. This model relates ``variables'' $\RV{F}$, $\RV{M}$ and $\RV{A}$. As \citet{feynman_feynman_1979} noted, in order to understand this law, we must bring some pre-existing understanding of force, mass and acceleration independent of the law itself. Furthermore, we contend, this knowledge cannot be expressed in any purely mathematical statement. In order to say wha the net force on a given object is, even a highly knowledgeable physicist will have to go and do some measurements, which is a procedure that they carry out involving interacting with the real world somehow and obtaining as a result a vector representing the net forces on that object.

That is, the variables $\RV{F}$, $\RV{M}$ and $\RV{A}$ are referring to the \emph{results of measurement procedures}. We will introduce a separate notation to refer to these measurement procedures -- $\proc{F}$ is the procedure for measuring force, $\proc{M}$ and $\proc{A}$ for mass and acceleration respectively. A measurement procedure $\proc{F}$ is akin to \citet{menger_random_2003}'s notion of variables as ``consistent classes of quantities'' that consist of pairing between real-world objects and quantities of some type. Force $\proc{F}$ itself is not a well-defined mathematical thing, as measurement procedures are not mathematically well-defined. At the same time, the set of values it may yield \emph{are} well-defined mathematical things. No actual procedure can be guaranteed to return elements of a mathematical set known in advance -- anything can fail -- but we assume that we can study procedures reliable enough that we don't lose much by making this assumption.

Note that, because $\proc{F}$ is not a purely mathematical thing, we cannot perform mathematical reasoning with $\proc{F}$ directly. Rather, we introduce a variable $\RV{F}$ which, as we will see, is a well-defined mathematical object, assert that it corresponds to $\proc{F}$ and conduct our reasoning using $\RV{F}$.

\subsection{Measurment procedures}\label{sec:mprocs}

\begin{definition}[Measurement procedure]
A \emph{measurement procedure} $\proc{B}$ is a procedure that involves interacting with the real world somehow and delivering an element of a mathematical set $X$ as a result. A procedure is given the font $\proc{B}$, we say it takes values in $X$.
\end{definition}

\begin{definition}[Values yielded by procedures]
$\proc{B}\yields x$ is the proposition that the the procedure $\proc{B}$ will yield the value $x\in X$. $\proc{B}\yields A$ for $A\subset X$ is the proposition $\lor_{x\in A} \proc{B}\yields x$.
\end{definition}

\begin{definition}[Equivalence of procedures]\label{def:equality}
Two procedures $\proc{B}$ and $\proc{C}$ are equal if they both take values in $X$ and $\proc{B}\yields x\iff \proc{C}\yields x$ for all $x\in X$. If they involve different measurment actions in the real world but still necessarily yield the same result, we say they are equal.
\end{definition}

It is worth noting that this notion of equivalence identifies procedures with different real-world actions. For example, ``measure the force'' and ``measure everything, then discard everything but the force'' are often different -- in particular, it might be possible to measure the force only before one has measured everything else. Thus the result yielded by the first procedure could be available before the result of the second. However, if the first is carried out in the course of carrying out the second, they both yield the same result in the end and so we treat them as equivalent. 

Measurement procedures are like functions without well-defined domains. Just like we can compose functions with other functions to create new functions, we can compose measurement procedures with functions to produce new measurement procedures.

\begin{definition}[Composition of functions with procedures]
Given a procedure $\proc{B}$ that takes values in some set $B$, and a function $f:B\to C$, define the ``composition'' $f\circ \proc{B}$ to be any procedure $\proc{C}$ that yields $f(x)$ whenever $\proc{B}$ yields $x$. We can construct such a procedure by describing the steps: first, do $\proc{B}$ and secondly, apply $f$ to the value yielded by $\proc{B}$.
\end{definition}

For example, $\proc{MA}$ is the composition of $h:(x,y)\mapsto xy$ with the procedure $(\proc{M},\proc{A})$ that yields the mass and acceleration of the same object. Measurement procedure composition is associative:

\begin{align}
    (g\circ f)\circ\proc{B}\text{ yields } x &\iff B\text{ yields } (g\circ f)^{-1}(x) \\
    &\iff B\text{ yields } f^{-1}(g^{-1}(x))\\
    &\iff f\circ B \text{ yields } g^{-1}(x)\\
    &\iff g\circ(f\circ B)\text{ yields } x
\end{align}


One might whether there is also some kind of ``append'' operation that takes a standalone $\proc{M}$ and a standalone $\proc{A}$ and returns a procedure $(\proc{M},\proc{A})$. Unlike function composition, this would be an operation that acts on two procedures rather than a procedure and a function. Thus this ``append'' combines real-world operations somehow, which might introduce additional requirements (we can't just measure mass and acceleration; we need to measure the mass and acceleration of the same object at the same time), and may be under-specified. For example, measuring a subatomic particle's position and momentum can be done separately, but if we wish to combine the two procedures then the we can get different results depending on the order in which we combine them.

Our approach here is to suppose that there is some complete measurement procedure $\proc{S}$ to be modeled, which takes values in the observable sample space $(\Psi,\sigalg{E})$ and for all measurement procedures of interest there is some $f$ such that the procedure is equivalent to $f\circ \proc{S}$ for some $f$. In this manner, we assume that any problems that arise from a need to combine real world actions have already been solved in the course of defining $\proc{S}$.

Given that measurement processes are in practice finite precision and with finite range, $\Psi$ will generally be a finite set. We can therefore equip $\Psi$ with the collection of measurable sets given by the power set $\sigalg{E}:=\mathscr{P}(\Psi)$, and $(\Psi,\sigalg{E})$ is a standard measurable space. $\sigalg{E}$ stands for a complete collection of logical propositions we can generate that depend on the results yielded by the measurement procedure $\proc{S}$.

In probability theory, another standard kind of measurable space considered is isomorphic to $(\mathbb{R},\mathcal{B}(\mathbb{R}))$, i.e. the reals with the Borel sigma-algebra. It is not obvious to us why this would be a natural choice to represent possible results of an actual measurement. It is possible that a Borel measurable space is an appropriate idealisation of ``floating point'' measurements, but we don't have a precise argument for this.

\subsection{Observable variables}

Our total procedure $\proc{S}$ represents a large collection of subprocedures of interest, each of which can be obtained by composition of some function with $\proc{S}$. We call the pair consisting of a subprocedure of interest $\proc{X}$ along with the variable $\RV{X}$ used to obtain it from $\proc{S}$ an \emph{observable variable}.

\begin{definition}[Observable variable]
Given a measurement procedure $\proc{S}$ taking values in $(\Psi,\sigalg{E})$, an observable variable is a pair $(\RV{X}\circ \proc{S},\RV{X})$ where $\RV{X}:(\Psi,\sigalg{E})\to (X,\sigalg{X})$ is a measurable function and $\proc{X}:=\RV{X}\circ \proc{S}$ is the measurement procedure induced by $\RV{X}$ and $\proc{S}$.
\end{definition}

For the model $\RV{F}=\RV{MA}$, for example, suppose we have a complete measurement procedure $\proc{S}$ that yields a triple (force, mass, acceleration) taking values in the sets $X$, $Y$, $Z$ respectively. Then we can define the ``force'' variable $(\proc{F},\RV{F})$ where $\proc{F}:=\RV{F}\circ \proc{S}$ and $\RV{F}:X\times Y\times Z\to X$ is the projection function onto $X$.

A measurement procedure yields a particular value when it is completed. We will call a proposition of the form ``$\proc{X}$ yields $x$'' an \emph{observation}. Note that $\proc{X}$ need not be a complete procedure here. Given the complete procedure $\proc{S}$, a variable $\RV{X}:\Psi\to X$ and the corresponding procedure $\proc{X}=\RV{X}\circ\proc{S}$, the proposition ``$\proc{X}$ yields $x$'' is equivalent to the proposition ``$\proc{S}$ yields a value in $\RV{X}^{-1}(x)$''. Because of this, we define the \emph{event} $\RV{X}\yields x$ to be the set $\RV{X}^{-1}(x)$.

\begin{definition}[Event]
Given the complete procedure $\proc{S}$ taking values in $\Psi$ and an observable variable $(\RV{X}\circ \proc{S},\RV{X})$ for $\RV{X}:\Psi\to X$, the \emph{event} $\RV{X}\yields x$ is the set $\RV{X}^{-1}(x)$ for any $x\in X$.
\end{definition}

If we are given an observation ``$\proc{X}$ yields $x$'', then the corresponding event $\RV{X}\yields x$ is \emph{compatible with this observation}.

It is common to use the symbol $=$ instead of $\bowtie$ to stand for ``yields'', but we want to avoid this because $\RV{Y}=y$ already has a meaning, namely that $\RV{Y}$ is a constant function everywhere equal to $y$.

An \emph{impossible event} is the empty set. If $\RV{X}\yields x=\emptyset$ this means that we have identified no possible outcomes of the measurement process $\proc{S}$ compaible with the observatoin ``$\proc{X}$ yields $x$''. 

\subsection{Model variables}

Observable variables are special in the sense that they are tied to a particular measurement procedure $\proc{S}$. However, the measurment procedure $\proc{S}$ does not enter into our mathematical reasoning; it guides our construction of a mathematical model, but once this is done mathematical reasoning proceeds entirely with mathematical objects like sets and functions, with no further reference to the measurement procedure.

A \emph{model variable} is what we are left with if we take an observable variable and discard most of the complete measurement procedure $\proc{S}$, retaining only its set of possible values $(\Psi,\sigalg{E})$. A model variable is simply a measurable function with domain $\Psi$.

Model variables do not have to be derived from observable variables. We may instead choose a sample space for our model $(\Omega,\sigalg{F})$ that does not correspond to the possible values that $\proc{S}$ might yield. In that case, we require a surjective model variable $\RV{S}:\Omega\to \Psi$ called the complete observable variable, and every observable variable $(\RV{X}'\circ \proc{S},\RV{X}')$ is associated with the model variable $\RV{X}:=\RV{X}'\circ \RV{S}$.

An \emph{unobserved variable} is a variable whose set of possible values is not constrained by the results of the measurement procedure.

\begin{definition}[Unobserved variable]\label{def:unobserved_variable}
Given a sample space $(\Omega,\sigalg{F})$ and a complete observable variable $\RV{S}:\Omega\to\Psi$, a model variable $\RV{Y}:\Omega\to Y$ is \emph{unobserved} if $\RV{Y}(\RV{S}\yields s)=Y$ for all $s\in \Psi$.
\end{definition}

\subsection{Variable sequences}

Given $\RV{Y}:\Omega\to X$, we can define a sequence of variables: $(\RV{X},\RV{Y}):=\omega\mapsto (\RV{X}(\omega),\RV{Y}(\omega))$. $(\RV{X},\RV{Y})$ has the property that $(\RV{X},\RV{Y})\yields (x,y)= \RV{X}\yields x\cap \RV{Y}\yields y$, which supports the interpretation of $(\RV{X},\RV{Y})$ as the values yielded by $\RV{X}$ and $\RV{Y}$ together.

\subsection{Decision procedures}\label{sec:actions}

Our central problems are those in which we aim to decide on one choice from a set of possible choices, and this involves comparing the consequences that we expect to arise from each choice. A basic principle we adopt is that models are informed by the measurement procedure -- the question of whether or not a mathematical model is appropriate depends on the measurement procedure it is modelling. We do not prescribe how this dependency plays out, but we do hold that one cannot decide a mathematical model to be appropriate in the absence of a description of the measurement procedure.

Putting both of these together, this means that in order to find an appropriate model of a decision problem we need a description of a measurement procedure for each possible choice. We could in principle describe a single measurement procedure that first determines the outcome of the decision, and then for each potential choice specifies how to conduct the rest of the procedure. However, we can often often make decisions without including the decision making process in the model, and trying to include the process for making a decision in the model creates some difficult problems.

We avoid these problems by not including the procedure fo making a decision. A \emph{decision procedure} is a collection of measurement procedures, one for each element of a set of potential choices $A$. We have a background understanding -- and maybe even a precise algorithm -- for deciding on an element of $A$, but we leave this out of our model of consequences. 

\begin{definition}[Decision procedure]
A decision procedure is a collection $\{\proc{S}_\alpha\}_{\alpha\in A}$ of measurement procedures. By convention, we label sub-procedures wiht the same subscript $\proc{X}_\alpha=\RV{X}\circ\proc{S}_\alpha$
\end{definition}

Like measurement procedures, a decision procedure $\{\proc{S}_\alpha\}_{\alpha\in A}$ isn't a well-defined mathematical object; it's a ``set'' containing real-world actions. However, we think the meaning is clear enough to work with.