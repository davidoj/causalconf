%!TEX root = main.tex

\section{Appendix, needs to be organised}


\subsection{Markov categories}\label{sec:app_mcat}

\citet{fritz_synthetic_2020} defines Markov categories in the following way:

\begin{definition}\label{def:mcat}
A Markov category $C$ is a symmetric monoidal category in which every object $X \in C$ is equipped with a commutative comonoid structure given by a comultiplication $\text{copy}_X : X\to X\otimes X$ and a counit $\text{del}_X : X \to I$, depicted in string diagrams as
\begin{align}
    \text{del}_X&:=\begin{tikzpicture}[baseline={([yshift=-.5ex]current bounding box.center)}]
    \path (0,0) ++ (1,0) node (B) {};
    \draw[-{Rays[n=8]}] (A) -- (B);
\end{tikzpicture}
    \text{copy}_X&:=\begin{tikzpicture}[baseline={([yshift=-.5ex]current bounding box.center)}]
    \path (0,0) node (A) {} 
    ++ (0.5,0) node[copymap] (copy0) {}
    ++ (0.5,0.15) node (B) {}
    + (0,-0.3) node (C) {};
    \draw (A) -- (copy0) to [out=45,in=180] (B) (copy0) to [out=-45, in=180] (C);
\end{tikzpicture}
\end{align}
and satisfying the commutative comonoid equations
\begin{align}
    \tikzfig{ccom_lhs} = \tikzfig{ccom_rhs}
\end{align}
\begin{align}
    \tikzfig{ccom2_lhs} = \tikzfig{ccom2_mhs} = \tikzfig{ccom2_rhs}
\end{align}
\begin{align}
    \tikzfig{ccom3_lhs} = \tikzfig{ccom3_rhs}
\end{align}
as well as compatibility with the monoidal structure
\begin{align}
    \tikzfig{mstruct1_lhs} &= \tikzfig{mstruct1_rhs}\\
    \tikzfig{mstruct2_lhs} &= \tikzfig{mstruct2_rhs}
\end{align}
and the naturality of \emph{del}, which means that
\begin{align}
    \tikzfig{naturality_lhs} &= \tikzfig{naturality_rhs}
\end{align}
for every morphism $f$.
\end{definition}

\subsection{Existence of conditional probabilities}


\begin{lemma}[Conditional pushforward]\label{th:recurs_pushf}
Suppose we have a sample space $(\Omega,\sigalg{F})$, variables $\RV{X}:\Omega\to X$ and $\RV{Y}:\Omega\to Y$, $\RV{Z}:\Omega\to Z$ and a probability set $\prob{P}_{\{\}}$ with conditional $\prob{P}_{\{\}}^{\RV{X}|\RV{Y}}$ such that $\RV{Z}=f\circ \RV{Y}$ for some $f:Y\to Z$. Then there exists a conditional probability $\prob{P}_{\{\}}^{\RV{Z}|\RV{X}}=\prob{P}_{\{\}}^{\RV{Y}|\RV{X}}\kernel{F}_{f}$.
\end{lemma}

\begin{proof}
Note that $(\RV{X},\RV{Z})=(\text{id}_X\otimes f)\circ (\RV{X},\RV{Y})$. Thus, by Lemma \ref{lem:pushf_kprod}, for any $\prob{P}_\alpha\in \prob{P}_{\{\}}$

\begin{align}
    \prob{P}_\alpha^{\RV{XZ}} = \prob{P}_\alpha^{\RV{XY}}\kernel{F}_{\text{id}_X\otimes f}
\end{align}

Note also that for all $A\in\sigalg{X}$, $B\in \sigalg{Z}$, $x\in X$, $y\in Y$:

\begin{align}
\prob{F}_{\text{id}_X\otimes f}(A\times B|x,y)&=\delta_x(A)\delta_{f(y)}(B)\\
&= \prob{F}_{\text{id}_X} (A|x)\otimes \prob{F}_f(B|y)\\
\implies \prob{F}_{\text{id}_X\otimes f} &= \prob{F}_{\text{id}_X} \otimes \prob{F}_f
\end{align}

Thus

\begin{align}
    \prob{P}_\alpha^{\RV{XZ}} &= (\prob{P}_\alpha^{\RV{X}}\odot \prob{P}_{\{\}}^{\RV{Y}|\RV{X}})\kernel{F}_{\text{id}_X}\otimes \kernel{F}_f\\
    &= \tikzfig{conditional_pushforward}
\end{align}

Which implies $\prob{P}_{\{\}}^{\RV{Y}|\RV{X}}\kernel{F}_{f}$ is a version of $\prob{P}_{\alpha}^{\RV{Z}|\RV{X}}$. Because this holds for all $\alpha$, it is therefore also a version of $\prob{P}_{\{\}}^{\RV{Z}|\RV{X}}$.
\end{proof}

\begin{theorem}[Existence of regular conditionals]\label{th:reg_cond}
Suppose we have a sample space $(\Omega,\sigalg{F})$, variables $\RV{X}:\Omega\to X$ and $\RV{Y}:\Omega\to Y$ with $Y$ standard measurable and a probability model $\prob{P}_{\alpha}$ on $(\Omega,\sigalg{F})$. Then there exists a conditional $\prob{P}_\alpha^{\RV{Y}|\RV{X}}$.
\end{theorem}

\begin{proof}
This is a standard result, see for example \cite{cinlar_probability_2011} Theorem 2.18.
\end{proof}

\begin{theorem}[Existence of higher order valid conditionals with respect to probability sets]\label{th:ho_cond_psets}
Suppose we have a sample space $(\Omega,\sigalg{F})$, variables $\RV{X}:\Omega\to X$ and $\RV{Y}:\Omega\to Y$, $\RV{Z}:\Omega\to Z$ and a probability set $\prob{P}_{\{\}}$ with regular conditional $\prob{P}_{\{\}}^{\RV{YZ}|\RV{X}}$ and $Y$ and $Z$ standard measurable. Then there exists a regular $\prob{P}_{\{\}}^{\RV{Z}|(\RV{Y}|\RV{X})}$.
\end{theorem}

\begin{proof}
Given a Borel measurable map $m:X\kto Y\times Z$ let $f:Y\times Z\to Y$ be the projection onto $Y$. Then $f\circ (\RV{Y},\RV{Z})=\RV{Y}$. \citet{bogachev_kantorovich_2020}, Theorem 3.5 proves that there exists a Borel measurable map $n:X\times Y\kto Y\times Z$  such that 
\begin{align}
    n(f^{-1}(y)|x,y) = 1\label{eq:proper}\\
    m(\RV{Y}^{-1}(A)\cap B|x) = \int_A n(B|x,y) m\kernel{F}_{f}(dy|x)&\forall A\in \sigalg{Y},B\in\sigalg{Y\times Z}\label{eq:conditional1}
\end{align}
In particular, $\prob{P}_{\{\}}^{\RV{YZ}|\RV{X}}$ is a Borel measurable map $X\kto Y\times Z$. Thus equation \ref{eq:conditional1} implies for all $A\in \sigalg{Y},B\in\sigalg{Y\times Z}$

\begin{align}
    \prob{P}_{\{\}}^{\RV{YZ}|\RV{X}}(\RV{Y}^{-1}(A)\cap B|x) &= \int_A n(B|x,y) \prob{P}_{\{\}}^{\RV{YZ}|\RV{X}}\kernel{F}_{f}(dy|x)\\
    &=\int_A n(B|x,y) \prob{P}_{\{\}}^{\RV{Y}|\RV{X}}(dy|x)\label{eq:rec_push}
\end{align}

Where Equation \ref{eq:rec_push} follows from Lemma \ref{th:recurs_pushf}.

Then, for any $\prob{P}_\alpha\in\prob{P}_{\{\}}$

\begin{align}
    \prob{P}_{\{\}}^{\RV{YZ}|\RV{X}}(\RV{Y}^{-1}(A)\cap B|x) &= \int_A n(B|x,y) \prob{P}_{\alpha}^{\RV{Y}|\RV{X}}(dy|x)
\end{align}

which implies $n$ is a version of $\prob{P}_{\{\}}^{\RV{YZ}|(\RV{Y}|\RV{X})}$. By Lemma \ref{th:recurs_pushf}, $n\kernel{F}_{f}$ is a version of $\prob{P}_{\{\}}^{\RV{Z}|(\RV{Y}|\RV{X})}$.
\end{proof}

We might be motivated to ask whether the higher order conditionals in Theorem \ref{th:ho_cond_psets} can be chosen to be valid. Despite Lemma \ref{lem:proper_implies_valid} showing that the existence of proper conditional probabilities implies the existence of valid ones, we cannot make use of this in the above theorem because Equation \ref{eq:proper} makes $n$ proper with respect to the ``wrong'' sample space $(Y\times Z, \sigalg{Y}\otimes\sigalg{Z})$ while what we would need is a proper conditional probability with respect to $(\Omega,\sigalg{F})$.

We can choose higher order conditionals to be valid in the case of discrete sets, and whether we can choose them to be valid in more general measurable spaces is an open question.

\begin{theorem}[Higher order conditionals]\label{th:higher_order_conditionals}
Suppose we have a sample space $(\Omega,\sigalg{F})$, variables $\RV{X}:\Omega\to X$ and $\RV{Y}:\Omega\to Y$, $\RV{Z}:\Omega\to Z$ and a probability set $\prob{P}_{\{\}}$ with conditional $\prob{P}_{\{\}}^{\RV{YZ}|\RV{X}}$. Then $\prob{P}_{\{\}}^{\RV{Z}|(\RV{Y}|\RV{X})}$ is a version of $\prob{P}_{\{\}}^{\RV{Z}|\RV{Y}\RV{X}}$ 
\end{theorem}

\begin{proof}
For arbitrary $\prob{P}_{\alpha}\in \prob{P}_{\{\}}$
\begin{align}
    \prob{P}_\alpha^{\RV{YZ}|\RV{X}} &= \tikzfig{higher_order_disint}\\
    \implies \prob{P}_\alpha^{\RV{XYZ}} &= \tikzfig{higher_order_disint_0}\\
    &= \tikzfig{higher_order_disint_1}\\
    &= \tikzfig{higher_order_disint_2}
\end{align}
Thus $\prob{P}_{\{\}}^{\RV{Z}|(\RV{Y}|\RV{X})}$ is a version of $\prob{P}_{\alpha}^{\RV{Z}|\RV{Y}\RV{X}}$ for all $\alpha$ and hence also a version of $\prob{P}_{\{\}}^{\RV{Z}|\RV{Y}\RV{X}}$.
\end{proof}


\begin{theorem}
Given probability gap model $\prob{P}_{\{\}}$, $\RV{X}$, $\RV{Y}$, $\RV{Z}$ such that $\prob{P}_{\{\}}^{\RV{Z}|\RV{YX}}$ exists, $\prob{P}_{\{\}}^{\RV{Z}|\RV{Y}}$ exists iff $\RV{Z}\CI_{\prob{P}_{\{\}}} \RV{X}|\RV{Y}$.
\end{theorem}

\begin{proof}
If:
If $\RV{Z}\CI_{\prob{P}_{\{\}}} \RV{X}|\RV{Y}$ then by Theorem \ref{th:cho_ci_equiv}, for each $\prob{P}_\alpha\in \prob{P}_{\{\}}$ there exists $\prob{P}_{\alpha}^{\RV{Z}|\RV{Y}}$ such that
\begin{align}
    \prob{P}_\alpha^{\RV{Y}|\RV{WX}} &= \tikzfig{cond_indep_erase}
\end{align}
\end{proof}


\begin{theorem}[Valid higher order conditionals]
Suppose we have a sample space $(\Omega,\sigalg{F})$, variables $\RV{X}:\Omega\to X$ and $\RV{Y}:\Omega\to Y$, $\RV{Z}:\Omega\to Z$ and a probability set $\prob{P}_{\{\}}$ with regular conditional $\prob{P}_{\{\}}^{\RV{YZ}|\RV{X}}$, $Y$ discrete and $Z$ standard measurable. Then there exists a valid regular $\prob{P}_{\{\}}^{\RV{Z}|\RV{XY}}$.
\end{theorem}

\begin{proof}
By Theorem \ref{th:ho_cond_psets}, we have a higher order conditional $\prob{P}_{\{\}}^{\RV{Z}|(\RV{Y}|\RV{X})}$ which, by Theorem \ref{th:higher_order_conditionals} is also a version of $\prob{P}_{\{\}}^{\RV{Z}|\RV{XY}}$.

We will show that there is a Markov kernel $\kernel{Q}$ almost surely equal to $\prob{P}_{\{\}}^{\RV{Z}|\RV{XY}}$ which is also valid. For all $x,y\in X\times Y$, $A\in\sigalg{Z}$ such that $(\RV{X},\RV{Y},\RV{Z})\yields\{(x,y)\}\times A=\emptyset$, let $\kernel{Q}(A|x,y)=\prob{P}_{\{\}}^{\RV{Z}|\RV{XY}}(A|x,y)$.

By validity of $\prob{P}_{\{\}}^{\RV{YZ}|\RV{X}}$, $x\in \RV{X}(\Omega)$ and $(\RV{X},\RV{Y},\RV{Z})\yields\{(x,y)\}\times A=\emptyset$ implies $\prob{P}_{\{\}}^{\RV{YZ}|\RV{X}}(\{y\}\times A|x)=0$. Thus we need to show

\begin{align}
    \forall A\in \sigalg{Z}, x\in X, y\in Y: \prob{P}_{\{\}}^{\RV{YZ}|\RV{X}}(\{y\}\times A|x)=0 \implies \left(\prob{Q}(A|x,y) = 0\right) \lor \left((\RV{X},\RV{Y})\yields \{(x,y)\} = \emptyset\right)
\end{align}

For all $x,y$ such that $\kernel{P}_{\{\}}^{\RV{Y}|\RV{X}}(\{y\}|x)$ is positive, we have $\model{P}^{\RV{YZ}|\RV{X}}(\{y\}\times A|x)=0\implies \prob{P}_\square^{\RV{Z}|\RV{XY}}(A|x,y)=0=:\kernel{Q}(A|x,y)$.

Furthermore, where $\kernel{P}_{\{\}}^{\RV{Y}|\RV{X}}(\{y\}|x)=0$, we either have $(\RV{X},\RV{Y},\RV{Z})\yields\{(x,y)\}\times A= \emptyset$ or can choose some $\omega\in (\RV{X},\RV{Y},\RV{Z})\yields\{(x,y)\}\times A$ and let $\kernel{Q}(\RV{Z}(\omega)|x,y) = 1$. This is an arbitrary choice, and may differ from the original $\prob{P}_{\{\}}^{\RV{Z}|\RV{XY}}$. However, because $Y$ is discrete the union of all points $y$ where $\kernel{P}_{\{\}}^{\RV{Y}|\RV{X}}(\{y\}|x)=0$ is a measure zero set, and so $\kernel{Q}$ differs from $\kernel{P}_{\{\}}^{\RV{Y}|\RV{X}}$ on a measure zero set.
\end{proof}


\subsection{Validity}

Validity is related to \emph{proper} conditional probabilities. In particular, valid conditional probabilities exist when regular proper conditional probabilities exist.

\begin{definition}[Regular proper conditional probability]
Given a probability space $(\mu,\Omega,\sigalg{F})$ and a variable $\RV{X}:\Omega\to X$, a regular proper conditional probbability $\mu^{|\RV{X}}:X\kto \Omega$ is Markov kernel such that
\begin{align}
    \mu(A\cap \RV{X}^{-1}(B))&=\int_{B} \mu^{|\RV{X}}(A|x) \mu^{\RV{X}}(\mathrm{d}x) &\forall A\in \sigalg{X}, B\in \sigalg{F}\\
    &\iff\\
    \mu&= \tikzfig{disint_def}
\end{align}
and
\begin{align}
    \mu^{|\RV{X}}(\RV{X}^{-1}(A)|x) &= \delta_x(A)
\end{align}
\end{definition}

\begin{lemma}\label{lem:proper_implies_valid}
Given a probability space $(\mu,\Omega,\sigalg{F})$ and variables $\RV{X}:\Omega\to X$, $\RV{Y}:\Omega\to Y$, if there is a regular proper conditional probability $\mu^{|\RV{X}}:X\kto \Omega$ then there is a valid conditional distribution $\mu^{\RV{Y}|\RV{X}}$.
\end{lemma}

\begin{proof}
Take $\kernel{K}=\mu^{|\RV{X}}\kernel{F}_{\RV{Y}}$. We will show that $\kernel{K}$ is valid, and a version of $\mu^{\RV{Y}|\RV{X}}$.

Defining $\RV{O}:=\text{id}_{\Omega}$ (the identity function $\Omega\to \Omega$), $\mu^{|\RV{X}}$ is a version of $\mu^{\RV{O}|\RV{X}}$. Note also that $\RV{Y}=\RV{Y}\circ\RV{O}$. Thus by Lemma \ref{th:recurs_pushf}, $\kernel{K}$ is a version of $\mu^{\RV{Y}|\RV{X}}$.

It remains to be shown that $\kernel{K}$ is valid. Consider some $x\in X$, $A\in \sigalg{Y}$ such that $\RV{X}^{-1}(\{x\})\cap \RV{Y}^{-1}(A)=\emptyset$. Then by the assumption $\mu^{|\RV{X}}$ is proper
\begin{align}
    \kernel{K}(\RV{Y}\yields A|x) &= \delta_x(\RV{Y}^{-1}(A))\\
    &= 0
\end{align}

Thus $\kernel{K}$ is valid.
\end{proof}


\begin{theorem}[Validity]\label{th:completion}
Given $(\Omega,\sigalg{F})$, $\RV{X}:\Omega\to X$, $\kernel{J}\in \Delta(X)$ with $\Omega$ and $X$ standard measurable, there exists some $\mu\in \Delta(\Omega)$ such that $\mu^{\RV{X}}=\kernel{J}$ if and only if $\kernel{J}$ is a valid distribution.
\end{theorem}

\begin{proof}
If:
This is a Theorem 2.5 of \citet{ershov_extension_1975}.
Only if:
This is also found in \citet{ershov_extension_1975}, but is simple enough to reproduce here. Suppose $\kernel{J}$ is not a valid probability distribution. Then there is some $x\in X$ such that $\RV{X}\yields x = \emptyset$ but $\kernel{J}(x)>0$. Then
\begin{align}
    \mu^{\RV{X}}(x) &= \mu (\RV{X}\yields x)\\
    &= \sum_{x'\in X} \kernel{J}(x') \kernel{K}(\RV{X}\yields x|x')\\
    &= 0\\
    &\neq \kernel{J}(x)
\end{align}
\end{proof}


\begin{lemma}[Semidirect product defines an intersection of probability sets]\label{th:intersection}
Given $(\Omega,\sigalg{F})$, $\RV{X}:\Omega\to (X,\sigalg{X})$, $\RV{Y}:\Omega\to (Y,\sigalg{Y})$, $\RV{Z}:\Omega\to (Z,\sigalg{Z})$ all standard measurable and maximal probability sets $\prob{P}_{\{\}}^{\RV{Y}|\RV{X}[M]}$ and $\prob{Q}_{\{\}}^{\RV{Z}|\RV{YX}[M]}$ then defining
\begin{align}
    \prob{R}_{\{\}}^{\RV{YZ}|\RV{X}} := \prob{P}_{\{\}}^{\RV{Y}|\RV{X}}\odot \prob{Q}_{\{\}}^{\RV{Z}|\RV{YX}}
\end{align}
we have
\begin{align}
    \prob{R}_{\{\}}^{\RV{YZ}|\RV{X}[M]} &= \prob{P}_{\{\}}^{\RV{Y}|\RV{X}[M]}\cap \prob{Q}_{\{\}}^{\RV{Z}|\RV{YX}[M]}
\end{align}
\end{lemma}

\begin{proof}
For any $\prob{R}_a\in\prob{R}_{\{\}}$

\begin{align}
    \prob{R}_{a}^{\RV{XYZ}} &= \prob{R}_a^{\RV{X}}\odot \prob{P}_{\{\}}^{\RV{Y}|\RV{X}}\odot \prob{Q}_{\{\}}^{\RV{Z}|\RV{YX}}\\
    \implies \prob{R}_{a}^{\RV{XY}} &= \prob{R}_a^{\RV{X}}\odot \prob{P}_{\{\}}^{\RV{Y}|\RV{X}}\\
    \land \prob{R}_{a}^{\RV{XYZ}} &= \prob{R}_{a}^{\RV{XY}}\odot \prob{Q}_{\{\}}^{\RV{Z}|\RV{YX}}
\end{align}

Thus $\prob{P}_{\{\}}^{\RV{Y}|\RV{X}}$ is a version of $\prob{R}_{\{\}}^{\RV{Y}|\RV{X}}$ and $\prob{Q}^{\RV{Z}|\RV{YX}}$ is a version of $\prob{R}_{\{\}}^{\RV{Z}|\RV{YX}}$ so $\prob{R}_{\{\}}\subset \prob{P}_{\{\}}\cap\prob{Q}_{\{\}}$.

Suppose there's an element $\prob{S}$ of $\prob{P}_{\{\}}\cap\prob{Q}_{\{\}}$ not in $\prob{R}_{\{\}}$. Then by definition of $\prob{R}_{\{\}}$, $\prob{R}_{\{\}}^{\RV{YZ}|\RV{X}}$ is not a version of $\prob{S}_{\{\}}^{\RV{YZ}|\RV{X}}$. But by construction of $\prob{S}$, $\prob{P}_{\{\}}^{\RV{Y}|\RV{X}}$  is a version of $\prob{S}^{\RV{Y}|\RV{X}}$ and  $\prob{Q}^{\RV{Z}|\RV{YX}}$ is a version of $\prob{S}^{\RV{Z}|\RV{YX}}$. But then by the definition of disintegration, $\prob{P}_{\{\}}^{\RV{Y}|\RV{X}} \odot \prob{Q}_{\{\}}^{\RV{Z}|\RV{YX}}$ is a version of $\prob{S}_{\{\}}^{\RV{YZ}|\RV{X}}$ and so $\prob{R}_{\{\}}^{\RV{YZ}|\RV{X}}$ is a version of $\prob{S}_{\{\}}^{\RV{YZ}|\RV{X}}$, a contradiction.
\end{proof}


\begin{lemma}[Equivalence of validity definitions]\label{th:valid_agree}
Given $\RV{X}:\Omega\to X$, with $\Omega$ and $X$ standard measurable, a probability measure $\prob{P}^{\RV{X}}\in \Delta(X)$ is valid if and only if the conditional $\prob{P}^{\RV{X}|*}:=*\mapsto \prob{P}^{\RV{X}}$ is valid.
\end{lemma}

\begin{proof}
$*\yields *=\Omega$ necessarily. Thus validity of $\prob{P}^{\RV{X}|*}$ means 

\begin{align}
    \forall A\in \sigalg{X}: \RV{X}\yields A=\emptyset \implies \prob{P}^{\RV{X}|*}(A|*)&=0
\end{align}

But $\prob{P}^{\RV{X}|*}(A|*)=\prob{P}^{\RV{X}}(A)$ by definition, so this is equivalent to

\begin{align}
    \forall A\in \sigalg{X}: \RV{X}\yields A=\emptyset \implies \prob{P}^{\RV{X}}(A)&=0
\end{align}
\end{proof}


\begin{lemma}[Semidirect product of valid candidate conditionals is valid]\label{lem:valid_extendability}
Given $(\Omega,\sigalg{F})$, $\RV{X}:\Omega\to X$, $\RV{Y}:\Omega\to Y$, $\RV{Z}:\Omega\to Z$ (all spaces standard measurable) and any valid candidate conditional $\prob{P}^{\RV{Y}|\RV{X}}$ and $\prob{Q}^{\RV{Z}|\RV{Y}\RV{X}}$, $ \prob{P}^{\RV{Y}|\RV{X}}\odot \prob{Q}^{\RV{Z}|\RV{Y}\RV{X}}$ is also a valid candidate conditional.
\end{lemma}

\begin{proof}
Let $\prob{R}^{\RV{YZ}|\RV{X}}:=\prob{P}^{\RV{Y}|\RV{X}}\odot \prob{Q}^{\RV{Z}|\RV{Y}\RV{X}}$.

We only need to check validity for each $x\in \RV{X}(\Omega)$, as it is automatically satisfied for other values of $\RV{X}$.

For all $x\in \RV{X}(\Omega)$, $B\in \sigalg{Y}$ such that $\RV{X}\yields \{x\}\cap\RV{Y}\yields B=\emptyset$, $\prob{P}^{\RV{Y}|\RV{X}}(B|x)=0$ by validity. Thus for arbitrary $C\in \sigalg{Z}$
\begin{align}
    \prob{R}^{\RV{YZ}|\RV{X}}(B\times C|x) &= \int_B \prob{Q}^{\RV{Z}|\RV{YX}}(C|y,x)\prob{P}^{\RV{Y}|\RV{X}}(dy|x)\\
                                  &\leq \prob{P}^{\RV{Y}|\RV{X}}(B|x)\\
                                  &=0
\end{align}

For all $\{x\}\times B$such that $\RV{X}\yields \{x\}\cap\RV{Y}\yields B\neq \emptyset$ and $C\in \sigalg{Z}$ such that $(\RV{X},\RV{Y},\RV{Z})\yields \{x\}\times B\times C=\emptyset$, $\prob{Q}^{\RV{Z}|\RV{YX}}(C|y,x)=0$ for all $y\in B$ by validity. Thus:
\begin{align}
    \prob{R}^{\RV{YZ}|\RV{X}}(B\times C|x) &= \int_B \prob{Q}^{\RV{Z}|\RV{YX}}(C|y,x)\prob{P}^{\RV{Y}|\RV{X}}(dy|x)\\
                                            &=0
\end{align}
\end{proof}

\begin{corollary}[Valid conditionals are validly extendable to valid distributions]\label{corr:valid_extend_order1}
Given $\Omega$, $\RV{U}:\Omega\to U$, $\RV{W}:\Omega\to W$ and a valid conditional $\prob{T}^{\RV{W}|\RV{U}}$, then for any valid conditional $\prob{V}^{\RV{U}}$, $\prob{V}^{\RV{U}}\odot \prob{T}^{\RV{W}|\RV{U}}$ is a valid probability.
\end{corollary}

\begin{proof}
Applying Lemma \ref{lem:valid_extendability} choosing $\RV{X}=*$, $\RV{Y}=\RV{U}$, $\RV{Z}=\RV{W}$ and $\prob{P}^{\RV{Y}|\RV{X}}=\prob{V}^{\RV{U}|*}$ and $\prob{Q}^{\RV{Z}|\RV{YX}}=\prob{T}^{\RV{W}|\RV{U*}}$ we have $\prob{R}^{WU|*}:=\prob{V}^{\RV{U}|*}\odot \prob{T}^{\RV{W}|\RV{U}*}$ is a valid conditional probability. Then $\prob{R}^{\RV{WU}}\cong \prob{R}^{\RV{WU}|*}$ is valid by Theorem \ref{th:valid_agree}.
\end{proof}

\begin{theorem}[Validity of conditional probabilities]\label{th:valid_conditional_probability}
Suppose we have $\Omega$, $\RV{X}:\Omega\to X$, $\RV{Y}:\Omega\to Y$, with $\Omega$, $X$, $Y$ discrete. A conditional $\prob{T}^{\RV{Y}|\RV{X}}$ is valid if and only if for all valid candidate distributions $\prob{V}^{\RV{X}}$, $\prob{V}^{\RV{X}}\odot \prob{T}^{\RV{Y}|\RV{X}}$ is also a valid candidate distribution.
\end{theorem}

\begin{proof}
If: this follows directly from Corollary \ref{corr:valid_extend_order1}.

Only if: suppose $\prob{T}^{\RV{Y}|\RV{X}}$ is invalid. Then there is some $x\in X$, $y\in Y$ such that $\RV{X}\yields(x)\neq \emptyset$, $(\RV{X},\RV{Y})\yields(x,y)=\emptyset$ and $\prob{T}^{\RV{Y}|\RV{X}}(y|x)>0$. Choose $\prob{V}^{\RV{X}}$ such that $\prob{V}^{\RV{X}}(\{x\})=1$; this is possible due to standard measurability and valid due to $\RV{X}^{-1}(x)\neq \emptyset$. Then
\begin{align}
    (\prob{V}^{\RV{X}}\odot \prob{T}^{\RV{Y}|\RV{X}})(x,y) &= \prob{T}^{\RV{Y}|\RV{X}}(y|x) \prob{V}^{\RV{X}}(x)\\
                                                                     &= \prob{T}^{\RV{Y}|\RV{X}}(y|x)\\
                                                                     &>0
\end{align}
Hence $\prob{V}^{\RV{X}}\odot \prob{T}^{\RV{Y}|\RV{X}}$ is invalid.
\end{proof}

% \begin{theorem}[Existence of valid conditional probabilities]\label{th:valid_disint}
% Given a probability gap model $\prob{P}_\square:A\to \Delta(\Omega)$ along with a valid conditional probability $\model{P}_\square^{\RV{XY}|\RV{W}}$, there exists a valid conditional probability $\prob{P}_\square^{\RV{Y}|\RV{WX}}$.
% \end{theorem}

% \begin{proof}
% From Lemma \ref{lem:disint_exist}, we have the existence of some Markov kernel $\prob{P}_\square^{\RV{Y}|\RV{WX}}:W\times X\to Y$ such that
% \begin{align}
%     \prob{P}_\square^{\RV{XY}|\RV{W}}=\prob{P}_\square^{\RV{X}|\RV{W}}\odot \prob{P}_\square^{\RV{Y}|\RV{WX}}\label{eq:k_disint}
% \end{align}

% By definition of conditional probability , for any insert $\alpha\in A$ there exists $\prob{P}_\alpha^{\RV{W}}\in\Delta(W)$ such that

% \begin{align}
%     \prob{P}_\alpha^{\RV{WXY}}=\prob{P}_\alpha^{\RV{W}}\odot\prob{P}_\square^{\RV{XY}|\RV{W}}
% \end{align}

% Thus

% \begin{align}
% \prob{P}_\alpha^{\RV{WXY}}&= \prob{P}_\alpha^{\RV{W}}\odot(\prob{P}_\square^{\RV{X}|\RV{W}}\odot \prob{P}_\square^{\RV{Y}|\RV{WX}})\\
% &= (\prob{P}_\alpha^{\RV{W}}\odot\prob{P}_\square^{\RV{X}|\RV{W}})\odot \prob{P}_\square^{\RV{Y}|\RV{WX}})
% \end{align}

% Let $\text{erasef}_Y:Y\to \{*\}$ be the erase function on $Y$ and $\text{id}_{W\times X}$ be the identity function on $W\times X$. Noting that 
% \begin{align}
% (\RV{W},\RV{X})&=(\text{idf}_{W\times X}\otimes \text{erasef}_Y)\circ (\RV{W},\RV{X},\RV{Y})
% \end{align}
% By Lemma \ref{lem:prod_pushf} together with Theorem \ref{th:recurs_pushf} we have for all $\alpha$:

% \begin{align}
%     \prob{P}_\alpha^{\RV{XW}} &= \prob{P}_\alpha^{\RV{WXY}}(\text{id}_{W\times X}\otimes \text{erase}_Y)\\
%                               &= \prob{P}_\alpha^{\RV{W}}\odot(\prob{P}_\square^{\RV{X}|\RV{W}}\odot \prob{P}_\square^{\RV{Y}|\RV{WX}})(\text{id}_{W\times X}\otimes \text{erase}_Y)\\
%                               &= \prob{P}_\alpha^{\RV{W}}\odot\prob{P}_\square^{\RV{X}|\RV{W}}
% \end{align}

% Then

% \begin{align}
% \prob{P}_\alpha^{\RV{XWY}}&= (\prob{P}_\alpha^{\RV{XW}})\odot \prob{P}_\square^{\RV{Y}|\RV{WX}})
% \end{align}

% And so $\prob{P}_\square^{\RV{Y}|\RV{WX}})$ is a $\RV{Y}|\RV{WX}$ conditional probability. We also want it to be valid, so we will verify that it can be chosen as such.

% We also need to check that $\prob{P}_\square^{\RV{Y}|\RV{WX}}$ can be chosen so that it is valid. By validity of $\model{K}^{\RV{W,Y}|\RV{X}}$, $w\in \RV{W}(\Omega)$ and $(\RV{X},\RV{W},\RV{Y})\yields(x,w,y)=\emptyset \implies \model{P}_\square^{\RV{W,Y}|\RV{X}}=0$, so we only need to check for $(w,x,y)$ such that $\model{P}_\square^{\RV{W,Y}|\RV{X}}(w,y|x)=0$. For all $x,y$ such that $\kernel{K}^{\RV{Y}|\RV{X}}(y|x)$ is positive, we have $\model{P}^{\RV{W,Y}|\RV{X}}(w,y|x)=0\implies \prob{P}_\square^{\RV{Y}|\RV{WX}}(y|w,x)=0$. Furthermore, where $\model{K}^{\RV{W}|\RV{X}}(w|x)=0$, we either have $(\RV{W},\RV{X})\yields(w,x)=\emptyset$ or we can choose some $\omega\in (\RV{W},\RV{X})\yields(w,x)$ and let $\prob{P}^{\RV{Y}|\RV{WX}}(\RV{Y}(\omega)|w,x) = 1$.
% \end{proof}

\subsection{Conditional independence}\label{sec:cond_ind_app}
\begin{reptheorem}{th:cons_ci}
Given standard measurable $(\Omega,\sigalg{F})$, variables $\RV{W}:\Omega\to W$, $\RV{X}:\Omega\to X$, $\RV{Y}:\Omega\to Y$ and a probability set $\prob{P}_{C}$ with uniform conditional probability $\prob{P}_{C}^{\RV{Y}|\RV{WX}}$ and $\alpha\in C$ such that $\prob{P}_\alpha^{\RV{WX}}\gg \{\prob{P}_\beta^{\RV{WX}}|\beta\in C\}$, $\RV{Y}\CI_{\prob{P}_{\alpha}}\RV{X}|\RV{W}$ if and only if there is a version of $\prob{P}_{C}^{\RV{Y}|\RV{WX}}$ and $\kernel{K}:W\kto Y$ such that
\begin{align}
  \prob{P}_C^{\RV{Y}|\RV{WX}} &= \tikzfig{cond_indep_erase}
\end{align}
\end{reptheorem}

\begin{proof}
If:
By assumption, for every $\beta\in A$ we can write
\begin{align}
  \prob{P}_\beta^{\RV{Y}|\RV{WX}} &= \tikzfig{cond_indep_erase_beta}
\end{align}
And so, by Theorem \ref{th:cho_ci_equiv}, $\RV{Y}\CI_{\prob{P}_\beta}\RV{X}|\RV{W}$ for all $\beta\in A$, and in particular $\RV{Y}\CI_{\prob{P}_{\alpha}}\RV{X}|\RV{W}$.
Only if:
By Theorem \ref{th:cho_ci_equiv}, there exists a version of $\prob{P}_\alpha^{\RV{Y}|\RV{WX}}$ such that
\begin{align}
  \prob{P}_\alpha^{\RV{Y}|\RV{WX}} &= \tikzfig{cond_indep_erase_alpha}
\end{align}

Because $\prob{P}_{\alpha}^{\RV{WX}}$ dominates $\{\prob{P}_\beta^{\RV{WX}}|\beta\in C\}$ and the set of points on which $\prob{P}_\alpha^{\RV{Y}|\RV{WX}}$ differs from $\prob{P}_C^{\RV{Y}|\RV{WX}}$ is of $\prob{P}_\alpha$ measure 0, this set must also be of $\prob{P}_\beta$ measure 0 for all $\beta\in C$. Therefore $\prob{P}_\alpha^{\RV{Y}|\RV{WX}}$ is a version of $\prob{P}_{C}^{\RV{Y}|\RV{WX}}$, and so
\begin{align}
      \prob{P}_C^{\RV{Y}|\RV{WX}} &= \tikzfig{cond_indep_erase_alpha}
\end{align}
\end{proof}

% \subsection{Extended conditional independence}\label{ap:eci}

% \citet{constantinou_extended_2017} introduced the idea of \emph{extended conditional independence}, which is a notion of conditional independence with respect to a parametrised collection of probability measures. It is motivated in part by the observation that such parametrised collections can be used to model causal questions. Furthermore, probability sets are closely related to parametrised probability sets -- one can get the former from the latter by simply dropping the parameters. 

% \todo[inline]{This needs major revision, and is not a top priority right now}

% In the case of a probability gap model $(\prob{P}_{\square}^{\RV{V}|\RV{W}},A)$ where there is some $\alpha\in A$ dominating $A$, we can relate conditional independence with respect to $\prob{P}_\square$ to what , which is a notion they define with respect to a Markov kernel. These concepts may differ if $A$ is not dominated. Theorem 4.4 of \citet{constantinou_extended_2017} proves the following claim:


% \begin{theorem}\label{th:dawid_constantionou}
% Let $\RV{A}^*=\RV{A}\circ \RV{V}$, $\RV{B}^*=\RV{B}\circ\RV{V}$, $\RV{C}^*=\RV{C}\circ \RV{V}$ ($(\RV{A},\RV{B},\RV{C})$ are $\sigalg{V}$-measurable) and $\RV{D}^*=\RV{D}\circ \RV{W}$,$\RV{E}^*=\RV{E}\circ \RV{W}$ where $W$ is discrete and $\RV{W}=(\RV{D}^*,\RV{E}^*)$. In addition, let $\prob{P}_\alpha^{\RV{W}}$ be some probability distribution on $\RV{W}$ such that $w\in\RV{W}(\Omega)\implies \prob{P}_\alpha^{\RV{W}}(w)>0$. Then, denoting extended conditional independence with $\CI_{\prob{P},\text{ext}}$ and $\prob{P}_\alpha^{\RV{VW}}:=\prob{P}_\alpha^{\RV{W}}\odot \prob{P}^{\RV{V}|\RV{W}}$
% \begin{align}
%     \RV{A}\CI_{\prob{P},\text{ext}}(\RV{B},\RV{D})|(\RV{C},\RV{E})\iff \RV{A}^*\CI_{\prob{P}_\alpha}(\RV{B}^*,\RV{D}^*)|(\RV{C}^*,\RV{E}^*)
% \end{align}
% \end{theorem}

% This result implies a close relationship between order 1 conditional independence and extended conditional independence.

% \begin{theorem}
% Let $\RV{A}^*=\RV{A}\circ \RV{V}$, $\RV{B}^*=\RV{B}\circ\RV{V}$, $\RV{C}^*=\RV{C}\circ \RV{V}$ ($(\RV{A},\RV{B},\RV{C})$ are $\sigalg{V}$-measurable) and $\RV{D}^*=\RV{D}\circ \RV{W}$,$\RV{E}^*=\RV{E}\circ \RV{W}$ where $V,W$ are discrete and $\RV{W}=(\RV{D}^*,\RV{E}^*)$. Then letting $\prob{P}_\alpha^{\RV{VW}}:=\prob{P}_\alpha^{\RV{W}}\odot \prob{P}^{\RV{V}|\RV{W}}$
% \begin{align}
%     \RV{A}\CI^1_{\prob{P},\text{ext}}(\RV{B},\RV{D})|(\RV{C},\RV{E})\iff \RV{A}^*\CI_{\prob{P}}(\RV{B}^*,\RV{D}^*)|(\RV{C}^*,\RV{E}^*)
% \end{align}
% \end{theorem}

% \begin{proof}
% If:

% By assumption, $\RV{A}^*\CI_{\prob{P}_\alpha}(\RV{B}^*,\RV{D}^*)|(\RV{C}^*,\RV{E}^*)$ for all $\prob{P}_\alpha^{\RV{D^*E^*}}$. In particular, this holds for some $\prob{P}_\alpha^{\RV{D^*E^*}}$ such that $(d,e)\in (\RV{D}^*,\RV{E}^*)(\Omega)\implies \prob{P}_\alpha^{\RV{D^*E^*}}(d,e) >0$. Then by Theorem \ref{th:dawid_constantionou}, $\RV{A}\CI_{\prob{P},\text{ext}}(\RV{B},\RV{D})|(\RV{C},\RV{E})$.

% Only if:

% For any $\beta$, $\prob{P}_\beta^{\RV{ABC}|\RV{DE}}= \prob{P}_\beta^{\RV{DE}}\odot \prob{P}^{\RV{ABC}|\RV{DE}}$. By Lemma \ref{th:higher_order_conditionals}, we have $\prob{P}^{\RV{A}|\RV{BCDE}}$ such that

% \begin{align}
%     \prob{P}_\beta^{\RV{A^*B^*C^*}\RV{D^*E^*}} &= \prob{P}_\beta^{\RV{D^*E^*}}\odot \prob{P}^{\RV{B^*C^*}|\RV{D^*E^*}}\odot \prob{P}^{\RV{A^*}|\RV{B^*C^*D^*E^*}}\\
%                                       &= \prob{P}_\beta^{\RV{B^*C^*D^*E^*}}\odot \prob{P}^{\RV{A^*}|\RV{B^*C^*D^*E^*}}\\
%                                       &= \prob{P}_\beta^{\RV{C^*E^*}}\odot \prob{P}_\beta^{\RV{B^*D^*}|\RV{C^*E^*}}\odot \prob{P}^{\RV{A}^*|\RV{B^*C^*D^*E^*}}
% \end{align}

% By Theorem \ref{th:dawid_constantionou}, we have some $\alpha$ such that $\prob{P}_\alpha^{\RV{D^*E^*}}$ is strictly positive on the range of $(\RV{D}^*,\RV{E}^*)$ and $\RV{A}^*\CI_{\prob{P}_\alpha}(\RV{B}^*,\RV{D}^*)|(\RV{C}^*,\RV{E}^*)$.

% By independence, for some version of $\prob{P}^{\RV{A}|\RV{BCDE}}$:

% \begin{align}
%     \prob{P}_\alpha^{\RV{C^*E^*}}\odot \prob{P}_\alpha^{\RV{B^*D^*}|\RV{C^*E^*}}\odot \prob{P}^{\RV{A}^*|\RV{B^*C^*D^*E^*}} &= \tikzfig{indep_strengthen_1}\\
%     &= \tikzfig{indep_strengthen_2}\\
%     &= \prob{P}_\alpha^{\RV{C^*E^*}}\odot \prob{P}_\alpha^{\RV{B^*D^*}|\RV{C^*E^*}}\odot (\prob{P}_\alpha^{\RV{A}^*|\RV{C^*E^*}}\otimes\text{erase}_{BD})
% \end{align}

% Thus for any $(a,b,c,d,e)\in A\times B\times C\times D\times E$ such that $\prob{P}_\alpha^{\RV{B^*C^*D^*E^*}}(b,c,d,e)>0$, $\prob{P}^{\RV{A}^*|\RV{B^*C^*D^*E^*}}(a|b,c,d,e) = \prob{P}_\alpha^{\RV{A}^*|\RV{C^*E^*}}(a|c,e)$. However, by assumption, $\prob{P}_\alpha^{\RV{B^*C^*D^*E^*}}(b,c,d,e)=0 \implies \prob{P}_\beta^{\RV{B^*C^*D^*E^*}}(b,c,d,e)=0$, and so $\prob{P}_\beta^{\RV{A}^*|\RV{B^*C^*D^*E^*}}= \prob{P}_\alpha^{\RV{A}^*|\RV{C^*E^*}}(a|c,e)$ everywhere except a set of $\prob{P}_\beta$-measure 0. Thus
    
% \begin{align}
%     \prob{P}_\beta^{\RV{A^*B^*C^*}\RV{D^*E^*}} &= \tikzfig{indep_strengthen_3}\\
%     &= \tikzfig{indep_strengthen_4}
% \end{align}
% \end{proof}

% Conditional independence is a property of variables, we define ``unresponsiveness'' as a property of Markov kernels.



This result can fail to hold in the absence of the domination condition. Consider $A$ a collection of inserts that all deterministically set a variable $\RV{X}$; then for any variable $\RV{Y}$ $\RV{Y}\CI_{\prob{P}_\square} \RV{X}$ because $\RV{X}$ is deterministic for any $\alpha\in A$. But $\prob{P}_\square^{\RV{Y}|\RV{X}}$ is not necessarily unresponsive to $\RV{X}$.

Note that in the absence of the assumption of the existence of $\prob{P}_\square^{\RV{Y}|\RV{WX}}$, $\RV{Y}\CI_{\prob{P}_\square}\RV{X}|\RV{W}$ does \emph{not} imply the existence of $\prob{P}_\square^{\RV{Y}|\RV{W}}$. If we have, for example, $A=\{\alpha,\beta\}$ and $\prob{P}_\alpha^{\RV{XY}}$ is two flips of a fair coin while $\prob{P}_\beta^{\RV{XY}}$ is two flips of a biased coin, then $\RV{Y}\CI_{\prob{P}}\RV{X}$ but $\prob{P}^{\RV{Y}}$ does not exist.

\subsection{Maximal probability sets and valid conditionals}

We have defined probability sets and uniform conditional probabilities. Thus, if we start with a probability set, we know how to check if certain uniform conditional probabilities exist or not. However, there is a particular line of reasoning that comes up most often in the graphical models tradition of causal inference where we start with collections of conditional probabilities and assemble them into probability models as needed. A simple example of this is the causal Bayesian network given by the graph $\RV{X}\longrightarrowRHD \RV{Y}$ and some observational probability distribution $\prob{P}^{\RV{XY}}\in\Delta(X\times Y)$. Using the standard notion of ``hard interventions on $\RV{X}$'', this model induces a probability set which we could informally describe as the set $\prob{P}_\square:=\{\prob{P}_a^{\RV{XY}}|a\in X\cup\{*\}\}$ where $*$ is a special element corresponding to the observational setting. The graph $\RV{X}\longrightarrowRHD \RV{Y}$ implies the existence of the uniform conditional probability $\prob{P}_\square^{\RV{Y}|\RV{X}}$ under the nominated set of interventions, while the usual rules of hard interventions imply that $\prob{P}_a^{\RV{X}} = \delta_a$ for $a\in X$.

Reasoning ``backwards'' like this -- from uniform conditionals and marginals back to probability sets -- must be done with care. The probability set associated with a collection of conditionals and marginals may be empty or nonunique. Uniqueness may not always be required, but an empty probability set is clearly not a useful model.

Consider, for example, $\Omega=\{0,1\}$ with $\RV{X}=(\RV{Z},\RV{Z})$ for $\RV{Z}:=\text{id}_{\Omega}$ and any measure $\kappa\in \Delta(\{0,1\}^2)$ such that $\kappa(\{1\}\times \{0\})>0$. Note that $\RV{X}^{-1}(\{1\}\times \{0\})=\RV{Z}^{-1}(\{1\})\cap \RV{Z}^{-1}(\{0\})=\emptyset$. Thus for any probability measure $\mu\in \Delta(\{0,1\})$, $\mu^{\RV{X}}(\{1\}\times \{0\}) = \mu(\emptyset)=0 $ and so $\kappa$ cannot be the marginal distribution of $\RV{X}$ for any base measure at all.

We introduce the notion of \emph{valid distributions} and \emph{valid conditionals}. The key result here is: probability sets defined by collections of recursive valid conditionals and distributions are nonempty. While we suspect this condition is often satisfied by causal models in practice, we offer one example in the literature where it apparently is not. The problem of whether a probability set is valid is analogous to the problem of whether a probability distribution satisfying a collection of constraints exists discussed in \citet{vorobev_consistent_1962}. As that work shows, there are many questions of this nature that can be asked and that are not addressed by the criterion of validity.

There is also a connection between the notion of validity and the notion of \emph{unique solvability} in \citet{bongers_theoretical_2016}. We ask ``when can a set of conditional probabilities together with equations be jointly satisfied by a probability model?'' while Bongers et. al. ask when a set of equations can be jointly satisfied by a probability model.

\begin{definition}[Valid distribution]\label{def:valid_dist}
Given $(\Omega,\sigalg{F})$ and a variable $\RV{X}:\Omega\to X$, an $\RV{X}$-valid probability distribution is any probability measure $\prob{K}\in \Delta(X)$ such that $\RV{X}^{-1}(A)=\emptyset\implies \prob{K}(A) = 0$ for all $A\in\sigalg{X}$.
\end{definition}

\begin{definition}[Valid conditional]\label{def:valid_conditional_prob}
Given $(\Omega,\sigalg{F})$, $\RV{X}:\Omega\to X$, $\RV{Y}:\Omega\to Y$ a \emph{$\RV{Y}|\RV{X}$-valid conditional probability} is a Markov kernel $\prob{L}:X\kto Y$ that assigns probability 0 to impossible events, unless the argument itself corresponds to an impossible event:
\begin{align}
    \forall B\in \sigalg{Y}, x\in X: (\RV{X},\RV{Y})\yields \{x\}\times B = \emptyset \implies \left(\prob{L}(B|x) = 0\right) \lor \left(\RV{X}\yields \{x\} = \emptyset\right)
\end{align}
\end{definition}

\begin{definition}[Maximal probability set]
Given $(\Omega,\sigalg{F})$, $\RV{X}:\Omega\to X$, $\RV{Y}:\Omega\to Y$ and a $\RV{Y}|\RV{X}$-valid conditional probability $\prob{L}:X\kto Y$ the maximal probability set $\prob{P}_{\{\}}^{\RV{Y}|\RV{X}[M]}$ associated with $\prob{L}$ is the probability set such that for all $\prob{P}_\alpha\in \prob{P}_{\{\}}$, $\prob{L}$ is a version of $\prob{P}_\alpha^{\RV{Y}|\RV{X}}$.
\end{definition}

We use the notation $\prob{P}_{\{\}}^{\RV{Y}|\RV{X}[M]}$ as shorthand to refer to the probability set $\prob{P}_{\{\}}$ maximal with respect to $\prob{P}_{\{\}}^{\RV{Y}|\RV{X}}$.

Lemma \ref{lem:valid_extendability} shows that the semidirect product of any pair of valid conditoinal probabilities is itself a valid conditional. Suppose we have some collection of $\RV{X}_i|\RV{X}_{[i-1]}$-valid conditionals $\{\prob{P}_i^{\RV{X}_i|\RV{X}_{[i-1]}}|i\in [n]\}$; then recursively taking the semidirect product $\kernel{M}:=\prob{P}_1^{\RV{X}_1}\odot (\prob{P}_2^{\RV{X}_2|\RV{X}_{1}}\odot ...)$ yields a $\RV{X}_{[n]}$ valid distribution. Furthermore, the maximal probability set associated with $\kernel{M}$ is nonempty.

Collections of recursive conditional probabilities often arise in causal modelling -- in particular, they are the foundation of the structural equation modelling approach \citet{richardson2013single,pearl_causality:_2009}.

Note that validity is not a necessary condition for a conditional to define a non-empty probability set. The intuition for this is: if we have some $\kernel{K}:X\kto Y$, $\kernel{K}$ might be an invalid $\RV{Y}|\RV{X}$ conditional on all of $X$, but might be valid on some subset of $X$, and so we might have some probability model $\prob{P}$ that assigns measure 0 to the bad parts of $X$ such that $\kernel{K}$ is a version of $\prob{P}^{\RV{Y}|\RV{X}}$. On the other hand, if we want to take the product of $\kernel{K}$ with arbitrary valid $\RV{X}$ probabilities, then the validity of $\kernel{K}$ is necessary (Theorem \ref{th:valid_conditional_probability}).

\begin{example}
Body mass index is defined as a person's weight divided by the square of their height. Suppose we have a measurement process $\proc{S}=(\proc{W},\proc{H})$ and $\proc{B}=\frac{\proc{W}}{\proc{H}^2}$ - i.e. we figure out someone's body mass index first by measuring both their height and weight, and then passing the result through a function that divides the second by the square of the first. Thus, given the random variables $\RV{W},\RV{H}$ modelling $\proc{W},\proc{H}$, $\proc{B}$ is the function given by $\RV{B}=\frac{\RV{W}}{\RV{H}^2}$.

With this background, suppose we postulate a decision model in which body mass index can be directly controlled by a variable $\RV{C}$, while height and weight are not. Specifically, we have a probability set $\prob{P}_\square$ with
\begin{align}
    \prob{P}_\square^{\RV{B}|\RV{WHC}} &= \tikzfig{invalid_BMI_model} \label{eq:bmi_example}
\end{align}
Then pick some $w,h,x\in\mathbb{R}$ such that $\frac{w}{h^2}\neq x$ and $(\RV{W},\RV{H})\yields (w,h)\neq \emptyset$ (which is to say, our measurement procedure could potentially yield $(w,h)$ for a person's height and weight). We have $\prob{P}_\square^{\RV{B}|\RV{WHC}}(\{x\}|w,h,x)=1$, but 
\begin{align}
    (\RV{B},\RV{W},\RV{H})\yields \{(x,w,h)\} &= \{\omega|(\RV{W},\RV{H})(\omega) = (w,h),\RV{B}(\omega) = \frac{w}{h^2}\}\\
    &=\emptyset
\end{align}
so $\prob{P}_\square^{\RV{B}|\RV{WHC}}$ is invalid. Thus there is some valid $\mu^{\RV{WHC}}$ such that the probability set $\prob{P}_{\square}^{\RV{BWHC}} = \mu^{\RV{WHC}}\odot \prob{P}_\square^{\RV{Y}|\RV{X}}$ is empty.

Validity rules out conditional probabilities like \ref{eq:bmi_example}. We conjecture that in many cases this condition is implicitly taken into account -- it is obviously silly to posit a model in which body mass index can be controlled independently of height and weight. We note, however, that presuming the authors intended their model to be interpreted according to the usual semantics of causal Bayesian networks, the invalid conditional probability \ref{eq:bmi_example} would be used to evaluate the causal effect of body mass index in the causal diagram found in \citet{shahar_association_2009}.
\end{example}

\subsection{Causal contractibility}\label{sec:ccontract_appendix}

\begin{reptheorem}{th:no_implication}
Exchange commutativity does not imply locality of consequences or vise versa.
\end{reptheorem}

\begin{proof}
A conditional probability model that exhibits exchange commutativity but some choices have non-local consequences:

Suppose $D=Y=\{0,1\}$ and we have a probability set $\prob{P}_C$ with uniform conditional $\prob{P}_C^{\RV{Y}|\RV{D}}$, where $\RV{D}=(\RV{D}_1,\RV{D}_2)$, $\RV{Y}=(\RV{Y}_1,\RV{Y}_2)$.

Suppose the unique version of $\prob{P}_C^{\RV{Y}|\RV{D}}$ is
\begin{align}
    \prob{P}_C^{\RV{Y}|\RV{D}}(y_1,y_2|d_1,d_2) &= \llbracket (y_1,y_2)= (d_1+d_2,d_1+d_2) \rrbracket
\end{align}
then 
\begin{align}
    \prob{P}_C^{\RV{Y}_1|\RV{D}}(y_1|d_1,d_2) &= \llbracket y_1 = d_1+d_2 \rrbracket
\end{align}
and there is no function depending on $y_1$ and $d_1$ only that is equal to this. Thus $\prob{P}_C$ exhibits non-local consequences. 

However, taking $\rho$ to be the unique nontrivial swap $\{0,1\}\to \{0,1\}$
\begin{align}
    \text{swap}_{\rho(\RV{D})}\prob{P}_{C}^{\RV{Y}|\RV{D}}(y_1,y_2|d_1,d_2) &= \prob{P}_{C}^{\RV{Y}|\RV{D}}(y_1,y_2|d_2,d_1)\\
    &= \llbracket (y_1,y_2)= (d_2+d_1,d_2+d_1) \rrbracket\\
    &= \llbracket (y_1,y_2)= (d_1+d_2,d_1+d_2) \rrbracket\\
    &= \llbracket (y_2,y_1)= (d_1+d_2,d_1+d_2) \rrbracket\\
    &= \prob{P}_{C}^{\RV{Y}|\RV{D}}\text{swap}_{\rho(\RV{Y})}(y_1,y_2|d_1,d_2)
\end{align}
so $\prob{P}_\square$ commutes with exchange.

A conditional probability model that exhibits locality of consequences but does not commute with exchange follows. Suppose again $D=Y=\{0,1\}$ and we have a probability set $\prob{P}_C$ with uniform conditional $\prob{P}_C^{\RV{Y}|\RV{D}}$, where $\RV{D}=(\RV{D}_1,\RV{D}_2)$, $\RV{Y}=(\RV{Y}_1,\RV{Y}_2)$. This time, suppose the unique version of $\prob{P}_C^{\RV{Y}|\RV{D}}$ is
\begin{align}
    \prob{P}_C{\RV{Y}|\RV{D}}(y_1,y_2|d_1,d_2) &= \llbracket (y_1,y_2)= (0,1) \rrbracket
\end{align}

Then  If $\prob{P}_\alpha^{\RV{D}_S}=\prob{P}_\beta^{\RV{D}_S}$ for $S\subset\{0,1\}$ then:
\begin{align}
    \prob{P}_C^{\RV{Y}_1|\RV{D}}(y_1|d_1,d_2) &= \llbracket y_1= 0 \rrbracket\\
    &= \prob{P}_C^{\RV{Y}_1|\RV{D}_1}(y_1|d_1)\\
    \prob{P}_C^{\RV{Y}_2|\RV{D}}(y_2|d_1,d_2)&= \llbracket y_2= 1 \rrbracket\\
    &= \prob{P}_C^{\RV{Y}_2|\RV{D}_2}(y_2|d_2)
\end{align}
so $\prob{P}_C^{\RV{Y}|\RV{D}}$ exhibits consequence locality.

However, $\prob{P}_C$ does not commute with exchange.
\begin{align}
    \text{swap}_{\rho(\RV{D})} \prob{P}_C^{\RV{Y}|\RV{D}}(y_1,y_2|d_1,d_2) &= \prob{P}_C^{\RV{Y}|\RV{D}}(y_1,y_2|d_2,d_1)\\
    &=\llbracket (y_1,y_2)= (0,1) \rrbracket\\
    &\neq \llbracket (y_2,y_1)= (0,1) \rrbracket\\
    &= \prob{P}_C^{\RV{Y}|\RV{D}}\text{swap}_{\rho(\RV{D})}(y_1,y_2|d_1,d_2)
\end{align}
\end{proof}

\begin{reptheorem}{th:table_rep}
Given a probability set $\prob{P}_C$ such that $\RV{D}:=(\RV{D}_i)_{i\in \mathbb{N}}$ and $\RV{Y}:=(\RV{Y}_i)_{i\in \mathbb{N}}$, $\prob{P}_C$ is  $(\RV{D};\RV{Y})$-causally contractible if and only if there exists a column exchangeable probability distribution $\mu^{\RV{Y}^D}\in \Delta(Y^{|D|\times \mathbb{N}})$ such that
\begin{align}
    \prob{P}_C^{\RV{Y}|\RV{D}} &= \tikzfig{lookup_representation}\\
    &\iff\\
    \prob{P}_C^{\RV{Y}|\RV{D}}(y|(d_i)_{i\in \mathbb{N}}) &= \mu^{\RV{Y}^D} \Pi_{(d_i i)_{i\in\mathbb{N}}}(y)
\end{align}
Where $\Pi_{(d_i i)_{i\in\mathbb{N}}}:Y^{|D|\times \mathbb{N}}\to Y^{\mathbb{N}}$ is the function that projects the $(d_i,i)$ indices for all $i\in \mathbb{N}$ and $\prob{F}_{\text{ev}}$ is the Markov kernel associated with the evaluation map
\begin{align}
    \text{ev}:D^\mathbb{N}\times Y^{D\times \mathbb{N}}&\to Y\\
    ((d_i)_\mathbb{N},(y_{ij})_{i\in D,j\in \mathbb{N}})&\mapsto (y_{d_i i})_{i\in \mathbb{N}}
\end{align}
\end{reptheorem}

\begin{proof}
Only if:
Consider a probability set $\prob{P}_{C'}$ where $C'\supset C$ contains all $\alpha$ such that $\prob{P}_\alpha^{\RV{D}}$ is deterministic and $\prob{P}_{C'}^{\RV{Y}|\RV{D}}\overset{P_C}{\cong}\prob{P}_C^{\RV{Y}|\RV{D}}$. It can be constructed by adding to $\prob{P}_C$ probability sets with marginals $\delta_d\odot \prob{P}_{C'}^{\RV{Y}|\RV{D}}$ for all $d\in D$.

We will prove the result holds for $\prob{P}_{C'}$, and it will therefore also hold for $\prob{P}_C$.

For all $d\in D$, abuse notation to say that $\prob{P}_d$ is a probability set in $C'$ such that $\prob{P}_d^{\RV{D}}=\delta_d$. For any $\alpha\in C'$, we have

\begin{align}
    \prob{P}_\alpha^{\RV{DY}}(B\times C) &= \int_B \prob{P}_C^{\RV{Y}|\RV{D}}(C|d)\prob{P}_\alpha^{\RV{D}}(\mathrm{d}d)\\
                                &= \int_B \int_D \prob{P}_C^{\RV{Y}|\RV{D}}(C|d')\prob{P}_d^{\RV{D}}(\mathrm{d}d')\prob{P}_\alpha^{\RV{D}}(\mathrm{d}d)\\
                                &= \int_B \prob{P}_d^{\RV{Y}}(C)\prob{P}_\alpha^{\RV{D}}(\mathrm{d}d)
\end{align}

Thus $d\mapsto \prob{P}_d^{\RV{Y}}$ is a version of $\prob{P}_C^{\RV{Y}|\RV{C}}$.

Choose $e:=(e_i)_{i\in\mathbb{N}}$ such that $e_{|D|i+j}$ is the $i$th element of $D$ for all $i,j\in \mathbb{N}$.

Define
\begin{align}
    \mu^{\RV{Y}^D}((y_{ij})_{D\times \mathbb{N}}):=\prob{P}_e^{\RV{Y}}((y_{|D|i+j})_{i\in D, j\in \mathbb{N}})
\end{align}

Now consider any $d:=(d_i)_{i\in \mathbb{N}}\in D^{\mathbb{N}}$. By definition of $e$, $e_{|D|d_i + i}=d_i$ for any $i,j\in \mathbb{N}$.

Define
\begin{align}
    \prob{Q}:D\kto Y\\
    \prob{Q}:= \tikzfig{lookup_representation}
\end{align}

and consider some ordered sequence $A\subset \mathbb{N}$ and $B:= ((|D|d_i+i))_{i\in A}$. Note that $e_B:=(e_{|D|d_i +i})_{i\in B}=d_A=(d_i)_{i\in A}$. Then 

\begin{align}
    \sum_{y\in \RV{Y}^{-1}(y_A)} \prob{Q}(y|d) &= \sum_{y\in \RV{Y}^{-1}(y_A)} \mu^{(\RV{Y}^{D}_{d_ii})_{A}}(y) \\
    &= \sum_{y\in \RV{Y}^{-1}(y_A)} \prob{P}_e^{(\RV{Y}_{|D|d_i+i})_{A}}(y)\\
    &= \prob{P}_e^{\RV{Y}_{B}}(y_A)\\
    &= \prob{P}_{d}^{\RV{Y}_A}(y_A)&\text{by causal contractibility}
\end{align}

Because this holds for all $A\subset\mathbb{N}$, by the Kolmogorov extension theorem

\begin{align}
    \prob{Q}(y|d) &= \prob{P}_d^{\RV{Y}}(y)
\end{align}

And so $\prob{Q}$ is also a version of $\prob{P}_\square^{\RV{Y}|\RV{C}}$.

Next we will show $\mu^{\RV{Y}^D}$ is exchangeable. Consider any subsequences $\RV{Y}^D_S$ and $\RV{Y}^D_T$ of $\RV{Y}^D$ with $|S|=|T|$. Let $\rho(S)$ be the ``expansion'' of the indices $S$, i.e. $\rho(S)=(|D|i+j)_{i\in S,j\in D}$. Then by construction of $e$, $e_{\rho(S)}=e_{\rho(T)}$ and therefore

\begin{align}
    \mu^{\RV{Y}^D}\Pi_S&= \prob{P}_e^{\RV{Y}_{\rho(S)}})\\
    &= \prob{P}_e^{\RV{Y}_{\rho(T)}})&\text{by contractibility of }\prob{P}_C\text{ and the equality } e_{\rho(S)}=e_{\rho(T)}\\
    &= \mu^{\RV{Y}^D}\Pi_T
\end{align}


If:
Suppose 
\begin{align}
    \prob{P}_C^{\RV{Y}|\RV{D}} &= \tikzfig{lookup_representation}
\end{align}

and consider any two deterministic decision functions $d,d'\in D^{\mathbb{N}}$ such that some subsequences are equal $d_S=d'_T$.

Let $\RV{Y}^{d_S}=(\RV{Y}_{d_i i})_{i\in S}$.

By definition,

\begin{align}
    \prob{P}_C^{\RV{Y}_S|\RV{D}}(y_S|d) &= \sum_{y^D_S\in Y^{|D|\times |S|}}\mu^{\RV{Y}^D} \Pi_S(y^D_S)\prob{F}_{\text{ev}}(y_S|d,y^D_S)\\
    &= \sum_{y^D_S\in Y^{|D|\times |T|}}\prob{P}_C^{\RV{Y}^D_T}(y^D_S)\prob{F}_{\text{ev}}(y_S|d,y^D_S)&\text{ by contractibility of }\mu^{\RV{Y}^D}\Pi_T \\
    &= \prob{P}_C^{\RV{Y}_T|\RV{D}}(y_S|d)
\end{align}
\end{proof}

\begin{replemma}{lem:aug_cc}
Given a $(\RV{D};\RV{Y})$-causally contractible model $\prob{P}_C'$ on $(\Omega',\sigalg{F}')$, there exists an \emph{augmented} model $\prob{P}_C$ on $(\Omega,\sigalg{F}):=((\Omega'\times Y^D,\sigalg{F}'\otimes\sigalg{Y}^D)$ such that $\prob{P}_C\Pi_{\Omega'}=\prob{P}_C'$ and, defining $\RV{Y}^D:\Omega\times Y^D\to Y^D$ as the projection onto $Y^D$
\begin{align}
    \prob{P}_C^{\RV{Y}|\RV{D}} &= \tikzfig{lookup_representation_variablised}
\end{align}
\end{replemma}

\begin{proof}
Set $\prob{P}_C^{\RV{Y}|\RV{Y}^D\RV{D}}=\kernel{F}_{\mathrm{ev}}$ and $\prob{P}_C^{\RV{Y}^D|\RV{D}}=\mu^{\RV{Y}^D}\mathrm{erase}_{D^{\mathbb{N}}}\otimes \mathrm{erase}_D$ as in Theorem \ref{th:table_rep}. Then Equation \ref{eq:lup_rep_varb} follows Theorem \ref{th:table_rep}.

Let $\RV{W}=\Pi_{\Omega'}$ and for each $\alpha\in C$, set 
\begin{align}
    \prob{P}_\alpha^{\RV{W}|\RV{Y}^D\RV{D}} &= \tikzfig{augmented_ccontracible}
\end{align}
Then
\begin{align}
    \prob{P}_\alpha^{\RV{W}} &= \tikzfig{augmented_ccon2}\\
    &= \tikzfig{augmented_ccon3}\\
    &= \prob{P}_\alpha'
\end{align}
\end{proof}

\begin{reptheorem}{th:iid_rep}
Suppose we have a sample space $(\Omega,\sigalg{F})$ and a probability set $\prob{P}_C$ with uniform conditional $\prob{P}_C^{\RV{Y}|\RV{D}}$ where $\RV{D}:=(\RV{D}_i)_{i\in \mathbb{N}}$ and $\RV{Y}:=(\RV{Y}_i)_{i\in \mathbb{N}}$. $\prob{P}_C$ is $(\RV{D};\RV{Y})$-causally contractible if and only if there exists some $\RV{H}:\Omega\to H$ such that $\prob{P}^{\RV{H}}_C$ and $\prob{P}_C^{\RV{Y}_i|\RV{H}\RV{D}_i}$ exist for all $i\in \mathbb{N}$ and
\begin{align}
    \prob{P}_C^{\RV{Y}|\RV{D}} &= \tikzfig{do_model_representation}\\
    &\iff\\
    \RV{Y}_i&\CI^e_{\prob{P}_C} \RV{Y}_{\mathbb{N}\setminus i},\RV{D}_{\mathbb{N}\setminus i}C|\RV{H}\RV{D}_i&\forall i\in \mathbb{N}\\
    \land \RV{H} \CI^e_{\prob{P}_C} \RV{D} C\\
    \land \prob{P}_C^{\RV{Y}_i|\RV{H}\RV{D}_i} &= \prob{P}^{\RV{Y}_0|\RV{H}\RV{D}_0} & \forall i\in \mathbb{N}
\end{align}
Where $\Pi_{D,i}:D^\mathbb{N}\kto D$ is the $i$th projection map.
\end{reptheorem}

\begin{proof}
We make use of Lemma \ref{th:table_rep} to show that we can represent the conditional probability $\prob{P}_C^{\RV{Y}|\RV{D}}$ as
\begin{align}
        \prob{P}_C^{\RV{Y}|\RV{D}} &= \tikzfig{lookup_representation}
\end{align}

As a preliminary, we will show
\begin{align}
    \kernel{F}_{\mathrm{ev}} &= \tikzfig{lookup_rep_intermediate}\label{eq:ev_alternate_rep}
\end{align}

Where $\Pi_{Y^D,i}:Y^{D\times\mathbb{N}}\kto Y^D$ is the $i$th column projection map on $\RV{Y}^{D\times \mathbb{N}}$ and $\mathrm{ev}_{Y^D\times D}:Y^D\times D\to Y$ is the evaluation function
\begin{align}
    ((y_i)_{i\in D},d)\mapsto y_d
\end{align}

Recall that $\mathrm{ev}$ is the function

\begin{align}
    ((d_i)_\mathbb{N},(y_{ij})_{i\in D,j\in \mathbb{N}})&\mapsto (y_{d_i i})_{i\in \mathbb{N}}
\end{align}

By definition, for any $\{A_i\in\sigalg{Y}|i\in \mathbb{N}\}$

\begin{align}
    \kernel{F}_{\mathrm{ev}}(\prod_{i\in \mathbb{N}}A_i|(d_i)_\mathbb{N},(y_{ij})_{i\in D,j\in \mathbb{N}}) &= \delta_{(y_{d_i i})_{i\in \mathbb{N}}}(\prod_{i\in \mathbb{N}}A_i)\\
        &= \prod_{i\in \mathbb{N}} \delta_{y_{d_i i}} (A_i)\\
        &= \mathrm{copy}^{\mathbb{N}}\prod_{i\in \mathbb{N}}( \Pi_{D,i}\otimes \Pi_{Y,i})\kernel{F}_{\mathrm{ev}_{Y^D\times D}}
\end{align}

Which is what we wanted to show.

Only if:
We will assume without loss of generality that we are dealing with an augmented causally contractible model (Lemma \ref{lem:aug_cc}). Because we have an augmented causally contractible model, we have a variable $\RV{Y}^D=(\RV{Y}^D_i)_{i\in \mathbb{N}}$ exchangeable with respect to $\prob{P}^{\RV{Y}^D}_C$ (Lemma \ref{th:table_rep}). From \citet{kallenberg_basic_2005} we have a directing random measure $\RV{H}$ such that
\begin{align}
    \prob{P}_C^{\RV{Y}^D|\RV{H}} &= \tikzfig{de_finetti_representation}\\
    &\iff\\
    \prob{P}_C^{\RV{Y}^D|\RV{H}}(\prod_{i\in \mathbb{N}} A_i|h) &= \prod_{i\in \mathbb{N}} \prob{P}_C^{\RV{Y}_0^D|\RV{H}}(A_i|h)
\end{align}

Furthermore, because $\RV{Y}$ is a deterministic function of $\RV{D}$ and $\RV{Y}^D$, $\RV{Y}\CI_{\prob{P}_C} \RV{H}|(\RV{D},\RV{Y}^D)$ and by definition of $\RV{Y}^D$, $\RV{Y}^D\CI_{\prob{P}_C}\RV{D}$ and so

\begin{align}
    \prob{P}_C^{\RV{Y}|\RV{HD}} &= \prob{P}_C^{\RV{Y}^D|\RV{HD}}\odot \prob{P}_C^{\RV{Y}|\RV{Y}^D\RV{HD}}\\
                                      &= \tikzfig{Y_pushback_factorisation}
                                      &= \tikzfig{do_model_representation}
\end{align}

If:
By assumption
\begin{align}
    \prob{P}_C^{\RV{Y}|\RV{D}}(\prod_{i\in \mathbb{N}} A_i|h,(d_i)_{i\in \mathbb{N}}) &= \int_H \prod_{i\in \mathbb{N}}\prob{P}_C^{\RV{Y}_1|\RV{HD}_1}(A_i|h,d_i)\prob{P}_C^{\RV{H}}(\mathrm{d}h)
\end{align}

Consider $\alpha,\alpha'$ such that $\prob{P}^{\RV{D}_M}_\alpha = \prob{P}^{\RV{D}_L}_{\alpha'}$ for $L,M\subset \mathbb{N}$ with $|M|=|L|$, both finite. Then

\begin{align}
    \prob{P}_\alpha^{\RV{Y}_M}(A) &= \int_{D^{\mathbb{N}}} \prob{P}_\alpha^{\RV{Y}_M|\RV{D}}(A|d)\prob{P}_\alpha^{\RV{D}}(\mathrm{d}d)\\
                                  &= \int_H\int_{D^{\mathbb{N}}} \prod_{i\in M}\prob{P}_C^{\RV{Y}_1|\RV{HD}_1}(A_i|h,d_i)\prob{P}_\alpha^{\RV{D}}(\mathrm{d}d)\prob{P}_C^{\RV{H}}(\mathrm{d}h)\\
                                  &= \int_H\int_{D^{|M|}} \prod_{i\in M}\prob{P}_C^{\RV{Y}_1|\RV{HD}_1}(A_i|h,d_i)\prob{P}_\alpha^{\RV{D}_M}(\mathrm{d}d_M)\prob{P}_C^{\RV{H}}(\mathrm{d}h)\\
                                  &= \int_H\int_{D^{|M|}} \prod_{i\in M}\prob{P}_C^{\RV{Y}_1|\RV{HD}_1}(A_i|h,d_i)\prob{P}_{\alpha'}^{\RV{D}_N}(\mathrm{d}d_N)\prob{P}_C^{\RV{H}}(\mathrm{d}h)\\
                                  &= \int_H\int_{D^{\mathbb{N}}} \prod_{i\in M}\prob{P}_C^{\RV{Y}_1|\RV{HD}_1}(A_i|h,d_i)\prob{P}_{\alpha'}^{\RV{D}}(\mathrm{d}d)\prob{P}_C^{\RV{H}}(\mathrm{d}h)\\
                                  &= \prob{P}_{\alpha'}^{\RV{Y}_M}(A)
\end{align}

\end{proof}


\subsection{Body mass index revisited}\label{sec:bmi_revis}

\begin{lemma}\label{lem:eci_cc_extend}
Suppose we have a probability set $\prob{P}_C$ that is $(\RV{D},\RV{X};\RV{Y})$-causally contractible where $\RV{D}:=(\RV{D}_i)_{i\in M}$ and similarly for $\RV{X}$ and $\RV{Y}$. If $\RV{Y}_i\CI^e_{\prob{P}_C} \RV{D}_i C|\RV{X}_i\RV{H}$, then $\prob{P}_C$ is also $(\RV{X};\RV{Y})$-causally contractible.
\end{lemma}

\begin{proof}
From causal contractibility we have
\begin{align}
\RV{Y}_i\CI_{\prob{P}_C}^e (\RV{Y}_{\{i\}^C},\RV{X}_{\{i\}^C},\RV{D}_{\{i\}^C}) C |\RV{H}\RV{D}_i\RV{X}_i
\end{align}
Combining this with the assumption $\RV{Y}_i\CI^e_{\prob{P}_C} \RV{D}_i C|\RV{X}_i\RV{H}$ we have, by contraction,
\begin{align}
    \RV{Y}_i\CI_{\prob{P}_C}^e (\RV{Y}_{\{i\}^C},\RV{X}_{\{i\}^C}) C |\RV{H}\RV{X}_i
\end{align}

Furthermore, also from causal contractibility, for all $i,j\in M$
\begin{align}
    \prob{P}_C^{\RV{Y}_i|\RV{X}_i\RV{D}_iH} &\overset{\prob{P}_C}{\cong} \prob{P}_C^{\RV{Y}_j|\RV{X}_j\RV{D}_jH}\\
    \implies \prob{P}_C^{\RV{Y}_i|\RV{X}_iH} &\overset{\prob{P}_C}{\cong} \prob{P}_C^{\RV{Y}_j|\RV{X}_jH}
\end{align}
\end{proof}

\begin{reptheorem}{lem:proxy_control}
Suppose we have a probability set $\prob{P}_C$ that is $(\RV{D};\RV{X},\RV{Y})$-causally contractible, where $\RV{D}:=(\RV{D}_i)_{i\in M}$ and similarly for $\RV{X}$ and $\RV{Y}$. If there exists $\alpha\in C$ such that $\prob{P}_\alpha^{\RV{D}}\gg \{\prob{P}_\beta^{\RV{D}}|\beta\in C\}$ and $\RV{Y}_i\CI_{\prob{P}_\alpha} \RV{D}_i|\RV{HX}_i$ for all $i\in M$, then $\prob{P}_C$ is also $(\RV{Y};\RV{X})$-causally contractible.
\end{reptheorem}

\begin{proof}
By Corollary \ref{cor:ci_cp_exist} and the existence of $\prob{P}_C^{\RV{Y}_i\RV{X}_i|\RV{H}\RV{D}_i}$ for all $i\in M$, $\prob{P}_C^{\RV{Y}_i|\RV{H}\RV{X}_i\RV{D}_i}$ also exists for all $i$. Furthermore, because $\prob{P}_C^{\RV{Y}_i\RV{X}_i|\RV{H}\RV{D}_i}=\prob{P}_C^{\RV{Y}_j\RV{X}_j|\RV{H}\RV{D}_j}$ for all $i,j\in M$, $\prob{P}_C^{\RV{Y}_i|\RV{H}\RV{X}_i\RV{D}_i}=\prob{P}_C^{\RV{Y}_j|\RV{H}\RV{X}_j\RV{D}_j}$ for all $i,j\in M$.

From causal contractibility we have
\begin{align}
(\RV{X}_i,\RV{Y}_i)\CI_{\prob{P}_\alpha} (\RV{X}_{\{i\}^C},\RV{Y}_{\{i\}^C},\RV{D}_{\{i\}^C}) |\RV{H}\RV{D}_i\label{eq:cc}\\
\RV{Y}_i\CI_{\prob{P}_\alpha} (\RV{Y}_{\{i\}^C},\RV{X}_{\{i\}^C}) |\RV{H}\RV{D}_i\RV{X}_i\label{eq:wu}
\end{align}

Where Eq. \ref{eq:wu} follows from \ref{eq:cc} by weak union. By Theorem \ref{th:cons_ci}, $\RV{Y}_i\CI_{\prob{P}_\alpha} (\RV{Y}_{\{i\}^C},\RV{X}_{\{i\}^C}) |\RV{H}\RV{D}_i\RV{X}_i$ also, and so $\prob{P}_C$ is $(\RV{D},\RV{X};\RV{Y})$-causally contractibile. 

$\RV{Y}_i\CI_{\prob{P}_\alpha} \RV{D}_i |(\RV{H},\RV{X}_i)$ for all $i\in M$ implies $\RV{Y}_i\CI_{\prob{P}_C}^e \RV{D}_i C|(\RV{H},\RV{X}_i)$ for all $i\in M$ by Theorem \ref{th:cons_ci}. The result follows by noting that $\prob{P}_C$ is also $(\RV{D},\RV{X};\RV{Y})$-causally contractible by higher order conditionals, and therefore $(\RV{X};\RV{Y})$-causally contractible by Lemma \ref{lem:eci_cc_extend}.
\end{proof}