%!TEX root = main.tex


\subsubsection{Extended conditional independence}\label{ap:eci}

\todo[inline]{Needs a support condition}

In the case of a probability gap model $(\prob{P}_{\square}^{\RV{V}|\RV{W}},A)$ where there is some $\alpha\in A$ dominating $A$, we can relate conditional independence with respect to $\prob{P}_\square$ to what \citet{constantinou_extended_2017} \emph{extended conditional independence}, which is a notion they define with respect to a Markov kernel. These concepts may differ if $A$ is not dominated. Theorem 4.4 of \citet{constantinou_extended_2017} proves the following claim:

\begin{theorem}\label{th:dawid_constantionou}
Let $\RV{A}^*=\RV{A}\circ \RV{V}$, $\RV{B}^*=\RV{B}\circ\RV{V}$, $\RV{C}^*=\RV{C}\circ \RV{V}$ ($(\RV{A},\RV{B},\RV{C})$ are $\sigalg{V}$-measurable) and $\RV{D}^*=\RV{D}\circ \RV{W}$,$\RV{E}^*=\RV{E}\circ \RV{W}$ where $W$ is discrete and $\RV{W}=(\RV{D}^*,\RV{E}^*)$. In addition, let $\prob{P}_\alpha^{\RV{W}}$ be some probability distribution on $\RV{W}$ such that $w\in\RV{W}(\Omega)\implies \prob{P}_\alpha^{\RV{W}}(w)>0$. Then, denoting extended conditional independence with $\CI_{\prob{P},\text{ext}}$ and $\prob{P}_\alpha^{\RV{VW}}:=\prob{P}_\alpha^{\RV{W}}\odot \prob{P}^{\RV{V}|\RV{W}}$
\begin{align}
    \RV{A}\CI_{\prob{P},\text{ext}}(\RV{B},\RV{D})|(\RV{C},\RV{E})\iff \RV{A}^*\CI_{\prob{P}_\alpha}(\RV{B}^*,\RV{D}^*)|(\RV{C}^*,\RV{E}^*)
\end{align}
Where $\CI_{\prob{P}_\alpha}$ is order 0 conditional independence.
\end{theorem}

This result implies a close relationship between order 1 condtional independence and extended conditional independence.

\begin{theorem}
Let $\RV{A}^*=\RV{A}\circ \RV{V}$, $\RV{B}^*=\RV{B}\circ\RV{V}$, $\RV{C}^*=\RV{C}\circ \RV{V}$ ($(\RV{A},\RV{B},\RV{C})$ are $\sigalg{V}$-measurable) and $\RV{D}^*=\RV{D}\circ \RV{W}$,$\RV{E}^*=\RV{E}\circ \RV{W}$ where $V,W$ are discrete and $\RV{W}=(\RV{D}^*,\RV{E}^*)$. Then letting $\prob{P}_\alpha^{\RV{VW}}:=\prob{P}_\alpha^{\RV{W}}\odot \prob{P}^{\RV{V}|\RV{W}}$
\begin{align}
    \RV{A}\CI^1_{\prob{P},\text{ext}}(\RV{B},\RV{D})|(\RV{C},\RV{E})\iff \RV{A}^*\CI_{\prob{P}}(\RV{B}^*,\RV{D}^*)|(\RV{C}^*,\RV{E}^*)
\end{align}
\end{theorem}

\begin{proof}
If:

By assumption, $\RV{A}^*\CI_{\prob{P}_\alpha}(\RV{B}^*,\RV{D}^*)|(\RV{C}^*,\RV{E}^*)$ for all $\prob{P}_\alpha^{\RV{D^*E^*}}$. In particular, this holds for some $\prob{P}_\alpha^{\RV{D^*E^*}}$ such that $(d,e)\in (\RV{D}^*,\RV{E}^*)(\Omega)\implies \prob{P}_\alpha^{\RV{D^*E^*}}(d,e) >0$. Then by Theorem \ref{th:dawid_constantionou}, $\RV{A}\CI_{\prob{P},\text{ext}}(\RV{B},\RV{D})|(\RV{C},\RV{E})$.

Only if:

For any $\beta$, $\prob{P}_\beta^{\RV{ABC}|\RV{DE}}= \prob{P}_\beta^{\RV{DE}}\odot \prob{P}^{\RV{ABC}|\RV{DE}}$. By Lemma \ref{lem:disint_exist}, we have $\prob{P}^{\RV{A}|\RV{BCDE}}$ such that

\begin{align}
    \prob{P}_\beta^{\RV{A^*B^*C^*}\RV{D^*E^*}} &= \prob{P}_\beta^{\RV{D^*E^*}}\odot \prob{P}^{\RV{B^*C^*}|\RV{D^*E^*}}\odot \prob{P}^{\RV{A^*}|\RV{B^*C^*D^*E^*}}\\
                                      &= \prob{P}_\beta^{\RV{B^*C^*D^*E^*}}\odot \prob{P}^{\RV{A^*}|\RV{B^*C^*D^*E^*}}\\
                                      &= \prob{P}_\beta^{\RV{C^*E^*}}\odot \prob{P}_\beta^{\RV{B^*D^*}|\RV{C^*E^*}}\odot \prob{P}^{\RV{A}^*|\RV{B^*C^*D^*E^*}}
\end{align}

By Theorem \ref{th:dawid_constantionou}, we have some $\alpha$ such that $\prob{P}_\alpha^{\RV{D^*E^*}}$ is strictly positive on the range of $(\RV{D}^*,\RV{E}^*)$ and $\RV{A}^*\CI_{\prob{P}_\alpha}(\RV{B}^*,\RV{D}^*)|(\RV{C}^*,\RV{E}^*)$.

By independence, for some version of $\prob{P}^{\RV{A}|\RV{BCDE}}$:

\begin{align}
    \prob{P}_\alpha^{\RV{C^*E^*}}\odot \prob{P}_\alpha^{\RV{B^*D^*}|\RV{C^*E^*}}\odot \prob{P}^{\RV{A}^*|\RV{B^*C^*D^*E^*}} &= \tikzfig{indep_strengthen_1}\\
    &= \tikzfig{indep_strengthen_2}\\
    &= \prob{P}_\alpha^{\RV{C^*E^*}}\odot \prob{P}_\alpha^{\RV{B^*D^*}|\RV{C^*E^*}}\odot (\prob{P}_\alpha^{\RV{A}^*|\RV{C^*E^*}}\otimes\text{erase}_{BD})
\end{align}

Thus for any $(a,b,c,d,e)\in A\times B\times C\times D\times E$ such that $\prob{P}_\alpha^{\RV{B^*C^*D^*E^*}}(b,c,d,e)>0$, $\prob{P}^{\RV{A}^*|\RV{B^*C^*D^*E^*}}(a|b,c,d,e) = \prob{P}_\alpha^{\RV{A}^*|\RV{C^*E^*}}(a|c,e)$. However, by assumption, $\prob{P}_\alpha^{\RV{B^*C^*D^*E^*}}(b,c,d,e)=0 \implies \prob{P}_\beta^{\RV{B^*C^*D^*E^*}}(b,c,d,e)=0$, and so $\prob{P}_\beta^{\RV{A}^*|\RV{B^*C^*D^*E^*}}= \prob{P}_\alpha^{\RV{A}^*|\RV{C^*E^*}}(a|c,e)$ everywhere except a set of $\prob{P}_\beta$-measure 0. Thus
    
\begin{align}
    \prob{P}_\beta^{\RV{A^*B^*C^*}\RV{D^*E^*}} &= \tikzfig{indep_strengthen_3}\\
    &= \tikzfig{indep_strengthen_4}
\end{align}
\end{proof}

We can deduce conditional independences in probability combs when conditional probabilities exist and they are \emph{unresponsive} to some input variables.

\begin{definition}[Unresponsiveness]
Given discrete $\Omega$, a probability gap model $\prob{P}_\square:A\to \Delta(\Omega)$, variables $\RV{W}:\Omega\to W$, $\RV{X}:\Omega\to X$, $\RV{Y}:\Omega\to Y$, if there is some version of the conditional probability $\prob{P}^{\RV{Y}|\RV{WX}}$ and $\prob{P}_\square^{\RV{Y}|\RV{W}}$ such that
\begin{align}
  \prob{P}_\square^{\RV{Y}|\RV{WX}} &= \tikzfig{cond_indep_erase} \label{eq:higherorder_ci_erase}
\end{align}
then $\prob{P}_\square^{\RV{Y}|\RV{WX}}$ is \emph{unresponsive} to $\RV{X}$.
\end{definition}

\begin{definition}[Domination]
Given a probability gap model $\prob{P}_\square:A\to \Delta(\Omega)$, $\alpha\in A$ dominates $A$ if $\prob{P}_\beta(B)>0\implies \prob{P}_\alpha(B)>0$ for all $\beta in A$, $B\in \sigalg{F}$.
\end{definition}

\begin{theorem}[Conditional independence from kernel unresponsiveness]\label{th:cons_ci}
Given discrete $\Omega$, variables $\RV{W}:\Omega\to W$, $\RV{X}:\Omega\to X$, $\RV{Y}:\Omega\to Y$ and a probability gap model $\prob{P}_\square:A\to \Delta(\Omega)$ with conditional probability $\prob{P}_\square^{\RV{Y}|\RV{WX}}$ and such that there is $\alpha\in A$ dominating $A$, $\RV{Y}\CI_{\prob{P}_\square}\RV{X}|\RV{W}$ if and only if $\prob{P}_\square^{\RV{Y}|\RV{WX}}$ is unresponsive to $\RV{W}$. 
\end{theorem}

\begin{proof}
If:
For every $\alpha\in A$ we can write
\begin{align}
  \prob{P}_\alpha^{\RV{Y}|\RV{WX}} &= \tikzfig{cond_indep_erase_alpha}
\end{align}
And so, by Theorem \ref{th:cho_ci_equiv}, $\RV{Y}\CI_{\prob{P}_\alpha}\RV{X}|\RV{W}$ for all $\alpha\in A$, and so $\RV{Y}\CI_{\prob{P}_\square}\RV{X}|\RV{W}$.
Only if:
For $\alpha$ dominating $A$, by Theorem \ref{th:cho_ci_equiv}, there exists a version of $\prob{P}_\alpha^{\RV{Y}|\RV{WX}}$ unresponsive to $\RV{W}$. Because $\alpha$ dominates $A$, every version of $\prob{P}_\alpha^{\RV{Y}|\RV{WX}}$ is a version of $\prob{P}_\beta^{\RV{Y}|\RV{WX}}$ for all $\beta in A$, thus it is a version of $\prob{P}_\square^{\RV{Y}|\RV{WX}}$ also.
\end{proof}

Note that $\RV{Y}\CI_{\prob{P}_\square}\RV{X}|\RV{W}$ does \emph{not} imply the existence of $\prob{P}_\square^{\RV{Y}|\RV{WX}}$. If we have, for example, $A=\{\alpha,\beta\}$ and $\prob{P}_\alpha^{\RV{AB}}$ is two flips of a fair coin while $\prob{P}_\beta^{\RV{AB}}$ is a flip of a biased coin followed by a flip of a fair coin, then $\RV{A}\CI_{\prob{P}}\RV{B}$ but $\prob{P}^{\RV{AB}}$ does not exist.

We also need the domination condition. Consider $A$ a collection of inserts that all deterministically set a variable $\RV{X}$; then for any variable $\RV{Y}$ $\RV{Y}\CI_{\prob{P}_\square} \RV{X}$ because $\RV{X}$ is deterministic for any $\alpha\in A$. But $\prob{P}_\square^{\RV{Y}|\RV{X}}$ is not necessarily unresponsive to $\RV{X}$.


\subsubsection{Graphical properties of conditional independence}

It is well-known that directed acyclic graphs are able to represent some conditional independence properties of probability models via the graphical property of \emph{d-separation}. String diagrams are similar to directed acyclic graphs, and string diagrams can be translated into directed acyclic graphs and vise-versa \citep{fong_causal_2013}. Thus we expect that a property analogous to d-separation can be defined for string diagrams.

We can reason from graphical properties of model disintegrations to graphical properties of models as Theorem \ref{th:cons_ci}. A general theory akin to d-separation for string diagrams may facilitate a more general understanding of how conditional independence properties of a model relate to conditional independence properties of its components.


\subsection{Results I use that don't really fit into the flow of the text}

\subsubsection{Repeated variables}

Lemmas \ref{lem:nocopy1} and \ref{lem:nocopy2} establish that models of repeated variables must connect the repetitions with a copy map.

\begin{lemma}[Output copies of the same variable are identical]\label{lem:nocopy1}
For any $\Omega$, $\RV{X},\RV{Y},\RV{Z}$ random variables on $\Omega$ and conditional probability $\model{K}^{\RV{YZ}|\RV{X}}$, there is a conditional probability $\kernel{K}^{\RV{YYZ}|\RV{X}}$ unique up to impossible values of $\RV{X}$ such that
\begin{align}
    \tikzfig{kyyz} = \model{K}^{\RV{YZ}|\RV{X}}
\end{align}
and it is given by
\begin{align}
        \kernel{K}^{\RV{YYZ}|\RV{X}} &= \tikzfig{compose_with_copymap}\\
        &\iff \\
        \kernel{K}^{\RV{YYZ}|\RV{X}}(y,y',z|x) &= \llbracket y=y' \rrbracket\kernel{K}^{\RV{YZ}|\RV{X}}(y,z|x)\\
\end{align}
\end{lemma}

\begin{proof}
If we have a valid $\model{K}^{\RV{YYZ}|\RV{X}}$, it must be the pushforward of $(\RV{Y},\RV{Y},\RV{Z})$ under some $\model{K}^{\RV{I}|\RV{X}}$. Furthermore, $\model{K}^{\RV{YZ}|\RV{X}}$ must be the pushforward of $(*,\RV{Y},\RV{Z})\cong (\RV{Y},\RV{Z})$ under the same $\model{K}^{\RV{I}|\RV{X}}$.

For any $x\in \RV{X}(\Omega)$, validity requires $(\RV{X},\RV{Y},\RV{Y},\RV{Z})\yields (x,y,y',z)=\emptyset \implies \model{K}^{\RV{YYZ}|\RV{X}}(y,y',z|x)=0$. Clearly, whenever $y\neq y'$, $\model{K}^{\RV{YYZ}|\RV{X}}(y,y',z|x)=0$. Because $\model{K}^{\RV{YYZ}|\RV{X}}$ is a Markov kernel, there is some $\model{L}:X\kto X\times Z$ such that
\begin{align}
    \model{K}^{\RV{YYZ}|\RV{X}}(y,y',z|x) = \llbracket y=y' \rrbracket \model{L}(y,z|x)\\
\end{align}
But then
\begin{align}
    \model{K}^{\RV{YZ}|\RV{X}}(y,z|x) &= \sum_{y'\in Y} \model{K}^{\RV{YYZ}|\RV{X}}(y,y',z|x)\\
    &= \model{L}(y,z|x)\\
\end{align}
\end{proof}

\begin{lemma}[Copies shared between input and output are identical]\label{lem:nocopy2}

\todo[inline]{This got mixed up at some point and needs ot be unmixed-up}
For any $\kernel{K}:(\RV{X},\RV{Y})\kto (\RV{X},\RV{Z})$, $\kernel{K}$ is a model iff there exists some $\model{L}:(\RV{X},\RV{Y})\kto \RV{Z}$ such that
\begin{align}
     \kernel{K} &= \tikzfig{precompose_with_copymap}\\
     &\iff\\
     \kernel{K}_{x,y}^{\prime x',z} &= \llbracket x=x'\rrbracket \kernel{L}_{\prime x,y}^{z}
\end{align}

For any $\Omega$, $\RV{X},\RV{Y},\RV{Z}$ random variables on $\Omega$ and conditional probability $\model{K}^{\RV{Z}|\RV{XY}}$, there is a conditional probability $\kernel{K}^{\RV{XZ}|\RV{XY}}$ unique up to impossible values of $(\RV{X},\RV{Y})$ such that
\begin{align}
    \tikzfig{kxyxz} = \kernel{K}^{\RV{XZ}|\RV{XY}}
\end{align}
and it is given by
\begin{align}
        \kernel{K}^{\RV{XZ}|\RV{XY}} &= \tikzfig{compose_with_copymap}\\
        &\iff \\
        \kernel{K}^{\RV{XZ}|\RV{XY}}(x,z|x',y) &= \llbracket x=x' \rrbracket\kernel{K}^{\RV{Z}|\RV{XY}}(z|x',y)\\
\end{align}

\end{lemma}

\begin{proof}
If we have a valid $\model{K}^{\RV{XZ}|\RV{XY}}$, it must be the pushforward of $(\RV{X},\RV{Z})$ under some $\model{K}^{\RV{I}|\RV{XY}}$. Furthermore, $\model{K}^{\RV{Z}|\RV{XY}}$ must be the pushforward of $(*,\RV{Z})\cong (\RV{Z})$ under the same $\model{K}^{\RV{I}|\RV{X}}$.

For any $(x,y)\in (\RV{X},\RV{Y})(\Omega)$, validity requires $(\RV{X},\RV{Y},\RV{X},\RV{Z})\yields (x,y,x',z)=\emptyset \implies \model{K}^{\RV{XZ}|\RV{XY}}(x',z|x,y)=0$. Clearly, whenever $x\neq x'$, $\model{K}^{\RV{XZ}|\RV{XY}}(x',z|x,y)=0$. Because $\model{K}^{\RV{XZ}|\RV{XY}}$ is a Markov kernel, there is some $\model{L}:X\times Y\kto Z$ such that
\begin{align}
    \model{K}^{\RV{XZ}|\RV{XY}}(x',z|x,y)=0 = \llbracket x=x' \rrbracket \model{L}(z|x,y)\\
\end{align}
But then
\begin{align}
    \model{K}^{\RV{Z}|\RV{XY}}(y,z|x) &= \sum_{x'\in X} \model{K}^{\RV{XZ}|\RV{XY}}(x',z|x,y)\\
    &= \model{L}(z|x,y)\\
\end{align}
\end{proof}


\begin{theorem}[Existence of valid conditional probabilities]\label{th:valid_disint}
Given a probability gap model $\prob{P}_\square:A\to \Delta(\Omega)$ along with a valid conditional probability $\model{P}_\square^{\RV{XY}|\RV{W}}$, there exists a valid conditional probability $\prob{P}_\square^{\RV{Y}|\RV{WX}}$.
\end{theorem}

\begin{proof}
From Lemma \ref{lem:disint_exist}, we have the existence of some Markov kernel $\prob{P}_\square^{\RV{Y}|\RV{WX}}:W\times X\to Y$ such that
\begin{align}
    \prob{P}_\square^{\RV{XY}|\RV{W}}=\prob{P}_\square^{\RV{X}|\RV{W}}\odot \prob{P}_\square^{\RV{Y}|\RV{WX}}\label{eq:k_disint}
\end{align}

By definition of conditional probability , for any insert $\alpha\in A$ there exists $\prob{P}_\alpha^{\RV{W}}\in\Delta(W)$ such that

\begin{align}
    \prob{P}_\alpha^{\RV{WXY}}=\prob{P}_\alpha^{\RV{W}}\odot\prob{P}_\square^{\RV{XY}|\RV{W}}
\end{align}

Thus

\begin{align}
\prob{P}_\alpha^{\RV{WXY}}&= \prob{P}_\alpha^{\RV{W}}\odot(\prob{P}_\square^{\RV{X}|\RV{W}}\odot \prob{P}_\square^{\RV{Y}|\RV{WX}})\\
&= (\prob{P}_\alpha^{\RV{W}}\odot\prob{P}_\square^{\RV{X}|\RV{W}})\odot \prob{P}_\square^{\RV{Y}|\RV{WX}})
\end{align}

Let $\text{erasef}_Y:Y\to \{*\}$ be the erase function on $Y$ (as opposed to the erase kernel) and $\text{idf}_{W\times X}$ be the identity function on $W\times X$. Noting that 
\begin{align}
(\RV{W},\RV{X})&=(\text{idf}_{W\times X}\otimes \text{erasef}_Y)\circ (\RV{W},\RV{X},\RV{Y})
\end{align}
By Lemma \ref{lem:prod_pushf} together with Theorem \ref{th:recurs_pushf} we have for all $\alpha$:

\begin{align}
    \prob{P}_\alpha^{\RV{XW}} &= \prob{P}_\alpha^{\RV{WXY}}(\text{id}_{W\times X}\otimes \text{erase}_Y)\\
                              &= \prob{P}_\alpha^{\RV{W}}\odot(\prob{P}_\square^{\RV{X}|\RV{W}}\odot \prob{P}_\square^{\RV{Y}|\RV{WX}})(\text{id}_{W\times X}\otimes \text{erase}_Y)\\
                              &= \prob{P}_\alpha^{\RV{W}}\odot\prob{P}_\square^{\RV{X}|\RV{W}}
\end{align}

Then

\begin{align}
\prob{P}_\alpha^{\RV{XWY}}&= (\prob{P}_\alpha^{\RV{XW}})\odot \prob{P}_\square^{\RV{Y}|\RV{WX}})
\end{align}

And so $\prob{P}_\square^{\RV{Y}|\RV{WX}})$ is a $\RV{Y}|\RV{WX}$ conditional probability. We also want it to be valid, so we will verify that it can be chosen as such.

We also need to check that $\prob{P}_\square^{\RV{Y}|\RV{WX}}$ can be chosen so that it is valid. By validity of $\model{K}^{\RV{W,Y}|\RV{X}}$, $w\in \RV{W}(\Omega)$ and $(\RV{X},\RV{W},\RV{Y})\yields(x,w,y)=\emptyset \implies \model{P}_\square^{\RV{W,Y}|\RV{X}}=0$, so we only need to check for $(w,x,y)$ such that $\model{P}_\square^{\RV{W,Y}|\RV{X}}(w,y|x)=0$. For all $x,y$ such that $\kernel{K}^{\RV{Y}|\RV{X}}(y|x)$ is positive, we have $\model{P}^{\RV{W,Y}|\RV{X}}(w,y|x)=0\implies \prob{P}_\square^{\RV{Y}|\RV{WX}}(y|w,x)=0$. Furthermore, where $\model{K}^{\RV{W}|\RV{X}}(w|x)=0$, we either have $(\RV{W},\RV{X})\yields(w,x)=\emptyset$ or we can choose some $\omega\in (\RV{W},\RV{X})\yields(w,x)$ and let $\prob{P}^{\RV{Y}|\RV{WX}}(\RV{Y}(\omega)|w,x) = 1$.
\end{proof}


\subsection{Validity}

\begin{theorem}[Validity]\label{th:completion}
Given $(\Omega,\sigalg{F})$, $\RV{X}:\Omega\to X$, $\kernel{J}\in \Delta(X)$ with $\Omega$ and $X$ standard measurable, there exists some $\mu\in \Delta(\Omega)$ such that $\mu^{\RV{X}}=\kernel{J}$ if and only if $\kernel{J}$ is a valid distribution.
\end{theorem}

\begin{proof}
If:
This is a Theorem 2.5 of \citet{ershov_extension_1975}.
Only if:
This is also found in \citet{ershov_extension_1975}, but is simple enough to reproduce here. Suppose $\kernel{J}$ is not a valid probability distribution. Then there is some $x\in X$ such that $\RV{X}\yields x = \emptyset$ but $\kernel{J}(x)>0$. Then
\begin{align}
    \mu^{\RV{X}}(x) &= \mu (\RV{X}\yields x)\\
    &= \sum_{x'\in X} \kernel{J}(x') \kernel{K}(\RV{X}\yields x|x')\\
    &= 0\\
    &\neq \kernel{J}(x)
\end{align}
\end{proof}


\begin{lemma}[Copy-product is an intersection of probability sets]\label{th:intersection}
Given $(\Omega,\sigalg{F})$, $\RV{X}:\Omega\to (X,\sigalg{X})$, $\RV{Y}:\Omega\to (Y,\sigalg{Y})$, $\RV{Z}:\Omega\to (Z,\sigalg{Z})$ all standard measurable and valid candidate conditionals $\prob{P}_{\{\}}^{\RV{Y}|\RV{X}}$ and $\prob{Q}_{\{\}}^{\RV{Z}|\RV{YX}}$ defining probability sets $\prob{P}_{\{\}}$ and $\prob{Q}_{\{\}}$, then the probability set $\prob{R}_{\{\}}$ defined by $\prob{R}_{\{\}}^{\RV{YZ}|\RV{X}}:=\prob{P}_{\{\}}^{\RV{Y}|\RV{X}}\odot \prob{Q}_{\{\}}^{\RV{Z}|\RV{YX}}$ is equal to $\prob{P}_{\{\}}\cap\prob{Q}_{\{\}}$.
\end{lemma}

\begin{proof}
By assumption

\begin{align}
    \prob{R}_{\{\}}^{\RV{YZ}|\RV{X}}:=\prob{P}_{\{\}}^{\RV{Y}|\RV{X}}\odot \prob{Q}_{\{\}}^{\RV{Z}|\RV{YX}}
\end{align}

Therefore for any $\prob{R}_a\in\prob{R}_{\{\}}$

\begin{align}
    \prob{R}_{a}^{\RV{XYZ}} &= \prob{R}_a^{\RV{X}}\odot \prob{P}_{\{\}}^{\RV{Y}|\RV{X}}\odot \prob{Q}_{\{\}}^{\RV{Z}|\RV{YX}}\\
    \implies \prob{R}_{a}^{\RV{XY}} &= \prob{R}_a^{\RV{X}}\odot \prob{P}_{\{\}}^{\RV{Y}|\RV{X}}\\
    \land \prob{R}_{a}^{\RV{XYZ}} &= \prob{R}_{a}^{\RV{XY}}\odot \prob{Q}_{\{\}}^{\RV{Z}|\RV{YX}}
\end{align}

Thus $\prob{P}_{\{\}}^{\RV{Y}|\RV{X}}$ is a version of $\prob{R}_{\{\}}^{\RV{Y}|\RV{X}}$ and $\prob{Q}^{\RV{Z}|\RV{YX}}$ is a version of $\prob{R}_{\{\}}^{\RV{Z}|\RV{YX}}$ so $\prob{R}_{\{\}}\subset \prob{P}_{\{\}}\cap\prob{Q}_{\{\}}$.

Suppose there's an element $\prob{S}$ of $\prob{P}_{\{\}}\cap\prob{Q}_{\{\}}$ not in $\prob{R}_{\{\}}$. Then by definition of $\prob{R}_{\{\}}$, $\prob{R}_{\{\}}^{\RV{YZ}|\RV{X}}$ is not a version of $\prob{S}_{\{\}}^{\RV{YZ}|\RV{X}}$. But by construction of $\prob{S}$, $\prob{P}_{\{\}}^{\RV{Y}|\RV{X}}$  is a version of $\prob{S}^{\RV{Y}|\RV{X}}$ and  $\prob{Q}^{\RV{Z}|\RV{YX}}$ is a version of $\prob{S}^{\RV{Z}|\RV{YX}}$. But then by the definition of disintegration, $\prob{P}_{\{\}}^{\RV{Y}|\RV{X}} \odot \prob{Q}_{\{\}}^{\RV{Z}|\RV{YX}}$ is a version of $\prob{S}_{\{\}}^{\RV{YZ}|\RV{X}}$ and so $\prob{R}_{\{\}}^{\RV{YZ}|\RV{X}}$ is a version of $\prob{S}_{\{\}}^{\RV{YZ}|\RV{X}}$, a contradiction.
\end{proof}


\begin{lemma}[Equivalence of validity definitions]\label{th:valid_agree}
Given $\RV{X}:\Omega\to X$, with $\Omega$ and $X$ standard measurable, a probability measure $\prob{P}^{\RV{X}}\in \Delta(X)$ is valid if and only if the conditional $\prob{P}^{\RV{X}|*}:=*\mapsto \prob{P}^{\RV{X}}$ is valid.
\end{lemma}

\begin{proof}
$*\yields *=\Omega$ necessarily. Thus validity of $\prob{P}^{\RV{X}|*}$ means 

\begin{align}
    \forall A\in \sigalg{X}: \RV{X}\yields A=\emptyset \implies \prob{P}^{\RV{X}|*}(A|*)&=0
\end{align}

But $\prob{P}^{\RV{X}|*}(A|*)=\prob{P}^{\RV{X}}(A)$ by definition, so this is equivalent to

\begin{align}
    \forall A\in \sigalg{X}: \RV{X}\yields A=\emptyset \implies \prob{P}^{\RV{X}}(A)&=0
\end{align}
\end{proof}


\begin{lemma}[Copy-product of valid candidate conditionals is valid]\label{lem:valid_extendability}
Given $(\Omega,\sigalg{F})$, $\RV{X}:\Omega\to X$, $\RV{Y}:\Omega\to Y$, $\RV{Z}:\Omega\to Z$ (all spaces standard measurable) and any valid candidate conditional $\prob{P}^{\RV{Y}|\RV{X}}$ and $\prob{Q}^{\RV{Z}|\RV{Y}\RV{X}}$, $ \prob{P}^{\RV{Y}|\RV{X}}\odot \prob{Q}^{\RV{Z}|\RV{Y}\RV{X}}$ is also a valid candidate conditional.
\end{lemma}

\begin{proof}
Let $\prob{R}^{\RV{YZ}|\RV{X}}:=\prob{P}^{\RV{Y}|\RV{X}}\odot \prob{Q}^{\RV{Z}|\RV{Y}\RV{X}}$.

We only need to check validity for each $x\in \RV{X}(\Omega)$, as it is automatically satisfied for other values of $\RV{X}$.

For all $x\in \RV{X}(\Omega)$, $B\in \sigalg{Y}$ such that $\RV{X}\yields \{x\}\cap\RV{Y}\yields B=\emptyset$, $\prob{P}^{\RV{Y}|\RV{X}}(B|x)=0$ by validity. Thus for arbitrary $C\in \sigalg{Z}$
\begin{align}
    \prob{R}^{\RV{YZ}|\RV{X}}(B\times C|x) &= \int_B \prob{Q}^{\RV{Z}|\RV{YX}}(C|y,x)\prob{P}^{\RV{Y}|\RV{X}}(dy|x)\\
                                  &\leq \prob{P}^{\RV{Y}|\RV{X}}(B|x)\\
                                  &=0
\end{align}

For all $\{x\}\times B$such that $\RV{X}\yields \{x\}\cap\RV{Y}\yields B\neq \emptyset$ and $C\in \sigalg{Z}$ such that $(\RV{X},\RV{Y},\RV{Z})\yields \{x\}\times B\times C=\emptyset$, $\prob{Q}^{\RV{Z}|\RV{YX}}(C|y,x)=0$ for all $y\in B$ by validity. Thus:
\begin{align}
    \prob{R}^{\RV{YZ}|\RV{X}}(B\times C|x) &= \int_B \prob{Q}^{\RV{Z}|\RV{YX}}(C|y,x)\prob{P}^{\RV{Y}|\RV{X}}(dy|x)\\
                                            &=0
\end{align}
\end{proof}

\begin{corollary}[Valid conditionals are validly extendable to valid distributions]\label{corr:valid_extend_order1}
Given $\Omega$, $\RV{U}:\Omega\to U$, $\RV{W}:\Omega\to W$ and a valid candidate conditional $\prob{T}^{\RV{W}|\RV{U}}$, then for any valid candidate conditional $\prob{V}^{\RV{U}}$, $\prob{V}^{\RV{U}}\odot \prob{T}^{\RV{W}|\RV{U}}$ is a valid candidate probability.
\end{corollary}

\begin{proof}
Applying Lemma \ref{lem:valid_extendability} choosing $\RV{X}=*$, $\RV{Y}=\RV{U}$, $\RV{Z}=\RV{W}$ and $\prob{P}^{\RV{Y}|\RV{X}}=\prob{V}^{\RV{U}|*}$ and $\prob{Q}^{\RV{Z}|\RV{YX}}=\prob{T}^{\RV{W}|\RV{U*}}$ we have $\prob{R}^{WU|*}:=\prob{V}^{\RV{U}|*}\odot \prob{T}^{\RV{W}|\RV{U}*}$ is a valid conditional probability. Then $\prob{R}^{\RV{WU}}\cong \prob{R}^{\RV{WU}|*}$ is valid by Theorem \ref{th:valid_agree}.
\end{proof}

\begin{theorem}[Validity of conditional probabilities]\label{th:valid_conditional_probability}
Suppose we have $\Omega$, $\RV{X}:\Omega\to X$, $\RV{Y}:\Omega\to Y$, with $\Omega$, $X$, $Y$ discrete. A conditional $\prob{T}^{\RV{Y}|\RV{X}}$ is valid if and only if for all valid candidate distributions $\prob{V}^{\RV{X}}$, $\prob{V}^{\RV{X}}\odot \prob{T}^{\RV{Y}|\RV{X}}$ is also a valid candidate distribution.
\end{theorem}

\begin{proof}
If: this follows directly from Corollary \ref{corr:valid_extend_order1}.

Only if: suppose $\prob{T}^{\RV{Y}|\RV{X}}$ is invalid. Then there is some $x\in X$, $y\in Y$ such that $\RV{X}\yields(x)\neq \emptyset$, $(\RV{X},\RV{Y})\yields(x,y)=\emptyset$ and $\prob{T}^{\RV{Y}|\RV{X}}(y|x)>0$. Choose $\prob{V}^{\RV{X}}$ such that $\prob{V}^{\RV{X}}(\{x\})=1$; this is possible due to standard measurability and valid due to $\RV{X}^{-1}(x)\neq \emptyset$. Then
\begin{align}
    (\prob{V}^{\RV{X}}\odot \prob{T}^{\RV{Y}|\RV{X}})(x,y) &= \prob{T}^{\RV{Y}|\RV{X}}(y|x) \prob{V}^{\RV{X}}(x)\\
                                                                     &= \prob{T}^{\RV{Y}|\RV{X}}(y|x)\\
                                                                     &>0
\end{align}
Hence $\prob{V}^{\RV{X}}\odot \prob{T}^{\RV{Y}|\RV{X}}$ is invalid.
\end{proof}

\subsection{Combs}\label{ap:combs}

\begin{reptheorem}{th:comb_equiv}[Equivalence of comb representations]
Given sample space $(\Omega,\sigalg{F})$, a finite collection of variables $\RV{X}_{i}:\Omega\to (X_i,\sigalg{X}_i)$ for $i\in [n]$, $X_i$ discrete, and a disassembled probability comb $\{\prob{P}_\square^{\RV{X}_{i}|\RV{X}_{[i-1]}}|i\in \RV{X}_{[n]_{\text{odd}}}\}$, for any $l\in [n]_{\mathrm{odd}}$ and any $\kernel{K}:X_{[l-1]}\kto X_l$
\begin{align}
    (\bigcombprod_{j\in [l-1]_{\mathrm{odd}}} \prob{P}_\square^{\RV{X}_{j}|\RV{X}_{[j-1]}}) \combprod \kernel{K} &\overset{\prob{P}_\square}{\cong} (\bigcombprod_{j\in [l]_{\mathrm{odd}}} \prob{P}_\square^{\RV{X}_{j}|\RV{X}_{[j-1]}})\\
    \implies \kernel{K} &\overset{\prob{P}_\square}{\cong} \prob{P}_\square^{\RV{X}_{l}|\RV{X}_{[l-1]}}
\end{align}
\end{reptheorem}

\begin{proof}
Equality is trivial for $l=1$. 

For a sequence $x_{[l-2]}\in X_{[l-2]}$, let $e_{[l-2]}$ be the even indices of $x_{[l-2]}$ and $o_{[l-2]}$ be the odd indices.

For any $e_{l-1}\in X_{l-1}$, $A\in X_{l}$ let $C^>_{A,e_{l-1}}\in \sigalg{X}_{[l-2]}$ be the set of points $C^>_{A,e_{l-1}}:=\{x_{[l-2]}| \kernel{K}(A|e_{l-1},x_{[l-2]})> \prob{P}_\square^{\RV{X}_{j}|\RV{X}_{[j-1]}}(A|e_{l-1},x_{[l-2]})\}$, and $C^<_{A,e_{l-1}}$ the obvious analog. Then, definining $C_{A,e_{l-1}}=C^>_{A,e_{l-1}}\cup C^<_{A,e_{l-1}}$,
\begin{align}
    (\bigcombprod_{j\in [l-1]_{\mathrm{odd}}} \prob{P}_\square^{\RV{X}_{j}|\RV{X}_{[j-1]}})(A\times C^>_{A,e_{l-1}}|e_{[l-3]},e_{l-1}) &= 0\\
    (\bigcombprod_{j\in [l-1]_{\mathrm{odd}}} \prob{P}_\square^{\RV{X}_{j}|\RV{X}_{[j-1]}})(A\times C^<_{A,e_{l-1}}|e_{[l-3]},e_{l-1}) &= 0\\
    \implies (\bigcombprod_{j\in [l-1]_{\mathrm{odd}}} \prob{P}_\square^{\RV{X}_{j}|\RV{X}_{[j-1]}})(A\times C_{A,e_{l-1}}|e_{[l-3]},e_{l-1}) &= 0\\
    &= \sum_{o_{[l-2]}\in C_{A,e_{l-1},\text{odd}}} \kernel{K}(A|e_{l-1},x_{[l-2]}) \prod_{j\in [l-2]_{\text{odd}}} \prob{P}_\square^{\RV{X}_{j}|\RV{X}_{[j-1]}}(o_{j}|o_{[j-2]},e_{[j-1]})
\end{align}
for all $e_{[l-3]}\in X_{[l-3]_{\text{even}}}$.

Consider arbitrary $\prob{P}_\alpha\in \prob{P}_\square$, $A\subset X_l$, $C\subset X_{[l-1]}$:
\begin{align}
    \prob{P}_\alpha^{\RV{X}_{[l]}} (A\times C) =&  \sum_{x_{[l-2]}\in C} \kernel{P}_\square(A|e_{l-1},x_{[l-2]}) \prod_{j\in [l-2]_{\text{odd}}} \prob{P}_\square^{\RV{X}_{j}|\RV{X}_{[j-1]}}(x_{j}|o_{[j-2]},e_{[j-1]}) \prob{P}_\alpha^{\RV{X}_{j+1}|\RV{X}_{[j]}}(x_{j+1}|o_{[j]},e_{[j-1]})\\
    =& \sum_{x_{[l-2]}\in C_{A,e_{l-1}}} \kernel{P}_\square(A|e_{l-1},x_{[l-2]}) \prod_{j\in [l-2]_{\text{odd}}} \prob{P}_\square^{\RV{X}_{j}|\RV{X}_{[j-1]}}(x_{j}|o_{[j-2]},e_{[j-1]}) \prob{P}_\alpha^{\RV{X}_{j+1}|\RV{X}_{[j]}}(x_{j+1}|o_{[j]},e_{[j-1]}) \\
    &+ \sum_{x_{[l-2]}\in C^C_{A,e_{l-1}}} \kernel{P}_\square(A|e_{l-1},x_{[l-2]}) \prod_{j\in [l-2]_{\text{odd}}} \prob{P}_\square^{\RV{X}_{j}|\RV{X}_{[j-1]}}(x_{j}|o_{[j-2]},e_{[j-1]}) \prob{P}_\alpha^{\RV{X}_{j+1}|\RV{X}_{[j]}}(x_{j+1}|o_{[j]},e_{[j-1]})\\
    =& 0 + \sum_{x_{[l-2]}\in C^C_{A,e_{l-1}}} \kernel{K}(A|e_{l-1},x_{[l-2]}) \prod_{j\in [l-2]_{\text{odd}}} \prob{P}_\square^{\RV{X}_{j}|\RV{X}_{[j-1]}}(x_{j}|o_{[j-2]},e_{[j-1]}) \prob{P}_\alpha^{\RV{X}_{j+1}|\RV{X}_{[j]}}(x_{j+1}|o_{[j]},e_{[j-1]})\\
    =& \prob{P}_\alpha^{\RV{X}_{[l-1]}}\odot \kernel{K} (A\times C)\\
    \implies \kernel{K} \overset{\prob{P}_\square}{\cong}& \prob{P}_\square^{\RV{X}_{l}|\RV{X}_{[l-1]}}
\end{align}
\end{proof}

\subsection{Comb conditional correspondence}\label{ap:ccc}

\begin{reptheorem}{th:comb_conditional_correspondence}[Comb-conditional correspondence]
Given a probability comb $\{\prob{P}_\square^{\RV{X}_{i}|\RV{X}_{[i-1]}}|i\in \RV{X}_{D_{\text{odd}}}\}$ and a blind choice $\alpha$
\begin{align}
\prob{P}_\square^{\RV{X}_{D_{\text{odd}}}\combbreak \RV{X}_{D_{\text{even}}}}\cong{\prob{P}_\alpha}{=}\prob{P}_\alpha^{\RV{X}_{D_{\text{odd}}}| \RV{X}_{D_{\text{even}}}}
\end{align}
\end{reptheorem}

\begin{proof}
Consider $n\in D$. The correspondence is immediate for $n=1$:
\begin{align}
    \prob{P}_\square^{\RV{X}_{1}\combbreak \RV{X}_{0}}\overset{\prob{P}_\alpha}{\cong}\prob{P}_\alpha^{\RV{X}_{1}| \RV{X}_{0}}
\end{align}

Suppose for induction the correspondence holds for odd $n-2$. For any blind $\alpha$
\begin{align}
    \prob{P}_\alpha^{\RV{X}_{[n]}|\RV{X}_0} &= \tikzfig{comb_condi_corresp_1}\\
    &= \tikzfig{comb_condi_corresp_2}\\
    &= \tikzfig{comb_condi_corresp_3}\\
    &= \tikzfig{comb_condi_corresp_4}\\
    &= \tikzfig{comb_condi_corresp_5}
\end{align}

and we also have
\begin{align}
    (\prob{P}_\alpha^{\RV{X}_{[n]_{\text{even}}}|\RV{X}_0} \odot\prob{P}_\square^{\RV{X}_{D_{\text{odd}}}\combbreak \RV{X}_{D_{\text{even}}}})(A\times B|x) &= 
\end{align}
\end{proof}