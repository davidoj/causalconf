%!TEX root = main.tex

\section{Appendix, needs to be organised}

\subsection{Existence of conditional probabilities}


\begin{lemma}[Conditional pushforward]\label{th:recurs_pushf}
Suppose we have a sample space $(\Omega,\sigalg{F})$, variables $\RV{X}:\Omega\to X$ and $\RV{Y}:\Omega\to Y$, $\RV{Z}:\Omega\to Z$ and a probability set $\prob{P}_{\{\}}$ with conditional $\prob{P}_{\{\}}^{\RV{X}|\RV{Y}}$ such that $\RV{Z}=f\circ \RV{Y}$ for some $f:Y\to Z$. Then there exists a conditional probability $\prob{P}_{\{\}}^{\RV{Z}|\RV{X}}=\prob{P}_{\{\}}^{\RV{Y}|\RV{X}}\kernel{F}_{f}$.
\end{lemma}

\begin{proof}
Note that $(\RV{X},\RV{Z})=(\text{id}_X\otimes f)\circ (\RV{X},\RV{Y})$. Thus, by Lemma \ref{lem:prod_pushf}, for any $\prob{P}_\alpha\in \prob{P}_{\{\}}$

\begin{align}
    \prob{P}_\alpha^{\RV{XZ}} = \prob{P}_\alpha^{\RV{XY}}\kernel{F}_{\text{id}_X\otimes f}
\end{align}

Note also that for all $A\in\sigalg{X}$, $B\in \sigalg{Z}$, $x\in X$, $y\in Y$:

\begin{align}
\prob{F}_{\text{id}_X\otimes f}(A\times B|x,y)&=\delta_x(A)\delta_{f(y)}(B)\\
&= \prob{F}_{\text{id}_X} (A|x)\otimes \prob{F}_f(B|y)\\
\implies \prob{F}_{\text{id}_X\otimes f} &= \prob{F}_{\text{id}_X} \otimes \prob{F}_f
\end{align}

Thus

\begin{align}
    \prob{P}_\alpha^{\RV{XZ}} &= (\prob{P}_\alpha^{\RV{X}}\odot \prob{P}_{\{\}}^{\RV{Y}|\RV{X}})\kernel{F}_{\text{id}_X}\otimes \kernel{F}_f\\
    &= \tikzfig{conditional_pushforward}
\end{align}

Which implies $\prob{P}_{\{\}}^{\RV{Y}|\RV{X}}\kernel{F}_{f}$ is a version of $\prob{P}_{\alpha}^{\RV{Z}|\RV{X}}$. Because this holds for all $\alpha$, it is therefore also a version of $\prob{P}_{\{\}}^{\RV{Z}|\RV{X}}$.
\end{proof}

\begin{theorem}[Existence of regular conditionals]
Suppose we have a sample space $(\Omega,\sigalg{F})$, variables $\RV{X}:\Omega\to X$ and $\RV{Y}:\Omega\to Y$ with $Y$ standard measurable and a probability model $\prob{P}_{\alpha}$ on $(\Omega,\sigalg{F})$. Then there exists a conditional $\prob{P}_\alpha^{\RV{Y}|\RV{X}}$.
\end{theorem}

\begin{proof}
This is a standard result, see for example \cite{cinlar_probability_2011} Theorem 2.18.
\end{proof}

\begin{theorem}[Existence of higher order valid conditionals with respect to probability sets]\label{th:ho_cond_psets}
Suppose we have a sample space $(\Omega,\sigalg{F})$, variables $\RV{X}:\Omega\to X$ and $\RV{Y}:\Omega\to Y$, $\RV{Z}:\Omega\to Z$ and a probability set $\prob{P}_{\{\}}$ with valid regular conditional $\prob{P}_{\{\}}^{\RV{YZ}|\RV{X}}$ and $Y$ and $Z$ standard measurable. Then there exists a valid regular $\prob{P}_{\{\}}^{\RV{Z}|(\RV{Y}|\RV{X})}$.
\end{theorem}

\begin{proof}
Given a Borel measurable map $m:X\kto Y\times Z$ let $f:Y\times Z\to Y$ be the projection onto $Y$. Then $f\circ (\RV{Y},\RV{Z})=\RV{Y}$. \citet{bogachev_kantorovich_2020}, Theorem 3.5 proves that there exists a Borel measurable map $n:X\times Y\kto Y\times Z$  such that 
\begin{align}
    n(\RV{Y}^{-1}(y)|x,y) = 1\label{eq:proper}\\
    m(\RV{Y}^{-1}(A)\cap B|x) = \int_A n(B|x,y) m\kernel{F}_{f}(dy|x)&\forall A\in \sigalg{Y},B\in\sigalg{Y\times Z}\label{eq:conditional}
\end{align}
In particular, $\prob{P}_{\{\}}^{\RV{YZ}|\RV{X}}$ is a Borel measurable map $X\kto Y\times Z$. Thus equation \ref{eq:conditional} implies for all $A\in \sigalg{Y},B\in\sigalg{Y\times Z}$

\begin{align}
    \prob{P}_{\{\}}^{\RV{YZ}|\RV{X}}(\RV{Y}^{-1}(A)\cap B|x) &= \int_A n(B|x,y) \prob{P}_{\{\}}^{\RV{YZ}|\RV{X}}\kernel{F}_{f}(dy|x)\\
    &=\int_A n(B|x,y) \prob{P}_{\{\}}^{\RV{Y}|\RV{X}}(dy|x)\label{eq:rec_push}
\end{align}

Where Equation \ref{eq:rec_push} follows from Lemma \ref{th:recurs_pushf}.

Then, for any $\prob{P}_\alpha\in\prob{P}_{\{\}}$

\begin{align}
    \prob{P}_{\{\}}^{\RV{YZ}|\RV{X}}(\RV{Y}^{-1}(A)\cap B|x) &= \int_A n(B|x,y) \prob{P}_{\alpha}^{\RV{Y}|\RV{X}}(dy|x)
\end{align}

which implies $n$ is a version of $\prob{P}_{\{\}}^{\RV{YZ}|(\RV{Y}|\RV{X})}$. Define $g:Y\times Z\to Z$ as the projection of $Y\times Z$ onto $Z$. By Lemma \ref{th:recurs_pushf}, $n\kernel{F}_{g}$ is a version of $\prob{P}_{\{\}}^{\RV{Z}|(\RV{Y}|\RV{X})}$.

We need to show that $n\kernel{F}_{g}$ is valid. We have validity of $\prob{P}_{\{\}}^{\RV{YZ}|\RV{X}}$ already. 
\end{proof}


\begin{theorem}[Higher order conditionals]\label{th:higher_order_conditionals}
Suppose we have a sample space $(\Omega,\sigalg{F})$, variables $\RV{X}:\Omega\to X$ and $\RV{Y}:\Omega\to Y$, $\RV{Z}:\Omega\to Z$ and a probability set $\prob{P}_{\{\}}$ with conditional $\prob{P}_{\{\}}^{\RV{YZ}|\RV{X}}$. Then $\prob{P}_{\{\}}^{\RV{Z}|(\RV{Y}|\RV{X})}$ is a version of $\prob{P}_{\{\}}^{\RV{Z}|\RV{Y}\RV{X}}$ 
\end{theorem}

\begin{proof}
For arbitrary $\prob{P}_{\alpha}\in \prob{P}_{\{\}}$
\begin{align}
    \prob{P}_\alpha^{\RV{YZ}|\RV{X}} &= \tikzfig{higher_order_disint}\\
    \implies \prob{P}_\alpha^{\RV{XYZ}} &= \tikzfig{higher_order_disint_0}\\
    &= \tikzfig{higher_order_disint_1}\\
    &= \tikzfig{higher_order_disint_2}
\end{align}
Thus $\prob{P}_{\{\}}^{\RV{Z}|(\RV{Y}|\RV{X})}$ is a version of $\prob{P}_{\alpha}^{\RV{Z}|\RV{Y}\RV{X}}$ for all $\alpha$ and hence also a version of $\prob{P}_{\{\}}^{\RV{Z}|\RV{Y}\RV{X}}$.
\end{proof}


\begin{theorem}
Given probability gap model $\prob{P}_{\{\}}$, $\RV{X}$, $\RV{Y}$, $\RV{Z}$ such that $\prob{P}_{\{\}}^{\RV{Z}|\RV{YX}}$ exists, $\prob{P}_{\{\}}^{\RV{Z}|\RV{Y}}$ exists iff $\RV{Z}\CI_{\prob{P}_{\{\}}} \RV{X}|\RV{Y}$.
\end{theorem}

\begin{proof}
If:
If $\RV{Z}\CI_{\prob{P}_{\{\}}} \RV{X}|\RV{Y}$ then by Theorem \ref{th:cho_ci_equiv}, for each $\prob{P}_\alpha\in \prob{P}_{\{\}}$ there exists $\prob{P}_{\alpha}^{\RV{Z}|\RV{Y}}$ such that
\begin{align}
    \prob{P}_\alpha^{\RV{Y}|\RV{WX}} &= \tikzfig{cond_indep_erase}
\end{align}
\end{proof}


\subsection{Validity}

Validity is related to \emph{proper} conditional probabilities. In particular, valid conditional probabilities exist when regular proper conditional probabilities exist.

\begin{definition}[Regular proper conditional probability]
Given a probability space $(\mu,\Omega,\sigalg{F})$ and a variable $\RV{X}:\Omega\to X$, a regular proper conditional probbability $\mu^{|\RV{X}}:X\kto \Omega$ is Markov kernel such that
\begin{align}
    \mu(A\cap \RV{X}^{-1}(B))&=\int_{B} \mu^{|\RV{X}}(A|x) \mu^{\RV{X}}(\mathrm{d}x) &\forall A\in \sigalg{X}, B\in \sigalg{F}\\
    &\iff\\
    \mu&= \tikzfig{disint_def_proper}\label{eq:conditional} 
\end{align}
and
\begin{align}
    \mu^{|\RV{X}}(\RV{X}^{-1}(A)|x) &= \delta_x(A)
\end{align}
\end{definition}

\begin{lemma}
Given a probability space $(\mu,\Omega,\sigalg{F})$ and variables $\RV{X}:\Omega\to X$, $\RV{Y}:\Omega\to Y$, if there is a regular proper conditional probbability $\mu^{|\RV{X}}:X\kto \Omega$ then there is a valid conditional distribution $\mu^{\RV{Y}|\RV{X}}$.
\end{lemma}

\begin{proof}
Take $\kernel{K}=\mu^{|\RV{X}}\kernel{F}_{\RV{Y}}$. We will show that $\kernel{K}$ is valid, and a version of $\mu^{\RV{Y}|\RV{X}}$.

Defining $\RV{O}:=\text{id}_{\Omega}$ (the identity function $\Omega\to \Omega$), $\mu^{|\RV{X}}$ is a version of $\mu^{\RV{O}|\RV{X}}$. Note also that $\RV{Y}=\RV{Y}\circ\RV{O}$. Thus by Lemma \ref{th:recurs_pushf}, $\kernel{K}$ is a version of $\mu^{\RV{Y}|\RV{X}}$.

It remains to be shown that $\kernel{K}$ is valid. Consider some $x\in X$, $A\in \sigalg{Y}$ such that $\RV{X}^{-1}(\{x\})\cap \RV{Y}^{-1}(A)=\emptyset$. Then by the assumption $\mu^{|\RV{X}}$ is proper
\begin{align}
    \kernel{K}(\RV{Y}\yields A|x) &= \delta_x(\RV{Y}^{-1}(A))\\
    &= 0
\end{align}

Thus $\kernel{K}$ is valid.
\end{proof}

\begin{theorem}[Validity]\label{th:completion}
Given $(\Omega,\sigalg{F})$, $\RV{X}:\Omega\to X$, $\kernel{J}\in \Delta(X)$ with $\Omega$ and $X$ standard measurable, there exists some $\mu\in \Delta(\Omega)$ such that $\mu^{\RV{X}}=\kernel{J}$ if and only if $\kernel{J}$ is a valid distribution.
\end{theorem}

\begin{proof}
If:
This is a Theorem 2.5 of \citet{ershov_extension_1975}.
Only if:
This is also found in \citet{ershov_extension_1975}, but is simple enough to reproduce here. Suppose $\kernel{J}$ is not a valid probability distribution. Then there is some $x\in X$ such that $\RV{X}\yields x = \emptyset$ but $\kernel{J}(x)>0$. Then
\begin{align}
    \mu^{\RV{X}}(x) &= \mu (\RV{X}\yields x)\\
    &= \sum_{x'\in X} \kernel{J}(x') \kernel{K}(\RV{X}\yields x|x')\\
    &= 0\\
    &\neq \kernel{J}(x)
\end{align}
\end{proof}


\begin{lemma}[Semidirect product is an intersection of probability sets]\label{th:intersection}
Given $(\Omega,\sigalg{F})$, $\RV{X}:\Omega\to (X,\sigalg{X})$, $\RV{Y}:\Omega\to (Y,\sigalg{Y})$, $\RV{Z}:\Omega\to (Z,\sigalg{Z})$ all standard measurable and valid candidate conditionals $\prob{P}_{\{\}}^{\RV{Y}|\RV{X}}$ and $\prob{Q}_{\{\}}^{\RV{Z}|\RV{YX}}$ defining probability sets $\prob{P}_{\{\}}$ and $\prob{Q}_{\{\}}$, then the probability set $\prob{R}_{\{\}}$ defined by $\prob{R}_{\{\}}^{\RV{YZ}|\RV{X}}:=\prob{P}_{\{\}}^{\RV{Y}|\RV{X}}\odot \prob{Q}_{\{\}}^{\RV{Z}|\RV{YX}}$ is equal to $\prob{P}_{\{\}}\cap\prob{Q}_{\{\}}$.
\end{lemma}

\begin{proof}
By assumption

\begin{align}
    \prob{R}_{\{\}}^{\RV{YZ}|\RV{X}}:=\prob{P}_{\{\}}^{\RV{Y}|\RV{X}}\odot \prob{Q}_{\{\}}^{\RV{Z}|\RV{YX}}
\end{align}

Therefore for any $\prob{R}_a\in\prob{R}_{\{\}}$

\begin{align}
    \prob{R}_{a}^{\RV{XYZ}} &= \prob{R}_a^{\RV{X}}\odot \prob{P}_{\{\}}^{\RV{Y}|\RV{X}}\odot \prob{Q}_{\{\}}^{\RV{Z}|\RV{YX}}\\
    \implies \prob{R}_{a}^{\RV{XY}} &= \prob{R}_a^{\RV{X}}\odot \prob{P}_{\{\}}^{\RV{Y}|\RV{X}}\\
    \land \prob{R}_{a}^{\RV{XYZ}} &= \prob{R}_{a}^{\RV{XY}}\odot \prob{Q}_{\{\}}^{\RV{Z}|\RV{YX}}
\end{align}

Thus $\prob{P}_{\{\}}^{\RV{Y}|\RV{X}}$ is a version of $\prob{R}_{\{\}}^{\RV{Y}|\RV{X}}$ and $\prob{Q}^{\RV{Z}|\RV{YX}}$ is a version of $\prob{R}_{\{\}}^{\RV{Z}|\RV{YX}}$ so $\prob{R}_{\{\}}\subset \prob{P}_{\{\}}\cap\prob{Q}_{\{\}}$.

Suppose there's an element $\prob{S}$ of $\prob{P}_{\{\}}\cap\prob{Q}_{\{\}}$ not in $\prob{R}_{\{\}}$. Then by definition of $\prob{R}_{\{\}}$, $\prob{R}_{\{\}}^{\RV{YZ}|\RV{X}}$ is not a version of $\prob{S}_{\{\}}^{\RV{YZ}|\RV{X}}$. But by construction of $\prob{S}$, $\prob{P}_{\{\}}^{\RV{Y}|\RV{X}}$  is a version of $\prob{S}^{\RV{Y}|\RV{X}}$ and  $\prob{Q}^{\RV{Z}|\RV{YX}}$ is a version of $\prob{S}^{\RV{Z}|\RV{YX}}$. But then by the definition of disintegration, $\prob{P}_{\{\}}^{\RV{Y}|\RV{X}} \odot \prob{Q}_{\{\}}^{\RV{Z}|\RV{YX}}$ is a version of $\prob{S}_{\{\}}^{\RV{YZ}|\RV{X}}$ and so $\prob{R}_{\{\}}^{\RV{YZ}|\RV{X}}$ is a version of $\prob{S}_{\{\}}^{\RV{YZ}|\RV{X}}$, a contradiction.
\end{proof}


\begin{lemma}[Equivalence of validity definitions]\label{th:valid_agree}
Given $\RV{X}:\Omega\to X$, with $\Omega$ and $X$ standard measurable, a probability measure $\prob{P}^{\RV{X}}\in \Delta(X)$ is valid if and only if the conditional $\prob{P}^{\RV{X}|*}:=*\mapsto \prob{P}^{\RV{X}}$ is valid.
\end{lemma}

\begin{proof}
$*\yields *=\Omega$ necessarily. Thus validity of $\prob{P}^{\RV{X}|*}$ means 

\begin{align}
    \forall A\in \sigalg{X}: \RV{X}\yields A=\emptyset \implies \prob{P}^{\RV{X}|*}(A|*)&=0
\end{align}

But $\prob{P}^{\RV{X}|*}(A|*)=\prob{P}^{\RV{X}}(A)$ by definition, so this is equivalent to

\begin{align}
    \forall A\in \sigalg{X}: \RV{X}\yields A=\emptyset \implies \prob{P}^{\RV{X}}(A)&=0
\end{align}
\end{proof}


\begin{lemma}[Copy-product of valid candidate conditionals is valid]\label{lem:valid_extendability}
Given $(\Omega,\sigalg{F})$, $\RV{X}:\Omega\to X$, $\RV{Y}:\Omega\to Y$, $\RV{Z}:\Omega\to Z$ (all spaces standard measurable) and any valid candidate conditional $\prob{P}^{\RV{Y}|\RV{X}}$ and $\prob{Q}^{\RV{Z}|\RV{Y}\RV{X}}$, $ \prob{P}^{\RV{Y}|\RV{X}}\odot \prob{Q}^{\RV{Z}|\RV{Y}\RV{X}}$ is also a valid candidate conditional.
\end{lemma}

\begin{proof}
Let $\prob{R}^{\RV{YZ}|\RV{X}}:=\prob{P}^{\RV{Y}|\RV{X}}\odot \prob{Q}^{\RV{Z}|\RV{Y}\RV{X}}$.

We only need to check validity for each $x\in \RV{X}(\Omega)$, as it is automatically satisfied for other values of $\RV{X}$.

For all $x\in \RV{X}(\Omega)$, $B\in \sigalg{Y}$ such that $\RV{X}\yields \{x\}\cap\RV{Y}\yields B=\emptyset$, $\prob{P}^{\RV{Y}|\RV{X}}(B|x)=0$ by validity. Thus for arbitrary $C\in \sigalg{Z}$
\begin{align}
    \prob{R}^{\RV{YZ}|\RV{X}}(B\times C|x) &= \int_B \prob{Q}^{\RV{Z}|\RV{YX}}(C|y,x)\prob{P}^{\RV{Y}|\RV{X}}(dy|x)\\
                                  &\leq \prob{P}^{\RV{Y}|\RV{X}}(B|x)\\
                                  &=0
\end{align}

For all $\{x\}\times B$such that $\RV{X}\yields \{x\}\cap\RV{Y}\yields B\neq \emptyset$ and $C\in \sigalg{Z}$ such that $(\RV{X},\RV{Y},\RV{Z})\yields \{x\}\times B\times C=\emptyset$, $\prob{Q}^{\RV{Z}|\RV{YX}}(C|y,x)=0$ for all $y\in B$ by validity. Thus:
\begin{align}
    \prob{R}^{\RV{YZ}|\RV{X}}(B\times C|x) &= \int_B \prob{Q}^{\RV{Z}|\RV{YX}}(C|y,x)\prob{P}^{\RV{Y}|\RV{X}}(dy|x)\\
                                            &=0
\end{align}
\end{proof}

\begin{corollary}[Valid conditionals are validly extendable to valid distributions]\label{corr:valid_extend_order1}
Given $\Omega$, $\RV{U}:\Omega\to U$, $\RV{W}:\Omega\to W$ and a valid conditional $\prob{T}^{\RV{W}|\RV{U}}$, then for any valid conditional $\prob{V}^{\RV{U}}$, $\prob{V}^{\RV{U}}\odot \prob{T}^{\RV{W}|\RV{U}}$ is a valid probability.
\end{corollary}

\begin{proof}
Applying Lemma \ref{lem:valid_extendability} choosing $\RV{X}=*$, $\RV{Y}=\RV{U}$, $\RV{Z}=\RV{W}$ and $\prob{P}^{\RV{Y}|\RV{X}}=\prob{V}^{\RV{U}|*}$ and $\prob{Q}^{\RV{Z}|\RV{YX}}=\prob{T}^{\RV{W}|\RV{U*}}$ we have $\prob{R}^{WU|*}:=\prob{V}^{\RV{U}|*}\odot \prob{T}^{\RV{W}|\RV{U}*}$ is a valid conditional probability. Then $\prob{R}^{\RV{WU}}\cong \prob{R}^{\RV{WU}|*}$ is valid by Theorem \ref{th:valid_agree}.
\end{proof}

\begin{theorem}[Validity of conditional probabilities]\label{th:valid_conditional_probability}
Suppose we have $\Omega$, $\RV{X}:\Omega\to X$, $\RV{Y}:\Omega\to Y$, with $\Omega$, $X$, $Y$ discrete. A conditional $\prob{T}^{\RV{Y}|\RV{X}}$ is valid if and only if for all valid candidate distributions $\prob{V}^{\RV{X}}$, $\prob{V}^{\RV{X}}\odot \prob{T}^{\RV{Y}|\RV{X}}$ is also a valid candidate distribution.
\end{theorem}

\begin{proof}
If: this follows directly from Corollary \ref{corr:valid_extend_order1}.

Only if: suppose $\prob{T}^{\RV{Y}|\RV{X}}$ is invalid. Then there is some $x\in X$, $y\in Y$ such that $\RV{X}\yields(x)\neq \emptyset$, $(\RV{X},\RV{Y})\yields(x,y)=\emptyset$ and $\prob{T}^{\RV{Y}|\RV{X}}(y|x)>0$. Choose $\prob{V}^{\RV{X}}$ such that $\prob{V}^{\RV{X}}(\{x\})=1$; this is possible due to standard measurability and valid due to $\RV{X}^{-1}(x)\neq \emptyset$. Then
\begin{align}
    (\prob{V}^{\RV{X}}\odot \prob{T}^{\RV{Y}|\RV{X}})(x,y) &= \prob{T}^{\RV{Y}|\RV{X}}(y|x) \prob{V}^{\RV{X}}(x)\\
                                                                     &= \prob{T}^{\RV{Y}|\RV{X}}(y|x)\\
                                                                     &>0
\end{align}
Hence $\prob{V}^{\RV{X}}\odot \prob{T}^{\RV{Y}|\RV{X}}$ is invalid.
\end{proof}

\begin{theorem}[Existence of valid conditional probabilities]\label{th:valid_disint}
Given a probability gap model $\prob{P}_\square:A\to \Delta(\Omega)$ along with a valid conditional probability $\model{P}_\square^{\RV{XY}|\RV{W}}$, there exists a valid conditional probability $\prob{P}_\square^{\RV{Y}|\RV{WX}}$.
\end{theorem}

\begin{proof}
From Lemma \ref{lem:disint_exist}, we have the existence of some Markov kernel $\prob{P}_\square^{\RV{Y}|\RV{WX}}:W\times X\to Y$ such that
\begin{align}
    \prob{P}_\square^{\RV{XY}|\RV{W}}=\prob{P}_\square^{\RV{X}|\RV{W}}\odot \prob{P}_\square^{\RV{Y}|\RV{WX}}\label{eq:k_disint}
\end{align}

By definition of conditional probability , for any insert $\alpha\in A$ there exists $\prob{P}_\alpha^{\RV{W}}\in\Delta(W)$ such that

\begin{align}
    \prob{P}_\alpha^{\RV{WXY}}=\prob{P}_\alpha^{\RV{W}}\odot\prob{P}_\square^{\RV{XY}|\RV{W}}
\end{align}

Thus

\begin{align}
\prob{P}_\alpha^{\RV{WXY}}&= \prob{P}_\alpha^{\RV{W}}\odot(\prob{P}_\square^{\RV{X}|\RV{W}}\odot \prob{P}_\square^{\RV{Y}|\RV{WX}})\\
&= (\prob{P}_\alpha^{\RV{W}}\odot\prob{P}_\square^{\RV{X}|\RV{W}})\odot \prob{P}_\square^{\RV{Y}|\RV{WX}})
\end{align}

Let $\text{erasef}_Y:Y\to \{*\}$ be the erase function on $Y$ and $\text{id}_{W\times X}$ be the identity function on $W\times X$. Noting that 
\begin{align}
(\RV{W},\RV{X})&=(\text{idf}_{W\times X}\otimes \text{erasef}_Y)\circ (\RV{W},\RV{X},\RV{Y})
\end{align}
By Lemma \ref{lem:prod_pushf} together with Theorem \ref{th:recurs_pushf} we have for all $\alpha$:

\begin{align}
    \prob{P}_\alpha^{\RV{XW}} &= \prob{P}_\alpha^{\RV{WXY}}(\text{id}_{W\times X}\otimes \text{erase}_Y)\\
                              &= \prob{P}_\alpha^{\RV{W}}\odot(\prob{P}_\square^{\RV{X}|\RV{W}}\odot \prob{P}_\square^{\RV{Y}|\RV{WX}})(\text{id}_{W\times X}\otimes \text{erase}_Y)\\
                              &= \prob{P}_\alpha^{\RV{W}}\odot\prob{P}_\square^{\RV{X}|\RV{W}}
\end{align}

Then

\begin{align}
\prob{P}_\alpha^{\RV{XWY}}&= (\prob{P}_\alpha^{\RV{XW}})\odot \prob{P}_\square^{\RV{Y}|\RV{WX}})
\end{align}

And so $\prob{P}_\square^{\RV{Y}|\RV{WX}})$ is a $\RV{Y}|\RV{WX}$ conditional probability. We also want it to be valid, so we will verify that it can be chosen as such.

We also need to check that $\prob{P}_\square^{\RV{Y}|\RV{WX}}$ can be chosen so that it is valid. By validity of $\model{K}^{\RV{W,Y}|\RV{X}}$, $w\in \RV{W}(\Omega)$ and $(\RV{X},\RV{W},\RV{Y})\yields(x,w,y)=\emptyset \implies \model{P}_\square^{\RV{W,Y}|\RV{X}}=0$, so we only need to check for $(w,x,y)$ such that $\model{P}_\square^{\RV{W,Y}|\RV{X}}(w,y|x)=0$. For all $x,y$ such that $\kernel{K}^{\RV{Y}|\RV{X}}(y|x)$ is positive, we have $\model{P}^{\RV{W,Y}|\RV{X}}(w,y|x)=0\implies \prob{P}_\square^{\RV{Y}|\RV{WX}}(y|w,x)=0$. Furthermore, where $\model{K}^{\RV{W}|\RV{X}}(w|x)=0$, we either have $(\RV{W},\RV{X})\yields(w,x)=\emptyset$ or we can choose some $\omega\in (\RV{W},\RV{X})\yields(w,x)$ and let $\prob{P}^{\RV{Y}|\RV{WX}}(\RV{Y}(\omega)|w,x) = 1$.
\end{proof}

\subsection{Extended conditional independence}\label{ap:eci}

\todo[inline]{Needs a support condition}

In the case of a probability gap model $(\prob{P}_{\square}^{\RV{V}|\RV{W}},A)$ where there is some $\alpha\in A$ dominating $A$, we can relate conditional independence with respect to $\prob{P}_\square$ to what \citet{constantinou_extended_2017} \emph{extended conditional independence}, which is a notion they define with respect to a Markov kernel. These concepts may differ if $A$ is not dominated. Theorem 4.4 of \citet{constantinou_extended_2017} proves the following claim:

\begin{theorem}\label{th:dawid_constantionou}
Let $\RV{A}^*=\RV{A}\circ \RV{V}$, $\RV{B}^*=\RV{B}\circ\RV{V}$, $\RV{C}^*=\RV{C}\circ \RV{V}$ ($(\RV{A},\RV{B},\RV{C})$ are $\sigalg{V}$-measurable) and $\RV{D}^*=\RV{D}\circ \RV{W}$,$\RV{E}^*=\RV{E}\circ \RV{W}$ where $W$ is discrete and $\RV{W}=(\RV{D}^*,\RV{E}^*)$. In addition, let $\prob{P}_\alpha^{\RV{W}}$ be some probability distribution on $\RV{W}$ such that $w\in\RV{W}(\Omega)\implies \prob{P}_\alpha^{\RV{W}}(w)>0$. Then, denoting extended conditional independence with $\CI_{\prob{P},\text{ext}}$ and $\prob{P}_\alpha^{\RV{VW}}:=\prob{P}_\alpha^{\RV{W}}\odot \prob{P}^{\RV{V}|\RV{W}}$
\begin{align}
    \RV{A}\CI_{\prob{P},\text{ext}}(\RV{B},\RV{D})|(\RV{C},\RV{E})\iff \RV{A}^*\CI_{\prob{P}_\alpha}(\RV{B}^*,\RV{D}^*)|(\RV{C}^*,\RV{E}^*)
\end{align}
Where $\CI_{\prob{P}_\alpha}$ is order 0 conditional independence.
\end{theorem}

This result implies a close relationship between order 1 condtional independence and extended conditional independence.

\begin{theorem}
Let $\RV{A}^*=\RV{A}\circ \RV{V}$, $\RV{B}^*=\RV{B}\circ\RV{V}$, $\RV{C}^*=\RV{C}\circ \RV{V}$ ($(\RV{A},\RV{B},\RV{C})$ are $\sigalg{V}$-measurable) and $\RV{D}^*=\RV{D}\circ \RV{W}$,$\RV{E}^*=\RV{E}\circ \RV{W}$ where $V,W$ are discrete and $\RV{W}=(\RV{D}^*,\RV{E}^*)$. Then letting $\prob{P}_\alpha^{\RV{VW}}:=\prob{P}_\alpha^{\RV{W}}\odot \prob{P}^{\RV{V}|\RV{W}}$
\begin{align}
    \RV{A}\CI^1_{\prob{P},\text{ext}}(\RV{B},\RV{D})|(\RV{C},\RV{E})\iff \RV{A}^*\CI_{\prob{P}}(\RV{B}^*,\RV{D}^*)|(\RV{C}^*,\RV{E}^*)
\end{align}
\end{theorem}

\begin{proof}
If:

By assumption, $\RV{A}^*\CI_{\prob{P}_\alpha}(\RV{B}^*,\RV{D}^*)|(\RV{C}^*,\RV{E}^*)$ for all $\prob{P}_\alpha^{\RV{D^*E^*}}$. In particular, this holds for some $\prob{P}_\alpha^{\RV{D^*E^*}}$ such that $(d,e)\in (\RV{D}^*,\RV{E}^*)(\Omega)\implies \prob{P}_\alpha^{\RV{D^*E^*}}(d,e) >0$. Then by Theorem \ref{th:dawid_constantionou}, $\RV{A}\CI_{\prob{P},\text{ext}}(\RV{B},\RV{D})|(\RV{C},\RV{E})$.

Only if:

For any $\beta$, $\prob{P}_\beta^{\RV{ABC}|\RV{DE}}= \prob{P}_\beta^{\RV{DE}}\odot \prob{P}^{\RV{ABC}|\RV{DE}}$. By Lemma \ref{lem:disint_exist}, we have $\prob{P}^{\RV{A}|\RV{BCDE}}$ such that

\begin{align}
    \prob{P}_\beta^{\RV{A^*B^*C^*}\RV{D^*E^*}} &= \prob{P}_\beta^{\RV{D^*E^*}}\odot \prob{P}^{\RV{B^*C^*}|\RV{D^*E^*}}\odot \prob{P}^{\RV{A^*}|\RV{B^*C^*D^*E^*}}\\
                                      &= \prob{P}_\beta^{\RV{B^*C^*D^*E^*}}\odot \prob{P}^{\RV{A^*}|\RV{B^*C^*D^*E^*}}\\
                                      &= \prob{P}_\beta^{\RV{C^*E^*}}\odot \prob{P}_\beta^{\RV{B^*D^*}|\RV{C^*E^*}}\odot \prob{P}^{\RV{A}^*|\RV{B^*C^*D^*E^*}}
\end{align}

By Theorem \ref{th:dawid_constantionou}, we have some $\alpha$ such that $\prob{P}_\alpha^{\RV{D^*E^*}}$ is strictly positive on the range of $(\RV{D}^*,\RV{E}^*)$ and $\RV{A}^*\CI_{\prob{P}_\alpha}(\RV{B}^*,\RV{D}^*)|(\RV{C}^*,\RV{E}^*)$.

By independence, for some version of $\prob{P}^{\RV{A}|\RV{BCDE}}$:

\begin{align}
    \prob{P}_\alpha^{\RV{C^*E^*}}\odot \prob{P}_\alpha^{\RV{B^*D^*}|\RV{C^*E^*}}\odot \prob{P}^{\RV{A}^*|\RV{B^*C^*D^*E^*}} &= \tikzfig{indep_strengthen_1}\\
    &= \tikzfig{indep_strengthen_2}\\
    &= \prob{P}_\alpha^{\RV{C^*E^*}}\odot \prob{P}_\alpha^{\RV{B^*D^*}|\RV{C^*E^*}}\odot (\prob{P}_\alpha^{\RV{A}^*|\RV{C^*E^*}}\otimes\text{erase}_{BD})
\end{align}

Thus for any $(a,b,c,d,e)\in A\times B\times C\times D\times E$ such that $\prob{P}_\alpha^{\RV{B^*C^*D^*E^*}}(b,c,d,e)>0$, $\prob{P}^{\RV{A}^*|\RV{B^*C^*D^*E^*}}(a|b,c,d,e) = \prob{P}_\alpha^{\RV{A}^*|\RV{C^*E^*}}(a|c,e)$. However, by assumption, $\prob{P}_\alpha^{\RV{B^*C^*D^*E^*}}(b,c,d,e)=0 \implies \prob{P}_\beta^{\RV{B^*C^*D^*E^*}}(b,c,d,e)=0$, and so $\prob{P}_\beta^{\RV{A}^*|\RV{B^*C^*D^*E^*}}= \prob{P}_\alpha^{\RV{A}^*|\RV{C^*E^*}}(a|c,e)$ everywhere except a set of $\prob{P}_\beta$-measure 0. Thus
    
\begin{align}
    \prob{P}_\beta^{\RV{A^*B^*C^*}\RV{D^*E^*}} &= \tikzfig{indep_strengthen_3}\\
    &= \tikzfig{indep_strengthen_4}
\end{align}
\end{proof}

Conditional independence is a property of variables, we define ``unresponsiveness'' as a property of Markov kernels.

\begin{definition}[Unresponsiveness]
Given discrete $\Omega$, a probability gap model $\prob{P}_\square:A\to \Delta(\Omega)$, variables $\RV{W}:\Omega\to W$, $\RV{X}:\Omega\to X$, $\RV{Y}:\Omega\to Y$, if there is some version of the conditional probability $\prob{P}^{\RV{Y}|\RV{WX}}$ and $\prob{P}_\square^{\RV{Y}|\RV{W}}$ such that
\begin{align}
  \prob{P}_\square^{\RV{Y}|\RV{WX}} &= \tikzfig{cond_indep_erase} \label{eq:higherorder_ci_erase}
\end{align}
then $\prob{P}_\square^{\RV{Y}|\RV{WX}}$ is \emph{unresponsive} to $\RV{X}$.
\end{definition}

\begin{definition}[Domination]
Given a probability set $\prob{P}_{\{\}}\subset \Delta(\Omega)$, $\prob{P}_\alpha$ dominates $\prob{P}_{\{\}}$ if $\prob{P}_\beta(B)>0\implies \prob{P}_\alpha(B)>0$ for all $\prob{P}_\beta in \prob{P}_{\{\}}$, $B\in \sigalg{F}$.
\end{definition}

\begin{theorem}[Conditional independence from kernel unresponsiveness]\label{th:cons_ci}
Given standard measurable $\Omega$, variables $\RV{W}:\Omega\to W$, $\RV{X}:\Omega\to X$, $\RV{Y}:\Omega\to Y$ and a probability set $\prob{P}_{\{\}}:A\to \Delta(\Omega)$ with conditional probability $\prob{P}_{\{\}}^{\RV{Y}|\RV{WX}}$ such that there is $\prob{P}_\alpha\in \prob{P}_{\{\}}$ dominating $\prob{P}_{\{\}}$, $\RV{Y}\CI_{\prob{P}_{\{\}}}\RV{X}|\RV{W}$ if and only if there is a version of $\prob{P}_{\{\}}^{\RV{Y}|\RV{WX}}$ unresponsive to $\RV{W}$. 
\end{theorem}

\begin{proof}
If:
For every $\alpha\in A$ we can write
\begin{align}
  \prob{P}_\alpha^{\RV{Y}|\RV{WX}} &= \tikzfig{cond_indep_erase_alpha}
\end{align}
And so, by Theorem \ref{th:cho_ci_equiv}, $\RV{Y}\CI_{\prob{P}_\alpha}\RV{X}|\RV{W}$ for all $\alpha\in A$, and so $\RV{Y}\CI_{\prob{P}_{\{\}}}\RV{X}|\RV{W}$.
Only if:
For $\prob{P}_{\alpha}$ dominating $\prob{P}_{\{\}}$, by Theorem \ref{th:cho_ci_equiv}, there exists a version of $\prob{P}_\alpha^{\RV{Y}|\RV{WX}}$ unresponsive to $\RV{W}$. Because $\prob{P}_{\alpha}$ dominates $\prob{P}_{\{\}}$, $\prob{P}_\alpha^{\RV{Y}|\RV{WX}}$ differs from $\prob{P}_\beta^{\RV{Y}|\RV{WX}}$ on a set of measure 0 for any $\prob{P}_{\beta}\in \prob{P}_{\{\}}$, thus $\prob{P}_\alpha^{\RV{Y}|\RV{WX}}$ is a version of $\prob{P}_{\{\}}^{\RV{Y}|\RV{WX}}$ also.
\end{proof}

\begin{corollary}\label{cor:ci_cp_exist}
Given standard measurable $\Omega$, variables $\RV{W}:\Omega\to W$, $\RV{X}:\Omega\to X$, $\RV{Y}:\Omega\to Y$ and a probability set $\prob{P}_{\{\}}:A\to \Delta(\Omega)$ with conditional probability $\prob{P}_{\{\}}^{\RV{Y}|\RV{WX}}$, $\prob{P}_{\{\}}^{\RV{Y}|\RV{W}}$ exists if $\RV{Y}\CI_{\prob{P}_{\{\}}}\RV{X}|\RV{W}$.
\end{corollary}

\begin{proof}
By Theorem \ref{th:cons_ci}, there is $\kernel{K}:W\kto Y$ such that for all $\alpha$
\begin{align}
    \prob{P}_{\alpha}^{\RV{WY}} &= \tikzfig{conditional_independence_cprob_exist}\\
    &= \tikzfig{conditional_independence_cprob_exist2}\\
    &= \tikzfig{conditional_independence_cprob_exist3}
\end{align}

Thus $\kernel{K}$ is a version of $\prob{P}_{\{\}}^{\RV{Y}|\RV{W}}$.
\end{proof}

This result can fail to hold in the absence of the domination condition. Consider $A$ a collection of inserts that all deterministically set a variable $\RV{X}$; then for any variable $\RV{Y}$ $\RV{Y}\CI_{\prob{P}_\square} \RV{X}$ because $\RV{X}$ is deterministic for any $\alpha\in A$. But $\prob{P}_\square^{\RV{Y}|\RV{X}}$ is not necessarily unresponsive to $\RV{X}$.

Note that in the absence of the assumption of the existence of $\prob{P}_\square^{\RV{Y}|\RV{WX}}$, $\RV{Y}\CI_{\prob{P}_\square}\RV{X}|\RV{W}$ does \emph{not} imply the existence of $\prob{P}_\square^{\RV{Y}|\RV{W}}$. If we have, for example, $A=\{\alpha,\beta\}$ and $\prob{P}_\alpha^{\RV{XY}}$ is two flips of a fair coin while $\prob{P}_\beta^{\RV{XY}}$ is two flips of a biased coin, then $\RV{Y}\CI_{\prob{P}}\RV{X}$ but $\prob{P}^{\RV{Y}}$ does not exist.

